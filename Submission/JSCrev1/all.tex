\documentclass[amsthm]{elsart}
\usepackage{yjsco}
\usepackage{natbib}
%\usepackage{bbm,fullpage}
\usepackage{bm}
\usepackage{amsmath,amssymb,amsthm}
\usepackage{amssymb}
\usepackage{algpseudocode}
\usepackage{algorithm}%, pseudocode}
\usepackage{mathrsfs}
\usepackage{enumitem}
\usepackage[titles]{tocloft}
\usepackage{yfonts}
\usepackage{xcolor}
\usepackage[pdftex,                %
    bookmarks         = true,%     % Signets
    bookmarksnumbered = true,%     % Signets numérotés
    pdfpagemode       = None,%     % Signets/vignettes fermé à l'ouverture
    pdfstartview      = FitH,%     % La page prend toute la largeur
    pdfpagelayout     = SinglePage,% Vue par page
    colorlinks        = true,%     % Liens en couleur
    linkcolor= blue, %    % couleur des liens internes
    anchorcolor= blue, %    % couleur des liens internes
    citecolor         =blue,
    urlcolor          = magenta,%  % Couleur des liens externes
    pdfborder         = {0 0 0}%   % Style de bordure : ici, pas de bordure
    ]{hyperref}%                   % Utilisation de HyperTeX

\hypersetup{
linkcolor=blue,
citecolor=red, %OrangeRed, %Emerald, %OliveGreen,
urlcolor=blue, %BlueViolet %NavyBlue
}
\hypersetup{colorlinks=true}
\makeatletter

%\usepackage{amsmath}
%\usepackage{easybmat}
%\usepackage{multirow,bigdelim}

\newcommand*\hexbrace[2]{%
  \underset{#2}{\underbrace{\rule{#1}{0pt}}}}

%\usepackage[indexonlyfirst,ucmark,toc]{glossaries}
%\renewcommand*{\glstextformat}[1]{\textcolor{black}{#1}}
%%\glsdisablehyper %pour enlever les liens hypertexte


\def\pollambda{{\textcolor{black}{\mathfrak{l}}}}
\def\polmu{{\textcolor{black}{\mathfrak{m}}}}
\def\linearmu{{\textcolor{black}{\mathfrak{h}}}}
\def\polnu{{\textcolor{black}{\mathfrak{n}}}}
\def\poldelta{{\textcolor{black}{\mathfrak{d}}}}
%% A simple open set
\def\singSBasicOpen{{\textcolor{black}{\ensuremath{\mathcal{U}}}}}
\def\BasicOpen{{\textcolor{black}{\ensuremath{\mathcal{O}}}}}
\def\UOpen{{\textcolor{black}{\ensuremath{\mathcal{U}}}}}
\def\openB{{\textcolor{black}{\ensuremath{\mathcal{B}}}}}
%locally closed set
\def\lcs{{\textcolor{black}{\ensuremath{V^\circ}}}}
\def\lcswi{{\textcolor{black}{\ensuremath{Y_i^\circ}}}}
\def\lcsw{{\textcolor{black}{\ensuremath{Y^\circ}}}}
\def\SlocallyclosediA{{\textcolor{black}{\ensuremath{S_{i,\mA}^\circ}}}}
\def\Slocallyclosed{{\textcolor{black}{\ensuremath{S^\circ}}}}
\def\algxAX{{\textcolor{black}{\mathfrak{X}_{\mA,\x}^\circ}}}
\def\algiAX{{\textcolor{black}{\mathfrak{X}_{i,\mA}^\circ}}}
\def\algAX{{\textcolor{black}{\mathfrak{X}_\mA^\circ}}}
\def\algX{{\textcolor{black}{\mathfrak{X}^\circ}}}
%notation a changer car dans la section E, \Lambda est utilise comme un polynome 
\def\lcLambda{{\textcolor{black}{Y^\circ}}}
%% Zariski closed
\def\closedX{{\textcolor{black}{X}}}
\def\closedZ{{\textcolor{black}{Z}}}
\def\algZ{{\textcolor{black}{Z}}}
\def\alg2Z{{\textcolor{black}{Z}}}
\def\Vmp{{\textcolor{black}{V^{mp}}}}

%%Prime ideal
\def\Prime{{\ensuremath{\mathcal{P}}}}

%%% tau racine de q
\def\roottau{{\alpha}}
%%% tau noeud de l'arbre
\def\nodetau{{\tau}}
\def\nodeoherletter{{\tilde\tau}}
%%% rho qui etait utilise dans geosolve-new pour la racine bivarie
\def\rootrho{{\xi}}
\def\depth{r}

%% Zariski open for A12
\def\ZOA12{\ensuremath{\mathscr{E}}}
\def\ZOpropRK{\ensuremath{\mathscr{A}}}
\def\ZOdeltapropRK{\ensuremath{\mathcal{U}}}
\def\ZOK{\ensuremath{\mathscr{K}}}
\def\ZOomega{\textcolor{black}{\ensuremath{\mathscr{M}}}}



\def\Clos#1{\ensuremath{\overline{\mathscr{U}#1}}}


\def\scrOpen{{\textcolor{black}{\ensuremath{\mathscr{G}}}}}
\def\scrIopen{{\textcolor{black}{\ensuremath{\mathscr{I}}}}}



\def\scrS{\ensuremath{\mathscr{S}}}

%% used to be K_4, becomes K_3
\def\Klastindex{3}
%% used to be K_5, becomes K_4
\def\Klastlastindex{4}

%% used in Section E. Should be changed? 
\def\scrX{\textcolor{black}{\ensuremath{\mathscr{X}}}}


\def\sfH{\mathsf{H}}

%\def\fbr{\ensuremath{\mathrm{fbr}}}
%\def\lag{\ensuremath{\mathsf{Lag}}}
%\def\reg{\ensuremath{{\rm reg}}}
%\def\freg{\ensuremath{V_{\rm reg}}}
%\def\oreg{\ensuremath{v_{\rm reg}}}
%\def\sing{\ensuremath{{\rm sing}}}
%\def\GL{\ensuremath{{\rm GL}}}
%\def\C {\ensuremath{\mathbf{C}}}
%\def\R {\ensuremath{\mathbf{R}}}
%\def\QQ {\ensuremath{\mathbf{Q}}}
%\def\scrQ{\ensuremath{\mathscr{Q}}}
% \def\Proj{\ensuremath{\mathscr{U}}}
% \def\Cons{\ensuremath{\mathscr{C}}}
% \def\Clos{\ensuremath{\mathscr{V}}}

%\def\scrP{\ensuremath{\mathscr{P}}} On peut l'enlever


\def\degQ{\ensuremath{\kappa}}
\def\degR{\ensuremath{\gamma}}
\def\degS{\ensuremath{\sigma}}
\def\degC{\ensuremath{\mu}}
\def\degB{\ensuremath{\beta}}
\def\degfiber{\ensuremath{\gamma}}

\def\mult{\ensuremath{\mathsf{M}}}
\def\softO{\ensuremath{{O}{\,\tilde{ }\,}}}
\def\dalgo{{\ensuremath{\tilde d}}}
\def\myminor{{\ensuremath{\Delta}}}
\def\minor{{\ensuremath{m'}}}
\def\pminor{{\ensuremath{m''}}}




\def\dinit{\ensuremath{{d_{\rho}}}}
\def\frakD{\ensuremath{\mathfrak{D}}}

\def\Vc{\ensuremath{\mathcal{V}}}
\def\A {\ensuremath{\mathbb{A}}}
\def\B {\ensuremath{\mathbb{B}}}
\def\Co {\ensuremath{\mathbb{C}}}
\def\CC {\ensuremath{\overline{\mathbf{K}}}}
\def\PP {\ensuremath{\mathsf{P}}}
\def\BB {\ensuremath{H}}
\def\P {\ensuremath{\mathbb{P}}}
\def\Q {\ensuremath{\mathbb{Q}}}
\def\N {\ensuremath{\mathbb{N}}}
\def\Re {\ensuremath{\mathbb{R}}}
\def\RP {\ensuremath{\tilde{\mathbf{R}}}}
\def\CP {\ensuremath{\tilde{\mathbf{C}}}}
\def\Z {\ensuremath{\mathbb{Z}}}
\def\RR {\ensuremath{\mathbb{K}}}
\def\k{\ensuremath{\mathbb{k}}}
\def\kbar{\ensuremath{\bar{\mathbb{k}}}}

\def\K {\ensuremath{\mathbb{K}}}
\def\KKbar {\ensuremath{\overline{\mathbf{K}}}}
\def\BS {\ensuremath{{B}}}
\def\CS {\ensuremath{{C}}}
\def\RS {\ensuremath{\mathsf{R}}}
\def\LL {\ensuremath{\mathbf{L}}}
\def\GG {\ensuremath{\mathbf{G}}}
\def\KK {\ensuremath{\mathbf{K}}}
\def\D {\ensuremath{\mathbf{D}}}
\def\n {\ensuremath{\mathbf{n}}}
\def\d {\ensuremath{\mathbf{d}}}
\def\p {\ensuremath{\mathbf{p}}}
\def\I {\ensuremath{\mathbf{I}}}
\def\mI {\ensuremath{\mathbf{I}}}
\def\mS {\ensuremath{\mathbf{S}}}
\def\mzero {\ensuremath{\mathbf{0}}}
\def\mT {\ensuremath{\mathbf{T}}}
\def\mJ {\ensuremath{\mathbf{J}}}
\def\JJ {\ensuremath{\mathbf{J}}}
\def\sfB {\ensuremath{\sf B}}

\def\MM {\ensuremath{\mathbb{M}}}
\def\mm {\ensuremath{\mathbf{m}}}
\def\v {\ensuremath{\mathbf{v}}}
\def\vc {\ensuremath{\mathbf{c}}}
\def\vd {\ensuremath{\mathbf{d}}}

\def\mA{\ensuremath{\mathbf{A}}}
\def\mG{\ensuremath{\mathbf{G}}}
\def\mH{\ensuremath{\mathbf{H}}}
\def\mB{\ensuremath{\mathbf{B}}}
\def\mC{\ensuremath{\mathbf{B}}}% sinon conflit avec les complexes
\def\mD{\ensuremath{\mathbf{D}}}
\def\mR{\ensuremath{\mathbf{R}}}
\def\mM{\ensuremath{\mathbf{M}}}
\def\G {\ensuremath{\mathrm{GL}}}
\def\Mat{\ensuremath{{\rm M}}}
\def\grad{\ensuremath{{\rm grad}}}
\def\ker{\ensuremath{{\rm ker}}}
\def\rank{\ensuremath{{\rm rank}}}
\def\jac{\ensuremath{{\rm jac}}}
\def\ext{\ensuremath{{\rm ext}}}
\def\Zons{\ensuremath{\mathscr{D}}}
\def\scrG{\ensuremath{\mathscr{G}}}
\def\scrZ{\ensuremath{\mathscr{Z}}}
\def\scrA{\ensuremath{\mathcal{A}}}
\def\scrT{\ensuremath{\mathscr{T}}}
\def\scrL{\ensuremath{\mathscr{L}}}
\def\scrU{\ensuremath{\mathcal{U}}}
\def\scrZ{\ensuremath{\mathscr{Z}}}
\def\scrR{\ensuremath{\mathscr{R}}}
\def\scrY{\ensuremath{\mathscr{Y}}}
\def\scrB{\ensuremath{\mathscr{B}}}

\def\y {\ensuremath{\mathbf{y}}}
\def\e {\ensuremath{\mathbf{e}}}
\def\a {\ensuremath{\mathbf{a}}}
\def\b {\ensuremath{\mathbf{b}}}
\def\z {\ensuremath{\mathbf{z}}}
\def\w {\ensuremath{\mathbf{w}}}
\def\f {\ensuremath{\mathbf{f}}}
\def\r {\ensuremath{\mathbf{r}}}
\def\s {\ensuremath{\mathbf{s}}}
\def\L {\ensuremath{\mathbf{L}}}
\def\F {\ensuremath{\mathbf{F}}}
\def\G {\ensuremath{\mathbf{G}}}
\def\E {\ensuremath{\mathbf{E}}}
\def\X {\ensuremath{\mathbf{X}}}
\def\Y {\ensuremath{\mathbf{Y}}}
\def\H {\ensuremath{\mathbf{H}}}
\def\bfA {\ensuremath{\mathbf{A}}}
\def\m {\ensuremath{\mathfrak{m}}}
\def\v {\ensuremath{\mathbf{v}}}
\def\u {\ensuremath{\mathbf{u}}}
\def\q {\ensuremath{\mathbf{q}}}
\def\U {\ensuremath{\mathbf{U}}}
\def\V {\ensuremath{\mathbf{V}}}
\def\t {\ensuremath{\mathbf{t}}}


\def\blambda{\mbox{\boldmath$\lambda$}}
\def\bdelta{\mbox{\boldmath$\delta$}}
\def\bzeta{\mbox{\boldmath$\zeta$}}
\def\bkappa{\mbox{\boldmath$\kappa$}}
\def\bmu{\mbox{\boldmath$\mu$}}
\def\bsigma{\mbox{\boldmath$\sigma$}}
\def\bpsi{\mbox{\boldmath$\psi$}}
\def\bphi{\mbox{\boldmath$\phi$}}
\def\sbdelta{\mbox{\scriptsize \boldmath$\delta$}}
\def\brho{\mbox{\boldmath$\rho$}}
\def\bell{\mbox{\boldmath$\ell$}}
\def\sbell{\mbox{\scriptsize \boldmath$\ell$}}

\def\Vfiber{\ensuremath{V''}}
\def\fiber2{\ensuremath{Q''}}
\def\fibersing2{\ensuremath{S''}}
\def\param2{\ensuremath{\mathscr{Q}''}}
\def\paramsing2{\ensuremath{\mathscr{S}''}}

\def\gcd{\ensuremath{\mathrm{GCD}}}

\def\cstn {\ensuremath{7}}
\def\CONST {\ensuremath{4}}


\def\control{{C}}
\def\controleps{{C_\eps}}

\def\rmp{{R}}
\def\rmplift{{R_\eps}}
\def\rmpproj{{S_\eps}}
\def\rmpprojspec#1{{S_{#1}}}

\def\limeps {\lim_{\varepsilon\rightarrow 0}}
\def\eps{\varepsilon}
\def\Qeps{\mathbf{Q}(\varepsilon)}
\def\Reps{\mathbf{R}\langle\varepsilon\rangle}
\def\Ceps{\mathbf{C}\langle\varepsilon\rangle}
\def\Veps{V_\varepsilon}
\def\ProjVeps{S_\varepsilon}

\def\liftplus#1{{#1}_+}
\def\liftmoins#1{{#1}_-}
\def\lext#1{\mathcal{#1}}

\def\elim {\mathsf{Elim}}
\def\newn{{k}}
\def\myr{{r}}
\def\myrank{{r}}
\def\newg#1 {\mathsf{ElimLagrange}(#1) }%{{\tilde{\mathbf{g}}}}
\def\newf#1 {\mathsf{ElimPolar}(#1 ) } % {{\tilde{\mathbf{f}}}}
\def\frakg{{\tilde{\mathfrak{g}}}}
\def\frakf{{\tilde{\mathfrak{f}}}}
\def\bm{{\mathbf{m}}}
\def\mybound {\mathscr{B}}
\def\newset {Y}
\def\leftkernel {K}

\def\xxx{\textbf{(xxx)}}
\def\why{\textbf{(why?)}}
\def\todo#1{(\textbf{todo:} #1)}
\def\check{\textbf{(check!)}}
\def\adeg{\ensuremath{\varepsilon\deg}}

\def\Bez#1#2{\ensuremath{\mathscr{C}_{#1}(#2)}}

\def\KKbar {\ensuremath{\overline{\mathbf{K}}}}
 
\DeclareBoldMathCommand{\bM}{M}
\DeclareBoldMathCommand{\bX}{X}
\DeclareBoldMathCommand{\bi}{i}
\DeclareBoldMathCommand{\bY}{Y}
\DeclareBoldMathCommand{\bn}{n}
\DeclareBoldMathCommand{\bbb}{b}
\DeclareBoldMathCommand{\d}{d}
\DeclareBoldMathCommand{\bc}{c}
\DeclareBoldMathCommand{\be}{e}
\DeclareBoldMathCommand{\bh}{h}
\DeclareBoldMathCommand{\bu}{u}
\DeclareBoldMathCommand{\bx}{x}
\DeclareBoldMathCommand{\by}{y}
\DeclareBoldMathCommand{\btheta}{\vartheta}
\DeclareBoldMathCommand{\f}{f}
\DeclareBoldMathCommand{\g}{g}
\DeclareBoldMathCommand{\h}{h}
\DeclareBoldMathCommand{\x}{x}
\DeclareBoldMathCommand{\bell}{\ell}
\DeclareBoldMathCommand{\bbeta}{\beta}
\DeclareBoldMathCommand{\balpha}{\alpha}
\def\coeff{\ensuremath{\mathrm{coeff}}}
\def\mdeg{\ensuremath{\mathrm{mdeg}}}
\def\pollambda{\lambda}
\def\roottau{{\alpha}}
\def\Zeroes{Z}

\def\mult{\ensuremath{\mathrm{mult}}}





\def\NOTE#1#2{{\begin{quote}\marginpar[\hfill{#1}]{{#1}}{{\textsf{[\![{#2}]\!]}}}\end{quote}}}
\def\respond#1{\NOTE{\textcircled{\textsc{a}}}{Note:~{#1}}}


%% \institute{J.D. Hauenstein \at
%% Department of Applied and Computational Mathematics and Statistics, 
%%  University of Notre Dame, USA \\
%%  \email{hauenstein@nd.edu}
%%  \and M. {Safey El Din} \at
%%  Sorbonne Universit\'e, CNRS, INRIA, Laboratoire d'Informatique de Paris 6, PolSys, Paris, France \\
%%  \email{mohab.safey@lip6.fr}
%%  \and 
%%  \'E. Schost \at
%%  David Cheriton School of Computer Science, University of Waterloo, ON, Canada \\
%%  \email{eschost@uwaterloo.ca}
%%  \and T.X. Vu
%%  David Cheriton School of Computer Science, University of Waterloo, ON, Canada \\
%%  Sorbonne Universit\'e, CNRS, INRIA, Laboratoire d'Informatique de Paris 6, PolSys, Paris, France \\
%%  \email{Thi-Xuan.Vu@lip6.fr}
%% }

%% \date{Received: date / Accepted: date}

\newtheorem{pbm}{Problem}
\newtheorem{definition}{Definition}
\newtheorem{theorem}[definition]{Theorem}
\newtheorem{corollary}[definition]{Corollary}
\newtheorem{proposition}[definition]{Proposition}
\newtheorem*{propositionnonumber}{Proposition}
\newtheorem{lemma}[definition]{Lemma}
\newtheorem{remark}[definition]{Remark}
\newtheorem{example}[definition]{Example}

\def\added#1{\textcolor{red}{#1}}

\begin{document}
\begin{frontmatter}
\title{Solving determinantal systems using 
homotopy techniques}
\author{Jon D. Hauenstein}
\address{Department of Applied and Computational Mathematics and Statistics, University of Notre Dame, USA}
\ead{hauenstein@nd.edu}
\author{Mohab {Safey El Din}}
\address{Sorbonne Universit\'e, CNRS, Laboratoire d'Informatique de Paris 6, PolSys, Paris, France}
\ead{mohab.safey@lip6.fr}
\author{\'Eric Schost}
\address{David Cheriton School of Computer Science, University of Waterloo, ON, Canada}
\ead{eschost@uwaterloo.ca}
\author{Thi Xuan Vu}
\address{Sorbonne Universit\'e, CNRS, Laboratoire d'Informatique de Paris
  6, PolSys, Paris, France and David Cheriton School of Computer Science, University of Waterloo, ON, Canada}
\ead{txvu@uwaterloo.ca}

\begin{abstract}
  Let $\KK$ be a field of characteristic zero and let $\KKbar$ be an algebraic
  closure of $\KK$. Consider a sequence of polynomials $\mG=(g_1,\dots,g_s)$ in
  $\KK[X_1,\dots,X_n]$ \added{with $s < n$}, a polynomial matrix $\mF=[f_{i,j}] \in
  \KK[X_1,\dots,X_n]^{p \times q}$, with $p \leq q$ \added{and $n=q-p+s+1$}, and the algebraic set
  $\VpFG{p}{\mF}{\mG}$ of points in $\KKbar$ at which all polynomials in $\mG$
  and all $p$-minors of $\mF$ vanish. Such polynomial systems appear naturally
  in polynomial optimization or computational geometry.

  We provide bounds on the number of isolated points in $\VpFG{p}{\mF}{\mG}$
  depending on the maxima of the degrees in rows (resp.\ columns) of $\mF$ and
  we design \added{probabilistic} homotopy algorithms for computing those
  points. These algorithms take advantage of the determinantal structure of the
  system defining $\VpFG{p}{\mF}{\mG}$. In particular, the algorithms run in
  time that is polynomial in the bound on the number of isolated points.
\end{abstract}
\begin{keyword}
  Polynomial system solving, Homotopy, Symbolic Computation, Determinantal
  algebraic sets, Complexity
\end{keyword}
\end{frontmatter}
\endNoHyper

\section{Introduction}\label{sec:intro}
Throughout this paper, $\KK$ is a field of characteristic zero with algebraic closure
$\KKbar$, $(X_1, \ldots, X_n)$ is a set of $n$ variables, and
$\KK[X_1,\dots,X_n]$ is the multivariate polynomial ring in $n$ variables with
coefficients in $\KK$. With this setup, let $\mF=[f_{i,j}] \in
\KK[X_1,\dots,X_n]^{p \times q}$ be a polynomial matrix, with $p \leq q$ and
$\mG = (g_1, \ldots, g_s)$ in $\KK[X_1, \ldots, X_n]$.

The central question which interests us here is to describe the set
$$\VpFG{p}{\mF}{\mG} = \{\bx \in \KKbar{}^n \mid
\mathrm{rank}(\mF({\bx})) < p \text{~and~} g_1(\bx)=\cdots=g_s(\bx)=0
\}$$

For any matrix $\mF$ over a ring $R$, and for any integer $r$,
$M_r(\mF)$ will denote the set of $r$-minors of $\mF$. For any subset $I$ in
$\KK[X_1,\dots,X_n]$, $V(I)$ will denote the zero-set of $I$ in
$\KKbar{}^n$, and for a matrix $\mF$ with entries in
$\KK[X_1,\dots,X_n]$, we will write $V_r(\mF)=V(M_r(\mF))$. Thus, for
$\mF$ of size $p \times q$, with $p \le q$, 
$$\VpF{p}{\mF}=\{\bx \in \KKbar{}^n \mid \mathrm{rank}(\mF({\bx})) < p\}$$
and $\VpFG{p}{\mF}{\mG}$ is an algebraic set as it is the intersection of
$V(\mG)$ with $\VpF{p}{\mF}$.

% We refer to e.g.~\cite[Chap.\ I and II]{Shafarevich77} for notions related to
% dimension and irreducible components of algebraic sets.
Results due to Macaulay~\cite{Macaulay16} and Eagon and Northcott~\cite{EN62}
imply that all irreducible components of $\VpF{p}{\mF}$ have co-dimension at
most $q-p+1$. Hence, the irreducible components of $\VpFG{p}{\mF}{\mG}$ have
co-dimension at most $q-p+s+1$. Hence, for the above problem, it is natural to
assume that $n = q-p+s+1$.

Even under this assumption, $\VpFG{p}{\mF}{\mG}$ may have positive dimensional
components but we will be interested in describing only its {\em isolated
  points}, that is, the points in the irreducible components of
$\VpFG{p}{\mF}{\mG}$ of dimension zero (this notion makes sense for any
algebraically closed field $\KK$; when $\KK=\mathbb{C}$, these points are indeed
isolated for the metric topology).
Our main problem is then the following.
\begin{pbm} \label{problem2} For $\mF \in \KK[X_1,\dots,X_n]^{p \times q}$ and
  $\mG=(g_1,\dots,g_s)$ in $\KK[X_1,\dots,X_n]$ such that $p \leq q$ and $n =
  q-p+s+1$, compute the isolated points of $\VpFG{p}{\mF}{\mG}$.
\end{pbm}

This problem appears in a variety of context; prominent examples are
optimization problems~\cite{GSZ10,JP14,BGHS14,GS14,NDS06}, and related questions
in real algebraic
geometry~\cite{ARS,BaGiHeMb01,BaGiHePa05,BGHSS,BRSS,RealDecompICMS,CellDecompSurface,BertiniReal,RealNumerical,SaSc03,SaSc11,SaSc17},
where $\mF$ consists of the Jacobian matrix of $\mG$, together with one extra
row, corresponding to the gradient of a function that we want to optimize on
$V(\mG)$. We will refer to this particular class of inputs as systems {\em
  coming from optimization}. %  (in such cases, $q=n$ and $p=s+1$, so our assumption
% $n=q-p+s+1$ holds).

In several of these situations, we are only interested in the solutions of the
system made of minors $M_p(\mF)$ and $\mG=(g_1,\dots,g_s)$ at which the
associated Jacobian matrix has full rank. This set of solutions is finite and is
always contained in the set of isolated points of $\VpFG{p}{\mF}{\mG}$
\cite[Theorem 16.19]{Eisenbud95}; we call these points {\em simple points}. % For
% instance, the set of simple points coincides with $\VpFG{p}{\mF}{\mG}$ when the
% minors $M_p(\mF)$ and $g_1,\dots,g_s$ generate a radical ideal of dimension
% zero; this case appears frequently in algorithms in real algebraic
% geometry~\cite{BGHSS}. 
Hence, we also consider the following variant of
Problem~\eqref{problem2}.

\begin{pbm} \label{problem3} For 
  $\mF \in \KK[X_1,\dots,X_n]^{p \times q}$ and 
  $\mG=(g_1,\dots,g_s)$ in $\KK[X_1,\dots,X_n]$ with $p \leq q$ and
  $n = q-p+s+1$, compute the simple  points of~$\VpFG{p}{\mF}{\mG}$.
\end{pbm}

We will represent the output of our algorithms using univariate polynomials. Let
$V \subset \KKbar{}^n$ be a zero-dimensional algebraic set defined over $\KK$. A
\emph{zero-dimensional parametrization} $\scrR = ((w,v_1, \ldots, v_n),
\lambda)$ of $V$ consists of polynomials $(w,v_1, \ldots, v_n)$ such that $w$ 
is in
$\KK[Y]$ ($Y$ is a new variable), monic and squarefree, all $v_i$'s are in
$\KK[Y]$ and satisfy $\deg(v_i) < \deg(w)$, and $\lambda$ is a $\KK$-linear form
in $n$ variables, such that
\begin{itemize}
\item $\lambda(v_1, \ldots, v_n) = Yw'$ mod $w$, with $w'=\frac{d w}{d Y}$;
\item we have $V = Z(\scrR)$, with $$Z(\scrR)= \left\{\left(\frac{v_1(\tau)}{w'(\tau)}, \ldots, \frac{v_n(\tau)}{w'(\tau)}\right) \ | \ w(\tau) = 0\right\}.$$
\end{itemize}
% The constraint on $\lambda$ then says that the roots of $w$ are the
% values taken by $\lambda$ on $V$. 
This representation was introduced in~\cite{Kronecker82,Macaulay16}, and has
been used in a variety of
algorithms~\cite{GiMo89,GiHeMoPa95,ABRW,GiHeMoMoPa98,Rouillier99,GiLeSa01}. We
use a rational parametrization, with $w'$ as a denominator, as
in~\cite{ABRW,Rouillier99,GiLeSa01} to control precisely the bit-size of the
coefficients when $\KK=\Q$ or the degrees in $T$ when $\KK=k(T)$, for a field
$k$ (see~\cite{Schost03,DaSc04}). 

{\bf Main results.} Our first result gives bounds on the number of solutions of
$\VpFG{p}{\mF}{\mG}$, counted with {\em multiplicities}. We will consider two
degree measures for the matrix $\mF$ which have been used before in~\cite{NieRan09,MiSt04}. For
$i=1,\dots,p$, we will write $\rdeg(\mF,i)$ for the degree of the $i$th row of
$\mF$, that is, $\rdeg(\mF,i)=\max(\deg(f_{i,j}))_{1 \le j \le q}$; similarly,
for $j=1,\dots,q$, we write $\cdeg(\mF,j)$ for the degree of the $j$th column of
$\mF$, that is, $\cdeg(\mF,j)=\max(\deg(f_{i,j}))_{1 \le i \le p}$. Further, for
$k \ge 0$ and integers $\delta_1,\dots,\delta_q$,
$$E_k(\delta_1,\dots,\delta_q)=\sum_{1\leq i_1 < \cdots < i_k \leq
  q}\delta_{i_1} \cdots \delta_{i_k}$$ is the elementary symmetric
polynomial of degree $k$ in $(\delta_1, \ldots, \delta_q)$; 
for integers $\alpha_1,\dots,\alpha_p$,
$$S_k(\alpha_1,\dots,\alpha_p) = \sum_{i_1+\cdots+i_p=k, i_j \geq
  0}\alpha_1^{i_1}\cdots\alpha_p^{i_p}$$ is the $k$th complete
symmetric polynomial in $(\alpha_1,\dots,\alpha_p)$.

Finally, we recall the notion of multiplicity of a point $\bx$ with respect to
an ideal $I$ in $\KKbar[X_1,\dots,X_n]$; it extends to ideals in
$\KK[X_1, \ldots, X_n]$ by considering their extension in $\KKbar[X_1, \ldots,
X_n]$. The ideal $I$ can be written
as % a ``minimal'' intersection of finitely many primary components, that is,
$I=Q_1\cap\cdots \cap Q_r$ for some primary ideals $Q_1,\dots,Q_r$; with
$V(Q_i)\neq V(Q_j)$ for $i\neq j$. Take $\bx$ isolated in $V(I)$; then there
exists a unique primary component $Q_i$, which must have dimension zero, such
that $\bx$ is in $V(Q_i)$. Because we take a primary decomposition over
$\KKbar$, we actually have $V(Q_i)=\{\bx\}$. % Although minimal primary
% decompositions are not unique, t
The fact that $\bx$ is isolated implies
that $Q_i$ does not depend on the primary decomposition of $I$ we
consider. Then, the \emph{multiplicity} of $\bx$ is defined as the
dimension of $\KKbar[X_1,\dots,X_n]/Q_i$. % When $\bx=0\in\KKbar{}^n$,
% the dimension of $\KKbar[X_1,\dots,X_n]/Q_i$ is the same as that of
% $\KKbar[[X_1, \ldots, X_n]]/I$, where $\KK[[X_1, \ldots, X_n]]$
% denotes the formal power series ring in $X_1, \ldots, X_n$ with
% coefficients in $\KKbar$~\cite[Theorem 4.2.2]{CLO_UAG}.

%All this being said, the following is our first~result.

\begin{theorem}\label{theo:1}
  Let $\mF$ be in $\KK[X_1,\dots,X_n]^{p \times q}$ and let
  $\mG=(g_1,\dots,g_s)$ be in $\KK[X_1,\dots,X_n]$, with $p \le q$ and
  $n=q-p+s+1$. Then, the sum of the multiplicities of the isolated
  points of the ideal generated by the $p$-minors of $\mF$ and 
$ g_1,\dots,g_s$ is at most
  $\min(c,c')$ with
$$c=\deg(g_1) \cdots \deg(g_s) E_{n-s}(\cdeg(\mF,1), \ldots, \cdeg(\mF,q))$$
and
$$c'=\deg(g_1) \cdots \deg(g_s) S_{n-s}(\rdeg(\mF,1), \ldots, \rdeg(\mF,p)).$$
\end{theorem}

When $\rdeg(\mF,i)=\cdeg(\mF,j)=d$ for all $i,j$, both bounds given above
coincide to $\deg(g_1) \cdots \deg(g_s) {q \choose {p-1}} d^{n-s}$. Else,
either of the two expressions can be the minimum.

Our second result gives bounds on the cost of computing a
zero-dimensional paramet\-rization of the isolated solutions of
$\VpFG{p}{\mF}{\mG}$. Our
algorithms take as input a \emph{straight-line program} (that is, a
sequence of elementary operations $+, -, \times$) that computes the
entries of $\mF$ and $\mG$; the
\emph{length $\sigma$} of the input is the number of operations it
performs. This assumption is not restrictive as $\mF$
and $\mG$ can be computed by a straight-line program (a
naive solution would consist in computing and adding all monomials in
$\mF$ and $\mG$).

\begin{theorem}\label{theo:2}
  Suppose that $\mF \in \KK[X_1,\dots,X_n]^{p \times q}$ and 
  $\mG=(g_1,\dots,g_s)$ in $\KK[X_1,\dots,X_n]$ are given by a straight-line
  program of length $\sigma$ and that $\deg(g_1),\dots,\deg(g_s)$, as well
  as $\cdeg(\mF,1), \ldots, \cdeg(\mF,q)$ and $\rdeg(\mF,1), \ldots,
  \rdeg(\mF,p)$ are all at least equal to $1$.

  Then, there exist randomized algorithms that solve
  Problem~\eqref{problem2} in either
   $$\softO\left (
     {q \choose p} c(e+c^5 )\big(\sigma + q \delta + \gamma  \big )
   \right)$$
  operations in $\KK$, with
  \begin{align*}
    c&=\deg(g_1)\cdots\deg(g_s)\ E_{n-s}(\cdeg(\mF,1), \ldots, \cdeg(\mF,q))\\
    e&=(\deg(g_1)+1)\cdots(\deg(g_s)+1)\ E_{n-s}(\cdeg(\mF,1)+1, \ldots, \cdeg(\mF,q)+1),\\
    \gamma&= \max(\deg(g_i), 1\leq i \leq s)\\
    \delta &= \max(\cdeg(\mF,i), 1\leq i \leq q)
  \end{align*}
  or 
   $$\softO\left (
     {q \choose p} c'(e'+{c'}^5 )\big(\sigma + p \alpha  +\gamma \big )
   \right)$$
  operations in $\KK$, with 
\begin{align*}
  c'&=\deg(g_1)\cdots\deg(g_s)\ S_{n-s}(\rdeg(\mF,1), \ldots, \rdeg(\mF,p))\\
  e'&=(\deg(g_1)+1)\cdots(\deg(g_s)+1)\ S_{n-s}(\rdeg(\mF,1)+1, \ldots, \rdeg(\mF,p)+1),\\
    \gamma&= \max(\deg(g_i), 1\leq i \leq s)\\
    \alpha &= \max(\rdeg(\mF,j), 1\leq j \leq p).
\end{align*}
\end{theorem}
\added{The assumption that all degrees are at least $1$ is not a strong
  restriction. If $\deg(g_i)=0$ for some $i$, $g_i$ is a constant. So either the
  system is inconsistent (if $g_i \ne 0$) or $g_i$ can be discarded, $s$
  decreases and $n > q-p+s+1$ after this update which implies that there is no
  isolated point in $\VpFG{p}{\mF}{\mG}$. Similarly, if say $\cdeg(\mF,i)=0$,
  the $i$th column of $\mF$ consists of constants. After applying linear
  combinations with coefficients in $\KK$ to the rows of $\mF$, we may assume
  that all entries in the $i$th column, except at most one, are zero without
  changing the column degrees. The $i$th column of $\mF$ and the row of the
  non-zero entry can then be discarded ; $p$ and $q$ decrease by $1$ and we
  still have $n= q-p+s+1$.}

Remark further that in the common situation where all degrees $\deg(g_i)$,
$\rdeg(\mF,i)$ and $\cdeg(\mF,j)$ involved in the formulas above are at least
equal to $2$, we have the inequalities $e \le c^2$, $e' \le {c'}{}^2$ and
$\binom{q}{p}\leq c$, $\binom{q}{p}\leq c'$. As a result, the runtimes are {\em
  polynomial} in $c,\sigma$ and $c',\sigma$: they respectively become $\softO
(c^8 \sigma)$ and $\softO ({c'}^8 \sigma)$ \added{(but recall that $c, c'$
  become $d^n\binom{q}{p-1}$ when $\cdeg(\mF, i)=\rdeg(\mF, j)=\deg(g_k)=d$ for
  all $i,j,k$)}. Note that by Theorem~\ref{theo:1}, 
that $\min(c,c')$ is an upper bound for the output size of
such~algorithms.

For solving Problem~\eqref{problem3}, one obtains slightly better
complexity estimates. 

\begin{theorem}\label{theo:3}
  Suppose that the matrix $\mF \in \KK[X_1,\dots,X_n]^{p \times q}$
  and polynomials $\mG=(g_1,\dots,g_s)$ in $\KK[X_1,\dots,X_n]$ are
  given by a straight-line program of length $\sigma$. Assume that
  $\deg(g_1),\dots,\deg(g_s)$, as well as
  $\cdeg(\mF,1), \ldots, \cdeg(\mF,q)$ and
  $\rdeg(\mF,1), \ldots, \rdeg(\mF,p)$ are all at least equal to $1$.

  Then, there exist randomized algorithms that solve
  Problem~\eqref{problem3} in either
   $$\softO\left (
     {q \choose p} ce\big(\sigma + q \delta + \gamma  \big )
   \right)$$
or 
   $$\softO\left (
     {q \choose p} c'e'\big(\sigma + p \alpha  +\gamma \big )
   \right)$$
  operations in $\KK$, 
  all notations being as in Theorem~\ref{theo:3}.
\end{theorem}
As above, in the common situation where all degrees involved are at
least $2$, the runtimes are {\it polynomial} in $c, \sigma$ and
$c',\sigma'$; they respectively become $\softO (c^5 \sigma)$ and $\softO
({c'}^5 \sigma)$.


The probabilistic aspects are as follows: at several steps, the
algorithms on which Theorems~\ref{theo:2} and~\ref{theo:3} rely will
draw elements from the base field at random. In all cases, there
exists an algebraic hypersurface $\cal H$ of the parameter space such
that success is guaranteed for all choices of parameters not
in~$\cal H$.

{\bf Prior works.} Although results in a vein similar to Theorem~\ref{theo:1}
have already been published, we are not aware of previous statements as above,
with no assumption on the dimension of $V_p(\mF,\mG)$, and that take into
account multiplicities as is done in Theorem~\ref{theo:1}.

Pioneering work of Giambelli-Thom-Porteous (see e.g. \cite{FP06} or
\cite{Fu92}) already established similar bounds under regularity
assumptions (when $V(\mG)$ is smooth and/or $V_p(\mF, \mG)$ has the
expected dimension). Previous work by Miller and
Sturmfels~\cite[Chapter~15]{MiSt04} proved general results on the
multi-degrees of determinantal ideals built from matrices with
indeterminate entries (here $s=0$, but the assumption
$n=q-p+1$ does not hold); they obtain analogues (and
generalizations) of the result in Theorem~\ref{theo:1} in that
context.

Nie and Ranestad proved in~\cite{NieRan09} that the bounds in
Theorem~\ref{theo:1} are tight for two families of polynomials
% (in a similar context where the polynomials are homogeneous in
% $n+1$ variables):
\begin{itemize}
\item when entries of $\mF$ are generic and homogeneous, and
 such that $\deg(f_{i,j}) = \cdeg(\mF,j)$ for all $i,j$, the ideal
 generated by $M_p(\mF)$ has degree $E_{n}(\cdeg(\mF,1), \ldots, \cdeg(\mF,q))$;
\item when entries of $\mF$ are generic and homogeneous, and such that
  $\deg(f_{i,j}) = \rdeg(\mF,i)$ for all $i,j$, the ideal generated by
  $M_p(\mF)$ has degree $S_{n}(\rdeg(\mF,1), \ldots, \rdeg(\mF,p))$.
\end{itemize}
From this, they deduce that the degree of the ideal generated by the
$p$-minors of $\mF$ and $ g_1,\dots,g_s $ is at most \sloppy
$\deg(g_1) \cdots \deg(g_s) S_{n-s}(\rdeg(\mF,1), \ldots,
\rdeg(\mF,p))$, for systems coming from optimization, assuming that
this ideal has dimension zero. Spaenlehauer also gave
in~\cite{Spa14} explicitely the Hilbert function of the
ideal above, for a generic input.


Our algorithms are based on a {\em symbolic homotopy continuation}. Following
early work in the 1930's, such as~\cite{Lahaye34}, homotopy continuation
algorithms have become a foundational tool for numerical algorithms. We mention for
instance~\cite{AlGe03} for an extensive list of references, Shub and Smale's
work on the complexity of these techniques (see e.g.~\cite{ShSm93}), or
works by Morgan, Sommese, Wampler emphasizing the underlying algebraic geometry
(see e.g.~\cite{BertiniBook,SoWa05}), as well
as~\cite{Ver94,Ver09,AdVe13} for dedicated numerical homotopy algorithms to take
into account sparsity in polynomial systems.

By contrast, the usage of homotopy methods in symbolic contexts is
more recent% , even though some early results, such as Bernstein's proof
% of the so-called BKK theorem, already involve Puiseux series
% manipulations~\cite{Bernstein75}
. References such
as~\cite{HeKrPuSaWa99,BoMaWaWa04} deal with systems with no particular
structure, or systems with no zeros at infinity~\cite{PaSa04}.
Further work extended this idea to sparse systems (in the polyhedral
sense)~\cite{JeMaSoWa09,HeJeSa10,HeJeSa13,HeJeSa14} and
multihomogeneous systems~\cite{HeJeSaSo02,SaSc16}.  In~\cite{SaSc16},
these techniques are used to solve Problem~\eqref{problem3}, but the
complexity estimates obtained there depend on multi-homogeneous
B\'ezout bounds involving the maxima of $\rdeg(\mF, 1), \ldots,
\rdeg(\mF, p)$ or $\cdeg(\mF,1), \ldots, \cdeg(\mF, q)$.

Most aforementioned algorithms solve {\em square} systems, that is, systems with
as many equations as unknowns; though extensions can deal with systems of
positive dimension by using variants of algorithms for square systems. Note that
using slack variables as in \cite{SoVe00}, polyhedral homotopies apply to
overdetermined systems but the control of their complexities is not known. Some
dedicated homotopies have been designed for special overdetermined systems as in
\cite{BV00, SVV10}.
% One
% notable exception is \cite{SVV10}, providing dedicated homotopy algorithms to
% solve Schubert problems which are formulated with rank conditions on some
% special matrices~\cite{LDSVV18}. These work through the Littlewood Richardson
% rule and some combinatorial construction. 
As far as we know, they cannot be used
to solve determinantal systems as the ones we tackle in this paper.

In this paper, we deal with determinantal
systems of equations, which are in essence over-determined; this is
made possible by the algebraic properties of determinantal ideals.

Starting from the determination of the Hilbert function of a determinantal ring
due to Conca and Herzog~\cite{CH94}, complexity estimates are given in
\cite{FSS13,FSS12} for computing Gr\"obner bases of ideals generated by either
$M_r(\mF)$ when $r\leq p\leq q$, or $M_{p}(\mF),g_1,\dots,g_s$, (for inputs
coming from optimization problems), but under some genericity assumptions on the
entries of $\mF$ or $\mG$; also the input polynomials must have all the same
degree. Thes works culminated with the result by Spaenlehauer in \cite{Spa14}:
he removes this latter degree assumption and provides sharp complexity
statements, still under genericity assumptions.

Systems encoding rank defects in polynomial matrices have also been studied in
the scope of the so-called geometric resolution algorithm in
\cite{BaGiHeLeMaSo15} and \cite{SaSp16}. The algorithms in these references
solve only Problem~\eqref{problem3} (isolated solutions which are not simple are
not considered in that line of work). \added{Computing isolated points of
  $\VpFG{p}{\mF}{\mG}$ could be done using Lecerf's equidimensional
  decomposition algorithm, still based on the geometric resolution \cite{Lecerf2000}. }
The cost of these algorithms is quadratic in certain geometric quantities (the
degree of algebraic sets defined by subsystems of the determinantal equations we
are dealing with); this is to be compared with the runtimes in
Theorem~\ref{theo:3}, where the main contributions are the products ${q \choose
  p} c e$, respectively ${q \choose p} c' e'$, and where $ce$, resp.\ $c'e'$,
are also of a geometric nature. Further work is needed to compare the degrees
involved in these complexity estimates with ours, and the resulting runtimes.

{\bf Overview of the paper and illustration of the main results.} We start with
$\mF$ and $\mG$ as above and build the equations
$\bC=(c_1,\dots,c_{s},\dots,c_m)$ that we want to solve:
$(c_1,\dots,c_{s})=(g_1,\dots,g_s)$, and $(c_{s+1},\dots,c_{m})$ are the
$p$-minors of $\mF$.

\begin{example}\label{ex:1}
  Throughout the paper, we will consider the following example, with $\KK=\Q$,
  $n=2$, $p=2$ and $q=3$ (so $n=q-p+1$), and we let $\mF \in \K[X_1,X_2]^{2
    \times 3}$ be given by
  $$\mF=\begin{bmatrix}
  X_1+X_2-1 & 3X_1+5X_2+2  & 10X_1+X_2-1\\
  X_2^2+X_1+10X_2+3   & X_1^2+3X_1X_2+X_1-1  & X_1^2-4X_1X_2+X_2^2+3
  \end{bmatrix}.$$
  The maximal minors $\bC=M_2(\mF)$ of this matrix are
  \begin{align*}
   c_1&= -7X_1^3 - 38X_1^2X_2 - 7X_1^2 - 20X_1X_2^2 - 6X_1X_2 + 20X_1 + 5X_2^3 +    2X_2^2 + 16X_2 + 5,\\
   c_2&=    X_1^3 - 3X_1^2X_2 - 11X_1^2 - 13X_1X_2^2 - 97X_1X_2 - 26X_1 - 10X_2^2 +    10X_2,\\
   c_3&=    X_1^3 + 4X_1^2X_2 - 3X_1^2 - 37X_1X_2 - 13X_1 - 5X_2^3 - 52X_2^2 - 36X_2 - 5.
  \end{align*}
%  They have $7$ common solutions, which we will describe below.
\end{example}


As a preliminary, we will need an algorithm which takes as input polynomials
$\bC$ and a point $\bx$ in the zero-set of $\bC$, and which decides whether
$\bx$ is an isolated point of $V(\bC)$ (this will be used to solve
Problem~\eqref{problem2}). When a bound $\mu$ is known on the multiplicity of
$\bx$ as a root of $\bC$, it becomes possible to solve this problem in time
polynomial in the number of equations $m$, the number of variables $n$, the
bound $\mu$, and the complexity of evaluation $\sigma$ of $\bC$. This is
detailed in Section~\ref{sec:isolated}, where we explain how to modify an
algorithm by Mourrain~\cite{Mourrain97} and adapt it to our context.

\begin{example}\label{ex:2}
  In Example~\ref{ex:1}, the maximal minors $\bC = M_2(\mF)$ generate a radical
  ideal of dimension zero, so Problems~\ref{problem2} and~\ref{problem3} admit the same
  answer. There are 7 solutions, which are described by means of the univariate
  representation $\scrR=((w,v_1,v_2),\lambda)$, with 
{\small  \begin{align*}
w &= Y^7 + \frac{5249}{285}Y^6 + \frac{5899}{76}Y^5 - \frac{32593}{950}Y^4 - \frac{719401}{5700}Y^3 
        - \frac{302473}{5700}Y^2 - \frac{1243}{475}Y + \frac{379}{1140}\\[1mm]
v_1&= -\frac{461}{114}Y^6 - \frac{39047}{380}Y^5 - \frac{2431807}{2850}Y^4 - \frac{87697}{76}Y^3 - \frac{560363}{1900}Y^2 
      + \frac{64121}{570}Y + \frac{1341}{76}\\[1mm]
v_2 &= -\frac{5249}{285}Y^6 - \frac{5899}{38}Y^5 + \frac{97779}{950}Y^4 + \frac{719401}{1425}Y^3 
       + \frac{302473}{1140}Y^2 + \frac{7458}{475}Y - \frac{2653}{1140}.
         \end{align*}}
       and $\lambda = X_2$. The coordinates of the solutions are the values taken by $(v_1/w',v_2/w')$
       at the roots of $w$.

    In our example, we have no polynomials $\mG$, so $s=0$. The column degrees of
  $\mF$ are $(\cdeg(\mF,1),\cdeg(\mF,2),\cdeg(\mF,3))=(2,2,2)$, whereas its
  row degrees are $(\rdeg(\mF,1),\rdeg(\mF,2))=(1,2)$. 
  Using Theorem~\ref{theo:1}, the column degree bound is $c=E_2(2,2,2) = 2^2 + 2^2 +
  2^2 =12$, the row degree bound is $c'=S_2(1,2) = 1^2 + 1\cdot 2 +
  2^2 = 7$. The latter is sharp ; it can be used for the bound $\mu$ we
  mentioned above.
\end{example}


In order to compute the isolated points, or the simple points, of
$V(\bC)$, we work with a deformation of these equations.  We let $T$ be a
new variable, and we define 
polynomials $\bV=(v_1,\dots,v_s)$ of the form
\begin{equation}\label{eqdef:bV}
\bV = (1-T) \cdot \bM + T \cdot \mG,  
\end{equation}
that connect certain polynomials $\bM=(m_1,\dots,m_s)$ to the target
system $\mG$, together with 
the matrix
\begin{equation}\label{eqdef:bU}
\bU = (1-T)\cdot \bL + T \cdot \mF \in \KK[T, X_1,\dots,X_n]^{p \times q}  
\end{equation}
that connects a suitable \emph{start matrix} $\bL$ to the target matrix $\mF$.
\begin{itemize}
\item The {\em start system} $\bA=(a_1,\dots,a_m)$ will be defined by
  taking $(a_1,\dots,a_s) = (m_1,\dots,m_s)$, and by letting
  $(a_{s+1},\dots,a_m)$ be the $p$-minors of $\bL$; these polynomials 
  are in $\KK[X_1,\dots,X_n]$. 
\item The parametric system $\bB=(b_1,\dots,b_m)$ will be defined by
  taking $(b_1,\dots,b_s) = (v_1,\dots,v_s)$, and by letting
  $(b_{s+1},\dots,b_m)$ be the $p$-minors of $\bU$; these polynomials 
  are in $\KK[T,X_1,\dots,X_n]$. 
\end{itemize}
In particular, setting $T=0$ in $\bB$ gives us $\bA$, and 
setting $T=1$ in it recovers $\bC$.

In Sections~\ref{sec:check} and~\ref{sec:homotopy}, we prove a few
properties of the ideal generated by $\bB$, independently of the
choices of $\bL$ and $\bM$.  Then, in Section~\ref{sec:homotalgo}, we
give symbolic homotopy algorithms which take as input the sequence of
polynomials $\bB$, together with a description of $V(\bA)$ (under
certain regularity assumptions), and computes a zero-dimensional
parametrization of either the isolated solutions, or the simple
solutions of $\bC$. % This is done by lifting the points of $V(\bA)$
% (that correspond to $T=0$) into a curve $\cal C$ parametrized by
% $T$. The isolated points of $V(\bC)$ all belong to the fiber of $\cal
% C$ above $T=1$, but some points in this fiber can actually lie in
% positive dimensional components of $V(\bC)$; the algorithm of
% Section~\ref{sec:isolated} will filter out such points. To find simple
% points, the procedure will be slightly simpler.

To give concrete algorithms, we will have to specify how to define
polynomials $\bM$ and matrix $\bL$, and how to find the solutions of 
$\bA=0$.
The main difficulty lies in the definition of a matrix
$\bL$ that will respect either the column-degree or the row-degree of
$\mF$, while satisfying all assumptions needed for the algorithm of
Section~\ref{sec:homotalgo} and allowing us to solve
the resulting system $\bA=0$ easily.
  The column-degree case is treated in
Section~\ref{sec:columndegree} in a rather straightforward way,
whereas the row-degree case is more delicate, and is treated in
Sections~\ref{sec:prel-row} and~\ref{sec:rowdegree}. The proofs of
some properties needed in the latter sections are postponed to the
appendix of the paper.


%%%%%%%%%%%%%%%%%%%%%%%%%%%%%%%%%%%%%%%%%%%%%%%%%%%%%%%%%%%%

% \newpage
% Throughout, $\KK$ is a field of characteristic zero with algebraic
% closure $\KKbar$, $(X_1, \ldots, X_n)$ is a set of $n$ variables, and
% $\KK[X_1,\dots,X_n]$ is the multivariate polynomial ring in $n$
% variables with coefficients in $\KK$.  With this setup, let
% $\mF=[f_{i,j}] \in \KK[X_1,\dots,X_n]^{p \times q}$ be a polynomial
% matrix, with $p \leq q$. The first question which will interest us in
% this paper is to describe the set of points $\x \in \KKbar{}^n$ at
% which the evaluation of the matrix $\mF$ has rank less than $p$.  In
% the particular case $p=1$, this simply means finding all common
% solutions of $f_{1,1},\dots,f_{1,q}$.

% For any matrix $\mF$ over a ring $R$, and for any integer $r$,
% $M_r(\mF)$ will denote the set of $r$-minors of $\mF$. For any subset $I$ in
% $\KK[X_1,\dots,X_n]$, $V(I)$ will denote the zero-set of $I$ in
% $\KKbar{}^n$, and for a matrix $\mF$ with entries in
% $\KK[X_1,\dots,X_n]$, we will write $V_r(\mF)=V(M_r(\mF))$. Thus, for
% $\mF$ of size $p \times q$, with $p \le q$, the set of points
% introduced in the previous paragraph is
% $$\VpF{p}{\mF}=\{\bx \in \KKbar{}^n \mid \mathrm{rank}(\mF({\bx})) < p\}.$$
% This is an algebraic set, since it is defined by the vanishing of
% all maximal minors of $\mF$. 

% We will discuss below dimension properties of $\VpF{p}{\mF}$.  Recall
% that any algebraic set $V$ is the finite union of its
% \emph{irreducible components}: these are the maximal irreducible
% algebraic sets contained in it (an algebraic set is irreducible if it
% is not the union of two proper algebraic sets). The {\em dimension} of
% an algebraic set $V$ is the largest integer $d$ such that intersecting
% $V$ with $d$ generic hyperplanes yields finitely many points; those
% algebraic sets with all irreducible components of the same dimension
% are called {\em equidimensional}. We refer to
% e.g.~\cite[Chap.\ I and II]{Shafarevich77} for these notions.

% For the problem above, it is natural to consider the case where $n = q-p+1$.
% Indeed, results due to Macaulay~\cite{Macaulay16} and Eagon and
% Northcott~\cite{EN62} imply that all irreducible components of $\VpF{p}{\mF}$
% have dimension at least $n-(q-p+1)$; furthermore, in the case $n = q-p+1$,
% $\VpF{p}{\mF}$ has dimension zero for a generic choice of the entries of $\mF$
% (this is proved for instance in~\cite{Spa14}). Of course, even if we assume $n =
% q-p+1$, $\VpF{p}{\mF}$ may have components of positive dimension; in this case,
% we will be interested in describing only its {\em isolated points}, that is, the
% points in the irreducible components of $\VpF{p}{\mF}$ of dimension zero (this
% notion makes sense for any algebraically closed field $\KK$; when
% $\KK=\mathbb{C}$, these points are indeed isolated for the metric topology).

% \begin{example}\label{ex:1}
%   Several algorithms in this paper will be illustrated by the
%   following example, where we take $\KK=\Q$, $n=2$, $p=2$ and $q=3$
%   (so $n=q-p+1$), and we let $\mF \in \K[X_1,X_2]^{2 \times 3}$ be
%   given by
%   $$\mF=\begin{bmatrix}
%   X_1+X_2-1 & 3X_1+5X_2+2  & 10X_1+X_2-1\\
%   X_2^2+X_1+10X_2+3   & X_1^2+3X_1X_2+X_1-1  & X_1^2-4X_1X_2+X_2^2+3
%   \end{bmatrix}.$$
%   The maximal minors $\bC=M_2(\mF)$ of this matrix are
%   \begin{align*}
%    c_1&= -7X_1^3 - 38X_1^2X_2 - 7X_1^2 - 20X_1X_2^2 - 6X_1X_2 + 20X_1 + 5X_2^3 +    2X_2^2 + 16X_2 + 5,\\
%    c_2&=    X_1^3 - 3X_1^2X_2 - 11X_1^2 - 13X_1X_2^2 - 97X_1X_2 - 26X_1 - 10X_2^2 +    10X_2,\\
%    c_3&=    X_1^3 + 4X_1^2X_2 - 3X_1^2 - 37X_1X_2 - 13X_1 - 5X_2^3 - 52X_2^2 - 36X_2 - 5.
%   \end{align*}
%   They have $7$ common solutions, which we will describe below.
% \end{example}



% Studying the set $\VpF{p}{\mF}$ is a particular case of a slightly
% more general question. In addition to matrix $\mF$, we may indeed take
% into account further equations of the form $g_1 =\cdots=g_s=0$, for
% some $\mG=(g_1,\dots,g_s)$ in $\KK[X_1,\dots,X_n]$. In this setting, the
% natural relation between the number $n$ of variables, the size of
% $\mF$ and the number $s$ of polynomials in $\mG$ is now
% $n=q-p+s+1$. Then, we define the algebraic set
% $$\VpFG{p}{\mF}{\mG} = \{\bx \in \KKbar{}^n \mid
% \mathrm{rank}(\mF({\bx})) < p \text{~and~} g_1(\bx)=\cdots=g_s(\bx)=0
% \};$$ this is thus the zero-set of the ideal generated by the
% $p$-minors of $\mF$ and $ g_1,\dots,g_s$. Our main problem is then the following.
% \begin{pbm} \label{problem2} 
%   For a field $\KK$, a matrix $\mF \in \KK[X_1,\dots,X_n]^{p \times q}$ and
%   polynomials $\mG=(g_1,\dots,g_s)$ in $\KK[X_1,\dots,X_n]$ such that 
%   $p \leq q$ and   $n = q-p+s+1$, compute the isolated points of $\VpFG{p}{\mF}{\mG}$.
% \end{pbm}
% This problem appears in a variety of context; prominent examples are
% optimization problems~\cite{GSZ10,JP14,BGHS14,GS14,NDS06}, and related
% questions in real algebraic
% geometry~\cite{ARS,BaGiHeMb01,BaGiHePa05,BGHSS,BRSS,RealDecompICMS,CellDecompSurface,BertiniReal,RealNumerical,SaSc03,SaSc11,SaSc17},
% where $\mF$ consists of the Jacobian matrix of $\mG$, together with
% one extra row, corresponding to the gradient of a function that we
% want to optimize on $V(\mG)$. Because they show up several times in
% this introduction, we will refer to this particular class of inputs as
% systems {\em coming from optimization} (in such cases, $q=n$ and $p=s+1$, 
% so our assumption $n=q-p+s+1$ holds).

% In several of these situations, we are only interested in the
% solutions of the system made of minors $M_p(\mF)$ and
% $\mG=(g_1,\dots,g_s)$ at which the associated Jacobian matrix has full
% rank. This set of solutions is finite and is always contained in the
% set of isolated points of $\VpFG{p}{\mF}{\mG}$ \cite[Theorem
%   16.19]{Eisenbud95}; we call these points {\em simple points}. For
% instance, the set of simple points coincides with $\VpFG{p}{\mF}{\mG}$
% when the minors $M_p(\mF)$ and $g_1,\dots,g_s$ generate a radical
% ideal of dimension zero; this case appears frequently in the context
% of algorithms in real algebraic geometry~\cite{BGHSS}.  Hence, it also
% makes sense to look at the following slight variant of
% Problem~\eqref{problem2}.

% \begin{pbm} \label{problem3} For a field $\KK$, a matrix
%   $\mF \in \KK[X_1,\dots,X_n]^{p \times q}$ and polynomials
%   $\mG=(g_1,\dots,g_s)$ in $\KK[X_1,\dots,X_n]$ with $p \leq q$ and
%   $n = q-p+s+1$, compute the simple  points of~$\VpFG{p}{\mF}{\mG}$.
% \end{pbm}

% We will represent the output of our algorithms using univariate polynomials. Let
% $V \subset \KKbar{}^n$ be a zero-dimensional algebraic set defined over $\KK$. A
% \emph{zero-dimensional parametrization} $\scrR = ((w,v_1, \ldots, v_n),
% \lambda)$ of $V$ consists of polynomials $(w,v_1, \ldots, v_n)$ such that $w$ 
% is in
% $\KK[Y]$ ($Y$ is a new variable), monic and squarefree, all $v_i$'s are in
% $\KK[Y]$ and satisfy $\deg(v_i) < \deg(w)$, and $\lambda$ is a $\KK$-linear form
% in $n$ variables, such that
% \begin{itemize}
% \item $\lambda(v_1, \ldots, v_n) = Yw'$ mod $w$, with $w'=\frac{d w}{d Y}$;
% \item we have $V = Z(\scrR)$, with $$Z(\scrR)= \left\{\left(\frac{v_1(\tau)}{w'(\tau)}, \ldots, \frac{v_n(\tau)}{w'(\tau)}\right) \ | \ w(\tau) = 0\right\}.$$
% \end{itemize}
% The constraint on $\lambda$ then says that the roots of $w$ are the
% values taken by $\lambda$ on $V$. This representation was introduced
% in~\cite{Kronecker82,Macaulay16}, and has been used in a variety of
% algorithms, such as those
% in~\cite{GiMo89,GiHeMoPa95,ABRW,GiHeMoMoPa98,Rouillier99,GiLeSa01}.
% The reason why we use a rational parametrization, with $w'$ as a
% denominator, goes back to~\cite{ABRW,Rouillier99,GiLeSa01}: when
% $\KK=\Q$, this allows us to control precisely the bit-size of the
% coefficients, using bounds such as those
% in~\cite{Schost03,DaSc04}. The same phenomenon holds with $\KK=k(T)$,
% for a field $k$, in which case we want to control degrees in $T$ of
% the numerators and denominators of the coefficients of $\scrR$.


% \begin{example}\label{ex:2}
%   In Example~\ref{ex:1}, the maximal minors $\bC = M_2(\mF)$ generate a radical
%   ideal of dimension zero, so Problems~\ref{problem2} and~\ref{problem3} admit the same
%   answer. There are 7 solutions, which are described by means of the univariate
%   representation $\scrR=((w,v_1,v_2),\lambda)$, with $\lambda = X_2$ and
% {\small  \begin{align*}
% w &= Y^7 + \frac{5249}{285}Y^6 + \frac{5899}{76}Y^5 - \frac{32593}{950}Y^4 - \frac{719401}{5700}Y^3 
%         - \frac{302473}{5700}Y^2 - \frac{1243}{475}Y + \frac{379}{1140}\\[1mm]
% v_1&= -\frac{461}{114}Y^6 - \frac{39047}{380}Y^5 - \frac{2431807}{2850}Y^4 - \frac{87697}{76}Y^3 - \frac{560363}{1900}Y^2 
%       + \frac{64121}{570}Y + \frac{1341}{76}\\[1mm]
% v_2 &= -\frac{5249}{285}Y^6 - \frac{5899}{38}Y^5 + \frac{97779}{950}Y^4 + \frac{719401}{1425}Y^3 
%        + \frac{302473}{1140}Y^2 + \frac{7458}{475}Y - \frac{2653}{1140}.
%   \end{align*}}
%   The coordinates of the solutions are the values taken by $(v_1/w',v_2/w')$
%   at the roots of $w$. 
% \end{example}


% Our first result gives bounds on the number of solutions of
% $\VpFG{p}{\mF}{\mG}$, counted with multiplicities. To state it, we
% need the following notation.  Take $\mF=[f_{i,j}]_{1 \le i \le p, 1
%   \le j \le q}$ in $\KK[X_1,\dots,X_n]^{p \times q}$.  We will
% consider two degree measures for the matrix $\mF$; these have been used
% before for determinantal ideals, see for
% instance~\cite{NieRan09,MiSt04}. For $i=1,\dots,p$, we will write
% $\rdeg(\mF,i)$ for the degree of the $i$th row of $\mF$, that is,
% $\rdeg(\mF,i)=\max(\deg(f_{i,j}))_{1 \le j \le q}$; similarly, for
% $j=1,\dots,q$, we write $\cdeg(\mF,j)$ for the degree of the $j$th
% column of $\mF$, that is, $\cdeg(\mF,j)=\max(\deg(f_{i,j}))_{1 \le i
%   \le p}$. Further, for $k \ge 0$ and integers $\delta_1,\dots,\delta_q$,
% $$E_k(\delta_1,\dots,\delta_q)=\sum_{1\leq i_1 < \cdots < i_k \leq
%   q}\delta_{i_1} \cdots \delta_{i_k}$$ is the elementary symmetric
% polynomial of degree $k$ in $(\delta_1, \ldots, \delta_q)$; 
% for integers $\alpha_1,\dots,\alpha_p$,
% $$S_k(\alpha_1,\dots,\alpha_p) = \sum_{i_1+\cdots+i_p=k, i_j \geq
%   0}\alpha_1^{i_1}\cdots\alpha_p^{i_p}$$ is the $k$th complete
% symmetric polynomial in $(\alpha_1,\dots,\alpha_p)$.

% Finally, we recall the notion of multiplicity of a point $\bx$ with
% respect to an ideal $I$ in
% $\KKbar[X_1,\dots,X_n]$; this notion
% extends to ideals in $\KK[X_1, \ldots, X_n]$ by considering their
% extension in $\KKbar[X_1, \ldots, X_n]$. The ideal $I$ can be written
% as the intersection of finitely many primary components, that is,
% $I=Q_1\cap\cdots \cap Q_r$ for some primary ideals $Q_1,\dots,Q_r$;
% this decomposition is said to be minimal when $V(Q_i)\neq V(Q_j)$ for
% $i\neq j$. Take $\bx$ isolated in $V(I)$; then there exists a unique
% primary component $Q_i$, which must have dimension zero, such that
% $\bx$ is in $V(Q_i)$; because we take a primary decomposition over
% $\KKbar$, we actually have $V(Q_i)=\{\bx\}$. Although minimal primary
% decompositions are not unique, the fact that $\bx$ is isolated implies
% that $Q_i$ does not depend on the primary decomposition of $I$ we
% consider; then, the \emph{multiplicity} of $\bx$ is defined as the
% dimension of $\KKbar[X_1,\dots,X_n]/Q_i$. When $\bx=0\in\KKbar{}^n$,
% the dimension of $\KKbar[X_1,\dots,X_n]/Q_i$ is the same as that of
% $\KKbar[[X_1, \ldots, X_n]]/I$, where $\KK[[X_1, \ldots, X_n]]$
% denotes the formal power series ring in $X_1, \ldots, X_n$ with
% coefficients in $\KKbar$~\cite[Theorem 4.2.2]{CLO_UAG}.

% All this being said, the following is our first~result.

% \begin{theorem}\label{theo:1}
%   Let $\mF$ be in $\KK[X_1,\dots,X_n]^{p \times q}$ and let
%   $\mG=(g_1,\dots,g_s)$ be in $\KK[X_1,\dots,X_n]$, with $p \le q$ and
%   $n=q-p+s+1$. Then, the sum of the multiplicities of the isolated
%   points of the ideal generated by the $p$-minors of $\mF$ and 
% $ g_1,\dots,g_s$ is at most
%   $\min(c,c')$ with
% $$c=\deg(g_1) \cdots \deg(g_s) E_{n-s}(\cdeg(\mF,1), \ldots, \cdeg(\mF,q))$$
% and
% $$c'=\deg(g_1) \cdots \deg(g_s) S_{n-s}(\rdeg(\mF,1), \ldots, \rdeg(\mF,p)).$$
% \end{theorem}
% When $\rdeg(\mF,i)=\cdeg(\mF,j)=d$ for all $i,j$, the two bounds given
% above coincide, with common value $\deg(g_1) \cdots \deg(g_s) {q \choose {p-1}} d^{n-s}$; otherwise, either of the two expressions
% $E_{n-s}(\cdeg(\mF,1), \ldots, \cdeg(\mF,q))$ and
% $S_{n-s}(\rdeg(\mF,1), \ldots, \rdeg(\mF,p))$ can be the minimum. 

% %% For 
% %% instance, consider the case where $s=0$ (so there are no equations $\mG$),
% %% and where the degrees of the entries in $\mF$ are 
% %% $$ \begin{bmatrix}
% %%     2 & 1 & 5 & 7 \\
% %%     2 & 1 & 5 & 7 \\
% %%     2 & 1 & 5 & 7 
% %%   \end{bmatrix}.$$
% %% Here, we have $p=3, q=4, s=0$ and $n=2$. Then, 
% %% the quantity $c$ is $c=E_2(2,1,5,7) = 2\cdot1+2\cdot5+2\cdot7+1\cdot5+1\cdot7+5\cdot7 = {73}$,
% %% whereas $c'=6 \cdot 7^2=294.$ On the other hand, if we 
% %% take $\mF$ with degree profile
% %% $$ \begin{bmatrix}
% %%     2 & 2 & 2 & 2 \\
% %%     1 & 1 & 1 & 1 \\
% %%     5 & 5 & 5 & 5 
% %%   \end{bmatrix},$$
% %% with the same values of $p,q,s,n$, we get $c=6 \cdot 7^2=294$ and
% %% $c'=S_2(2,1,5) = 2^2+2\cdot 1 + 2\cdot 5 + 1^2 + 1 \cdot 5 + 5^2 =
% %% {47}$.

% \begin{example}
%   In our example, we have no polynomials $\mG$, so $s=0$. The column degrees of
%   $\mF$ are $(\cdeg(\mF,1),\cdeg(\mF,2),\cdeg(\mF,3))=(2,2,2)$, whereas its
%   row degrees are $(\rdeg(\mF,1),\rdeg(\mF,2))=(1,2)$. 

%   The column degree bound is $c=E_2(2,2,2) = 2^2 + 2^2 +
%   2^2 =12$, the row degree bound is $c'=S_2(1,2) = 1^2 + 1\cdot 2 +
%   2^2 = 7$. The latter is sharp.
% \end{example}
% For systems coming from optimization, where $\mF$ is a Jacobian
% matrix, we are in a situation similar to our example, where the $i$th
% row degree of $\mF$ is simply the degree of the corresponding
% equation, minus one.

% Although results in a similar vein have already been published, we are
% not aware of previous statements as above, with no assumption on the
% dimension of $V_p(\mF,\mG)$, and that take into account multiplicities
% as is done in Theorem~\ref{theo:1}.

% Pioneering work of Giambelli-Thom-Porteous (see e.g. \cite{FP06} or
% \cite{Fu92}) already established similar bounds under regularity
% assumptions (when $V(\mG)$ is smooth and/or $V_p(\mF, \mG)$ has the
% expected codimension). Previous work by Miller and
% Sturmfels~\cite[Chapter~15]{MiSt04} proved very general results on the
% multi-degrees of determinantal ideals built from matrices with
% indeterminate entries (in which case we have $s=0$, but the assumption
% $n=q-p+1$ does not hold); in particular, they obtain analogues (and
% generalizations) of the result in Theorem~\ref{theo:1} in that
% context.

% Nie and Ranestad proved in~\cite{NieRan09} that the bounds in
% Theorem~\ref{theo:1} are tight for two families of polynomials
% (in a similar context where the polynomials are homogeneous in
% $n+1$ variables):
% \begin{itemize}
% \item when entries of $\mF$ are generic and homogeneous, and
%  such that $\deg(f_{i,j}) = \cdeg(\mF,j)$ for all $i,j$, the ideal
%  generated by the $p$-minors of $\mF$ has degree $E_{n}(\cdeg(\mF,1), \ldots, \cdeg(\mF,q))$;
% \item when entries of $\mF$ are  generic and homogeneous, and
%   such that $\deg(f_{i,j}) = \rdeg(\mF,i)$ for all $i,j$, the
%   ideal generated by the $p$-minors of $\mF$
%  has degree $S_{n}(\rdeg(\mF,1), \ldots, \rdeg(\mF,p))$.
% \end{itemize}
% From this, they deduce that the degree of the ideal generated by the
% $p$-minors of $\mF$ and $ g_1,\dots,g_s $ is at most \sloppy
% $\deg(g_1) \cdots \deg(g_s) S_{n-s}(\rdeg(\mF,1), \ldots,
% \rdeg(\mF,p))$, for systems coming from optimization, assuming that
% this ideal has dimension zero. In this context, Spaenlehauer also gave
% in~\cite{Spa14} an explicit expression for the Hilbert function of the
% ideal above, for a generic input.

% \medskip

% Our second result gives bounds on the cost of computing a
% zero-dimensional parametrization of the isolated solutions of
% $\VpFG{p}{\mF}{\mG}$. Our
% algorithms take as input a \emph{straight-line program} (that is, a
% sequence of elementary operations $+, -, \times$) that computes the
% entries of $\mF$ and $\mG$ from the input variables $X_1,\dots,X_n$; the
% \emph{length $\sigma$} of the input is the number of operations it
% performs. This assumption is not restrictive, since any matrix $\mF$
% and polynomials $\mG$ can be computed by a straight-line program (a
% naive solution would consist in computing and adding all monomials in
% $\mF$ and $\mG$).

% \begin{theorem}\label{theo:2}
%   Suppose that matrix $\mF \in \KK[X_1,\dots,X_n]^{p \times q}$ and
%   polynomials $\mG=(g_1,\dots,g_s)$ in $\KK[X_1,\dots,X_n]$ are given by
%   a straight-line program of length $\sigma$. Assume that
%   $\deg(g_1),\dots,\deg(g_s)$, as well as
%   $\cdeg(\mF,1), \ldots, \cdeg(\mF,q)$ and
%   $\rdeg(\mF,1), \ldots, \rdeg(\mF,p)$ are all at least equal to $1$.

%   Then, there exist randomized algorithms that solve
%   Problem~\eqref{problem2} in either
%    $$\softO\left (
%      {q \choose p} c(e+c^5 )\big(\sigma + q \delta + \gamma  \big )
%    \right)$$
%   operations in $\KK$, with
%   \begin{align*}
%     c&=\deg(g_1)\cdots\deg(g_s)\ E_{n-s}(\cdeg(\mF,1), \ldots, \cdeg(\mF,q))\\
%     e&=(\deg(g_1)+1)\cdots(\deg(g_s)+1)\ E_{n-s}(\cdeg(\mF,1)+1, \ldots, \cdeg(\mF,q)+1),\\
%     \gamma&= \max(\deg(g_i), 1\leq i \leq s)\\
%     \delta &= \max(\cdeg(\mF,i), 1\leq i \leq q)
%   \end{align*}
%   or 
%    $$\softO\left (
%      {q \choose p} c'(e'+{c'}^5 )\big(\sigma + p \alpha  +\gamma \big )
%    \right)$$
%   operations in $\KK$, with 
% \begin{align*}
%   c'&=\deg(g_1)\cdots\deg(g_s)\ S_{n-s}(\rdeg(\mF,1), \ldots, \rdeg(\mF,p))\\
%   e'&=(\deg(g_1)+1)\cdots(\deg(g_s)+1)\ S_{n-s}(\rdeg(\mF,1)+1, \ldots, \rdeg(\mF,p)+1),\\
%     \gamma&= \max(\deg(g_i), 1\leq i \leq s)\\
%     \alpha &= \max(\rdeg(\mF,j), 1\leq j \leq p).
% \end{align*}
% \end{theorem}
% \textcolor{brown}{The assumption that all degrees are at least $1$ is not a restriction.
% If $\deg(g_i)=0$ for some $i$, $g_i$ is a constant, so either the
% system is inconsistent (if $g_i \ne 0$) or $g_i$ can be
% discarded. Similarly, if say $\cdeg(\mF,i)=0$, the $i$th column of
% $\mF$ consists of constants; after applying linear combinations with
% coefficients in $\KK$ to the rows of $\mF$, we may assume that all
% entries in the $i$th column, except at most one, are non-zero without
% changing the column degrees. The $i$th column of $\mF$ (and the row of
% the non-zero entry, if there is one) can then be discarded.}

% Remark further that in the common situation where all degrees
% $\deg(g_i)$, $\rdeg(\mF,i)$ and $\cdeg(\mF,j)$ involved in the
% formulas above are at least equal to $2$, we have the inequalities $e
% \le c^2$, $e' \le {c'}{}^2$ and $\binom{q}{p}\leq c$,
% $\binom{q}{p}\leq c'$. As a result, the runtimes are {\em polynomial}
% in $c,\sigma$ and $c',\sigma$: they respectively become
% $\softO (c^8 \sigma)$ and $\softO ({c'}^8 \sigma)$.  This is to be
% compared with Theorem~\ref{theo:1}, which shows that $\min(c,c')$ is a
% natural upper bound for the output size of such~algorithms.

% For solving Problem~\eqref{problem3}, one obtains slightly better
% complexity estimates. 

% \begin{theorem}\label{theo:3}
%   Suppose that the matrix $\mF \in \KK[X_1,\dots,X_n]^{p \times q}$
%   and polynomials $\mG=(g_1,\dots,g_s)$ in $\KK[X_1,\dots,X_n]$ are
%   given by a straight-line program of length $\sigma$. Assume that
%   $\deg(g_1),\dots,\deg(g_s)$, as well as
%   $\cdeg(\mF,1), \ldots, \cdeg(\mF,q)$ and
%   $\rdeg(\mF,1), \ldots, \rdeg(\mF,p)$ are all at least equal to $1$.

%   Then, there exist randomized algorithms that solve
%   Problem~\eqref{problem3} in either
%    $$\softO\left (
%      {q \choose p} ce\big(\sigma + q \delta + \gamma  \big )
%    \right)$$
% or 
%    $$\softO\left (
%      {q \choose p} c'e'\big(\sigma + p \alpha  +\gamma \big )
%    \right)$$
%   operations in $\KK$, 
%   all notations being as in Theorem~\ref{theo:3}.
% \end{theorem}
% As above, in the common situation where all degrees involved are at
% least $2$, the runtimes are {\it polynomial} in $c, \sigma$ and
% $c',\sigma'$; precisely, they respectively become $\softO (c^5 \sigma)$ and $\softO
% ({c'}^5 \sigma)$.


% The probabilistic aspects are as follows: at several steps, the
% algorithms on which Theorems~\ref{theo:2} and~\ref{theo:3} rely will
% draw elements from the base field at random. In all cases, there
% exists an algebraic hypersurface $\cal H$ of the parameter space such
% that success is guaranteed for all choices of parameters not
% in~$\cal H$.


% %% The number ${q \choose p}$ is the number of elements in the
% %% determinantal system of $\mF$. In several cases, we can improve the
% %% algorithm and replace this by the number $q$ of columns of $\mF$, for
% %% instance, when all isolated solutions of $I_\mF$ are known to have
% %% multiplicity one.

% As already said, our algorithms are based on a {\em symbolic homotopy
%   continuation}. Following early work in the 1930's, such
% as~\cite{Lahaye34}, homotopy continuation algorithms have become a
% foundational tools for numerical algorithms; see for
% instance~\cite{AlGe03} for an extensive list of references.  We
% mention in particular Shub and Smale's work on the complexity of these
% techniques, starting from~\cite{ShSm93}, or work by Morgan, Sommese,
% Wampler (as summarized, for instance, in~\cite{BertiniBook,SoWa05}),
% with an emphasis on the underlying algebraic geometry. In this
% context, dedicated numerical homotopy algorithms have also been
% developed to take into account sparsity in polynomial
% systems~\cite{Ver94,Ver09,AdVe13}.

% By contrast, the usage of homotopy methods in symbolic contexts is
% more recent, even though some early results, such as Bernstein's proof
% of the so-called BKK theorem, already involve Puiseux series
% manipulations~\cite{Bernstein75}. References such
% as~\cite{HeKrPuSaWa99,BoMaWaWa04} deal with systems with no particular
% structure, or systems with no zeros at infinity~\cite{PaSa04}.
% Further work extended this idea to sparse systems (in the polyhedral
% sense)~\cite{JeMaSoWa09,HeJeSa10,HeJeSa13,HeJeSa14} and
% multihomogeneous systems~\cite{HeJeSaSo02,SaSc16}.  In~\cite{SaSc16},
% these techniques are used to solve Problem~\eqref{problem3}, but the
% complexity estimates obtained there depend on multi-homogeneous
% B\'ezout bounds involving the maxima of $\rdeg(\mF, 1), \ldots,
% \rdeg(\mF, p)$ or $\cdeg(\mF,1), \ldots, \cdeg(\mF, q)$.

% Most algorithms in the previous references have in common that they
% solve {\em square} systems, that is, systems with as many equations as
% unknowns; extensions of these methods can deal with systems of
% positive dimension by essentially using variants of the algorithm for
% square systems.  One notable exception is given in \cite{SVV10}, where
% dedicated homotopy algorithms are given to solve Schubert problems
% which consist in determining linear spaces of prescribed dimension
% which meet a set of fixed linear subspaces in specified
% dimensions; such problems are formulated with rank
% conditions on some special matrices~\cite{LDSVV18}. These
% algorithms strongly exploit and are dedicated to the structure of the
% Schubert problem through the Littlewood Richardson rule and an
% associated combinatorial construction. Hence, as far as we know, they
% cannot be used to solve determinantal systems of equations expressing
% that a given matrix with polynomial entries is rank deficient. 

% One of the contributions in this paper is to deal with determinantal
% systems of equations, which are in essence over-determined; this is
% made possible by the algebraic properties of determinantal ideals.

% It is well known that Gr\"obner bases behave rather well on
% over-determined systems. Starting from the determination of the
% Hilbert function of a determinantal ring due to Conca and
% Herzog~\cite{CH94}, complexity estimates are given in
% \cite{FSS13,FSS12} for computing Gr\"obner bases of ideals generated
% by either $M_r(\mF)$ when $r\leq p\leq q$, or
% $M_{p}(\mF),g_1,\dots,g_s$, (for inputs coming from optimization
% problems), but under some genericity assumptions on the entries of
% $\mF$ or $\mG$; the input polynomials are also assumed to all have the
% same degree.  This series of works culminated with the result obtained
% by Spaenlehauer in \cite{Spa14}, where he removes this latter degree
% assumption and provides sharp complexity statements, still under
% genericity~assumptions.

% Systems encoding rank defects in polynomial matrices have also been
% studied in the scope of the so-called geometric resolution algorithm
% in \cite{BaGiHeLeMaSo15}, with a slight generalization in
% \cite{SaSp16}. The algorithms in these references work for generic
% inputs and solve only our second problem, computing simple solutions
% (isolated solutions which are not simple are not considered in that
% line of work). The cost of these algorithms is quadratic in certain
% geometric quantities (the degree of algebraic sets defined by
% subsystems of the determinantal equations we are dealing with); this
% is to be compared with the runtimes in Theorem~\ref{theo:3}, where the
% main contributions are the products ${q \choose p} c e$, respectively
% ${q \choose p} c' e'$, and where $ce$, resp.\ $c'e'$, are also of a
% geometric nature.  In cases where the results
% of~\cite{BaGiHeLeMaSo15,SaSp16} apply, further work is needed to
% compare the degrees involved in these complexity estimates with ours, and the
% resulting runtimes.

% \medskip In the following paragraphs, we describe our results in more
% detail, and introduce notation in use in all the paper. We start from
% polynomials $\mG=(g_1,\dots,g_s)$ in $\KK[\bX]=\KK[X_1,\dots,X_n]$ and
% polynomial matrix $\mF \in \KK[\bX]^{p \times q}$; the equations
% $\bC=(c_1,\dots,c_{s},\dots,c_m)$ that we want to solve are defined as
% follows: $(c_1,\dots,c_{s})=(g_1,\dots,g_s)$, and
% $(c_{s+1},\dots,c_{m})$ are the $p$-minors of $\mF$.

% As a preliminary, we will need an algorithm which takes as input
% polynomials $\bC$ and a point $\bx$ in the zero-set of $\bC$, and
% which decides whether $\bx$ is an isolated points of $V(\bC)$ (this
% will be used to solve Problem~\eqref{problem2}).  Without any other
% information, this decision problem is difficult to solve in a good
% complexity. However, when a bound $\mu$ is known on the multiplicity
% of $\bx$ as a root of $\bC$, it becomes possible to solve this problem
% in time polynomial in the number of equations $m$, the number of
% variables $n$, the bound $\mu$, and the complexity of evaluation
% $\sigma$ of $\bC$. This is detailed in Section~\ref{sec:isolated},
% where we explain how to modify an algorithm by
% Mourrain~\cite{Mourrain97} and adapt it to our context.

% In order to compute the isolated points, or the simple points, of
% $V(\bC)$, we work with a deformation of these equations.  We let $T$ be a
% new variable, and we define 
% polynomials $\bV=(v_1,\dots,v_s)$ of the form
% \begin{equation}\label{eqdef:bV}
% \bV = (1-T) \cdot \bM + T \cdot \mG,  
% \end{equation}
% that connect certain polynomials $\bM=(m_1,\dots,m_s)$ to the target
% system $\mG$, together with 
% the matrix
% \begin{equation}\label{eqdef:bU}
% \bU = (1-T)\cdot \bL + T \cdot \mF \in \KK[T, X_1,\dots,X_n]^{p \times q}  
% \end{equation}
% that connects a suitable \emph{start matrix} $\bL$ to the target matrix $\mF$.
% \begin{itemize}
% \item The {\em start system} $\bA=(a_1,\dots,a_m)$ will be defined by
%   taking $(a_1,\dots,a_s) = (m_1,\dots,m_s)$, and by letting
%   $(a_{s+1},\dots,a_m)$ be the $p$-minors of $\bL$; these polynomials 
%   are in $\KK[X_1,\dots,X_n]$. 
% \item The parametric system $\bB=(b_1,\dots,b_m)$ will be defined by
%   taking $(b_1,\dots,b_s) = (v_1,\dots,v_s)$, and by letting
%   $(b_{s+1},\dots,b_m)$ be the $p$-minors of $\bU$; these polynomials 
%   are in $\KK[T,X_1,\dots,X_n]$. 
% \end{itemize}
% In particular, setting $T=0$ in $\bB$ gives us $\bA$, and 
% setting $T=1$ in it recovers $\bC$.

% In Sections~\ref{sec:check} and~\ref{sec:homotopy}, we prove a few
% properties of the ideal generated by $\bB$, independently of the
% choices of $\bL$ and $\bM$.  Then, in Section~\ref{sec:homotalgo}, we
% give symbolic homotopy algorithms which take as input the sequence of
% polynomials $\bB$, together with a description of $V(\bA)$ (under
% certain regularity assumptions), and computes a zero-dimensional
% parametrization of either the isolated solutions, or the simple
% solutions of $\bC$. This is done by lifting the points of $V(\bA)$
% (that correspond to $T=0$) into a curve $\cal C$ parametrized by
% $T$. The isolated points of $V(\bC)$ all belong to the fiber of $\cal
% C$ above $T=1$, but some points in this fiber can actually lie in
% positive dimensional components of $V(\bC)$; the algorithm of
% Section~\ref{sec:isolated} will filter out such points. To find simple
% points, the procedure will be slightly simpler.

% To give concrete algorithms, we will have to specify how to define
% polynomials $\bM$ and matrix $\bL$, and how to find the solutions of 
% $\bA=0$.
% The main difficulty lies in the definition of a matrix
% $\bL$ that will respect either the column-degree or the row-degree of
% $\mF$, while satisfying all assumptions needed for the algorithm of
% Section~\ref{sec:homotalgo} and allowing us to solve
% the resulting system $\bA=0$ easily.
%   The column-degree case is treated in
% Section~\ref{sec:columndegree} in a rather straightforward way,
% whereas the row-degree case is more delicate, and is treated in
% Sections~\ref{sec:prel-row} and~\ref{sec:rowdegree}. The proofs of
% some properties needed in the latter sections are postponed to the
% appendix of the paper.


% %%%%%%%%%%%%%%%%%%%%%%%%%%%%%%%%%%%%%%%%%%%%%%%%%%%%%%%%%%%%

%%%%%%%%%%%%%%%%%%%%%%%%%%%%%%%%%%%%%%%%%%%%%%%%%%%%%%%%%%%%
%%%%%%%%%%%%%%%%%%%%%%%%%%%%%%%%%%%%%%%%%%%%%%%%%%%%%%%%%%%%

\section{A local dimension test} \label{sec:isolated}

Let $\LL$ be a field containing the field $\KK$ and $\LLbar$ be an
algebraic closure of $\LL$.  Let $\bC=(c_1,\dots,c_m)$ be polynomials
in $\KK[\bX]$, with $\bX=(X_1,\dots,X_n)$; we will apply the result in
this section to $\bC$ as defined in the introduction, but the
algorithm accepts arbitrary inputs. Given a point $\bx$ with
coordinates in $\LL$ that belongs to the zero-set $V(\bC)\subset
\LLbar{}^n$, we discuss here how to decide whether $\bx$ is an
isolated point in $V(\bC)$.  Without loss of generality, we assume
that $m\ge n$ (otherwise, $\bx$ cannot be an isolated solution).

Our result is the following proposition, whose proof occupies the rest
of this section.
\begin{proposition}\label{prop:testisolated}
  Suppose that $\bC$ is given by a straight-line program of length
  $\sigma$, and that we are given an integer $\mu$ such that
  either $\bx$ is isolated in $V(\bC)$, with multiplicity at most
  $\mu$ with respect to the ideal $\langle \bC\rangle$, or $\bx$
  belongs to a positive-dimensional component of $V(\bC)$.
  Then, we can decide whether $\bx$ is an isolated point of $V(\bC)$
  using
$$O(n^4 \mu^4 + n^2 m \mu^3 + n \sigma \mu^4) \subset (\mu\,\sigma\,m)^{O(1)}$$ operations in~$\LL$.
\end{proposition}
\textcolor{red}{A few remarks are in order here on how to apply such a result.
  Of course, the easiest way is to do so is when $\LL=\KK$. Further, in
  Section~\ref{sec:homotalgo}, we will apply it when $\LL$ the fraction field of
  the integral ring $\KK[Y] / \langle w \rangle$ where $Y$ is a new variable and
  $w \in \KK[Y]$ is irreducible. In this setting, the coordinates of $\bx$ are
  expressed through the evaluation of a polynomial in $\KK[Y]$ at one root of
  $w$. Classical arithmetic operation such as addition, subtraction and
  multiplication are performed modulo $w$ ; inverting a non-zero element in
  $\KK[Y] / \langle w \rangle$ boils down to applying the Extended Euclidean
  Algorithm (see \cite{GaGe03}). }


Reference~\cite{BaHaPeSo09} gives an algorithm to compute the
dimension of $V(\bC)$ at $\bx$, but its complexity is not known to us,
as it relies on linear algebra with matrices of potentially large size
(not necessarily polynomial in $\mu,\sigma,m$).  Instead, we use an
adaptation of a prior result by Mourrain~\cite{Mourrain97}, which
allows us to control the size of the matrices we handle. We only give
detailed proofs for new ingredients that are specific to our context,
a key difference being the cost analysis in the straight-line program
model: Mourrain's original result depends on the number of monomials
appearing when we expand $\bC$, which would be too high for the
applications we will make of this result. Remark that the assumption
that $\KK$ (and thus $\LL$) have characteristic zero is needed for 
Mourrain's algorithm.

We assume henceforth that $\bx=0$; this is done by replacing $\bC$ by
the polynomials $\bC(\bX+\bx)$, which can be computed by a
straight-line program of length $\sigma'=\sigma+n$.  The basis of our
algorithm is the following remark.

\begin{lemma}
  Let $I$ be the zero-dimensional ideal
  $\langle \bC \rangle + \m^{\mu+1}$, where
  $\m=\langle X_1,\dots,X_n\rangle$ is the maximal ideal at the
  origin. Then, $0$ is isolated in $V(\bC)$ if and only if the
  multiplicity $d$ of $I$ at the origin is at most $\mu$.
\end{lemma}
\begin{proof}
  This follows from the following
  result~\cite[Theorem~A.1]{BaHaPeSo09}.  For $k \ge 1$, let $I_k$ be
  the zero-dimensional ideal $\langle \bC \rangle + \m^{k}$, and let
  $\nu_k$ be the multiplicity of the origin with respect to this
  ideal. Then, the reference above proves that the sequence
  $(\nu_k)_{k \ge 1}$ is non-decreasing, and that $0$ is isolated in
  $V(\bC)$ if and only if there exists $k\ge 1$ such that
  $\nu_k=\nu_{k+i}$ for any $i\geq 0$.
  \begin{itemize}
  \item If $0$ is isolated in $V(\bC)$, then by assumption 
    its multiplicity with respect to $\langle \bC\rangle$ is at most $\mu$,
    and its multiplicity $d$ with respect to $I$ cannot be larger.
  \item Otherwise, by the result above, $\nu_{k+1} > \nu_k$ holds for
    all $k \ge 1$, so that $\nu_k \ge k$ holds for all such $k$ (since
    $\nu_1=1$). In particular, the multiplicity $d$ of 
 $I$ at the origin, which is $\nu_{\mu+1}$, is at least $\mu+1$.
    \qedhere
  \end{itemize}
\end{proof}

Hence, we are left with deciding whether $d$, the multiplicity of the
ideal $I$ at the origin, is at most $\mu$; remark that this
multiplicity is equal to the dimension of $\LL[\bX]/I$, since $I$ is
$\m$-primary.  We do this by following and slightly modifying
Mourrain's algorithm for the computation of the orthogonal
$I^{\perp}$, that is, the set of $\LL$-linear forms $\LL[\bX] \to \LL$
that vanish on $I$; this is a $\LL$-vector space naturally identified
with the dual of $\LL[\bX]/I$, so it has dimension $d$, the
multiplicity of $I$ at the origin.

We do not need to give all details of the algorithm, let alone proof
of correctness; we just mention the key ingredients for the cost
analysis in our setting. 

The algorithm represents the elements in $I^{\perp}$ by means of {\em
  multiplication matrices}. An important feature of $I^{\perp}$ is
that it admits the structure of a $\LL[\bX]$-module: for $k$ in
$\{1,\dots,n\}$ and $\beta$ in $I^{\perp}$, the $\LL$-linear form $X_k
\cdot \beta: f \mapsto \beta(X_k f)$ is easily seen to still lie in
$I^{\perp}$.  In particular, if
$\bbeta=(\beta_1,\dots,\beta_d)$ is an $\LL$-basis of
$I^{\perp}$, then for all $k$ as above, and all $i$ in
$\{1,\dots,d\}$, $X_k \cdot \beta_i$ is a linear combination of
$\beta_1,\dots,\beta_d$. Mourrain's algorithm computes a basis
$\bbeta=(\beta_1,\dots,\beta_d)$ with the following features:
\begin{itemize}
\item for $i$ in $\{1,\dots,d\}$ and $k$ in $\{1,\dots,n\}$, we have
  $X_k \cdot \beta_i=\sum_{0 \le j < i} \lambda^{(k)}_{i,j} \beta_j$
  (hence $\lambda^{(k)}_{i,j}$ may be non-zero 
  only for $j<i$);
\item $\beta_1$ is the evaluation at $0$, $f \mapsto f(0)$;
\item for $i$ in $\{2,\dots,d\}$, $\beta_i(1)=0$.
\end{itemize}
The following lemma shows that the coefficients $(\lambda^{(k)}_{i,j})$
are sufficient to evaluate  the linear forms $\beta_i$ at any $f$ in
$\LL[\bX]$. More precisely, knowing only their values for $j < i \le s$,
for any $s \le d$, allows us to evaluate $\beta_1,\dots,\beta_s$ at such an $f$.
The lemma follows~\cite{Mourrain97} in its description
of the matrices $\bM_{k,s}$; the (rather straightforward) complexity analysis 
in the straight-line program model is new.
\begin{lemma}\label{lemma:evalbeta}
   Let $s$ be in $1,\dots,d$, and suppose that the coefficients
  $\lambda^{(k)}_{i,j}$ are known for $i=1,\dots,s$, $j=0,\dots,i-1$
  and $k=1,\dots,n$. Given a straight-line program $\Gamma$ of length
  $\sigma$ that computes some polynomials $\h=(h_1,\dots,h_R)$, one can compute
  $\beta_i(h_r)$, for all $i=1,\dots,s$ and $r=1,\dots,R$, using
  $O(s^3\,\sigma)$ operations in $\LL$.
\end{lemma}
\begin{proof}
  By definition, for $h$ in $\LL[\bX]$ and $k=1,\dots,n$, the following equality
  holds:
$$
  \begin{bmatrix}
    \beta_1(X_k h)\\
    \vdots\\
    \beta_s(X_k h)
  \end{bmatrix}=
\bM_{k,s}
  \begin{bmatrix}
    \beta_1(h)\\
    \vdots\\
    \beta_s(h)
  \end{bmatrix},
\quad\text{with}\quad
\bM_{k,s}= \begin{bmatrix}
    \lambda^{(k)}_{1,1} & \cdots & \lambda^{(k)}_{s,1}\\
    \vdots && \vdots \\
    \lambda^{(k)}_{1,s} & \cdots & \lambda^{(k)}_{s,s}
  \end{bmatrix}.
$$ 
 Remark that the matrices $\bM_{k,s}$ all commute with each other. Indeed, 
for any $k,k'$ in $\{1,\dots,n\}$, and $h$ as above, the relation above implies
that 
$$
\Delta_{k,k',s}
  \begin{bmatrix}
    \beta_1(h)\\
    \vdots\\
    \beta_s(h)
  \end{bmatrix} =
  \begin{bmatrix}
0\\ \vdots \\ 0 
  \end{bmatrix},
$$
where $\Delta_{k,k',s} = \bM_{k,s}\bM_{k',s}-\bM_{k',s}\bM_{k,s}.$ Because 
the linear forms $\beta_1,\dots,\beta_s$ are linearly independent, this implies
that all rows of $\Delta_{k,k',s}$ must be zero, as claimed.
We then deduce that for any polynomial $h$ in $\LL[\bX]$, we have
the equality
$$  \begin{bmatrix}
    \beta_1(h)\\
    \vdots\\
    \beta_s(h)
  \end{bmatrix} =
h(\bM_{1,s},\dots,\bM_{n,s})   \begin{bmatrix}
    \beta_1(1)\\
    \vdots\\
    \beta_s(1)
  \end{bmatrix}. $$ On the other hand, our assumptions imply that the
  sequence $(\beta_1(1),\dots,\beta_s(1))$ is simply $(1,0,\dots,0)$.
  To prove the lemma, it is then enough to note that the evaluations \sloppy
  $h_1(\bM_{1,s},\dots,\bM_{n,s}),\dots,h_R(\bM_{1,s},\dots,\bM_{n,s})$
  can be computed using the straight-line program doing
  $O(s^3\,\sigma)$ operations.
\end{proof}

Mourrain's algorithm proceeds in an iterative manner, starting from
$\bbeta^{(1)}=(\beta_{1})$ (and setting $e_1=1$), and computing
successively $\bbeta^{(2)}=(\beta_{e_1+1},\dots,\beta_{e_2})$,
$\bbeta^{(3)}=(\beta_{e_2+1},\dots,\beta_{e_3})$, \dots for some
integers $e_1 \le e_2 \le e_3 \dots$ The algorithm stops when
$e_{\ell+1}=e_{\ell}$, in which case $\beta_1,\dots,\beta_{e_\ell}$ is
an $\LL$-basis of $I^\perp$, and $e_\ell=d$. In our case, we
are not interested in computing this multiplicity, but only in
deciding whether it is less than or equal to the parameter $\mu$. We do it as follows: assume that we have
computed $\bbeta^{(1)},\bbeta^{(2)},\dots,\bbeta^{(\ell)}$, together
with the corresponding integers $e_1,e_2,\dots,e_\ell$, with $e_1 <
\cdots < e_\ell \le \mu$. We compute $\bbeta^{(\ell+1)}$ and $e_{\ell+1}$,
and continue according to the following:
\begin{itemize}
\item if $e_{\ell+1}=e_{\ell}$, we conclude that the multiplicity
  $d$ of $I$ at the origin is $e_\ell \le \mu$; we stop the
  algorithm;
\item if $e_{\ell+1} > \mu$, we conclude that this multiplicity is greater 
  than $\mu$; we stop the algorithm;
\item else, when $e_\ell < e_{\ell+1} \le \mu$, we do another loop.
\end{itemize}
Because the $e_\ell$'s are an increasing sequence of integers, they
satisfy $e_\ell \ge \ell$; hence, every time we enter the loop above we
have $\ell \le \mu$. To finish the analysis of the algorithm, it
remains to explain how to compute $\bbeta^{(\ell+1)}$ from
$(\bbeta^{(1)},\bbeta^{(2)},\dots,\bbeta^{(\ell)})=(\beta_{1},\dots,\beta_{e_\ell})$.

As per our description above, at any step of the algorithm,
$\beta_{1},\dots,\beta_{e_\ell}$ are represented by means of the
coefficients $\lambda^{(k)}_{i,j}$, for $0 \le j < i \le e_{\ell}$ and
$1 \le k \le n$.  At step $\ell$, Mourrain's algorithm solves a homogeneous linear system
$T_\ell$ with $n(n-1) e_\ell/2+m'$ equations and $n e_\ell$ unknowns,
where $m'$ is the number of generators of the ideal $I= \langle \bC
\rangle + \m^{\mu+1}$. Remark that $m'$ is not polynomial in $\mu$ 
and $n$, so the size of $T_\ell$ is {\em a priori} too large to 
fit our cost bound; we will explain below how to resolve this issue.

The null space dimension of this linear system gives us the cardinality
$e_{\ell+1}-e_{\ell}$ of $\bbeta^{(\ell+1)}$. Similarly, the coordinates of
the $e_{\ell+1}-e_{\ell}$ vectors in a null space basis are precisely
the coefficients $\lambda^{(k)}_{i,j}$ for
$i=e_{\ell}+1,\dots,e_{\ell+1}$, $j=1,\dots,e_\ell$ and $k=1,\dots,n$
(we have $\lambda^{(k)}_{i,j}=0$ for $j=e_{\ell}+1,\dots,i-1$). For
all $\ell \ge 2$, all linear forms $\beta$ in $\bbeta^{(\ell)}$ are
such that for all $k$ in $\{1,\dots,n\}$, $X_k \cdot \beta$ belongs to
the span of $\bbeta^{(1)},\dots,\bbeta^{(\ell-1)}$; in particular, a
quick induction shows that all linear forms in
$\bbeta^{(1)},\dots,\bbeta^{(\ell)}$ vanish on all monomials of degree
at least $\ell$.

There remains the question of setting up the system $T_\ell$. For $k$
in $\{1,\dots,n\}$ and an $\LL$-linear form $\beta$, we denote by
$X_k^{-1} \cdot \beta$ the $\LL$-linear form defined by $\LL$-linearity as follows:
\begin{itemize}
\item $(X_k^{-1} \cdot \beta)(X_k f) = \beta(f)$ for any monomial $f$ in $\LL[\bX]$,
\item $(X_k^{-1} \cdot \beta)(f)=0$ if $f\in \LL[\bX]$ is a monomial which does not depend on $X_k$.
\end{itemize}
In other words,
$(X_k^{-1} \cdot \beta)(f)=\beta(\delta_k(f))$ holds for all $f$,
where $\delta_k:\LL[\bX] \to \LL[\bX]$ is the $k$th divided difference
operator
$$f\mapsto \frac
{f(X_1,\dots,X_n)-f(X_1,\dots,X_{k-1},0,X_{k+1},\dots,X_n)}{X_k}.$$
One verifies that, as the notation suggests, $X_k \cdot (X_k^{-1}
\cdot \beta)$ is equal to $\beta$. This being said, we can then
describe what the entries of $T_\ell$ are:
\begin{itemize}
\item the first $n(n-1) e_\ell/2$ equations involve only the coefficients 
  $\lambda^{(k)}_{i,j}$ previously computed (we refer to~\cite[Section~4.4]{Mourrain97} for details of how exactly 
these entries are distributed in $T_\ell$, as we do not need such details here).
\item each of the other $m'$ equations has coefficient vector
{\small
$$v_f = \big (\
 (X_k^{-1} \cdot \beta_1)(f(X_1,\dots,X_k,0,\dots,0)),\dots,\ (X_k^{-1} \cdot \beta_{e_\ell})(f(X_1,\dots,X_k,0,\dots,0))\
\big )_{1 \le k \le n},$$}
where $f$ is a generator of $I=\langle \bC \rangle +\m^{\mu+1}$.
\end{itemize}
We claim that only those equations corresponding to generators
$c_1,\dots,c_m$ of the input system $\bC$ are useful, as all others
are identically zero.

We pointed out above that any linear form $\beta_i$ in
$\beta_1,\dots,\beta_{e_\ell}$ vanishes on all monomials of degree at
least $\ell$. Since we saw that we must have $\ell \le \mu$, all
$\beta_i$ as above vanish on monomials of degree $\mu$; this implies
that $X_k^{-1}\cdot \beta_i$ vanishes on all monomials of degree
$\mu+1$. The generators $f$ of $\m^{\mu+1}$ have degree $\mu+1$, and
for any such $f$, $f(X_1,\dots,X_k,0,\dots,0)$ is either zero, or of
degree $\mu+1$ as well. Hence, for any $k$, $\beta_i$ in
$\beta_1,\dots,\beta_{e_\ell}$ and $f$ as above, $(X_k^{-1} \cdot
\beta_i)(f(X_1,\dots,X_k,0,\dots,0))$ vanishes. This implies that the
vector $v_f$ is identically zero for such an $f$, and that the
corresponding equation can be discarded.

Altogether, as claimed above, we see that we have to compute the
values
$$(X_k^{-1} \cdot \beta_i)(c_j(X_1,\dots,X_k,0,\dots,0)),$$ for
$k=1,\dots,n$, $i=1,\dots,e_\ell$ and $j=1,\dots,m$.  Fixing $k$, we
let $\bC_k = (c_{j,k})_{1 \le j \le m}$, where $c_{j,k}$ is the
polynomial $c_j(X_1,\dots,X_k,0,\dots,0)$; note that the system
$\bC_k$ can be computed by a straight-line program of length
$\sigma'= \sigma+n$. Then, applying the following lemma with
$s=e_\ell \le \mu$ and $\h = \bC_k$, we deduce that the values
$(X_k^{-1} \cdot \beta_i)(c_j(X_1,\dots,X_k,0,\dots,0))$, for $k$
fixed, can be computed in time $O(\mu^3 (\sigma+n))$.


\begin{lemma}
  Let $s$ be in $1,\dots,d$, and suppose that the coefficients
  $\lambda^{(k)}_{i,j}$ are known for $i=1,\dots,s$, $j=0,\dots,i-1$
  and $k=1,\dots,n$. Given a straight-line program $\Gamma$ of length
  $\sigma$ that computes $\h=(h_1,\dots,h_R)$ and given $k$ in
  $\{1,\dots,n\}$, one can compute $(X_k^{-1}\cdot \beta_i)(h_r)$, for
  all $i=1,\dots,s$ and $r=1,\dots,R$, using $O(s^3 (\sigma+n))$
  operations in $\LL$.
\end{lemma}
\begin{proof}
  In view of the formula $(X_k^{-1} \cdot
  \beta)(f)=\beta(\delta_k(f))$, and of Lemma~\ref{lemma:evalbeta}, it is
  enough to prove the existence of a straight-line program of length
  $O(\sigma+n)$ that computes $(\delta_k(h_1),\dots,\delta_k(h_R))$.

  To do this, we replace all polynomials
  $\gamma_{-n+1},\dots,\gamma_\sigma$ computed by $\Gamma$ by terms
  $\eta_{-n+1},\dots,\eta_\sigma$ and $\nu_{-n+1},\dots,\nu_\sigma$,
  with
  $\eta_\ell=\gamma_\ell(X_1,\dots,X_{k-1},0,X_{k+1},\dots,X_n)$
  and $\nu_\ell$ in $\LL[\bX]$ such that
  $\gamma_\ell= \eta_\ell+X_k \nu_\ell$ holds for all $\ell$, so
  that in particular $\nu_\ell=\delta_k(\gamma_\ell)$.  To compute
  $\eta_\ell$ and $\nu_\ell$, assuming all previous
  $\eta_{\ell'}$ and $\nu_{\ell'}$ are known, we proceed as
  follows:
  \begin{itemize}
  \item if $\gamma_\ell=X_k$, we set $\eta_\ell=0$ and $\nu_\ell=1$;
  \item if $\gamma_\ell=X_{k'}$, with $k' \ne k$, we set $\eta_\ell=X_{k'}$ and $\nu_\ell=0$;
  \item if $\gamma_\ell =c_\ell$, with $c_\ell \in \LL$,
    then we set $\eta_\ell=c_\ell$ and  $\nu_\ell=0$;
  \item if $\gamma_\ell = \gamma_{a_\ell} \pm \gamma_{b_\ell}$,
    for some indices $a_\ell,b_\ell < \ell$, 
    then we set $\eta_\ell=\eta_{a_\ell}\pm\eta_{b_\ell}$
    and $\nu_\ell=\nu_{a_\ell}\pm\nu_{b_\ell}$;
\item if $\gamma_\ell = \gamma_{a_\ell} \gamma_{b_\ell}$,
      for some indices $a_\ell,b_\ell < \ell$,
    then we set $\eta_\ell=\eta_{a_\ell} \eta_{b_\ell}$
    and $$\nu_\ell=
\eta_{a_\ell} \nu_{b_\ell}
+
\nu_{a_\ell} \eta_{b_\ell}
+
X_k\nu_{a_\ell} \nu_{b_\ell}.$$
\end{itemize}
One verifies that in all cases, the relation $\gamma_\ell=
\eta_\ell+X_k \nu_\ell$ still holds. Since the previous
construction allows us to compute $\eta_\ell$ and $\nu_\ell$ in
$O(1)$ operations from the knowledge of all previous $\eta_{\ell'}$
and $\nu_{\ell'}$, we deduce that all $\eta_\ell$ and $\nu_\ell$,
for $\ell=-n+1,\dots,\sigma$, can be computed by a straight-line program of
length $O(\sigma+n)$.
\end{proof}

Taking all values of $k$ into account, we see that we can compute all
entries we need to set up the linear system $T_\ell$ using $O(\mu^3
n(\sigma+n))$ operations in $\LL$. After discarding the useless equations
described above, the numbers of equations and unknowns in the system
$T_\ell$ are respectively at most $n^2 \mu+m$ and $n \mu$; this
implies that we can find a null space basis of it in time $O(n^4 \mu^3
+ n^2 m \mu^2)$. Altogether, the time spent to find
$\bbeta^{(\ell+1)}$ from
$(\bbeta^{(1)},\bbeta^{(2)},\dots,\bbeta^{(\ell)})=(\beta_{1},\dots,\beta_{e_\ell})$
is $O(n^4 \mu^3 + n^2 m \mu^2 + n \sigma \mu^3)$.

Since we saw that we do at most $\mu$ such loops, the cumulative time
is $O(n^4 \mu^4 + n^2 m \mu^3 + n \sigma \mu^4)$, and
Proposition~\ref{prop:testisolated} is proved.

%%%%%%%%%%%%%%%%%%%%%%%%%%%%%%%%%%%%%%%%%%%%%%%%%%%%%%%%%%%%
%%%%%%%%%%%%%%%%%%%%%%%%%%%%%%%%%%%%%%%%%%%%%%%%%%%%%%%%%%%%
%%%%%%%%%%%%%%%%%%%%%%%%%%%%%%%%%%%%%%%%%%%%%%%%%%%%%%%%%%%%

\section{Some properties of determinantal ideals}\label{sec:check}

In the next sections, we develop our homotopy algorithms for
determinantal systems. These algorithms rely on certain dimension and
unmixedness properties; we establish some of these properties in this
section.

Let $T$ and $\bX=(X_1,\dots,X_n)$ be variables, let $\mV=(v_1,\dots,v_s)$ be
polynomials in $\KK[T,\bX]$, with $s \le n$, and let $\mU$ be a polynomial
matrix in $\KK[T,\bX]^{p \times q}$, with $p \le q$. Later on, $\mV$ and $\mU$
will be defined as in~\eqref{eqdef:bV} and~\eqref{eqdef:bU}, but this is not
required for the moment. Define $\bB=(b_1,\dots,b_m)$ by taking $(b_1,\dots,b_s)
= (v_1,\dots,v_s)$, and by letting $(b_{s+1},\dots,b_m)$ be the $p$-minors of
$\bU$, and let $J$ be the ideal generated by $\bB$ in $\KKbar[T,\bX]$. Below, we
denote by $\bar J$ the ideal generated by only the minors
 $(b_{s+1},\dots,b_m)$.
\begin{proposition}\label{prop:KH1H2}
  If $n=q-p+s+1$, the ideal $J$ satisfies the following properties:
\begin{itemize}[leftmargin=8mm]
\item Any irreducible component of $V(J) \subset
  \KKbar{}^{n+1}$ has dimension at least one.
\item For any maximal ideal $\m \subset\KKbar[T,\bX]$,
  if the localization $J_\m \subset \KKbar[T,\bX]_\m$ has height $n$,
  then it is unmixed (that is, all associated primes have height $n$).
\end{itemize}
\end{proposition}
Remark first that when $p=1$, $J$ is defined by $q+s$ polynomials ($q$
entries from $\mU$ and $s$ polynomials from $\mV$), and $q+s=n=n-p+1$ in
this case. Then, these properties are well-known: the first one is is
Krull's theorem, the second one is Macaulay's unmixedness theorem in
the Cohen-Macaulay ring
$\KKbar[T,\bX]_\m$~\cite[Corollary~18.14]{Eisenbud95}.

The proof in the general case occupies the rest of this section. We start by
recalling, without proof, results from~\cite[Section~6]{EN62}.

\begin{lemma}\label{lemma:EN}
  Let $R$ be a Cohen-Macaulay ring and let $I$ be the ideal generated
  by all $p$-minors of a matrix in $R^{p\times q}$,
  with $p \le q$. Then:
  \begin{itemize}
  \item if $I \ne R$, then the height of $I$ is at most $q-p+1$;
  \item if $I$ has height $q-p+1$, then $I$ is unmixed (all associated
    primes have height $q-p+1$).
\end{itemize}
\end{lemma}

To prove Proposition~\ref{prop:KH1H2}, let $W_1,\dots,W_k$ be the
$\KKbar$-irreducible components of $V(\bar J)\subset
\KKbar{}^{n+1}$. We prove in the next paragraph that $\dim(W_i) \ge
(n+1) -(q-p+1)$ holds for all $i$. Of course, we can assume that
$V(\bar J)\ne \emptyset$, so that $\bar J \ne \KKbar[T,\bX]$, otherwise the
proposition itself would be vacuously true.

First, remark that for a point $\bx$ in $V(\bar J)\subset \KKbar{}^{n+1}$, and
writing $\m \subset \KKbar[T,\bX]$ for the maximal ideal at $\bx$, the height of
$\bar J_\m$ in $\KKbar[T,\bX]_\m$ is equal to $(n+1)-\max\{ \dim(W_i) \mid 1 \le i
\le k, \bx \in W_i\}$. For $i=1,\dots,k$, let then $\bx_i$ be a point in $W_i$
that does not belong to any other $W_{i'}$, $i' \ne i$, and let $\m_i$ be the
corresponding maximal ideal; then, the previous equality becomes ${\rm
  height}(\bar J_{\m_i})=(n+1)-\dim(W_i)$. Applying the first item in the lemma above
in $\KKbar[T,\bX]_{\m_i}$ (which is Cohen-Macaulay), we deduce that
$(n+1)-\dim(W_i) \le q-p+1$, that is, $\dim(W_i) \ge (n+1) -(q-p+1)$.

Notice that we can rewrite $(n+1)-(q-p+1)$ as $s+1$.  Since $\mV$
consists of $s$ polynomials, all irreducible components of $V(J)$ must
have dimension at least $1$, by Krull's theorem. The first property
 is proved.

For the second one, let $J_\m=Q_1 \cap \cdots \cap Q_t$ be an
irredundant primary decomposition of $J_\m$ in $\KKbar[T,\bX]_\m$, and
let $P_1,\dots,P_t$ be the corresponding primes; we assume that the
height of $J_\m$ is $n$, and our goal is to prove that all $P_i$'s
have height~$n$.

Of course, we can restrict to an ideal $\m$ containing $J$; $\m$ is
then the maximal ideal at a point $\x \in \KKbar{}^{n+1}$ that belongs
to $V(J)$. The height of the localization
$J_\m \subset \KKbar[T,\bX]_\m$ can be rewritten as
$(n+1)-\dim(V_\x)$, where $V_\x$ is the union of the irreducible
components of $V(J)$ passing through $\x$. Our assumption 
is that the height of $J_\m$ is $n$, that is, that
$\dim(V_\x)=1$. Thus, every irreducible component of $V(J)$ containing
$\x$ has dimension~$1$.

Let $W$ be an irreducible component of $V(\bar J)$ containing $\x$.  We
claim that $\dim(W)=s+1$. Indeed, we saw in the first paragraph
that $\dim(W) \ge s+1$. If $\dim(W) > s+1$, then by Krull's theorem,
every irreducible component of $W \cap V(\mV)$ has dimension greater
than $1$; since $W \cap V(\mV)$ is a subset of $V(J)$ and contains $\x$,
we have reached a contradiction. Now, the fact that $\dim(W)=s+1$ for
any irreducible component of $V(\bar J)$ containing $\x$ means that
$\bar J_\m$ has height $n-s=q-p+1$.  As a
result,~\cite[Theorem~18.18]{Eisenbud95} shows that
$\KKbar[T,\bX]_\m/\bar J_\m$ is Cohen-Macaulay.

For an ideal $I \subset \KKbar[T,\bX]_\m$, we denote by $\bar I$ its
image modulo $\bar J_\m$.  By the remarks
following~\cite[Theorem~IV.5.9]{ZaSa58},
$\bar Q_1 \cap \cdots \cap \bar Q_t$ is an irredundant primary
decomposition of $\bar J_\m$ in
$\KK[T,\bX]_\m/\bar J_\m$, with associated primes
$\bar P_1,\dots,\bar P_t$. In addition, if we let $P_1,\dots,P_u$ be
the minimal primes of $J_\m$, for some $s \le t$,
$\bar P_1,\dots,\bar P_u$ are the minimal primes of $\bar J_\m$.

Our assumption says that $P_1,\dots,P_u$ have height $n$. Because
$\KKbar[T,\bX]_\m/\bar J_\m$ is local and Cohen-Macaulay, for any
$i \le t$, we have 
$$\dim(\KKbar[T,\bX]_\m/\bar J_\m)=\dim((\KKbar[T,\bX]_\m/\bar J_\m) / \bar P_i) + {\rm height}(\bar P_i)$$
by~\cite[Theorem~17.4(i)]{Matsumura86}.
The factor ring $(\KKbar[T,\bX]_\m/\bar J_\m) / \bar P_i$ is simply
$\KKbar[T,\bX]_\m/P_i$, so this can be rewritten as
$$s+1 = \dim(\KKbar[T,\bX]_\m/P_i) + {\rm height}(\bar P_i).$$ For $i\le
u$, we have $\dim(\KKbar[T,\bX]_\m/P_i)=1$, so that ${\rm height}(\bar
P_i)=s$; for $i > u$, the height of $\bar P_i$ is necessarily
$s+1$. Because $\bar P_1,\dots,\bar P_u$ are the minimal primes of
$\bar J_\m$, the height of $\bar J_\m$ is thus $s$ as well.

The ideal $\bar J_\m$ is generated in $\KKbar[T,\bX]_\m/\langle \bB'
\rangle_\m$ by $\mV$. Since $\KKbar[T,\bX]_\m/\langle
\bB' \rangle_\m$ is Cohen-Macaulay, $\bar J_\m$ is unmixed, that is,
$u=t$ (second property in the previous lemma).  As a result, $Q_1 \cap
\cdots \cap Q_u$ is an irredundant primary decomposition of $J_\m$,
and $J_\m$ is unmixed. The proof of Proposition~\ref{prop:KH1H2} is complete.

%%%%%%%%%%%%%%%%%%%%%%%%%%%%%%%%%%%%%%%%%%%%%%%%%%%%%%%%%%%%
%%%%%%%%%%%%%%%%%%%%%%%%%%%%%%%%%%%%%%%%%%%%%%%%%%%%%%%%%%%%
%%%%%%%%%%%%%%%%%%%%%%%%%%%%%%%%%%%%%%%%%%%%%%%%%%%%%%%%%%%%

\section{Some properties of determinantal homotopies}\label{sec:homotopy}

We now consider polynomials $\mG=(g_1,\dots,g_s)$ in $\KK[\bX]$, with
$s \le n$, a polynomial matrix $\mF$ in $\KK[\bX]^{p \times q}$, with
$p \le q$ and we let $\bC=(c_1,\dots,c_{s},\dots,c_m)$ be polynomials
defined as follows: $(c_1,\dots,c_{s})=(g_1,\dots,g_s)$, and
$(c_{s+1},\dots,c_{m})$ are the $p$-minors of $\mF$, so that $m=s+{q
  \choose p}$. The set that interests us, $\VpFG{p}{\mF}{\mG}$, is the
zero-set of~$\bC$. As in the previous section, we assume $n=q-p+s+1$.

As in the introduction, suppose that we are given a polynomial matrix
$\bL \in \K[\bX]^{p \times q}$ and polynomial equations
$\bM=(m_1,\dots,m_s)$ in $\K[X_1,\dots,X_n]$. Our {\em start system}
$\bA=(a_1,\dots,a_s,\dots,a_m)$ is defined by taking $(a_1,\dots,a_s) =
(m_1,\dots,m_s)$, and letting $(a_{s+1},\dots,a_m)$ be the $p$-minors
of $\bL$; these polynomials are in $\KK[\bX]$. 

Let finally $T$ be our homotopy variable; we define the matrix
\[\bU = (1-T)\cdot \bL + T \cdot \mF \in \KK[T, \bX]^{p \times q}\]
and the polynomials $\bV=(v_1,\dots,v_s)$ in $\K[T,\bX]$ by
\[\bV = (1-T) \cdot \bM + T \cdot \mG.\]
This allows us to define polynomials $\bB=(b_1,\dots,b_s,\dots,b_m)$
in $\KK[T,\bX]$ by setting $(b_1,\dots,b_s)=(v_1,\dots,v_s)$
and letting $(b_{s+1},\dots,b_m)$ be the $p$-minors of $\mU$.

For $\tau$ in $\KKbar$, we write
$\bB_{T=\tau}=(b_{\tau,1},\dots,b_{\tau,m})=\bB(\tau,\bX)\subset
\KKbar[\bX]$. In particular, we have $\bB_{T=0}=\bA$ and
$\bB_{T=1}=\bC$.  The main result in this section is the following; it
relates the number of isolated solutions, counted with multiplicities,
of the system $\bC=\bB_{T=1}$, and more generally at any value $\tau
\in \KKbar$, to those for the start system $\bA=\bB_{T=0}$, assuming the latter
satisfies certain regularity properties.
\begin{proposition}\label{prop:degree_fiber}
  There exists an integer $c$ such that for all $\tau$ in $\KKbar$, the
  sum of the multiplicities of the isolated solutions of $\bB_{T=\tau}$
  is at most $c$. Suppose further that the following three conditions
  hold:
\begin{description}[leftmargin=*]
\item[$\assG_1.$] For $k=1,\dots,m$, $\deg_\bX(b_k)=\deg_\bX(a_k)$,
  where $\deg_\bX$ denotes the degree in $\bX$.
\item[$\assG_2.$] The only common solution to
  $a_1^H(0,\bX)=\cdots=a_m^H(0,\bX)=0$ is $(0,\dots,0)\in\KKbar{}^n$,
  where for $k=1,\dots,m$, $a_{k}^H$ is the polynomial in
  $\KKbar[X_0,\bX]$ obtained by homogenizing $a_{k}$ using a new
  variable $X_0$.
\item[$\assG_3.$] The ideal $\langle \bA \rangle$ is radical in $\KKbar[\bX]$.
\end{description}
Then, $\bA$ has exactly $c$ solutions, all of them having multiplicity
one.
\end{proposition}

The rest of the section is dedicated to proving this proposition.
In the course of the proof, we will give a precise characterization of
the integer $c$ mentioned in the proposition, although the statement
given in the proposition will actually be enough for our further
purposes.

Consider an irredundant primary decomposition of the ideal $J=\langle
\bB\rangle$ in $\KKbar[T,\bX]$, of the form $J=Q_1 \cap \cdots \cap
Q_r$, and let $P_1,\dots,P_r$ be the associated primes, that is, the
respective radicals of $Q_1,\dots,Q_r$. We assume that $P_1,\dots,P_\rho$
are the minimal primes, for some $s \le r$, so that
$V(P_1),\dots,V(P_\rho)$ are the (absolutely) irreducible components of
$V(J)\subset \KKbar{}^{n+1}$. By the first property of
Proposition~\ref{prop:KH1H2}, these irreducible components all have
dimension at least one. Refining further, we assume that $t \le s$ is
such that $V(P_1),\dots,V(P_t)$ are the irreducible components of
$V(J)$ of dimension one whose image by $\pi_T: (\tau,x_1,\dots,x_n)
\mapsto \tau$ is Zariski dense in $\KKbar$.

\begin{lemma}\label{lemma:vPi}
  Let $\tau$ be in $\KKbar$ and let $\bx \in \KKbar{}^n$ be an isolated
  solution of the system $\bB_{T=\tau}$. Then, $(\tau,\bx)$ belongs to $V(P_i)$
  for at least one index $i$ in $\{1,\dots,t\}$, and does not belong
  to $V(P_i)$ for any index $i$ in $\{t+1,\dots,r\}$.
\end{lemma}
\begin{proof}
  Because $(\tau,\bx)$ cancels $\bB$, it belongs to at least one of
  $V(P_1),\dots,V(P_r)$. It remains to rule out the possibility that
  $(\tau,\bx)$ belongs to $V(P_i)$ for some index $i$ in
  $\{t+1,\dots,r\}$.

  We first deal with indices $i$ in $\{t+1,\dots,s\}$. These are those
  primary components with minimal associated primes $P_i$ that either
  have dimension at least two, or have dimension one but whose image
  by $\pi_T$ is a single point. In both cases, all irreducible
  components of the intersection $V(P_i)\cap V(T-\tau)$ have dimension
  at least one. Since $\bx$ is isolated in $V(\bB_{T=\tau})$, $(\tau,\bx)$ is
  isolated in $V(\bB)\cap V(T-\tau)$, so it cannot belong to
  $V(P_i)\cap V(T-\tau)$ for any $i$ in $\{t+1,\dots,s\}$.
  
  We conclude by proving that $(\tau,\bx)$ does not belong to $V(P_i)$,
  for any of the embedded primes $P_{\rho+1},\dots,P_r$. We proceed by
  contradiction, assuming for definiteness that $(\tau,\bx)$ belongs to
  $V(P_{\rho+1})$. Because $P_{\rho+1}$ is an embedded prime, $V(P_{\rho+1})$
  is contained in (at least) one of $V(P_1),\dots,V(P_\rho)$. In view of
  the previous paragraph, it cannot be one of
  $V(P_{t+1}),\dots,V(P_\rho)$.  Now, all of $V(P_1),\dots,V(P_t)$ have
  dimension one, so $V(P_{\rho+1})$ has dimension zero (so it is the point $\{(\tau,\bx)\}$). For the same
  reason, if $(\tau,\bx)$ belonged to another $V(P_i)$, for some $i >
  s+1$, $V(P_i)$ would also be zero-dimensional, and thus equal to $\{(\tau,\bx)\}$; as a result, $V(P_i)$
  would be equal to $V(P_{\rho+1})$, and this would contradict the
  irredundancy of our decomposition.
  
  To summarize, $(\tau,\bx)$ belongs to $V(P_{\rho+1})$, together with
  $V(P_i)$ for some indices $P_i$ in $\{1,\dots,t\}$ (say
  $P_1,\dots,P_u$, up to reordering, for some $u \ge 1$), and avoids
  all other associated primes.  Let us localize the decomposition
  $J=Q_1 \cap \cdots \cap Q_r$ at
  $P_{\rho+1}$. By~\cite[Proposition~4.9]{AtMc},
  $J_{P_{\rho+1}}={Q_1}_{P_{\rho+1}} \cap \cdots \cap {Q_u}_{P_{\rho+1}}\cap
  {Q_{\rho+1}}_{P_{\rho+1}}$ is an irredundant primary decomposition of
  $J_{P_{\rho+1}}$ in $\KKbar[T,\bX]_{P_{\rho+1}}$; the minimal primes are
  ${P_1}_{P_{\rho+1}},\dots,{P_u}_{P_{\rho+1}}$.

  By Corollary~4 p.24 in~\cite{Matsumura86}, for any prime
  ${P_i}_{P_{\rho+1}}$, $i=1,\dots,u$ or $i=\rho+1$, the localization of
  $\KKbar[T,\bX]_{P_{\rho+1}}$ at ${P_i}_{P_{\rho+1}}$ is equal to
  $\KKbar[T,\bX]_{P_{i}}$. In particular, the height of
  ${P_i}_{P_{\rho+1}}$ in $\KKbar[T,\bX]_{P_{\rho+1}}$ is equal to that of
  $P_i$ in $\KKbar[T,\bX]_{P_{i}}$, that is, $n$ if $i=1,\dots,u$,
  since then $V(P_i)$ has dimension $1$, or $n+1$ if $i=\rho+1$. Since $u
  \ge 1$, this proves that $J_{P_{\rho+1}}$ has height $n$. As a result,
  the second property of Proposition~\ref{prop:KH1H2} implies that
  $J_{P_{\rho+1}}$ is unmixed, a contradiction.
\end{proof}

Let us write $J=J' \cap J''$, with $J'=Q_1 \cap \cdots \cap Q_t$ and
$J''=Q_{t+1} \cap \cdots \cap Q_r$. For $\tau$ in $\KKbar$, we denote
by $J_{T=\tau} \subset \KKbar[T,\bX]$ the ideal $J + \langle T-\tau \rangle$,
and similarly for $J'_{T=\tau}$ and $ J''_{T=\tau}$.

\begin{remark}\label{rem:Jprime}
  The zero-set $V(J')$ is the union of all one-dimensional irreducible
  components of $V(J)=V(\bB) \subset \KKbar{}^{n+1}$ whose projection on
  the $T$-axis is dense. As a consequence, we will call it the {\em
    homotopy curve}.
\end{remark}

\begin{lemma}\label{lemma:JJprime}
  Let $\tau$ and $\bx$ be as in Lemma~\ref{lemma:vPi}. Then, the
  multiplicities of the ideals $J_{T=\tau}$ and $J'_{T=\tau}$ at $(\tau,\bx)$
  are the same.
\end{lemma}
\begin{proof}
  Without loss of generality, assume that $\tau=0 \in \KKbar$ and
  $\bx=0 \in \KKbar{}^n$. We start from the equality $J=J' \cap J''$,
  which holds in $\KKbar[T,\bX]$, and we see it in the formal power
  series ring $\KKbar[[T,\bX]]$.  The previous lemma implies that
  there exists a polynomial in $J''$ that does not vanish at
  $(\tau,\bx)=0 \in \KKbar{}^{n+1}$.  This polynomial is a unit in
  $\KKbar[[T,\bX]]$, which implies that the extension of $J''$ in
  $\KKbar[[T,\bX]]$ is the trivial ideal $\langle 1 \rangle$, and
  finally that the equality of extended ideals $J=J'$ holds in
  $\KKbar[[T,\bX]]$. This implies the equality
  $J+\langle T \rangle =J'+\langle T \rangle $ in $\KKbar[[T,\bX]]$,
  and the conclusion follows.
\end{proof}

Our goal is now to give a bound on the sum of the multiplicities of
$\bB_{T=\tau}$ at all its isolated roots, for any $\tau$ in $\KKbar$.

To achieve this, we consider the Puiseux series field
$\SS = \KKbar\langle \langle T\rangle \rangle$ in $T$ with
coefficients in $\KKbar$.  Since $\KKbar$ is algebraically closed and
of characteristic $0$, $\SS$ is algebraically closed (actually, it is
an algebraic closure of $\KKbar(T)$) and hence a perfect field.

Next, we consider the extension $\frak{J}$ of $J$ in $\SS[\bX]$, and
similarly let $\frak{J}'$ and ${\frak J}''$ denote the extensions of $J'$ and
$J''$ in $\SS[\bX]$.

\begin{lemma}\label{lemma:dimJprime}
  The ideal $\frak{J}'$ has dimension zero and $V(\frak{J}') \subset
  \SS^n$ is the set of isolated solutions of
  $V(\frak{J}) \subset \SS^n$.
\end{lemma}
\begin{proof}
 From the equality $J=J' \cap J''$ and Corollary~3.4 in~\cite{AtMc},
 we deduce that $\frak{J}=\frak{J'} \cap \frak{J''}$. The properties
 of $J'$ (that the irreducible components of $V(J')$ are precisely those
 irreducible components of $V(J)$ that have dimension one and with a
 dense image by $\pi_T$) imply our claim.
\end{proof}


Let us write $c=\dim_{\SS}(\SS[\bX]/{\frak J}')$. Because $\SS$ is an
algebraic closure of $\KKbar(T)$, one has $\dim_{\KKbar(T)} (
\KKbar(T)[\bX]/\tilde{J'} ) = c$ where $\tilde{J'}$ is the extension
of $J'$ in $\KKbar(T)[\bX]$. The following lemma relates this
quantity to the multiplicities of the solutions in any fiber
$\bB_{T=\tau}$. This proves the first statement in
Proposition~\ref{prop:degree_fiber}.

 
\begin{lemma}\label{lemma:19}
  Let $\tau$ be in $\KKbar$. The sum of the multiplicities of the
  isolated solutions of $\bB_{T=\tau}$ is at most equal to $c$.
\end{lemma}
\begin{proof}
  The sum in the lemma is also the sum of the multiplicities of the
  ideal $J_{T=\tau}$ at all $(\tau,\bx)$, for $\bx$ an isolated solution
  of $\bB_{T=\tau}$.  By Lemma~\ref{lemma:JJprime}, this is also the sum
  of the multiplicities of $J'_{T=\tau}$ at all $(\tau,\bx)$, for $\bx$ an
  isolated solution of $\bB_{T=\tau}$. We prove below that the sum of the
  multiplicities of $J'_{T=\tau}$ at all $(\tau,\bx)$, for $\bx$ such that
  $(\tau,\bx)$ cancels $J'_{T=\tau}$, is at most $c$; this will be enough
  to conclude (for any isolated solution $\bx$ of $\bB_{T=\tau}$,
  $(\tau,\bx)$ is a root of $J'_{T=\tau}$, though the converse may not be
  true). Remark that the latter sum is simply the dimension of
  $\KKbar[T,\bX]/J'_{T=\tau}$.
  
  Let $k$ be the dimension of $\KKbar[T,\bX]/J'_{T=\tau}$ and $m_1,\dots,m_k$ be
  monomials that form a $\KKbar$-basis of $\KKbar[T,\bX]/J'_{T=\tau}$; since
  $T-\tau$ is in $J'_{T=\tau}$, these monomials can be assumed not to involve
  $T$. We will prove that they are still $\KKbar(T)$-linearly independent in
  $\KKbar(T)[\bX]/\tilde{J'}$; this will imply that $k \le c$, and finish the
  proof.
  
  Suppose that there exists a linear combination $A_1 m_1 + \cdots +
  A_k m_k$ in $\tilde{J}'$, with all $A_i$'s in $\KKbar(T)$, not
  all of them zero. Thus, we have an equality $A'_1/d_1\, m_1 + \cdots
  + A'_k/d_k\, m_k = A/d$, with $A_1,\dots,A_k$ and
  $d,d_1,\dots,d_k$ in $\KKbar[T]$ and $A$ in the ideal
  $J'$. Clearing denominators, we obtain a relation of the form $e_1
  m_1 +\cdots+ e_k m_k \in J'$, with not all $e_i$'s zero. Let
  $(T-\tau)^u$ be the highest power of $T-\tau$ that divides all
  $e_i$'s (this is well-defined, since not all $e_i$'s vanish) so that
  we can rewrite the above as $(T-\tau)^u (f_1 m_1 +\cdots+ f_k
  m_k) \in J'$, with $f_i=e_i/(T-\tau)^u \in \KKbar[T]$ for all $i$.
  In particular, our definition of $e_i$ implies that the values
  $f_i(\tau)$ are not all zero.

  Recall that the ideal $J'$ has the form $J'=Q_1 \cap \cdots \cap
  Q_t$. For $i=1,\dots,t$, since $Q_i$ is primary, the membership
  equality $(T-\tau)^u (f_1 m_1 +\cdots +f_k m_k) \in J'$ implies
  that either $f_1 m_1 +\cdots +f_k m_k$ or some power
  $(T-\tau)^{v}$, for some $v > 0$, is in $Q_i$. Since $Q_i$ does not
  contain non-zero polynomials in $\KKbar[T]$, $f_1 m_1 +\cdots+ f_k
  m_k$ belongs to all $Q_i$'s, that is, to $J'$. We can then
  evaluate this relation at $T=\tau$. We saw that the values
  $f_i(\tau)$ do not all vanish on the left, which is a contradiction
  with the independence of the monomials $m_1,\dots,m_k$ modulo
  $J'_{T=\tau}$.
\end{proof}


We now take $\tau=0$ and we discuss the geometry of $V(J)$ in a
neighbourhood of $T=0$.  We already emphasized that the field $\SS$ is an
algebraic closure of $\KKbar(T)$; we thus let $\Phi_1,\dots,\Phi_{c'}$
be the points of $V(\mathfrak{J}')$, with coordinates taken in
$\SS$. In particular, we see that $c' \le c$; we prove below that if
$\assG_1$, $\assG_2$ and $\assG_3$ hold, we actually have $c'=c$ (that
is, that $\mathfrak{J}'$ is radical).

Any non-zero series $\varphi$ in $\SS$ admits a well-defined {\em
  valuation} $\nu(\varphi)$, which is the smallest exponent that
appears in its expansion with a non-zero coefficient; we also set
$\nu(0)=\infty$. The valuation $\nu(\Phi)$, for a vector
$\Phi=(\varphi_1,\dots,\varphi_r)$ with entries in $\SS$, is the
minimum of the valuations of its exponents. We say that $\Phi$ is {\em
  bounded} if it has non-negative valuation; in this case,
$\lim_0(\Phi)$ is defined as the vector
$(\lim_0(\varphi_1),\dots,\lim_0(\varphi_s))$, with
$\lim_0(\varphi_i)={\rm coeff}(\varphi_i,T^0)$ for all $i$.

Without loss of generality, we assume that
$\Phi_1,\dots,\Phi_\kappa$ are bounded, and
$\Phi_{\kappa+1},\dots,\Phi_{c'}$ are not, for some $\kappa$ in
$\{0,\dots,c'\}$, and we define $\varphi_1,\dots,\varphi_\kappa$ by
$\varphi_i=\lim_0(\Phi_i)\in\KKbar{}^n$ for
$i=1,\dots,\kappa$.

\begin{lemma}\label{lemma:Z1}
  The equality $V(J' +\langle T \rangle)=\{\varphi_i \mid i=1,\dots,\kappa\}$ holds.
\end{lemma}
\begin{proof}
  Let $(s_1,\dots,s_h)$ be generators of the ideal $J'$ in
  $\KKbar[T,\bX]$; they also generate $\mathfrak{J}'$ in
  $\KKbar(T)[\bX]$. Then, the polynomials $s_{0,i}=s_i(0,\bX) \in
  \KKbar[\bX]$, for $i=1,\dots,h$, are such that $J'+\langle T\rangle
  = \langle T,s_{0,1},\dots,s_{0,h} \rangle$.  Consider $i \le
  \kappa$, and the corresponding vector of series $\Phi_i$. We know
  that for $j=1,\dots,h$, we have $s_j(\Phi_i)=0$.  Since all elements
  involved have non-negative valuation, we can take the coefficient of
  degree $0$ in $T$ in this equality and deduce
  $s_{0,j}(\varphi_i)=0$, as claimed. Hence, each $\varphi_i$, for $i
  \le \kappa$, is in $V(J' + \langle T \rangle)$.

  Conversely, take indeterminates $T_1,\dots,T_n$, and let $\LL$ be
  the algebraic closure of the field $\KKbar(T_1,\dots,T_n)$; let
  ${\cal C}\subset{\LL}{}^{n+1}$ be the zero-set of the ideal $J'\cdot
  \LL[T,\bX]$ and consider the projection ${\cal C} \to {\LL}{}^2$
  defined by $(\tau,x_1,\dots,x_n)\mapsto (\tau,T_1 x_1 + \cdots + T_n
  x_n)$. The Zariski closure ${\cal S}$ of the image of this mapping
  is a hypersurface, that is, a plane curve.  Since the ideal $J'$ is generated by polynomials
  with coefficients in $\KKbar$, one deduces that ${\cal S}$ admits a square-free
  defining equation in $\KKbar(T_1,\dots,T_n)[T,T_0]$.

  Consider such a polynomial, say $C$, and assume without loss of
  generality that $C$ belongs to 
  $\KKbar[T_1,\dots,T_n][T,T_0]$. Because ${\cal C}$ admits no irreducible
  component lying above $T=\tau$, for any $\tau$ in $\KKbar$, $C$
  admits no factor in $\KKbar[T]$; thus, $C(0,T_0)$ is non-zero.

  Let $\ell \in \KKbar[T_1,\dots,T_n,T]$ be the leading coefficient of
  $C$ with respect to $T_0$. Proposition~1 in~\cite{Schost03} proves
  that $C/\ell$, seen in $\KKbar(T_1,\dots,T_n,T)[T_0] \subset
  \LL(T)[T_0]$, is the minimal polynomial of $T_1 X_1 + \cdots +
  T_n X_n$ in $\LL(T)[\bX]/\sqrt{J'}\cdot \LL(T)[\bX]$. The latter ideal
  is also the extension of $\sqrt{\mathfrak{J}'}$ to $\LL(T)[\bX]$, 
  so $C/\ell$ factors as
  $$\frac C\ell = \prod_{1\le i \le c'}(T_0-T_1 \Phi_{i,1} - \cdots - T_n \Phi_{i,n})$$
  in $\LL' [T_0]$ where $\LL'$ is the generalized Power
  series ring in $T$ with coefficients in $\LL$.  This gives the
  equality
  $$C =\ell \prod_{1\le i \le  c'}(T_0-T_1 \Phi_{i,1} - \cdots - T_n
  \Phi_{i,n})$$ over $\SS[T_1,\dots,T_n,T_0]$. 

  Let us extend the valuation $\nu$ on $\SS$ to
  $\SS[T_1,\dots,T_n,T_0]$ in the direct manner, by setting
  $\nu(\sum_\alpha f_\alpha T_0^{\alpha_0} \cdots T_n^{\alpha_n}) =
  \min_\alpha \nu(f_\alpha)$. The fact that $C$ has no factor in
  $\KKbar[T]$ implies that $\nu(C)=0$. Using Gauss' Lemma, we see that
  the valuation of the right-hand side is $\nu(\ell) + \sum_{\kappa <
    i \le c}\mu_i$, with $\mu_i= \nu(\Phi_i)$ for all $i$; note that
  $\mu_i < 0$ for $i > \kappa$. Thus, we can rewrite
  $$C =\left ({T}^{-\nu(\ell)} \ell\right ) 
  \prod_{1 \le i \le \kappa}(T_0-T_1 \Phi_{i,1} - \cdots - T_n  \Phi_{i,n} )
  \prod_{\kappa < i \le c'} ({T}^{-\mu_i}T_0-{T}^{-\mu_i}T_1 \Phi_{i,1} - \cdots - {T}^{-\mu_i}T_n  \Phi_{i,n} ),$$
  where all terms appearing above have non-negative valuation.
  As a result, we can take the coefficient of ${T}^0$ term-wise,
  and obtain
  $$C(0,T_0) = s \prod_{1 \le i \le \kappa}(T_0-T_1 \varphi_{i,1} -
  \cdots - T_n \varphi_{i,n} ),$$ where $s$ is in $\KKbar[T_1,\dots,T_n]$;
  note that $s \ne 0$, since $C(0,T_0)$ is non-zero.
 By construction of $C$, for any
  $\bx=(x_1,\dots,x_n)$ in $V(J'+\langle T \rangle)$, $T_1 x_1 + \cdots + T_n x_n$
  cancels $C(0,T_0)$, so $\bx$ must be one of
  $\varphi_1,\dots,\varphi_{\kappa}$.
\end{proof}

To conclude the proof of Proposition~\ref{prop:degree_fiber}, we now
assume that $\assG_1$, $\assG_2$ and $\assG_3$ hold.

\begin{lemma}
   $\Phi_1,\dots,\Phi_{c'}$ are bounded; equivalently, $\kappa=c'$.
\end{lemma}
\begin{proof}
  We want to prove that $\Phi_1,\dots,\Phi_{c'}$ are bounded. Without
  loss of generality, one can assume that they are all non-zero (a
  zero vector is bounded).

  For $i=1,\dots,c'$, write
  $\Phi_i=1/T^{e_i} (\Psi_{i,1},\dots,\Psi_{i,n})$, for a vector
  $(\Psi_{i,1},\dots,\Psi_{i,n})$ of generalized power  series of valuation zero,
  that is, such that all $\Psi_{i,j}$ are bounded and
  $(\psi_{i,1},\dots,\psi_{i,n})=\lim_0(\Psi_{i,1},\dots,\Psi_{i,n})$
  is non-zero. Hence, $e_i=-\nu(\Phi_i)$, and we have to prove that
  $e_i \le 0$.  By way of contradiction, we assume that $e_i > 0$.

  The series $\Phi_i$ cancels $b_1,\dots,b_m$. For $k=1,\dots,m$, let
  $b_k^H \in \KKbar[T][X_0,\bX]$ be the homogenization of $b_k$ with
  respect to $\bX$. From the equality
  $b_k^H(T^{e_i},\Psi_{i,1},\dots,\Psi_{i,n})= T^{e_i}b_k(\Phi_i)$, we
  deduce that $b_k^H(T^{e_i},\Psi_{i,1},\dots,\Psi_{i,n})=0$ for all
  $k$. We can write $b_k = a_{k} + T \tilde b_k$, for some
  polynomial $\tilde b_k$ in $\KKbar[T,\bX]$, and $\assG_1$ implies
  that $\deg_\bX(\tilde b_k) \le \deg_\bX(a_{k})$. As a result, the
  homogenizations (with respect to $\bX$) of $b_{k},a_{k}$ and
  $\tilde b_k$ satisfy a relation of the form
  $b^H_k = a_{k}^H + X_0^{\delta_k} T \tilde b^H_k$, for some
  $\delta_k \ge 0$. This implies the equality
  $$a_{k}^H(T^{e_i},\Psi_{i,1},\dots,\Psi_{i,n}) + T^{\delta_k
    e_i+1}\tilde b_k^H(T^{e_i},\Psi_{i,1},\dots,\Psi_{i,n})=0.$$ The
  second term has positive valuation, so that
  $a_{k}^H(T^{e_i},\Psi_{i,1},\dots,\Psi_{i,n})$ has positive
  valuation as well. Taking the coefficient of $T^0$, this means that
  $a_{k}^H(0,\psi_{i,1},\dots,\psi_{i,n})=0$ (since $e_i > 0$),
  which implies that $(\psi_{i,1},\dots,\psi_{i,n})=(0,\dots,0)$, in
  view of $\assG_2$.  This however contradicts the definition of
  $(\psi_{i,1},\dots,\psi_{i,n})$.
\end{proof}

\begin{lemma}\label{lemma:Jprimerad}
  The ideal $\frak{J}'$ is radical; equivalently, $c'=c$.
\end{lemma}
\begin{proof}
  We know that $\frak{J}'$ has dimension zero
  (Lemma~\ref{lemma:dimJprime}), so it is enough to prove that for
  $i=1,\dots,c'$, the localization of $\SS[\bX]/\frak{J}'$ at the
  maximal ideal $\mathfrak{m}_{\Phi_i}$ is a field, or equi\-valently
  that the localization of $\SS[\bX]/\frak{J}$ at
  $\mathfrak{m}_{\Phi_i}$ is a field.  Recall that $\SS$ is
  algebraically closed, hence a perfect field. By the Jacobian
  criterion~\cite[Theorem~16.19.b]{Eisenbud95}, our claim holds if
  and only if the Jacobian matrix of $\bB$ with respect to $\bX$ has
  full rank $n$ at $\Phi_i$. We know that $\varphi_i=\lim_0(\Phi_i)$
  is a root of $\bB_{T=0}$ (Lemma~\ref{lemma:Z1}), and the Jacobian
  criterion conversely implies that since the ideal $\langle \bB_{T=0}
  \rangle$ is radical (by assumption $\assG_3$) and zero-dimensional
  (by assumption $\assG_2$), the Jacobian matrix of
  $\bB_{T=0}(\bX)=\bB(0,\bX)$ has full rank $n$ at $\varphi_i$. Since
  this matrix is the limit at zero of the Jacobian matrix of $\bB$
  with respect to $\bX$, taken at $\Phi_i$, the latter must have full
  rank $n$, and our claim that $\frak{J}'$ is radical is proved.
\end{proof}

To finish the proof of Proposition~\ref{prop:degree_fiber}, we have to
establish that $V(\bA)=V(\bB_{T=0})$ consists of exactly $c$
solutions, all with multiplicity one. Assumption $\assG_2$ implies
that $V(\bA)$ is finite, and $\assG_3$ shows that the multiplicities
of all solutions are all equal to one. We need to establish that the number
of solutions is indeed $c$.

Lemma~\ref{lemma:vPi} shows that $\bx$ is in $V(\bA)$ if and only if
$(0,\bx)$ is in $V(J' + \langle T\rangle)$. Next, remark that the two
previous lemmas taken together imply that $c=\kappa$; thus, in view of
Lemma~\ref{lemma:Z1}, to conclude, it is enough to prove that for
$i,i'$ in $\{1,\dots,c\}$, with $i \ne i'$, we have $\varphi_i \ne
\varphi_{i'}$.

Suppose to the contrary that $\varphi_i = \varphi_{i'}$. We know that
the Jacobian matrix of $\bA$ has full rank $n$ at $\varphi_i$; up to
reindexing, we assume that rows $1,\dots,n$ correspond to a maximal
non-zero minor. Let $\bB'=(b_1,\dots,b_n)$.

Let $z=\nu(\Phi_i-\Phi_{i'})$; since $\varphi_i = \varphi_{i'}$, we
have $z > 0$; it is finite else we would have $\Phi_i=\Phi_{i'}$ which
contradicts $i\neq i'$. We can thus write $\Phi_i=f + T^z \delta_i$
and $\Phi_{i'}=f + T^z \delta_{i'}$, for some vectors of bounded
series $f, \delta_i, \delta_{i'}$ such that all terms in $f$ have
valuation less than~$z$; in addition,
$\lim_0(\delta_i) \ne \lim_0(\delta_{i'})$. Write the Taylor expansion
of $\bB'$ at $f$ as
$$\bB'(\Phi_i) = \bB'(f) + \jac(\bB',\bX)_{\bX=f} T^z \delta_i + T^{2z} r_i =0$$
and
$$\bB'(\Phi_{i'}) = \bB'(f) + \jac(\bB',\bX)_{\bX=f} T^z \delta_{i'} + T^{2z}
r_{i'} =0,$$ for some vectors of bounded series $r_i,r_{i'}$.  By
subtraction and division by $T^z$, we obtain
$\jac(\bB',\bX)_{\bX=f} (\delta_i-\delta_{i'}) = T^z r$, for some vector of
bounded series $r$.  Since $\jac(\bB',\bX)_{\bX=f}$ is invertible, this
further gives $\delta_i-\delta_{i'} = T^z r'$, where again $r'$ is a
vector of bounded series.  However, by construction the left-hand side
has valuation zero, while the right-hand side has positive valuation
(since $z > 0$). Hence, we derived a contradiction to our assumption
that $\varphi_i = \varphi_{i'}$. The proof of
Proposition~\ref{prop:degree_fiber} is complete. (Although we do not
need it now, the linearization used above also implies that all
$\Phi_i$ are actually power series.)

%%%%%%%%%%%%%%%%%%%%%%%%%%%%%%%%%%%%%%%%%%%%%%%%%%%%%%%%%%%%
%%%%%%%%%%%%%%%%%%%%%%%%%%%%%%%%%%%%%%%%%%%%%%%%%%%%%%%%%%%%
%%%%%%%%%%%%%%%%%%%%%%%%%%%%%%%%%%%%%%%%%%%%%%%%%%%%%%%%%%%%

\section{A determinantal homotopy algorithm}\label{sec:homotalgo}

Notation being as in the previous section, we now describe our
algorithmic framework for computing either the isolated solutions, or
the simple solutions, of the polynomial system $\bC=(c_1,\dots,c_m)$,
assuming that conditions $\assG_1,\assG_2,\assG_3$ of
Proposition~\ref{prop:degree_fiber} are satisfied. The input of the algorithm
are the following items:
\begin{itemize}
\item a straight-line program $\Gamma$ that computes $\bB$; we denote
  its length by $\sigma$.
\item a zero-dimensional parametrization $\scrR_0
  =((w_{0},v_{0,1},\dots,v_{0,n}),\lambda)$ with coefficients in $\KK$
  of $V(\bA)=V(\bB_{T=0})$; the linear form $\lambda$ needs to satisfy some
  genericity requirements, that are described below. Note that $w_0$
  has degree $c$, where $c$ is defined in
  Proposition~\ref{prop:degree_fiber}.
\item an upper bound $e$ on the degree of the homotopy curve $V(J')
  \subset \KKbar{}^{n+1}$ (see Remark~\ref{rem:Jprime}).
\end{itemize}
%% In order to control the cost of the algorithm, we introduce the
%% following assumptions.
%% \begin{description}[leftmargin=*]
%% \item[${\assD}_1$.] We suppose that we know a description of
%%   $V(\bB_{0})$ by means of . 
%% \item[${\assD}_2$.]  Define the ideal $J=\langle \bB \rangle \subset
%%   \KKbar[T,\bX]$.  We know (we prove that $e \ge c$ in Lemma~\ref{lemma:e-geq-c}).
%% \item[${\assD}_3$.] We can compute $\bB$ using a straight-line program
%%   of length $\sigma$.
Then, the main result in this section is the following.
\begin{proposition}\label{prop:compute_isolated}
  Under the assumptions above, there exists a randomized algorithm
  ${\sf Homotopy}$ which computes a zero-dimensional parametrization
  of the isolated points of $V(\bC)$ using
  $$\softO(c^5 m n^2  + c(e+c^5) n(\sigma + n^3)) \subset (e\,\sigma\,m)^{O(1)}$$
  operations in~$\KK$. 
\end{proposition}
The variant below focuses on the computation of simple points. We
reuse the notations introduced above.
\begin{proposition}\label{prop:compute_regular}
  Under the assumptions of Proposition~\ref{prop:compute_isolated},
  there exists a randomized algorithm ${\sf Homotopy\_simple}$ which
  computes a zero-dimensional parametrization of the simple points
  of~$V(\bC)$ using
  $$\softO( c^2\, m \,n^2 + \,c\,e\,n (\sigma+n^2) )\subset (e\,\sigma\,m)^{O(1)}$$
  operations in~$\KK$.
\end{proposition}
In this section, our goal is to prove these propositions. Once this is
done, in order to obtain complete algorithms, we will still have to
specify how to define $\bB$ and how to solve the start system
$\bA=\bB_{T=0}$; this is the object of the next sections.

\paragraph*{Decomposing $\scrR_0$.}
Let $\scrR_0 =((w_{0},v_{0,1},\dots,v_{0,n}),\lambda)$ be the input
zero-dimensional parametrization of $V(\bB_{T=0})$, with $w_0$ and all
$v_{0,j}$ in $\KK[Y]$.  We start by decomposing $\scrR_0$ into
finitely many zero-dimensional parametrizations
$\scrR_{0,j}=((w_{0,j},v_{0,j,1},\dots,v_{0,j,n}),\lambda)_{1\le j\le
  t}$, all with coefficients in $\KK$, such that for $j$ in
$\{1,\dots,t\}$, we know indices $\bi_j=(i_{j,1},\dots,i_{j,n})$ such that
the Jacobian matrix of $(b_{0,i})_{i \in \bi_j}$ has full rank $n$ at
$\bx$, for all $\bx$ in $Z(\scrR_{0,j})$.

If $w_0$ were irreducible, we would simply evaluate the Jacobian
matrix of $\bB_{T=0}$ at the point $(v_{0,1}/w_0',\dots,v_{0,n}/w_0')$,
which has coordinates in the field $\LL=\KK[Y]/\langle w_0 \rangle$,
and find a non-zero minor of size $n$ in this matrix. It takes
$O(n \sigma)$ operations in $\LL$ to compute this Jacobian matrix, and
$O(mn^2)$ operations in $\LL$ to find an invertible minor, e.g. using
Gaussian elimination. The total time, under the assumption that $w_0$
is irreducible, is thus $O(mn^2 + n\sigma)$ operations in $\LL$, that is,
$\softO(c (mn^2 + n\sigma) )$ operations in $\KK$.

When $w_0$ is not irreducible, $\LL=\KK[Y]/\langle w_0 \rangle$ is a
product of fields. We can still apply the same process as in the
irreducible case; if the algorithm goes through, we have obtained our
answer. In general, one workaround would be to factor $w_0$, but we do
not want our runtime to depend on the cost of factoring polynomials
(else our analysis would depend on the bit size of the data when
$\KK=\mathbb{Q}$). Hence, we will use {\em dynamic evaluation
  techniques}, as in~\cite{D5}. Indeed, the only issue that may arise
is that we attempt to invert a zero-divisor in $\LL$. If this is the case, it
means we have found a non-trivial factor $r_0$ of $w_0$: we can then
replace $\scrR_0$ by two new zero-dimensional parametrizations,
$\scrR'_0=((r_0,(v_{0,1}/s_0) \bmod r_0,\dots,(v_{0,n}/s_0)\bmod
r_0),\lambda)$
and
$\scrR''_0=((s_0,(v_{0,1}/r_0) \bmod s_0,\dots,(v_{0,n}/ r_0)\bmod
s_0),\lambda)$,
with $s_0=w_0/r_0$, that define a partition of $Z(\scrR_0)$ into the
subsets $Z(\scrR'_0)$ and $Z(\scrR''_0)$ where $r_0$ vanishes, resp.\
is non-zero.

We can then start over again, from $\scrR'_0$ and $\scrR''_0$
independently. Overall, in the worst case, this splitting process
induces an extra factor $O(c)$ in the runtime compared to the case
where $w_0$ is irreducible, for a total of $\softO( c^2 (mn^2 + n\sigma) )$
operations in $\KK$.

\paragraph*{Lifting power series and rational reconstruction.}
For $j=1,\dots,t$, we can then apply Newton iteration to the system
$(b_i)_{i \in \bi_j}$ to lift
$\scrR_{0,j}=((w_{0,j},v_{0,j,1},\dots,v_{0,j,n}),\lambda)$ into a
zero-dimensional parametrization
$\scrR_{j}=((w_{j},v_{j,1},\dots,v_{j,n}),\lambda)$ with coefficients
in $\KK[[T]]/\langle T^{2e}\rangle$, where $e$ is the degree bound given
as input to the algorithm.

As explained in~\cite[Section~2.2]{SaSc16}, using the algorithm
of~\cite{GiLeSa01}, this can be done using $\softO(c\,e (\sigma+n^2)n)$
operations in $\KK$.  Using the Chinese Remainder Theorem, we can
combine all $\scrR_{j}$ into a single zero-dimensional parametrization
$\scrR$ with coefficients in $\KK[[T]]/\langle T^{2e}\rangle$, since
for $j\ne j'$, $w_{0,j}$ and $w_{0,j'}$ generate the unit ideal in
$\KK[[T]]/\langle T^{2e}\rangle$; this takes time 
$\softO(c\,e\,n)$.

Using the notation of the previous section, the zeros of $\scrR$ in
$\KKbar[[T]]/\langle T^{2e}\rangle$ are the truncations of the power
series roots $\Phi_1,\dots,\Phi_c$ of $\mathfrak{J}'$. Since $V(J')$ has
degree at most $e$, knowing $\scrR$ at precision $2e$ allows us to
reconstruct a zero-dimensional parametrization $\scrS$ with
coefficients in $\KK(T)$ such that $Z(\scrS)=V(\mathfrak{J}')$,
with all coefficients having numerator and denominator of degree at
most $e$~\cite[Theorem~1]{Schost03}.  This is done by applying
rational function reconstruction to all coefficients of $\scrR$, as
in~\cite{Schost03}, and takes time $\softO(c\,e\,n)$.

All in all, the total cost of this step is $\softO(c\,e\,n (\sigma+n^2))$.

\paragraph*{A finite set containing the isolated points of $V(\bC)$.}
As we did in the previous section for $T=0$, we let
$\Phi'_1,\dots,\Phi'_c$ be the roots of $\mathfrak{J}'$ in the field
of generalized power series in $T'$ with coefficients in $\KKbar$ at
$T=1$, with $T'=T-1$. Without loss of generality, we assume that
$\Phi'_1,\dots,\Phi'_{\kappa'}$ are bounded, and
$\Phi'_{\kappa'+1},\dots,\Phi'_c$ are not, for some $\kappa'$ in
$\{0,\dots,c\}$, and we define $\varphi'_1,\dots,\varphi'_{\kappa'}$
by $\varphi'_i=\lim_0(\Phi'_i)\in\KKbar{}^n$ for $i=1,\dots,\kappa'$.
By Lemma~\ref{lemma:Z1},
$V(J' + \langle T-1\rangle) = \{ \varphi'_i \mid i=1,\dots,\kappa'\}$.

We can now specify our requirements on the linear form $\lambda$.
Following~\cite{RRS} and~\cite{SaSc16}, we ask that $\lambda$ be a
{\em well-separating element}, that is:
\begin{enumerate}
\item $\lambda$ is separating for $V(\mathfrak{J}')=\{\Phi'_1,\dots,\Phi'_c\}$ (all values $\lambda(\Phi'_1),\dots,\lambda(\Phi'_c)$ are pairwise distinct)
\item $\lambda$ is separating for $V(J' + \langle T-1\rangle) = \{ \varphi'_1,\dots,\varphi'_{\kappa'}\}$
\item $\nu(\lambda(\Phi_i)) = \mu_i$ for all $i=1,\dots,c$, where $\nu$ denotes the $T'$-adic valuation.
\end{enumerate}
Applying Lemma~14 in~\cite[Section 3]{SaSc16}, we see that these
conditions are satisfied for a generic choice of $\lambda$. When this
is the case, Lemma~4.4 in~\cite{RRS} shows how to recover a
zero-dimensional parametrization
$\scrR_1=((w_1,v_{1,1},\dots,v_{1,n}),\lambda)$ with coefficients in
$\KK$ for the limit set $V(J' + \langle T-1\rangle) =\{\varphi'_i \mid
i=1,\dots,{\kappa'}\}$ starting from the previously computed rational
parametrization $\scrS$, in time $\softO(c\,e\,n)$.

When the chosen linear form $\lambda$ is not generic enough, the
algorithm may fail, or output a parametrization of a subset of the
zero-dimensional set we aim to compute. We refer to \cite[Remark
  14]{SaSc16} for a discussion on probabilistic aspects.

\paragraph*{Cleaning.}
Finally, summing all the previous costs, we perform
$$
\softO(c^2 (mn^2+n\sigma) + c\, e\,n (\sigma+n^2) )
$$
operations in $\KK$ for the first three steps (decomposition of
$\scrR_0$, lifting and rational reconstruction and getting a finite
set containing the isolated points of $V(\bC)$).

Let us first show how to prove
Proposition~\ref{prop:compute_isolated}.  Lemma~\ref{lemma:vPi}
implies that for any isolated solution $\bx$ of $\bC$, $(1,\bx)$ is in
$V(J' + \langle T-1\rangle)$, so we discard from $V(J' + \langle
T-1\rangle)$ those points that do not correspond to isolated points of
$V(\bC)$. We use the algorithm of Section~\ref{sec:isolated}; by
Proposition~\ref{prop:degree_fiber}, we can take $c$ as an upper bound
on the multiplicity of isolated solutions of~$\bC$.

\textcolor{red}{We apply the same dynamic evaluation techniques, as in the first
  paragraph above, to use the algorithm of Section~\ref{sec:isolated}, using
  polynomial gcd and polynomial inversion modulo $w_1$, as if $Z(\scrR_1)$ were
  an irreducible variety. When this is not the case, splittings are performed ;
  this number of splitting is bounded by $c$ (as is the degree of $w_1$).
  Computations are then performed over quotient rings $\KK[Y] / \langle
  \tilde{w}_1 \rangle$ where $\tilde{w}_1$ is a factor of $w_1$. They cost
  $\softO(c)$ ; hence the total overhead is $\softO(c^2)$. } The runtime deduced
from Proposition~\ref{prop:testisolated} is then
$$\softO(c^6 n^4  +  c^5 m n^2  +c^6 n \sigma )$$ operations in~$\KK$. Adding all
costs seen so far, we have proved
Proposition~\ref{prop:compute_isolated}. The resulting algorithm
is called $\mathsf{Homotopy}$.




\begin{algorithm}
\caption{$\mathsf{Homotopy}(\Gamma,\scrR_0,e)$}
{\bf Input}: a straight-line program $\Gamma$ of length $\sigma$ that computes $\bB \in \KK[T,\bX]^m$\\
\textcolor{white}{{\bf Input}:} a zero-dimensional parametrization $\scrR_0$ of the system $\bA=\bB_{T=0}$\\
\textcolor{white}{{\bf Input}:} an upper bound $e$ on the degree of the homotopy curve\\
{\bf Output}: a zero-dimensional parametrization of the isolated points of $V(\bC)$, with $\bC=\bB_{T=1}$
\begin{enumerate}
  \setlength\itemsep{0em}
\item decompose $\scrR_0$ into $(\scrR_{0,j})_{1 \leq j \leq t}$\\
$\text{\sf{cost:~}} \softO(c^2 (mn^2 + n\sigma) )$
\item lift $(\scrR_{0,j})_{1 \leq j \leq t}$ to $(\scrR_j)_{1\leq j \leq t}$ with 
  coefficients in $\KK[[T]]/\langle T^{2e}\rangle$\\
$\text{\sf{cost:~}} \softO(c\,e (\sigma+n^2)n)$
\item combine $(\scrR_j)_{1 \leq j \leq t}$ into  $\scrR$ with coefficients in $\KK[[T]]/\langle T^{2e}\rangle$\\
$\text{\sf{cost:~}} \softO(c\,e\,n)$
\item compute a zero-dimensional parametrization $\scrS$ with coefficients in $\KK(T)$ from $\scrR$\\
$\text{\sf{cost:~}} \softO(c\,e\,n)$
\item deduce a zero-dimensional parametrization $\scrR_1$ with coefficients in $\KK$ from $\scrS$\\
$\text{\sf{cost:~}} \softO(c\,e\,n)$
\item\label{step:homot:final} remove from $Z(\scrR_1)$ points that are not isolated in $V(\bC)$ \\
  $\text{\sf{cost:~}} \softO(c^6 n^4  + c^5 m n^2   + c^6 n \sigma )$
\end{enumerate}
\label{DetSys}
\end{algorithm}
 



The only difference to prove Proposition~\ref{prop:compute_regular} is
that we now need to discard from $V(J' + \langle T-1\rangle)$ those
points at which the Jacobian matrix associated to $\bC$ is not full
rank; this is easier than discarding those points which are not
isolated. We construct a straight-line program evaluating that
Jacobian matrix; it has length
$\sigma'\in O(n \, \sigma)$. Next, one evaluates this matrix modulo
$w_1$, as done previously when we were decomposing $\scrR_0$, and use
Gaussian elimination modulo $w_1$ to identify divisors of $w_1$ that
need to be removed. The overall cost is similar to that of decomposing
$\scrR_0$, that is, $\softO(c^2(mn^2+n\sigma))$ operations in $\KK$.
The final cleaning step is done using Algorithm $\mathsf{Clean}$ of
\cite{GiLeSa01}, whose cost is dominated by that of the previous computations.

All in all, the total cost is
$$
\softO(c^2(mn^2+n\sigma) + c\, e\,n (\sigma+n^2)  )
$$
operations in $\KK$. Taking into account the inequality
$e\geq c$ (Lemma~\ref{lemma:e-geq-c} below) this simplifies as 
$$
\softO(c^2 mn^2  + c\, e\,n (\sigma+n^2) ),
$$
which ends the proof of Proposition~\ref{prop:compute_regular}. In the
sequel, the resulting algorithm is called
$\mathsf{Homotopy\_simple}$. It differs from Algorithm
$\mathsf{Homotopy}$ at Step \ref{step:homot:final}, where the cleaning step we
just described replaces the one of $\mathsf{Homotopy}$.
We end this section with the proof of inequality $e\geq c$ used above.
\begin{lemma}\label{lemma:e-geq-c}
Under the above notations and assumptions, the inequality $e\geq c$ holds.   
\end{lemma}
\begin{proof}
  By definition, $e$ is greater than or equal to the degree of
  $V(J')$, which is an algebraic curve.  The degree of this curve is
  greater than or equal to the cardinality of any fiber
  $V(J'_{T=\tau})$; in particular, we have
  \[
  \sharp V(J'_{T=0}) \leq \deg(V(J')) \leq e.
  \]
  Besides, Proposition~\ref{prop:degree_fiber} establishes that the
  number of isolated points of $V(\bB_{T=0})$ equals $c$. By Lemma~\ref{lemma:vPi}, all
  these points lie in $V(J'_{T=0})$, which allows us to deduce $c\leq e$.
\end{proof}

%%%%%%%%%%%%%%%%%%%%%%%%%%%%%%%%%%%%%%%%%%%%%%%%%%%%%%%%%%%%
%%%%%%%%%%%%%%%%%%%%%%%%%%%%%%%%%%%%%%%%%%%%%%%%%%%%%%%%%%%%
%%%%%%%%%%%%%%%%%%%%%%%%%%%%%%%%%%%%%%%%%%%%%%%%%%%%%%%%%%%%

\section{The column-degree homotopy}\label{sec:columndegree}

We can now prove the first half of our main results, by giving a
complete homotopy algorithm that takes into account the column-degree
structure of our matrices. As input, we take $\mF \in \K[\bX]^{p\times q}$ and $\mG=(g_1,\dots,g_s)$,
and we show how to compute either the isolated points, or
the simple points, of
$$\VpFG{p}{\mF}{\mG} = \{\bx \in \KKbar{}^n \mid
\mathrm{rank}(\mF({\bx})) < p \text{~and~}
g_1(\bx)=\cdots=g_s(\bx)=0\}.$$ 
To match the notation of the previous sections, we also let
$\bC=(c_1,\dots,c_{s},\dots,c_m)$ be defined as follows:
$(c_1,\dots,c_{s})=(g_1,\dots,g_s)$, and $(c_{s+1},\dots,c_{m})$ are
the $p$-minors of $\mF$, so that $m=s+{q \choose p}$ and
$\VpFG{p}{\mF}{\mG}$ is the zero-set of~$\bC$.


In what follows, we use the following notation for the column-degrees
of $\mF$: $\delta_1=\cdeg(\mF,1),\dots,\delta_q=\cdeg(\mF,q)$; in
particular, $\deg(f_{i,j}) \leq \delta_j$ holds for all $i,j$.  We
will also write $\gamma_1=\deg(g_1),\dots,\gamma_s=\deg(g_s)$.  Recall
that for $k\geq 0$, $E_k(\delta_1,\dots,\delta_q)$ denotes the
elementary symmetric polynomial of degree $k$ in $(\delta_1, \ldots,
\delta_q)$. Then, the main results of this section are the following
propositions, which establish the first half of
Theorems~\ref{theo:1},~\ref{theo:2} and~\ref{theo:3}.

\begin{proposition}\label{prop:coldeg}
  The sum
  of the multiplicities of the isolated points of $\VpFG{p}{\mF}{\mG}$
is at most
  $c=\gamma_1\cdots\gamma_sE_{n-s}(\delta_1, \ldots, \delta_q)$.

  Suppose that the matrix $\mF \in \KK[X_1,\dots,X_n]^{p \times q}$
  and the polynomials $G=(g_1,\dots,g_s)$ in $\KK[X_1,\dots,X_n]$ are
  given by a straight-line program of length $\sigma$. 
  Assume that all $\gamma_i$'s and $\delta_j$'s are at least equal to
  $1$, and let
  $e=(\gamma_1+1)\cdots(\gamma_s+1) E_{n-s}(\delta_1+1, \ldots,
  \delta_q+1)$, $\gamma = \max(\gamma_1, \ldots, \gamma_s)$ and
  $\delta = \max(\delta_1, \ldots, \delta_q)$.
  Then, there exists a randomized algorithm that computes these isolated
  points using
  $$
  \softO\left ({q \choose p} c(e+c^5 )(\sigma +\gamma + q
    \delta)\right)
  $$
  operations in $\KK$.
\end{proposition}

The next proposition states a better complexity estimate when one only
computes simple points of $\VpFG{p}{\mF}{\mG}$.
\begin{proposition}\label{prop:coldeg_simple}
  Reusing the notations introduced above, 
  there exists a randomized algorithm that computes the simple points of $\VpFG{p}{\mF}{\mG}$ using
  $$ 
  \softO\left (   {q \choose p}c \,e (\sigma+\gamma+q\delta)) \right )
  $$ 
  operations in $\KK$.
\end{proposition}
In both cases, we apply homotopy algorithms given in the previous
section.  These algorithms do not specify how to choose the matrix
$\bL$ and the polynomials $\bM$ used to define the homotopy
deformation, or how to solve the start system of the homotopy. This is
what we do here, with constructions that respect the column degree
structure of matrix $\mF$. In the case where there are no polynomials
$\mG$, the construction used in this section is already in the
appendix of~\cite{NieRan09}, where it was used to bound the number of
solutions of determinantal systems.

Using the degrees $\gamma_1,\dots,\gamma_s$ and
$\delta_1,\dots,\delta_q$, we first construct a polynomial matrix $\bL
\in \KK[\bX]^{p \times q}$, and polynomials $\bM=(m_1,\dots,m_s)$ in
$\KK[\bX]$, to use as a starting point for the homotopy algorithm. For
any $1 \leq j \leq q$ and $1 \leq k \leq \delta_j$, define
$$\lambda_{j,k} = \lambda_{j,k,0} + \sum_{\ell =
  1}^{n}\lambda_{j,k,\ell}X_\ell,$$ where all $\lambda_{j,k,\ell}$ are
random elements in $\KK$. Then, for $j=1,\dots,q$, we define
$$\lambda_j = \prod_{k=1}^{\delta_j}\lambda_{j,k},$$
and we let  $\bL$ be the matrix
\begin{align}\label{eqdef:col}
\bL = 
\left( \begin{matrix}
\lambda_1 & 2\lambda_2 & \cdots & q\lambda_{q}\\
\lambda_1 & 2^2\lambda_2 & \cdots & q^2\lambda_q\\
\vdots & \vdots &  & \vdots \\
\lambda_1 & 2^p\lambda_2 & \cdots & q^p\lambda_q
\end{matrix} \right) \in \KK[\bX]^{p\times q}.
\end{align}
For $i=1,\dots,s$ and $k=1,\dots,\gamma_i$, let us further define
$$\mu_{i,k} =  \mu_{i,k,0} + \sum_{\ell = 1}^{n}\mu_{i,k,\ell}X_\ell,$$ where
all $\mu_{i,k,\ell}$ are random elements in $\KK$; then, we let
$\bM=(m_1,\dots,m_s)$, with
$$m_i=\prod_{k=1}^{\gamma_i} \mu_{i,k}, \quad i=1,\dots,s.$$ Our start
system $\bA=(a_1,\dots,a_s,\dots,a_m)$ is defined by taking
$(a_1,\dots,a_s) = (m_1,\dots,m_s)$, and letting $(a_{s+1},\dots,a_m)$
be the $p$-minors of $\bL$.

As in the previous sections, we define next the polynomials
$\mV=(v_1,\dots,v_s)$ by $\bV = (1-T) \cdot \bM + T \cdot \mG$, the
matrix $\mU=(1-T)\cdot \bL + T \cdot \mF \in \KK[T,\bX]^{p\times q}$,
and we let $\bB$ be the polynomials in $\KK[T,\bX]$ given by
$\bB=(b_1,\dots,b_s,\dots,b_m)$, where $b_i=v_i$ for $i=1,\dots,s$ and
$(b_{s+1},\dots,b_{m})$ are the $p$-minors of $\mU$.  Then, $\bB_{T=1}=\bC$ and $\bB_{T=0}=\bA$.

Finally, we define $J$ as the ideal generated by $\bB$ in
$\KKbar[T,\bX]$.
\begin{example}\label{ex:coldeg}
  We illustrate this construction with Example~\ref{ex:1} from the
  introduction. In this case, there are no polynomials $\mG$, so
  $s=0$. The column degrees $\delta_1,\delta_2,\delta_3$ of $\mF$
  are all equal to $2$, and we take
  \begin{align*}
    \lambda_1 &= (10X_1+X_2-1)(X_1+3X_2-5)\\
    \lambda_2 &= (2X_1-X_2-2)(3X_1+3X_2-1)\\
    \lambda_3 &= (-X_1+X_2-9)(-3X_1+X_2+5).
  \end{align*}
Then, $\bL$ is given by
\[\bL=\begin{bmatrix} 
\lambda_1 & 2 \lambda_2 & 3\lambda_3 \\
\lambda_1 & 4 \lambda_2 & 9\lambda_3 
\end{bmatrix}.\]
The start system $\bA$ is the set of 2-minors of this matrix;
up to non-zero constants, $\bA=(\lambda_1\lambda_2, \lambda_1\lambda_3, \lambda_2\lambda_3)$.
\end{example}




Having in mind to apply
Propositions~\ref{prop:degree_fiber},~\ref{prop:compute_isolated}
and~\ref{prop:compute_regular}, we now verify that all required
assumptions are satisfied. First, we check that properties
$\assG_1,\assG_2,\assG_3$ needed in
Proposition~\ref{prop:degree_fiber} hold.

\paragraph*{Property $\assG_1$.} We have to prove that for $i=1,\dots,m$,
$\deg_\bX(b_i)=\deg_\bX(a_i)$. 
For $i=1,\dots,s$, we have $a_i=m_i$, so
this amounts to proving that $\deg_\bX((1-T) m_i +
T g_i)=\deg_\bX(m_i)$. The latter is by construction equal to
$\gamma_i$. The former is at most $\gamma_i$ (since $b_i$ is the sum
of two polynomials of degree $\gamma_i$ in $\bX$), but since
evaluating $T$ at $0$ in $b_i$ gives us $g_i$, its degree in $\bX$
must be exactly $\gamma_i$.

To each index $i=s+1,\dots,m$ corresponds a sequence
$\bj_i=(j_{i,1},\dots,j_{i,p})$ such that $b_i$ and $a_i$ are the
minors built with columns indexed by $\bj_i$ in respectively
$\mU=(1-T)\cdot \bL + T \cdot \mF$ and $\bL$. In view of the shape of
$\bL$, the polynomial $a_i$ is equal to $c_i\lambda_{j_{i,1}}\cdots
\lambda_{j_{i,p}}$, with
$$c_i = \left | 
\begin{matrix}
j_{i,1} & j_{i,2} & \cdots & j_{i,p}\\
j_{i,1}^2 & j_{i,2}^2 & \cdots & j_{i,p}^2\\
\vdots & \vdots &  & \vdots \\
j_{i,1}^p & j_{i,2}^p & \cdots & j_{i,p}^p
\end{matrix}
\right |.$$
Because $\KK$ has characteristic zero,
 $c_i$ is a non-zero constant, so that $a_i$ has degree $\delta_{j_{i,1}} +
\cdots + \delta_{j_{i,p}}$.  Since the columns
$(j_{i,1},\dots,j_{i,p})$ of $\mU$ have respective degrees at most
$(\delta_{j_{i,1}},\dots,\delta_{j_{i,p}})$, $b_i$ has degree at most
$\delta_{j_{i,1}} + \cdots + \delta_{j_{i,p}}$. However, evaluating
$T$ at $0$ in $b_i$ gives us back the polynomial $a_i$, so $b_i$ must
have degree exactly $\delta_{j_{i,1}} + \cdots + \delta_{j_{i,p}}$.

\paragraph*{Property $\assG_2$.} We have to prove that the homogenization
of the system $\bA$ has no root at infinity. Thus, let $X_0$ be a new
variable, and let $\bA^H=(a_1^H,\dots,a_m^H)$ be the homogenization
of $\bA$. For $i=1,\dots,s$, we have
$$a_i^H=\prod_{k=1}^{\gamma_i} \mu^H_{i,k} \quad\text{with}\quad \mu^H_{i,k}=(\mu_{i,k,0}X_0 + \sum_{\ell = 1}^{n}\mu_{i,k,\ell}X_\ell),$$
whereas for $i=s+1,\dots,m$, 
$$a_i^H=c_i \lambda^H_{j_{i,1}}\ldots \lambda^H_{j_{i,p}}, \quad \text{~for~} \bj_i=(j_{i,1},\dots,j_{i,p}) \text{~as above},$$
where for $j=1,\dots,q$ we set 
$\lambda^H_j = \prod_{k=1}^{\delta_j}\lambda^H_{j,k}$,
with
$$\lambda^H_{j,k}=\lambda_{j,k,0}X_0 + \sum_{\ell = 1}^{n}\lambda_{j,k,\ell}X_\ell.$$
To prove  $\assG_2$, we start by writing down all projective
solutions of this system (this will be of use below), before adding
the constraint $X_0=0$.

Since all $a_i^H$ are products of linear forms, we find the solutions
of $\bA^H$ by setting some of these linear forms to zero. In order to
cancel $a_1^H,\dots,a_s^H$, we choose indices $\bu=(u_1,\dots,u_s)$,
with $u_1\in\{1,\dots,\gamma_1\}$, \dots,
$u_s\in\{1,\dots,\gamma_s\}$, and we consider the equations 
$$\mu^H_{i,u_i}=0, \quad \text{~that is,~} \quad \mu_{i,u_i,0}X_0 + \sum_{\ell = 1}^{n}\mu_{i,u_i,\ell}X_\ell =0,$$ for $i=1,\dots,s$.
In what follows, we fix such an $\bu$.
Then, for a generic choice of coefficients $\mu_{i,k,\ell}$, these equations
are equivalent to
$$X_{n-s+1}=\Phi_{n-s+1,\bu}(X_0,\dots,X_{n-s}),\dots,X_{n}=\Phi_{n,\bu}(X_0,\dots,X_{n-s}),$$
for some homogeneous linear forms
$\Phi_{n-s+1,\bu},\dots,\Phi_{n,\bu}$.  After applying this
substitution, for all $j=1,\dots,q$, $\lambda^H_j$ can be rewritten as
$$\lambda^H_{j,\bu}=\prod_{k=1}^{\delta_j}\lambda^H_{j,k,\bu},$$
where 
$$\lambda^H_{j,k,\bu}=\lambda_{j,k,0}X_0 + \sum_{\ell =
  1}^{n-s}\lambda_{j,k,\ell}X_\ell + \sum_{\ell =
  n-s+1}^{n}\lambda_{j,k,\ell}
\Phi_{\ell,\bu}(X_0,\dots,X_{n-s}).$$ Then,
$\bx=(x_0,\dots,x_n)$ cancels $a^H_{s+1},\dots,a^H_m$ if and only if
$\bx'=(x_0,\dots,x_{n-s})$ cancels the product
$\lambda^H_{j_1,\bu}\cdots \lambda^H_{j_p,\bu},$ for any choice of $p$ columns
$\bj=(j_1,\dots,j_p)$.

\begin{lemma}
  For $\bx'$ in $\P^{n-s}(\KKbar)$, the products
  $\lambda^H_{j_1,\bu}(\bx')\cdots \lambda^H_{j_p,\bu}(\bx')$
  vanish for all choices of columns $\bj=(j_1,\dots,j_p)$ if and only
  if there exists $\{j_1,\dots,j_{n-s}\} \subset \{1,\dots,q\}$ such 
  that $\lambda^H_{j_1,\bu}(\bx')=\cdots=\lambda^H_{j_{n-s},\bu}(\bx')=0$.
\end{lemma}
\begin{proof}
  Take an arbitrary representative $\bx^*$ of $\bx'$ in
  $\KKbar{}^{n+1}$, and consider the polynomial 
  $(1+\lambda^H_{1,\bu}(\bx^*)Y_1) \cdots (1+\lambda^H_{q,\bu}(\bx^*)Y_q),$
  for new variables $Y_1,\dots,Y_q$. The products
  $\lambda^H_{j_1,\bu}(\bx^*)\cdots \lambda^H_{j_p,\bu}(\bx^*)$ are all zero
  if and only if this polynomial has degree less than $p$, that is, if
  and only if $q-p+1=n-s$ terms among
  $\lambda^H_{1,\bu}(\bx^*),\dots,\lambda^H_{q,\bu}(\bx^*)$ vanish.
\end{proof}


For a given $\bu$ and generic coefficients $\lambda_{j,k,\ell}$ and $\mu_{i,k,\ell}$,
 the linear forms $\lambda^H_{j,k,\bu}$ are all pairwise distinct, so
the condition of the lemma holds if and only if there exist
$\bj=\{j_1,\dots,j_{n-s}\} \subset \{1,\dots,q\}$ and
$\bv=(v_1,\dots,v_{n-s})$, with $v_k$ in $\{1,\dots,\delta_k\}$ for all
$k$, such that $\lambda^H_{j_k,v_k,\bu}(\bx')=0$ 
for $k=1,\dots,n-s$.

This implies that for a fixed $\bu$, the possible values of $\bx'=(x_0,\dots,x_{n-s}) \in \P^{n-s}(\KKbar)$ are
determined as solutions of a linear system of size $n-s$. For a
generic choice of the coefficients $\lambda_{j,k,\ell}$ and
$\mu_{i,k,\ell}$, none of these points satisfies $X_0=0$, so that
$\assG_2$ holds.

\paragraph*{Property $\assG_3$.} From the previous paragraph we know that
the projective variety defined by $\bA^\mH$ has no point at infinity,
so it is finite; as a result, the affine algebraic set defined by
$\bA$ is finite as well. In addition, all the affine solutions to
$\bA$ are obtained by setting $X_0=1$ in the projective solutions of
$\bA^\mH$. In other words, they are obtained by choosing indices
$\bu=(u_1,\dots,u_s)$ with $u_k$ in $\{1,\dots,\gamma_k\}$ for all $k$,
column indices $\bj=(j_1,\dots,j_{n-s})$, and
$\bv=(v_1,\dots,v_{n-s})$, with $v_k$ in $\{1,\dots,\delta_k\}$
for all $k$, solving the affine linear system
$$\lambda_{j_1,v_1,\bu}(X_1,\dots,X_{n-s})=\cdots=\lambda_{j_{n-s},v_{n-s},\bu}(X_1,\dots,X_{n-s})=0$$ 
and using the expressions
$$X_{n-s+1}=\phi_{n-s+1,\bu}(X_1,\dots,X_{n-s}),\dots,X_{n}=\phi_{n,\bu}(X_1,\dots,X_{n-s}),$$
where $\phi_{k,\bu}(X_1,\dots,X_{n-s})=\Phi_{n-s+1,\bu}(1,X_1,\dots,X_{n-s})$
for all $k$.
To prove that the ideal generated by $\bA$ is radical, we prove that
at any point as described above, the Jacobian matrix of $\bA$ with
respect to $X_1,\dots,X_n$ has full rank.

Let thus $\bu$, $\bj$ and $\bv$ be as above, let $\bx \in \KKbar{}^n$
be the corresponding point in $V(\bA)$, and consider equations
$(a_1,\dots,a_s)$ first. Each such equation is a product of linear forms
such as $a_i=\prod_{k=1}^{\gamma_i} \mu_{i,k}$, with $\mu_{i,u_i}(\bx)=0$.
Since the coefficients $\mu_{i,k,\ell}$ are chosen generically, for
$i=1,\dots,s$ and $k \ne u_i$, $\mu_{i,k}(\bx)$ is non-zero; as a
result, in the local ring at $\bx$, the polynomials $(a_1,\dots,a_s)$
are equal (up to units) to the linear forms
$(\mu_{1,u_1},\dots,\mu_{s,u_s})$.

Next, we consider the $p$-minors of $\bL$; in what follows, we 
write $\bx'=(x_1,\dots,x_{n-s})$. Our starting point is that due to 
the genericity of the coefficients $\lambda_{j,k,\ell}$, since 
$$\lambda_{j_1,v_1,\bu}=\cdots=\lambda_{j_{n-s},v_{n-s},\bu}=0$$
only admits $\bx'$ as a solution,
none of the other linear forms $\lambda_{j,k,\bu}$ vanishes at $\bx'$.
Equivalently, none of the other linear forms $\lambda_{j,k}$ vanishes at $\bx$.

Recall that $n=q-p+s+1$, so that $n-s = q-(p-1)$. Hence, there are
exactly $p-1$ columns of $\bL$ not indexed by $\bj=(j_1,\dots,j_{n-s})$; call
them $\bj'=(j'_1,\dots,j'_{p-1})$. We can then consider the 
products
$$ \lambda_{j_1} \lambda_{j'_1} \cdots \lambda_{j'_{p-1}},\dots, \lambda_{j_{n-s}}
\lambda_{j'_1} \cdots \lambda_{j'_{p-1}};$$ each of them (up to a non-zero
constant) is a $p$-minor of $\bL$, so they appear as elements in the
sequence $(a_{s+1},\dots,a_m)$, say as
$(a_{e_1},\dots,a_{e_{n-s}})$. By the remark of the previous
paragraph, in the local ring at $\bx$, up to non-zero constants, these
polynomials are respectively equal to the linear forms
$\lambda_{j_1,v_1},\dots,\lambda_{j_{n-s},v_{n-s}}$.  

To summarize, we have found that the linear equations
$(\mu_{1,u_1},\dots,\mu_{s,u_s})$ and
$(\lambda_{j_1,v_1},\dots,\lambda_{j_{n-s},v_{n-s}})$ belong to the
ideal $\langle \bA \rangle_\m$, where $\m$ is the maximal ideal at
$\bx$. As a result, the Jacobian matrix of $\bA$ must be invertible
at $\bx$, and $\assG_3$ holds.

\begin{example}\label{ex:12pts}
  Let us see how the discussion above allows us to find all solutions
  to the system $\bA$ from Example~\ref{ex:coldeg}. Since $s=0$, we do
  not need to involve indices $\bu$ in our discussion: the matrix
  $\bL$ given in that example has rank less than $2$ at $\bx$ if and
  only if one of the conditions $\lambda_1(\bx)=\lambda_2(\bx)=0$,
  $\lambda_1(\bx)=\lambda_3(\bx)=0$, or
  $\lambda_2(\bx)=\lambda_3(\bx)=0$ holds. Since each $\lambda_i$
  is a product of two linear forms, we obtain a total of $3 \cdot 4 = 12$ 
  points in $V(\bA)$, namely
  \begin{align*}
&  (-\frac{11}2, \frac 72), (-\frac{13}3, \frac{14}3), (-2, \frac 73), (-\frac 8{11}, \frac{91}{11}), (\frac 2{27}, \frac 7{27}), \\ 
&  (\frac 14, -\frac 32), (\frac 6{13}, -\frac{47}{13}), (\frac 43, -1), (\frac {11}7, \frac 87), (2, 1), (3, 4), (11, 20).
  \end{align*}
\end{example}


\medskip

At this stage, we have established all assumptions necessary to apply
Proposition~\ref{prop:degree_fiber}. We deduce that the sum of the
multiplicities of the isolated solutions of $\bC=\bB_{T=1}$ is at most
$c$, where $c$ is the number of solutions of $\bA=\bB_{T=0}$.

\begin{lemma}\label{lemma:column:c_estimate}
  Under the above assumptions, $c=\gamma_1\cdots \gamma_s
  E_{n-s}(\delta_1, \ldots, \delta_q)$.
\end{lemma}
\begin{proof}
  To estimate $c$, note first that there are $\gamma_1\cdots
  \gamma_s$ choices of $\bu$. For each choice of $\bu$, there are
  $E_{n-s}(\delta_1, \ldots, \delta_q)$ ways to
  choose $\bj$ and $\bv$, where $E_{n-s}$ denotes the elementary
  symmetric polynomial of degree $n-s$.   
\end{proof}
This proves the first part of Proposition~\ref{prop:coldeg}.  In order
to apply the homotopy algorithms of Propositions~\ref{prop:compute_isolated}
and~\ref{prop:compute_regular}, we now need to ensure that we can
prepare the three inputs they need: a straight-line program for $\bB$,
a zero-dimensional representation of the solutions of $\bA=\bB_{T=0}$ and an upper
bound on the degree of the homotopy curve. For the cost analysis
below, as in Theorem~\ref{theo:2}, we assume that all $\gamma_i$'s and
$\delta_j$'s are at least equal to $1$.

\paragraph*{A straight-line program for $\bB$.} 
As input, we are only given a straight-line program $\Gamma$ of size
$\sigma$ that computes polynomials $\mG=(g_1,\dots,g_s)$ and the entries
of $\mF$. We need to build one that computes the polynomials
$\bB=(b_1,\dots,b_m)$.

First, we estimate the complexity of computing the polynomials
$(b_1,\dots,b_s)$. For $i \le s$, the $i$th polynomial $b_i$ is equal
to $(1-T)a_i + T g_i$, where $a_i$ is a product of $\gamma_i$ linear
forms in $n$ variables. The latter polynomial can be computed in $O(n
\gamma_i)$ operations in $\KK$, hence with a total of $O(n
(\gamma_1+\cdots+\gamma_s))$ operations for $(a_1,\dots,a_s)$, and
$O(\sigma+n (\gamma_1+\cdots+\gamma_s))$ for $(b_1,\dots,b_s)$.

The polynomials $(b_{s+1},\dots,b_m)$ are the $p$-minors of
$\mU=(1-T)\cdot\bL+T\cdot\mF$.  The polynomials $\lambda_1,\dots,\lambda_q$ can
be computed in $O(n (\delta_1+\cdots+\delta_q))$ operations, so that
the entries of $\mU$ can be computed in $O(\sigma +
n(\delta_1+\cdots+\delta_q))$ operations. From that, all $p$-minors of
$\mU$ can be deduced in $O({q \choose p} n^3)$ further steps.  To
summarize, all polynomials in $\bB$ can be computed by a straight-line
program $\Gamma'$ of size $\sigma'=O(\sigma + {q \choose p} n^3 +n(\gamma_1+\cdots+\gamma_s+\delta_1+\cdots+\delta_q))$.

\paragraph*{A zero-dimensional parametrization $\scrR_0$ of $V(\bA)$.} 
We do this by following the description of the solutions of $\bA$
given previously: for any choice of indices $\bu$, $\bj$ and $\bv$ as
above, the corresponding point $\bx \in \KKbar{}^n$ in $V(\bA)$ can be
found by solving the linear system of size $n$ given by
$(\mu_{1,u_1},\dots,\mu_{s,u_s})$ and
$(\lambda_{j_1,v_1},\dots,\lambda_{j_{n-s},v_{n-s}})$, so in time
$O(n^3)$. We repeat this procedure $c$ times, using a total of $O(c
n^3)$ operations in $\KK$.

Knowing all the points in $V(\bA)$, we can construct a
zero-dimensional parametrization $\scrR_0$ such that
$Z(\scrR_0)=V(\bA)$ in time $O\tilde{~}(c n)$ by means of fast
interpolation~\cite[Chapter~10]{GaGe03}.  The total cost here is in
$O(c n^3)$ operations in $\KK$.

\paragraph*{An upper bound $e$ on the degree of the homotopy curve.}  We claim that we
can take $e=(\gamma_1+1)\cdots(\gamma_s+1) E_{n-s}(\delta_1+1, \ldots,
\delta_q+1)$.  Let us write $V(\bB)=V(J') \cup V' \cup V''$, where
$V''$ is the union of the other components of dimension one of
$V(\bB)$ and $V''$ is the union of the components of higher dimension
(recall that $V(\bB)$ has no isolated point), and let $H$ be a
generic hyperplane in coordinates $T,X_1,\dots,X_n$. Then, $(V(J')
\cup V') \cap V(H)$ is a finite set consisting of $\deg(V(J')) +
\deg(V')$ points, whereas $V'' \cap V(H)$ consists only on components
of positive dimension; these two sets are disjoint. Thus, we can take
for $e$ the number of isolated points of $V(\bB)\cap V(H)$.

The hyperplane $H$ is defined by an equation
$h_0 + h_1 X_1 + \cdots + h_{n}X_{n} + h_{n+1} T=0$. This equation
allows us to rewrite $T$ as
$\eta(X_1,\dots,X_n)=-(h_0 + h_1 X_1 + \cdots +
h_{n}X_{n})/h_{n+1}$;
the points in $V(\bB)\cap V(H)$ are thus in one-to-one correspondence
with the solutions of the system
$(\beta_1,\dots,\beta_s,\beta_{s+1},\dots,\beta_m)$, where
$\beta_i=(1-\eta) a_i + \eta g_i$, for $i=1,\dots,s$, and
$\beta_{s+1},\dots,\beta_m$ are the $p$-minors of the matrix
${\nu}=(1-\eta)\, \bL + \eta \, \mF $.  Now, the polynomials
$(\beta_1,\dots,\beta_s)$ have respective degrees at most
$(\gamma_1+1),\dots,(\gamma_s+1)$, and the column degrees of ${\nu}$
are $\delta_1+1,\dots,\delta_q+1$.

We can then apply Proposition~\ref{prop:degree_fiber}, which shows we
can take for $e$ the integer $(\gamma_1+1)\cdots(\gamma_s+1)
E_{n-s}(\delta_1+1, \ldots, \delta_q+1)$.  

\begin{example}\label{ex:12param}
On our running example, starting from the points given in 
Example~\ref{ex:12pts},
we obtain the zero-dimensional parametrization
$\scrR_0=((w_0,v_{0,1},v_{0,2}),\lambda)$
for $V(\bA)$, with $\lambda = X_2$ and
{\footnotesize  \begin{align*}
w_0 &=Y^{12} - \frac{1055660}{27027}Y^{11} + \frac{53137069}{108108}Y^{10} - \frac{1093435073}{486486}Y^9 - 
    \frac{820013219}{972972}Y^8 + \frac{18538617847}{486486}Y^7 \\
  &~~~- \frac{2418753031}{24948}Y^6 - 
    \frac{1649924501}{162162}Y^5 + \frac{1528208159}{5148}Y^4 - \frac{16255281049}{69498}Y^3 - 
    \frac{5525925412}{34749}Y^2\\
  &~~~ + \frac{7236468568}{34749}Y - \frac{36111040}{891}\\[1mm]
v_{0,1}&=
\frac{770785}{108108}Y^{11} - \frac{20800447}{216216}Y^{10} + \frac{50442596}{81081}Y^9 - 
    \frac{28536694169}{5837832}Y^8 + \frac{30893680099}{1459458}Y^7 + \frac{3580073831}{216216}Y^6 -  \\
  &~~~
    \frac{8167305065}{27027}Y^5 + \frac{73892907181}{176904}Y^4 + \frac{60113381407}{138996}Y^3 - 
    \frac{13255132849}{16038}Y^2 + \frac{23446514308}{104247}Y  - \frac{5841976}{1287}\\[1mm]
v_{0,2}&=\frac{1055660}{27027}Y^{11} - \frac{53137069}{54054}Y^{10} + \frac{1093435073}{162162}Y^9 + 
    \frac{820013219}{243243}Y^8 - \frac{92693089235}{486486}Y^7+ \frac{2418753031}{4158}Y^6 \\ &~~~ + 
    \frac{1649924501}{23166}Y^5 - \frac{3056416318}{1287}Y^4 + \frac{16255281049}{7722}Y^3 + 
    \frac{55259254120}{34749}Y^2 -\frac{7236468568}{3159}Y + \frac{144444160}{297}.
  \end{align*}}
The degree bound for the homotopy curve is $e=3\cdot3 + 3 \cdot 3 + 3\cdot 3 = 36$.
\end{example}

\paragraph*{Completing the cost analysis.}
We can then apply Proposition~\ref{prop:compute_isolated},
whose runtime is $\softO(c^5 m n^2  + c(e+c^5) n(\sigma' + n^3))$ operations
in~$\KK$; since $m \le n + {q \choose p}$, this can be simplified as
$$\softO\left (c(e+c^5) n \left(\sigma + {q \choose p} n^3
+n(\gamma_1+\cdots+\gamma_s+\delta_1+\cdots+\delta_q)\right)\right).$$
Since $s \leq n$, $\gamma = \max(\gamma_1, \ldots, \gamma_s)$ and
$\delta = \max(\delta_1, \ldots, \delta_q)$, our bound becomes
$$\softO\left (c(e+c^5) n(\sigma + {q \choose p}n^3+n^2\gamma + nq
  \delta)\right).$$ This can also be rewritten as
$$\softO\left (c(e+c^5 )(\sigma + {q \choose p}n^3+n^2\gamma + nq
  \delta)\right),$$
since one easily checks that $e \ge 2^n$ (because by assumption we
have $\gamma_i\geq 1 $ and $\delta_i\geq 1$), so that
$n \in \softO(e)$. A last factorization shows that the bound can be
simplified to
\[
  \softO\left ({q \choose p}  c(e+c^5 )n^3(\sigma +\gamma + q
    \delta)\right).
\]
Using again  that $n\le\log_2(e)$, we can omit the factor
$n^3$ from the $\softO(\ )$, and we conclude the proof of Proposition~\ref{prop:coldeg}.
The resulting algorithm is called $\mathsf{ColumnDegree}$.

\begin{algorithm}[h]
\caption{$\mathsf{ColumnDegree}(\Gamma)$}
{\bf Input}: a straight-line program $\Gamma$ of length $\sigma$ that computes 
\begin{itemize}  
\setlength\itemsep{0em}
\item $\mF \in \KK[X_1, \ldots, X_n]^{p \times q}$ with $\deg(f_{i,j}) \leq \delta_j$ for all $j$ and $p \le q$
\item polynomials $\mG = (g_1, \ldots, g_s)$ in $\KK[X_1, \ldots, X_n]$, with $n=q-p+s+1$
\end{itemize}
{\bf Output}: a zero-dimensional parametrization of the isolated points of $\VpFG{p}{\mF}{\mG}$
\begin{enumerate}\setlength\itemsep{0em}
\item for any sequence $\bu=(u_1,\dots,u_s)$, with $u_j \in \{1,\dots,\gamma_j\}$ for all $j$
\begin{enumerate}\setlength\itemsep{0em}
\item for any subsequence $\bj=(j_1,\dots,j_{n-s})$ of $(1,\dots,q)$
\begin{enumerate}\setlength\itemsep{0em}
\item for any sequence $\bv=(v_1,\dots,v_{n-s})$, with $v_k$ in $\{1,\dots,\delta_k\}$ for all $k$
\begin{enumerate}\setlength\itemsep{0em}
 \item compute a zero-dimensional parametrization $\scrR_{\bi,\bj,\bv}$ of the solution of the system 
$$\mu_{1,u_1}=\cdots=\mu_{s,u_s}=\lambda_{j_1,v_1}=\cdots=\lambda_{j_{n-s},v_{n-s}}=0$$

\hfill $\text{\sf{cost:~}} O(cn^3)$, with $c=\gamma_1\cdots\gamma_s E_{n-s}(\delta_1,\dots,\delta_q)$
\end{enumerate}
\end{enumerate}
\end{enumerate}
\item combine all $(\scrR_{\bu,\bj,\bv})_{\bu,\bj,\bv}$ into a zero-dimensional parametrization $\scrR_0$

  \hfill $\text{\sf{cost:~}} \softO(cn)$

\item construct a straight-line program $\Gamma'$ that computes all polynomials $\bB$

\hfill length of $\Gamma'$ is $\sigma'=O(\sigma + {q \choose p} n^3 + n
(\alpha_1+\cdots+\alpha_p) + n(\gamma_1 + \cdots + \gamma_s))$

\item return $\mathsf{Homotopy}(\Gamma',\scrR_0,e)$ 

\hfill $\text{\sf{cost:~}} \softO\left (c^5 m n^2 + c(e+c^5)n (\sigma' + n^3)\right)$, 

\hfill with $e=(\gamma_1+1)\cdots(\gamma_s+1) E_{n-s}(\delta_1+1,\dots,\delta_q+1)$
\end{enumerate}
\label{ColHom}
\end{algorithm}


Finally, to prove Proposition~\ref{prop:coldeg_simple}, we design an
algorithm called $\mathsf{ColumnDegree\_simple}$, which differs from
$\mathsf{ColumnDegree}$ only at the last step, where Algorithm
$\mathsf{Homotopy\_simple}$ is called instead of $\mathsf{Homotopy}$.
Hence, one applies Proposition~\ref{prop:compute_regular}, which yields a
runtime $\softO(c^2\,m \,n^2  + c\,e\,n (\sigma'+n^2) )$ operations
in $\KK$. Using again $m \leq n+ \binom{q}{p}\leq n\binom{q}{p}$,
$\sigma'=O(\sigma + {q \choose p} n^3 +n(n\gamma+q\delta))$, we
obtain as a bound
\[
  \softO\left (
    {q \choose p}\, c^2\,n^3 +  c\, e\,n (\sigma+ {q \choose p}n^3 \, +n^2\gamma +nq \delta)
  \right ),
\]
which we simplify as
\[
 \softO\left (   {q \choose p}c\, e\,n^4 (\sigma+\gamma+q\delta) \right ),
\]
taking into account that $c \le e$.
Since $e \ge 2^n$, the term $n^4$ can be absorbed in 
the $\softO(\, )$.
This concludes the proof of Proposition~\ref{prop:coldeg_simple}. 

\begin{example}
  We apply the symbolic homotopy algorithm to the system
  $\bB=(b_1,b_2,b_3)$ of $2\times 2$ minors of matrix $\mU = (1-T) \bL
  + T\mF$, where $\mF$ is as in Example~\ref{ex:1} and $\bL$ as in
  Example~\ref{ex:coldeg}.  Starting from $\scrR_0$ as obtained in
  Example~\ref{ex:12param}, we obtain a zero-dimensional
  parametrization $\scrS$ of degree 12 with coefficients in $\Q(T)$
  that describes the homotopy curve defined by $\bB=0$.

  The polynomials in $\scrS$ are too large to be displayed here, all
  the more as we are only interested in the points that $\scrS$
  describes in the limit $T \to 1$. Note that one cannot simply
  substitute $T=1$ in $\scrS$, since most denominators vanish. Instead,
  we apply the procedure in~\cite{RRS}, which in this case amounts to
  multiplying the polynomials in $\scrS$ by $(T-1)^5$ before applying
  the substitution $T = 1$. This leaves us with a
  zero-dimensional parametrization of degree $12-5=7$, which is
  precisely the one given in Example~\ref{ex:2}.

  Note that the bound $e=36$ on the degree of the coefficients in
  $\scrS$ is rather pessimistic; all rational functions in $\Q(T)$
  appearing in it have degree at most $11$.
\end{example}


%%%%%%%%%%%%%%%%%%%%%%%%%%%%%%%%%%%%%%%%%%%%%%%%%%%%%%%%%%%%
%%%%%%%%%%%%%%%%%%%%%%%%%%%%%%%%%%%%%%%%%%%%%%%%%%%%%%%%%%%%
%%%%%%%%%%%%%%%%%%%%%%%%%%%%%%%%%%%%%%%%%%%%%%%%%%%%%%%%%%%%

\section{Preliminaries for the row-degree homotopy}\label{sec:prel-row}

In this section, the setting is slightly different from what we have
seen up to now.  We work with two families of matrices of size $p
\times q$, with $p \le q$, and with entries that are either
polynomials in $n=q-p+1$ variables $X_1,\dots,X_n$, or homogeneous in
$n+1$ variables $X_0,\dots,X_n$; we discuss several properties that
will be used in our row-degree homotopy algorithm. First, a notation:
for any ring $R$ and any matrix $\bP \in R^{p\times q}$, if $S$ is a
subsequence of $(1,\dots,p)$ and $U$ a subsequence of $(1,\dots,q)$,
$\bP_{S,U}$ is the submatrix of $\bP$ obtained by keeping rows indexed
by $S$ and columns indexed by $U$. We also call this the
$(S,U)$-submatrix of $\bP$.

Let $\balpha=(\alpha_1,\dots,\alpha_p)$ be positive integers. The
first matrices we consider are
\begin{align}\label{eqdef:type2}
\bP^H= \left( \begin{matrix}
\lambda^H_{1,1} & \lambda^H_{1,2} & \cdots & \lambda^H_{1, q}\\
 \lambda^H_{2,1} &  \lambda^H_{2,2} & \cdots & \lambda^H_{2, q}\\
 \vdots & & & \vdots\\
 \lambda^H_{p,1} &  \lambda^H_{p,2}& \cdots & \lambda^H_{p, q}
\end{matrix} \right)
\end{align}
and matrices of a more specialized kind of the form
\begin{align}\label{eqdef:type1}
\bL^H= \left( \begin{matrix}
\lambda^H_{1,1} & 0 & \cdots & 0 & \lambda^H_{1,p+1} & \cdots & \lambda^H_{1, q}\\
0 & \lambda^H_{2,2} & \cdots & 0 & \lambda^H_{2,p+1} & \cdots & \lambda^H_{2, q}\\
\vdots & \vdots & \ddots & \vdots & \vdots & \ddots & \vdots\\
0 & 0 & \cdots & \lambda^H_{p,p} & \lambda^H_{p,p+1} & \cdots & \lambda^H_{p, q}
\end{matrix} \right),
\end{align}
where the ${}^H$ superscript indicates that all entries are
homogeneous.  In both cases, for all $i,j$, the entry $\lambda^H_{i,j}$
is a product of $\alpha_i$ homogeneous linear forms in $n+1$ variables
$X_0,\dots,X_n$ with coefficients in $\KK$ (except when
$\lambda^H_{i,j}$ is explicitly set to zero in the second case), that
is, $\lambda^H_{i,j}=\prod_{k=1}^{\alpha_i} \lambda^H_{i,j,k}$.  

We are interested in describing the projective algebraic sets defined
in $\P^n(\KKbar)$ by the $p$-minors of $\bP^H$ and $\bL^H$ (note that
these minors are all homogeneous). In the rest of this section, if
$\bQ^H$ is any matrix with polynomial entries that are homogeneous in
$X_0,\dots,X_n$, we use the notation $V_p(\bQ^H)$ to denote the
projective set defined by its $p$-minors in $\P^n(\KKbar)$.  This is
the same notation as the one we use for affine algebraic sets, but
this should cause no confusion. The first key result stated in this
section is the following; since the proof is rather lengthy and
technical, we postpone it to the appendix of the paper.

\begin{proposition}\label{lemma:appendix}
  For generic choices of the coefficients of the linear forms
  $\lambda^H_{i,j,k}$, the following holds:
  \begin{itemize}
  \item the projective algebraic sets $V_p(\bP^H)$ and $V_p(\bL^H)$
    have no solution at infinity (that is, with $X_0=0$);
  \item the Jacobian matrices of the sets of $p$ minors 
    of $\bP^H$, resp.\ of $\bL^H$, has rank $n$ at every point 
    in $V_p(\bP^H)$, resp.\ $V_p(\bL^H)$.
\end{itemize}
\end{proposition}

While homogeneity is used at several steps in the proof, and will be
needed again when we apply this result, our main algorithm deals with
matrices without a homogeneous structure. Thus we will also consider 
matrices $\bP$ and $\bL$ as in~\eqref{eqdef:type1}, but with $X_0=1$. Explicitly,
we have
\begin{align}\label{eqdef:type2aff}
  \bP= \left( \begin{matrix}
    \lambda_{1,1} & \lambda_{1,2} & \cdots & \lambda_{1, q}\\
    \lambda_{2,1} &  \lambda_{2,2} & \cdots & \lambda_{2, q}\\
    \vdots & & & \vdots\\
    \lambda_{p,1} &  \lambda_{p,2}& \cdots & \lambda_{p, q}
  \end{matrix} \right),
\end{align}
and
\begin{align}\label{eqdef:type1aff}
\bL= \left( \begin{matrix}
\lambda_{1,1} & 0 & \cdots & 0 & \lambda_{1,p+1} & \cdots & \lambda_{1, q}\\
0 & \lambda_{2,2} & \cdots & 0 & \lambda_{2,p+1} & \cdots & \lambda_{2, q}\\
\vdots & \vdots & \ddots & \vdots & \vdots & \ddots & \vdots\\
0 & 0 & \cdots & \lambda_{p,p} & \lambda_{p,p+1} & \cdots & \lambda_{p, q}
\end{matrix} \right),
\end{align}
where for all $i,j$, $\lambda_{i,j}$ is the product of $\alpha_i$
degree-one polynomials $(\lambda_{i,j,k})_{1 \le k \le \alpha_i}$ with
coefficients in $\KK$, in variables $X_1,\dots,X_n$. By
Proposition~\ref{lemma:appendix}, we deduce that for a generic choice
of the coefficients of all $(\lambda_{i,j,k})$, both $V_p(\bP) \subset
\KKbar{}^n$ and $V_p(\bL) \subset \KKbar{}^n$ are finite sets.

The second result in this section is an algorithm called
$\mathsf{RowDegreeDiagonal}$, that takes as input 
$(\lambda_{i,j,k})$ as above and computes a zero-dimensional parametrization of
the latter set $V_p(\bL)$. The key to do so is to observe that $\bx$
belongs to $V_p(\bL)$ if and only if some diagonal terms of $\bL$
vanish at $\bx$, say
$\lambda_{i_1,i_1}(\bx)=\cdots=\lambda_{i_\kappa,i_\kappa}(\bx)=0$ for
some indices $1 \le i_1 < \cdots < i_\kappa \le p$, all other
$\lambda_{i,i}(\bx)$ being non-zero, and if the submatrix of $\bL$
obtained by keeping rows $i_1,\dots,i_k$ and columns $p+1,\dots,q$ has
rank less than $\kappa$ at $\bx$. Since for all $i,j$ we have the
factorization
$\lambda_{i,j}=\lambda_{i,j,1}\cdots\lambda_{i,j,\alpha_i}$, where the
right-hand factors are linear, $\lambda_{i,i}(\bx)=0$ if and only if
$\lambda_{i,i,r}(\bx)=0$, for some $r \in \{1,\dots,\alpha_i\}$.

Precisely, we prove in the appendix that for generic choices of the
coefficients of all $(\lambda_{i,j,k})$, to find all points
$\bx$ in $V_p(\bL)$, it is enough to do the following:
\begin{enumerate}
\item Consider all choices of indices $\bi = (1\le i_1 < \dots <
  i_\kappa\le p)$, with $1 \leq \kappa \leq\min(n,p)$.
\item For any such $\bi$, consider all $\br = (r_1,\dots,r_\kappa)$,
  with $r_k \in \{1,\dots,\alpha_{i_k}\}$ for all $k$.
\item For any such $\bi,\br$, apply Gaussian elimination to the linear
  system
  $\lambda_{i_1,i_1,r_1}=\dots=\lambda_{i_\kappa,i_\kappa,r_\kappa}=0$
  to rewrite $(X_{n-\kappa+1}, \ldots, X_n)$  linearly in $(X_1,
  \ldots, X_{n-\kappa})$.
\item If $\kappa = n$, we are done. Otherwise, 
  eliminate $(X_{n-\kappa+1}, \ldots, X_n)$ in the submatrix
  $\bL_{\bi,(p+1, \ldots, q)}$ of $\bL$ of row indices $\bi$ and
  column indices $p+1,\dots,q$. Find the values of $(X_1, \ldots,
  X_{n-\kappa})$ for which this matrix has rank less than $\kappa$,
  and the corresponding values of $(X_{n-\kappa+1}, \ldots, X_n)$ by
  back-substitution. Remark that the latter matrix is as
  in~\eqref{eqdef:type2aff}, but of size $\kappa \times (q-p)$, that
  is, $ \kappa\times (n-1)$.
\end{enumerate}
In addition, we will prove that, for generic choices of the
coefficients of all $(\lambda_{i,j,k})$, we never obtain 
twice the same point from for different choices of $\bi$ and $\br$.

\begin{example}\label{ex:N}
  In the next section, we will work with $\bL \in \K[X_1,X_2]^{2\times 3}$ given by
$$\bL = \left [\begin{matrix}
-X_1 + 10 X_2 + 1 & 0 & X_1+X_2 + 2\\
  0 & (X_1-3X_2+5)(X_1 + 2 X_2 - 4) & (3X_1 +2 X_2-1)(2X_1 -3X_2 +5)
    \end{matrix} \right ].$$
  Here, $p=2$, $q=3$ and $(\alpha_1,\alpha_2)=(1,2)$.
  Let us follow the procedure above for this particular example:
  \begin{itemize}
  \item With $\bi=(1)$, we can only take $\br = (1)$; the equation
    $-X_1 + 10 X_2 + 1=0$ gives the substitution $X_2=(X_1-1)/10$,
    which we inject in $\bL_{(1),(3)} = [ X_1+X_2 + 2 ]$, to find the 
    solution $(-19/11, -3/11)$.
  \item With $\bi=(2)$, we first take $\br=(1)$, so we use
    $X_1-3X_2+5$ to obtain $X_2 = (X_1+5)/3$; we inject this expression into
    $\bL_{(2),(3)}=[(3X_1 +2 X_2-1)(2X_1 -3X_2 +5)]$ to find two solutions, namely
    $(-7/11, 16/11)$ and $(0, 5/3)$. With $\br =(2)$, we find two
    other solutions, $(2/7, 13/7)$ and $(-3/2, 11/4)$.
  \item With $\bi=(1,2)$, we can take $\br=(1,1)$ or $\br=(1,2)$,
    which lead us to solve respectively $-X_1 + 10 X_2 +
    1=X_1-3X_2+5=0$ and $-X_1 + 10 X_2 + 1=X_1 + 2 X_2 - 4=0$; we
    obtain two solutions, $(-53/7, -6/7)$ and $(7/2, 1/4)$. Note that
    in Step 4, we are in the case $\kappa = n$, so there is no need to
    deal with the matrix $\bL_{\bi,(p+1, \ldots, q)}$.
  \end{itemize}
  Altogether, this gives us the 7 points where the rank of $\bL$ is
  not two.
\end{example}

In this example, the matrix $\bL_{\bi,(p+1, \ldots, q)}$ has one column,
which made it straightforward to deal with. In general, in order to
perform Step 4, we assume that we have a subroutine
$\mathsf{RowDegree\_simple}(\Gamma)$ which takes as input a
straight-line program $\Gamma$ that computes a polynomial matrix $\mF$
and a system of equations $\mG$, and returns the simple points of
$V_p(\mF,\mG)$ using a row-degree homotopy. In the context of Step 4,
we only need a particular case of it, where the input matrix belongs
to the family in~\eqref{eqdef:type2aff}, and there are no additional
equations~$\mG$.

We give such an algorithm $\mathsf{RowDegree\_simple}$ in the next section;
that algorithm will in turn use algorithm $\mathsf{RowDegreeDiagonal}$
from this section recursively. In terms of complexity, we denote by
$T_{\rm row}(\sigma,\bgamma,\balpha,q)$ the time spent by
$\mathsf{RowDegree\_simple}(\Gamma)$ on input a straight-line program
of length $\sigma$ that computes a polynomial matrix with row degrees
$\balpha=(\alpha_1,\dots,\alpha_p)$ and $q$ columns, and
$\mG=(g_1,\dots,g_s)$ of degrees $\bgamma=(\gamma_1,\dots,\gamma_s)$.
As we said above, for the moment, we only need a particular case of
this algorithm, since there are no additional equations $\mG$, and the
matrix is as matrix $\bP$ in~\eqref{eqdef:type2aff}. Each entry of
such a matrix $\bP$ can be computed in $O((q-p) \alpha_i)$ operations
in $\KK$ (since we multiply $\alpha_i$ linear forms in $q-p+1$
variables), so that the whole matrix $\bP$ can be computed by a
straight-line program of length
$\sigma=O((q-p)q(\alpha_1+\cdots+\alpha_p))$. Thus, we denote the cost
of Algorithm {\sf RowDegree\_simple} for such input by
\begin{align}
T_{\bP,{\rm row}}(\balpha,q)=T_{\rm row}((q-p)q(\alpha_1+\cdots+\alpha_p),(),\balpha,q).
\end{align}
We conclude this section with the detailed presentation and cost 
analysis of Algorithm $\mathsf{RowDegreeDiagonal}$. Cost estimates reported in the 
algorithm are explained within the proof. 
\begin{algorithm}[!t]
\caption{$\mathsf{RowDegreeDiagonal}((\lambda_{i,j,k})_{i,j,k})$}
{\bf Input}: linear forms $(\lambda_{i,j,k})_{i,j,k}$ making up 
the entries of 
$\bL \in \KK[X_1, \ldots, X_n]^{p \times q}$ as in~\eqref{eqdef:type1aff}, with $p \leq q$ and $n = q-p+1$\\
{\bf Output}: a zero-dimensional parametrization $\scrR$ of $V_p(\bL)$
\begin{enumerate}
\item for any subsequence $\bi = (i_1, \ldots, i_\kappa)$ of $(1, \ldots, p)$ with $1 \leq \kappa \leq\min(n-1,p)$
  \begin{enumerate}
  \item for any sequence $\br = (r_1, \ldots, r_\kappa)$, with $r_k$ in 
    $\{1,\dots,\alpha_{i_k}\}$ for all $k$
    \begin{enumerate}
    \item apply Gaussian elimination to the system 
      $\lambda_{i_1,i_1,r_1}=\dots=\lambda_{i_\kappa,i_\kappa,r_\kappa}=0$
      to rewrite $(X_{n-\kappa+1}, \ldots, X_n)$ as linear forms $(f_{j,\bi,\br})_{n-\kappa+1 \le j \le n}$ in $(X_1, \ldots, X_{n-\kappa})$. 

\hfill $\text{\sf{cost:~}} O(\sum_{\bi,\br} n^3)$
    \item\label{step:constSLP} construct a straight-line program $\Gamma_{\bi,\br}$ that computes the matrix $\bP_{(\bi,\br)}$ in $\KK[X_1, \dots, X_{n-\kappa}]^{\kappa \times (n-1)}$ obtained
      by substituting $(f_{j,\bi,\br})_{n-\kappa+1 \le j \le n}$ into $\bL_{\bi,(p+1, \ldots, q)}$. The length of $\Gamma_{\bi,\br}$ is $O((n-\kappa) n(\alpha_{i_1}+\cdots+\alpha_{i_\kappa}))$.

\hfill $\text{\sf{cost:~}} O(\sum_{\bi,\br} (\alpha_{i_1} + \cdots + \alpha_{i_\kappa}) n^3)$
    \item $\scrR_{\bi,\br}' \gets$ $\mathsf{RowDegree\_simple}(\Gamma_{\bi,\br})$  (points have coordinates $(X_1, \ldots, X_{n-\kappa})$)

\hfill $\text{\sf{cost:~}} \sum_{\bi,\br} T_{\bP,{\rm row}}((\alpha_{i_1},\dots,\alpha_{i_\kappa}), n-1)$
    \item\label{step:substdiag} deduce $\scrR_{\bi,\br}$ from $\scrR_{\bi,\br}'$ by adding the expressions for $(X_{n-\kappa+1}, \ldots, X_n)$

\hfill      $\text{\sf{cost:~}} O(\sum_{\bi,\br}  c'_{\bi,\br} n^2)$
  \end{enumerate}
  \end{enumerate}

\item if $n \le p$, for any subsequence $\bi = (i_1, \ldots, i_n)$ of $(1, \ldots, p)$
  \begin{enumerate}
  \item for any sequence $\br = (r_1, \ldots, r_n)$, with $r_k \in \{1,\dots,\alpha_k\}$ for all $k$
    \begin{enumerate}
    \item let $\bx_{\bi,\br}$ be the solution of the system $\lambda_{i_1,i_1,r_1}=\dots=\lambda_{i_n,i_n,r_n}=0$

\hfill      $\text{\sf{cost:~}} O(\sum_{\bi,\br}n^3)$
    \item create a zero-dimensional parametrization $\scrR_{\bi,\br}$ such that $Z(\scrR_{\bi,\br})=\{\bx_{\bi,\br}\}$

\hfill      $\text{\sf{cost:~}} O(\sum_{\bi,\br}n)$
  \end{enumerate}
\end{enumerate}
\item combine all $(\scrR_{\bi,\br})_{\bi,\br}$ into the output $\scrR$

\hfill  $\text{\sf{cost:~}} \softO(\sum_{\bi,\br}  c'_{\bi,\br} n)$
\end{enumerate}
\label{Row}
\end{algorithm}

\begin{lemma}\label{lemma:rowdegreediagonal}
  Let $S_n(\alpha_1,\dots,\alpha_p)$ be the degree $n$ complete
  symmetric function of $(\alpha_1,\dots,\alpha_p)$. For generic 
  choices of the coefficients of $(\lambda_{i,j,k})_{i,j,k}$,
  $\mathsf{RowDegreeDiagonal}((\lambda_{i,j,k})_{i,j,k})$
  computes a zero-dimensional parametrization of $V_p(\bL)$
  in time $$\sum_{\substack{\bi=(i_1,\dots,i_\kappa)\\ \kappa\le\min(n-1,p)}}
  \alpha_{i_1} \cdots \alpha_{i_\kappa} T_{\bP,{\rm
      row}}((\alpha_{i_1},\dots,\alpha_{i_\kappa}), n-1) + \softO\left (n^3(\rc + S_n(\alpha_1,\dots,\alpha_p))  \right),$$
  where $\rc$ is the cardinality of $V_p(\bL)$.
\end{lemma}
\begin{proof}
  The cost reported at each step in the pseudo-code is the total amount
  of time spent there, over all iterations (the sums in the first loop
  are for $\kappa \le \min(n-1,p)$, the ones in the second loop for
  $\kappa=n$ if $n \le p$). Several steps are straightforward to
  analyze; we briefly comment on a few others.

  Step~\ref{step:constSLP} uses the linear forms
  $(f_{j,\bi,\br})_{n-\kappa+1 \le j \le n}$ to construct a
  straight-line program $\Gamma_{\bi,\br}$ that computes the entries
  of $\bL_{\bi,(p+1, \ldots, q)}$, in which we replace $X_j$ by
  $f_{j,\bi,\br}(X_1, \ldots, X_{n-\kappa})$, for
  $j=n-\kappa+1,\dots,n$.  This is done by computing the coefficients
  of the linear forms in $(X_1,\dots,X_{n-\kappa})$ obtained after
  substitution. Each linear form requires a matrix-vector product with
  a matrix of size $(n-\kappa) \times n$, for $O(n^2)$ operations,
  whence a total of $O((\alpha_{i_1} + \cdots +\alpha_{i_\kappa})n^3)$
  for all entries.

  Step~\ref{step:substdiag} consists in adding $\kappa$ coordinates
  $(X_{n-\kappa+1},\dots,X_n)$ to a zero-dimensional parametrization
  in variables $X_1,\dots,X_{n-\kappa}$, where
  $(X_{n-\kappa+1},\dots,X_n)$ are known as linear forms
  $(f_{j,\bi,\br})_{n-\kappa+1 \le j \le n}$ in $(X_1, \ldots,
  X_{n-\kappa})$: this is done by means of a matrix product in size
  $(\kappa \times n-\kappa)$ by $(n-\kappa \times c'_{\bi,\br})$,
  where $c'_{\bi,\br}$ is the number of solutions we obtain from
  $\mathsf{RowDegree\_simple}(\Gamma_{\bi,\br})$. The cost is thus
  $O(c'_{\bi,\br}n^2 )$; the sum of these costs is $O(\rc n^2 )$,
  since the sum of all $c'_{\bi,\br}$ is equal to $\rc$.

  The combination in the last step is done by fast Chinese Remaindering,
  in quasi-linear time $\softO(\sum_{\bi,\br} c'_{\bi,\br}  n)$,
  which is $\softO(\rc n)$. Thus, the total runtime is 
  \begin{multline*}
 \sum_{\substack{\bi=(i_1,\dots,i_\kappa)\\ \br=(r_1,\dots,r_\kappa)\\\kappa\le\min(n-1,p)}}
T_{\bP,{\rm row}}((\alpha_{i_1},\dots,\alpha_{i_\kappa}), n-1)\\
+\softO\left (\rc n^2  + \sum_{\substack{\bi=(i_1,\dots,i_\kappa)\\ \br=(r_1,\dots,r_\kappa)\\\kappa\le\min(n-1,p)}}
  (\alpha_{i_1} + \cdots +\alpha_{i_\kappa}) n^3 
  +  \sum_{\substack{\bi=(i_1,\dots,i_n)\\ \br=(r_1,\dots,r_n)}} n^3\right).    
  \end{multline*}
  The costs reported in the sums do not depend on $\br$, so that 
  this can be rewritten as 
  \begin{multline*}
 \sum_{\substack{\bi=(i_1,\dots,i_\kappa)\\ \kappa\le\min(n-1,p)}}
  \alpha_{i_1}  \cdots \alpha_{i_\kappa}T_{\bP,{\rm row}}((\alpha_{i_1},\dots,\alpha_{i_\kappa}), n-1)\\
+\softO\left ( \rc n^2 + \sum_{\substack{\bi=(i_1,\dots,i_\kappa)\\ \kappa\le\min(n-1,p)}}
    \alpha_{i_1}  \cdots \alpha_{i_\kappa} (\alpha_{i_1} + \cdots +\alpha_{i_\kappa}) n^3 
  +  \sum_{\substack{\bi=(i_1,\dots,i_n)}   }\alpha_{i_1}  \cdots \alpha_{i_n} n^3\right).    
  \end{multline*}
  The final simplification comes from noting that $\sum_{\bi}
  \alpha_{i_1} \cdots \alpha_{i_\kappa}(\alpha_{i_1} + \cdots
  +\alpha_{i_\kappa})$, for $\bi$ a subsequence of $(1,\dots,p)$ of
  length $\kappa\le\min(n-1,p)$, is bounded  above by
  $S_n(\alpha_1,\dots,\alpha_p)$. The same holds for the second
  sum (which is empty if $n >p$).
\end{proof}


\begin{example}\label{ex:Npar}
  In Example~\ref{ex:N}, we already found the coordinates of all
  points in $V_2(\bL)$ (they may not be rational for other values of
  $p,q$).  The only thing left to do is to describe them by means of a
  univariate representation; we write it as $((w,v_1,v_2),X_2)$, with
{\small  \begin{align*}
w &=Y^7 - \frac{226}{33}Y^6 + \frac{1428899}{94864}Y^5 - \frac{2137943}{284592}Y^4 - \frac{1146637}{94864}Y^3 + 
    \frac{3111547}{284592}Y^2 + \frac{46547}{47432}Y - \frac{390}{539} \\[1mm]
v_1&= -\frac{589}{77}Y^6 + \frac{3963109}{71148}Y^5 - \frac{14713869}{94864}Y^4 + \frac{1345585}{6776}Y^3 - 
    \frac{9415375}{94864}Y^2 - \frac{843103}{71148}Y + \frac{138935}{6776}\\[1mm]
v_2 &=\frac{226}{33}Y^6 - \frac{1428899}{47432}Y^5 + \frac{2137943}{94864}Y^4 + \frac{1146637}{23716}Y^3 - 
\frac{15557735}{284592}Y^2 - \frac{139641}{23716}Y + \frac{390}{77}.
  \end{align*}}
\end{example}


%%%%%%%%%%%%%%%%%%%%%%%%%%%%%%%%%%%%%%%%%%%%%%%%%%%%%%%%%%%%
%%%%%%%%%%%%%%%%%%%%%%%%%%%%%%%%%%%%%%%%%%%%%%%%%%%%%%%%%%%%
%%%%%%%%%%%%%%%%%%%%%%%%%%%%%%%%%%%%%%%%%%%%%%%%%%%%%%%%%%%%

\section{The row-degree homotopy}\label{sec:rowdegree}

We now give algorithms to solve our main problems with  runtimes that
depend on the row-degrees of the input matrix $\mF$. These algorithms
are more complex than the ones in Section~\ref{sec:columndegree}, due
to their recursive nature: the start system we use for the homotopy
must itself be solved by means of several homotopies of smaller size,
using the algorithm given in the previous section.

Again, we are given a matrix $\mF =[f_{i,j}]\in \KK[X_1,\dots,X_n]^{p
  \times q}$ and polynomials $\mG=(g_1,\dots,g_s)$ in
$\KK[X_1,\dots,X_n]$, with $p \leq q$ and $n = q-p+s+1$, and we want
to compute the isolated points (or the simple points) of
$\VpFG{p}{\mF}{\mG}$, with
$$\VpFG{p}{\mF}{\mG} = \{\bx \in \KKbar{}^n \mid
\mathrm{rank}(\mF({\bx})) < p \text{~and~}
g_1(\bx)=\cdots=g_s(\bx)=0\}.$$ Now, we want algorithms whose cost
depends on the row degrees
$\alpha_1=\rdeg(\mF,1),\dots,\alpha_p=\rdeg(\mF,p)$; with this
notation, $\deg(f_{i,j}) \leq \alpha_i$ holds for all $i,j$. As in
Section~\ref{sec:columndegree}, we write
$\gamma_1=\deg(g_1),\dots,\gamma_s=\deg(g_s)$ and we let $\alpha =
\max(\alpha_1, \ldots, \alpha_p)$ and $\gamma = \max(\gamma_1, \ldots,
\gamma_s)$.  We start by stating our first result on computing the
isolated points of $\VpFG{p}{\mF}{\mG}$. Recall in what follows that
$S_{n-s}$ is the complete homogeneous symmetric function of degree
$n-s$.

\begin{proposition}\label{prop:rowdegree}
  The sum of the multiplicities of the isolated points of
  $\VpFG{p}{\mF}{\mG}$ is at most $\rc=\gamma_1\cdots\gamma_s
  S_{n-s}(\alpha_1, \ldots, \alpha_p)$.
  
  Suppose that the matrix $\mF \in \KK[X_1,\dots,X_n]^{p \times q}$
  and the polynomials $\mG=(g_1,\dots,g_s)$ in $\KK[X_1,\dots,X_n]$
  are given by a straight-line program of length $\sigma$.  Assume
  that all $\gamma_i$'s and $\alpha_j$'s are at least equal to $1$,
  and let $\re=(\gamma_1+1)\cdots(\gamma_s+1) S_{n-s}(\alpha_1+1,
  \ldots, \alpha_p+1)$, $\alpha = \max(\alpha_1, \ldots, \alpha_p)$
  and $\gamma = \max(\gamma_1, \ldots, \gamma_s)$. Then, there exists
  a randomized algorithm that computes the isolated points of
  $\VpFG{p}{\mF}{\mG}$ using
  $$\softO\left( {q \choose p} \rc (\re+\rc^5 )  (\sigma + \gamma+ p \alpha)  \right)$$
  operations in $\KK$.
\end{proposition}

We state now the complexity result for computing the simple
points of $\VpFG{p}{\mF}{\mG}$.

\begin{proposition}\label{prop:rowdegree_simple}
  Reusing the notations introduced above, there
  exists a randomized algorithm that computes the simple
  points of $\VpFG{p}{\mF}{\mG}$ using
  $$\softO\left({q \choose p} \rc \re (\sigma + \gamma+ p \alpha)
  \right)$$ operations in $\KK$.
\end{proposition}
These propositions complete the proofs of Theorems~\ref{theo:1},
\ref{theo:2} and~\ref{theo:3}. 

%%%%%%%%%%%%%%%%%%%%%%%%%%%%%%%%%%%%%%%%%%%%%%%%%%%%%%%%%%%%

\subsection{Setting up the homotopy}

We are again going to rely on the algorithm of
Section~\ref{sec:homotopy}. As in Section~\ref{sec:columndegree}, we
let $\bC=(c_1,\dots,c_s,c_{s+1}\dots,c_m)$ be such that
$(c_1,\dots,c_s)=(g_1,\dots,g_s)$ and $(c_{s+1},\dots,c_m)$ are the
$p$-minors of $\mF$. Our main concern is to design a sequence of
polynomials $\bB=(b_1,\dots,b_s,\dots,b_m)$ in $\KK[T,\bX]$ such that
$\bC=\bB_{T=1}$, such that we can solve efficiently the system
$\bA=\bB_{T=0}$, and such that $\bB$ has the same row-degree profile
as our target system~$\bC$.

As in the column-degree case, using the degrees
$\gamma_1,\dots,\gamma_s$ allows us to construct polynomials
$\bM=(m_1,\dots,m_s)$ in $\KK[\bX]$: for $i=1,\dots,s$ and $k=1,\dots,\gamma_i$,
we define
\begin{equation}\label{eqdef:ai}
\mu_{i,k} = \mu_{i,k,0} + \sum_{\ell = 1}^{n}\mu_{i,k,\ell}X_\ell,
\end{equation}
where all $\mu_{i,k,\ell}$ are random elements in $\KK$ and we let
$\bM=(m_1,\dots,m_s)$, with $m_i=\prod_{k=1}^{\gamma_i} \mu_{i,k}$ for
all $i$. The difference with the column-degree case lies in the
construction of the start matrix used in the homotopy. The
construction presented in Section~\ref{sec:columndegree} does not
carry over if we want to take row degrees into account. Instead, we
use a deformation that cancels out many off-diagonal terms; following
the construction in the previous section, we define $\bL$ as
in~\eqref{eqdef:type1aff}, that is
\begin{align*}
\bL = \left( \begin{matrix}
\lambda_{1,1} & 0 & \cdots & 0 & \lambda_{1,p+1} & \cdots & \lambda_{1, q}\\
0 & \lambda_{2,2} & \cdots & 0 & \lambda_{2,p+1} & \cdots & \lambda_{2, q}\\
\vdots & \vdots & \ddots & \vdots & \vdots & \ddots & \vdots\\
0 & 0 & \cdots & \lambda_{p,p} & \lambda_{p,p+1} & \cdots & \lambda_{p, q}
\end{matrix} \right), 
\end{align*} 
where for all $i,j$, $\lambda_{i,j}$ is a product of $\alpha_i$ linear
forms with random coefficients in $\KK$, of the form
$$\lambda_{i,j}= \prod_{k=1}^{\alpha_i}\lambda_{i,j,k},
\quad\text{with}\quad
\lambda_{i,j,k} =\lambda_{i,j,k,0} + \sum_{\ell=1}^n \lambda_{i,j,k,\ell}X_\ell.
$$ Our start system $\bA=(a_1,\dots,a_s,\dots,a_m)$ is defined by
taking $(a_1,\dots,a_s) = (m_1,\dots,m_s)$, and letting
$(a_{s+1},\dots,a_m)$ be the $p$-minors of $\bL$.

We can then define the polynomials $\mV=(v_1,\dots,v_s)$ by $\bV =
(1-T) \cdot \bM + T \cdot \mG$, the matrix $\mU=(1-T)\cdot \bL + T
\cdot \mF \in \KK[T,\bX]^{p\times q}$, and we let $\bB$ be the
polynomials in $\KK[T,\bX]$ given by $\bB=(b_1,\dots,b_s,\dots,b_m)$,
where $b_i=v_i$ for $i=1,\dots,s$ and $(b_{s+1},\dots,b_{m})$ are the
$p$-minors of $\mU$.  Then, $\bB_{T=1}=\bC$ and $\bB_{T=0}=\bA$.

Our next step is to prove that the assumptions of
Proposition~\ref{prop:degree_fiber} are satisfied for $\bB$, as long
as the coefficients of $m_1,\dots,m_s$ and $\bL$ are chosen
generically.

\paragraph*{Property $\assG_1$.} We have to prove that for $i=1,\dots,m$,
$\deg_\bX(b_i)=\deg_\bX(m_i)$. We already established it in
Section~\ref{sec:columndegree} for indices $i=1,\dots,s$. For
$i=s+1,\dots,m$, we can readily see that the degree of $b_i$ in $\bX$
is at most $\alpha_1 + \cdots + \alpha_p$, so it suffices to prove
that the degree of all $p$-minors $(a_{s+1},\dots,a_m)$ of $\bL$ is
$\alpha_1 + \cdots + \alpha_p$.

Indeed, any $p$-minor of $\bL$ is of the form $\lambda_{i_1,i_1}
\cdots \lambda_{i_\kappa,i_\kappa} \zeta$, for some sequence
$\bi=(i_1,\dots,i_\kappa) \subset (1,\dots,p)$ of length $\kappa \in
\{0,\dots,p\}$ and some $(p-\kappa)$-minor $\zeta$ of $\bL_{\bi
,(p+1,\dots,q)}$.  Since the entries of $\bL_{\bi
,(p+1,\dots,q)}$ are products of linear form with randomly
chosen coefficients $(\lambda_{i,j,k,\ell})$, for a generic choice of
these coefficients, the determinant $\zeta$ has degree
$\sum_{i' \notin \bi} \alpha_{i'}$. As a result, the corresponding
$p$-minor of $\bL$ has degree $\alpha_1 + \cdots + \alpha_p$, as
claimed.

\paragraph*{Property $\assG_2$.} Next, we prove that the system $\bA=\bB_{T=0}$ has no solution 
at infinity. As in Section~\ref{sec:columndegree}, we introduce a
homogenization variable $X_0$, and we consider the system
$\bA^H=(a_1^H,\dots,a_s^H,\dots,a_m^H)$ obtained by homogenizing all
equations in $\bA$. Thus we have
$$a_i^H=\prod_{k=1}^{\gamma_i} \mu^H_{i,k} \quad\text{with}\quad \mu^H_{i,k}=\mu_{i,k,0}X_0 + \sum_{\ell = 1}^{n}\mu_{i,k,\ell}X_\ell$$
for $i=1,\dots,s$, whereas $a_{s+1}^H,\dots,a_m^H$ are the $p$-minors of the matrix
\begin{align*}
\bL^H = \left( \begin{matrix}
\lambda^H_{1,1} & 0 & \cdots & 0 & \lambda^H_{1,p+1} & \cdots & \lambda^H_{1, q}\\
0 & \lambda^H_{2,2} & \cdots & 0 & \lambda^H_{2,p+1} & \cdots & \lambda^H_{2, q}\\
\vdots & \vdots & \ddots & \vdots & \vdots & \ddots & \vdots\\
0 & 0 & \cdots & \lambda^H_{p,p} & \lambda^H_{p,p+1} & \cdots & \lambda^H_{p, q}
\end{matrix} \right),  
\end{align*}
where $\lambda^H_{i,k}$ is the homogenization of
$\lambda_{i,j}$. (This latter property requires genericity of the
coefficients of the linear forms $\lambda^H_{i,k}$; it is enough that 
each $p$-minor of $\bL$ have degree $\alpha_1 + \cdots + \alpha_p$.)

The solutions of $\bA^H$ in $\P^n(\KKbar)$ are found by first solving
the equations $(a^H_1,\dots,a^H_s)$. As in Section~\ref{sec:columndegree}, all
$a_i^H$ are products of linear forms, so any solution of
$(a^H_1,\dots,a^H_s)$ is obtained by setting some of these linear forms
to zero (at least one for each $i=1,\dots,s$). We choose indices $\bu=(u_1,\dots,u_s)$, with
$u_1\in\{1,\dots,\gamma_1\}$, \dots, $u_s\in\{1,\dots,\gamma_s\}$, and
we solve
$$\mu^H_{i,u_i}=0, \quad \text{~that is,~} \quad \mu_{i,u_i,0}X_0 + \sum_{\ell = 1}^{n}\mu_{i,u_i,\ell}X_\ell =0,$$ for $i=1,\dots,s$.
Then, for a generic choice of coefficients $\mu_{i,k,\ell}$, these equations
are equivalent to
$$X_{n-s+1}=\Phi_{n-s+1,\bu}(X_0,\dots,X_{n-s}),\dots,X_{n}=\Phi_{n,\bu}(X_0,\dots,X_{n-s}),$$
for some homogeneous linear forms $\Phi_{n-s+1,\bu},\dots,\Phi_{n,\bu}$.
After applying this substitution, for all $i,j$,
$\bL^H$ can be rewritten as 
\begin{align*}
 \bL^H_\bu = \left( \begin{matrix}
\lambda^H_{1,1,\bu} & 0 & \cdots & 0 & \lambda^H_{1,p+1,\bu} & \cdots & \lambda^H_{1, q,\bu}\\
0 & \lambda^H_{2,2,\bu} & \cdots & 0 & \lambda^H_{2,p+1,\bu} & \cdots & \lambda^H_{2, q,\bu}\\
\vdots & \vdots & \ddots & \vdots & \vdots & \ddots & \vdots\\
0 & 0 & \cdots & \lambda^H_{p,p,\bu} & \lambda^H_{p,p+1,\bu} & \cdots & \lambda^H_{p, q,\bu}
\end{matrix} \right),
\end{align*}
with
$$\lambda^H_{i,j,\bu}=\prod_{k=1}^{\alpha_i}\lambda^H_{i,j,k,\bu},
\quad\text{and}\quad \lambda^H_{i,j,k,\bu}=\sum_{\ell =
  0}^{n-s}\lambda_{i,j,k,\ell}X_\ell + \sum_{\ell =
  n-s+1}^{n}\lambda_{i,j,k,\ell} \Phi_{\ell,\bu}(X_0,\dots,X_{n-s}).$$
Remark that the entries of $\bL^H_\bu$ are products of homogeneous
linear forms in $(n-s)+1$ variables $(X_0,\dots,X_{n-s})$, so that
this matrix has the form seen in~\eqref{eqdef:type1}. As a result, for
a generic choice of the coefficients $\mu_{i,k,\ell}$ and
$\lambda_{i,j,k,\ell}$, the first item in
Proposition~\ref{lemma:appendix} implies that there is no common projective
solution to the $p$-minors of 
$\bL^H_\bu$ satisfying $X_0=0$. Taking into account
all possible choices of $\bu$, we deduce that there is no projective
solution to $\bA^H$ satisfying $X_0=0$, and $\assG_2$ is proved.

\paragraph*{Property $\assG_3$.} Finally, we have to prove that the Jacobian
matrix of $\bA$ has full rank $n$ at any point in $V(\bA) \subset
\KKbar{}^n$. Let thus $\bx=(x_1,\dots,x_n)$ be in $V(\bA)$; in
particular, $\tilde \bx=(1,x_1,\dots,x_n)$ is a projective solution of
$\bA^H$.  Thus, there exists $\bu=(u_1,\dots,u_s)$ as above such that
$$x_{n-s+1}=\phi_{n-s+1,\bu}(x_1\dots,x_{n-s}),\dots,x_{n}=\phi_{n,\bu}(x_1,\dots,x_{n-s}),$$
where $\phi_{k,\bu}(X_1,\dots,X_{n-s})=\Phi_{k,\bu}(1,X_1,\dots,X_{n-s})$ for 
all $k$, and such that
and $\bL^H_\bu$ has rank less than $p$ at $\tilde\bx'=(1,x_1,\dots,x_{n-s})$.  The second item of
Proposition~\ref{lemma:appendix} shows that the Jacobian matrix of
the $p$-minors of 
$\bL^H_\bu$ with respect to $X_0,\dots,X_{n-s}$ has rank $n-s$
at $\tilde \bx'$. Since the first coordinate of $\tilde\bx'$ is non-zero,
and the $p$-minors  of $\bL^H_\bu$ are homogeneous,
Euler's relation implies that the Jacobian matrix of 
these minors
with respect to $X_1,\dots,X_{n-s}$
has full rank $n-s$ at $\bx'=(x_1,\dots,x_{n-s})$, where  
\begin{align}\label{eqdef:Lu}
 \bL_\bu = \left( \begin{matrix}
\lambda_{1,1,\bu} & 0 & \cdots & 0 & \lambda_{1,p+1,\bu} & \cdots & \lambda_{1, q,\bu}\\
0 & \lambda_{2,2,\bu} & \cdots & 0 & \lambda_{2,p+1,\bu} & \cdots & \lambda_{2, q,\bu}\\
\vdots & \vdots & \ddots & \vdots & \vdots & \ddots & \vdots\\
0 & 0 & \cdots & \lambda_{p,p,\bu} & \lambda_{p,p+1,\bu} & \cdots & \lambda_{p, q,\bu}
\end{matrix} \right),
\end{align}
with
$$\lambda_{i,j,\bu}=\prod_{k=1}^{\alpha_i}\lambda_{i,j,k,\bu},
\quad\text{and}\quad \lambda_{i,j,k,\bu}=\lambda_{i,j,k,0}+\sum_{\ell
  = 1}^{n-s}\lambda_{i,j,k,\ell}X_\ell + \sum_{\ell =
  n-s+1}^{n}\lambda_{i,j,k,\ell}
\phi_{\ell,\bu}(X_1,\dots,X_{n-s}).$$ We now prove
that the Jacobian matrix of $\bA$ with respect to $X_1,\dots,X_n$ has
full rank at $\bx$.

The first step is similar to what we did in Section~\ref{sec:columndegree}.  For
$i=1,\dots,s$, $a_i$ is a product of linear forms of the form
$a_i=\prod_{k=1}^{\gamma_i} \mu_{i,k}$, with $\mu_{i,u_i}(\bx)=0$.
Since the coefficients $\mu_{i,k,\ell}$ are chosen generically, for
$i=1,\dots,s$ and $k \ne u_i$, $\mu_{i,k}(\bx)$ is non-zero; as a
result, in the localization  $\mathcal{O}_\bx$ of the polynomial ring $\KK[X_1,\dots,X_{n}]$ at $\bx$,
 the polynomials $(a_1,\dots,a_s)$
are equal (up to units) to the linear forms
$(\mu_{1,u_1},\dots,\mu_{s,u_s})$. This further implies that
\begin{align}\label{eq:subst}
X_{n-s+1}-\phi_{n-s+1,\bu}(X_1,\dots,X_{n-s}),\dots,X_{n}-\phi_{n,\bu}(X_1,\dots,X_{n-s})
\end{align}
belong to the ideal generated by $(a_1,\dots,a_s)$ in
$\mathcal{O}_\bx$.

Next, we consider the $p$-minors $(a_{s+1},\dots,a_m)$ of $\bL$. Let $\zeta
\in\KK[X_1,\dots,X_n]$ be a $p$-minor of $\bL$, and let $\zeta_\bu \in
\KK[X_1,\dots,X_{n-s}]\subset \KK[X_1,\dots,X_{n}]$ be the polynomial obtained after applying the
substitution in~\eqref{eq:subst} in $\bL$.  Since
$\zeta$ and all polynomials in~\eqref{eq:subst} are in the ideal $\langle \bA
\rangle \cdot \mathcal{O}_\bx$, the polynomial $\zeta_\bu$ is in this ideal as
well. Now, note that $\zeta_\bu$ is a $p$-minor of $\bL_\bu$ as defined
in~\eqref{eqdef:Lu}, and that all its $p$-minors are obtained this way. We
pointed out above that the Jacobian matrix of these equations with respect to
$X_1,\dots,X_{n-s}$ has full rank $n-s$ at $\bx'$. As a result, taking all
$\zeta_\bu$ into account, together with the equations in~\eqref{eq:subst}, we
obtain a family of polynomials in $\langle \bA \rangle \cdot \mathcal{O}_\bx$
whose Jacobian matrix has rank $n$ at $\bx$, and $\assG_3$ is proved.

\medskip

In view of the previous paragraphs, we can then apply
Proposition~\ref{prop:degree_fiber}. We deduce that the sum of the
multiplicities of the isolated solutions of $\bC=\bB_{T=1}$ is at most
$\rc$, where $\rc$ is the number of solutions of $\bA$. Our next step
is to establish the value of $\rc$. This is done in
Corollary~\ref{coro:complete} below, and proves the first claim in
Proposition~\ref{prop:rowdegree}.

\begin{lemma}
  Let $\balpha=(\alpha_1,\dots,\alpha_p)$ be positive integers,  and
  let $S_{t}(\alpha_1,\dots,\alpha_p)$ be the complete symmetric
  function of degree $t$ in $\alpha_1,\dots,\alpha_p$. For generic $p
  \times q$ matrices $\bL$ as in~\eqref{eqdef:type1aff} or $\bP$ as
  in~\eqref{eqdef:type2aff}, with entries in $t=q-p+1$ variables,
$V_p(\bL)$ and $V_p(\bP)$ have
  cardinality $S_{t}(\alpha_1,\dots,\alpha_p)$.
\end{lemma}
\begin{proof}
  First, let us show that if the claim holds for $\bL$ in size $p
  \times q$, it holds for $\bP$ as well. To this effect, we set up a
  homotopy between $\bL$ and $\bP$, by considering the matrix
  $(1-T)\cdot\bL + T\cdot\bP$.  The discussion in the previous
  paragraphs shows that (for generic choices of the coefficients) this
  matrix satisfies properties $\assG_1,\assG_2,\assG_3$ at $T=0$.  We claim
  that the same properties (degree bound, no solution at infinity, no
  multiplicities) hold as well at $T=1$: the degree property at $T=1$
  is proved as we did for $T=0$ above, and the latter two are
  restatements of Proposition~\ref{lemma:appendix}. As a result, we
  can apply Proposition~\ref{prop:degree_fiber} to the specializations
  of $(1-T)\cdot\bL + T\cdot\bP$ at both $T=0$ and $T=1$, and conclude
  that $V_p(\bL)$ and $V_p(\bP)$ have the same cardinality, for
  generic choices of the coefficients of~$\bL$~and~$\bP$.

  We finish the proof by induction. If $p=q$, then $t=1$, $\bL$ is
  diagonal, and its determinant has degree $\alpha_1 + \cdots +
  \alpha_p = S_1(\alpha_1,\dots,\alpha_p)$, so our claim holds for
  $\bL$ (and thus for $\bP$). Suppose now that the claim is true for
  all $p'\le p$ and all $q' < q$ with $p' \le q'$ and for all choices
  of degrees $(\alpha_1,\dots,\alpha_{p'})$. Following Algorithm~{\sf
    RowDegreeDiagonal}, we obtain
  $$
  |V_p(\bL)| = \sum_{\substack{\bi=(i_1,\dots,i_\kappa)\\\br=(r_1,\dots,r_\kappa)}}  |V_\kappa(\bP_{\bi,\br})|,
  $$ for all subsequences $\bi=(i_1,\dots,i_\kappa)$ of length $\kappa
  \in \{1,\dots,\min(t-1,p)\}$ and $\br=(r_1,\dots,r_\kappa)$, with
  $r_k \in \{1,\dots,\alpha_k\}$ for all $k$, and where matrix
  $\bP_{(\bi,\br)}$ is from Step~\ref{step:constSLP} of that algorithm.
  It has $\kappa \le p$ rows and $t-1 < q$ columns, with row degrees
  $(\alpha_{i_1},\dots,\alpha_{i_p})$; in particular, we can apply our
  induction assumption to such matrices.  In addition, if $t \le p$,
  we should take into account one extra point for each subsequence
  $(i_1,\dots,i_t)$ of $(1,\dots,p)$. Altogether, we obtain
  $$
  |V_p(\bL)| = \sum_{\substack{\bi=(i_1,\dots,i_\kappa),\\\br=(r_1,\dots,r_\kappa)}}  S_{t-\kappa}(\alpha_{i_1},\dots,\alpha_{i_\kappa}),
  $$
  for $\kappa \in \{1,\dots,\min(t,p)\}$, since $S_0=1$.
  For any given $\bi=(i_1,\dots,i_\kappa)$, there are $\alpha_{i_1}\cdots \alpha_{i_\kappa}$ 
  choices of indices $\br$, so that we have
  $$
  |V_p(\bL)| = \sum_{\bi=(i_1,\dots,i_\kappa)} \alpha_{i_1}\cdots \alpha_{i_\kappa} S_{t-\kappa}(\alpha_{i_1},\dots,\alpha_{i_\kappa}),
  $$
  for $\bi=(i_1,\dots,i_\kappa)$ subsequence of $(1,\dots,p)$ with $\kappa \in \{1,\dots,\min(t,p)\}$.
  The latter sum is precisely $S_t(\alpha_1,\dots,\alpha_p)$, so we are done.
\end{proof}

\begin{corollary}\label{coro:complete}
  For a generic choice of coefficients $\mu_{i,k,\ell}$ and
  $\lambda_{i,j,k,\ell}$, the cardinality $\rc$ of the algebraic set
  $V(\bA)$ is $\gamma_1 \cdots \gamma_s S_{n-s}(\alpha_1,\dots,\alpha_p)$.
\end{corollary}
\begin{proof}
  For a sequence $\bu=(u_1,\dots,u_s)$ as above, let $V_\bu$ be the
  subset of $V(\bA)$ consisting of all those points $\bx$ such that
  $\mu_{i,u_i}(\bx)=0$ for all $i$. Remark first that the sets $V_\bu$
  are (generically) pairwise disjoint: we pointed out above that for
  $\bx$ in $V_\bu$, any index $i$ and any $k \ne u_i$,
  $\mu_{i,k}(\bx)$ is non-zero.
  
  Let us thus fix $\bu=(u_1,\dots,u_s)$. The cardinality of $V_\bu$ is
  equal to the number of points in $V_p(\bL_\bu)$; this is a
  polynomial matrix of size $p \times q$, with entries that are
  products of linear forms in $n-s=q-p+1$ variables and with row
  degrees $\alpha_1,\dots,\alpha_p$. The previous lemma then shows
  that for any $\bu$, for generic choices of the coefficients, $V_\bu$
  has cardinality $S_{n-s}(\alpha_1,\dots,\alpha_p)$; the conclusion
  follows.
\end{proof}

%%%%%%%%%%%%%%%%%%%%%%%%%%%%%%%%%%%%%%%%%%%%%%%%%%%%%%%%%%%%

\subsection{The homotopy algorithms}

Next, we show how to apply Proposition~\ref{prop:compute_regular} to
compute the {\em simple} points in $\VpFG{p}{\mF}{\G}$ first. Indeed, the
resulting algorithm $\mathsf{RowDegree\_simple}$ is used by algorithm
$\mathsf{RowDegreeDiagonal}$ of the previous section. In a second
time, by a minor modification of the algorithm, we show how to compute
the isolated points of $\VpFG{p}{\mF}{\mG}$, by means of
Proposition~\ref{prop:compute_isolated}.

As in Section~\ref{sec:columndegree}, we start by a description of the
inputs needed by the two propositions above. In what follows, we
assume that we are given a straight-line program $\Gamma$ of length
$\sigma$ that computes the input matrix $\mF$ and the input equations
$\G$.  Besides, we also assume that all $\gamma_i$'s and $\alpha_j$'s
are at least equal to $1$.

\paragraph*{A straight-line program for $\bB$.} 
As a preliminary, we construct a straight-line program $\Delta$ that
computes the entries of~$\bL$: for all $i,j$, $\Delta$ computes and
multiplies the values of the $\alpha_i$ linear forms involved in
$\lambda_{i,j}$ using $O(n \alpha_i)$ steps. Its total length is
$\sigma_\bL=O(n^2 (\alpha_1+\cdots+\alpha_p))$, which is $O(n^2 p
\alpha)$, with $\alpha = \max(\alpha_1, \ldots, \alpha_p)$.

For an extra $O({q \choose p} n^3)$ operations, we can compute all
entries of $\mU=(1-T)\cdot\bL+T\cdot\mF$ and all $p$-minors
$(b_{s+1},\dots,b_m)$ of this matrix.  Adding an extra $O(n(\gamma_1 +
\cdots + \gamma_s))\in O(n^2 \gamma)$ operations (with $\gamma =
\max(\gamma_1, \ldots, \gamma_s)$), we can also compute all
polynomials $(a_1,\dots,a_s)$, and thus $(b_1,\dots,b_s)$.
Altogether, we have obtained a straight-line program $\Gamma'$ that
computes $\bB=(b_1,\dots,b_m)$ using $\sigma'=\sigma + \sigma_\bL +
O({q \choose p} n^3 + n^2 \gamma)=\sigma +O( {q \choose p} n^3 + n^2 p
\alpha+ n^2\gamma)$ operations.

\paragraph*{A zero-dimensional parametrization $\scrR_0$ of $V(\bA)$.} 
To perform the homotopy, we need the solutions of the start system,
that is, a zero-dimensional parametrization of $V(\bA)$. We now
describe how to obtain it; the process is based on Algorithm {\sf
  RowDegreeDiagonal} given in the previous section, and makes up the
first two steps in Algorithm ${\sf RowDegree\_simple}$.

For any sequence $\bu=(u_1,\dots,u_s)$, with $u_j$ in
$\{1,\dots,\gamma_j\}$ for all $j$, we start by solving the equations
$\mu_{1,u_1} = \cdots = \mu_{s,u_s}=0$, to express
$(X_{n-s+1},\dots,X_n)$ as linear forms
$(\phi_{n-s+1,\bu},\dots,\phi_{n,\bu})$ in $(X_1,\dots,X_{n-s})$; this
takes a total of $O(\gamma_1 \cdots \gamma_s n^3 )$ operations in
$\KK$.

From this, we deduce a straight-line program $\Delta_\bu$ that
computes the entries of matrix $\bL_\bu$ from~\eqref{eqdef:Lu}: it
simply consists in $\Delta$ (from the previous paragraph), to which we
add $O(n^2)$ operations that evaluate
$(\phi_{n-s+1,\bu},\dots,\phi_{n,\bu})$. Given $\Delta_\bu$, we can
then apply Algorithm {\sf RowDegreeDiagonal} to compute a
zero-dimensional parametrization $\scrR'_\bu$ of $V_p(\bL_\bu)$.  The
number of points $\rc$ in the output is
$S_{n-s}(\alpha_1,\dots,\alpha_p)$ (Corollary~\ref{coro:complete}), so
by Lemma~\ref{lemma:rowdegreediagonal}, Algorithm {\sf
  RowDegreeDiagonal} takes time
\begin{eqnarray}
\mathscr{T}:=\sum_{\substack{\bi=(i_1,\dots,i_\kappa)\\ \kappa\le\min(n-s-1,p)}}
\alpha_{i_1} \cdots \alpha_{i_\kappa} T_{\bP,{\rm row}}((\alpha_{i_1},\dots,\alpha_{i_\kappa}), n-s-1)
+
\softO( S_{n-s}(\alpha_1,\dots,\alpha_p) n^3).  \label{eqdef:T}
\end{eqnarray}
Since there are $\gamma_1 \cdots \gamma_s$ choices of $\bu$, 
the total cost is $\gamma_1 \cdots \gamma_s \mathscr{T}$.


The next stage consists in adding to each $\scrR'_\bu$, which involves
only variables $(X_1,\dots,X_{n-s})$, the expressions of
$(X_{n-s+1},\dots,X_n)$ obtained from
$(\phi_{n-s+1,\bu},\dots,\phi_{n,\bu})$. As in the analysis of
Algorithm {\sf RowDegreeDiagonal}, the total runtime is $O(\gamma_1
\cdots \gamma_s S_{n-s}(\alpha_1,\dots,\alpha_p)  n^2)=O(\rc n^2)$. Finally, we
combine the resulting parametrizations $(\scrR_\bu)_\bu$ into a single
parametrization $\scrR$ using Chinese Remaindering, in time
$\softO(\gamma_1 \cdots \gamma_s  S_{n-s}(\alpha_1,\dots,\alpha_p) n)=\softO(\rc n)$.

Altogether, the overall time spent in computing the zero-dimensional
parametrization $\scrR$ of $V(\bA)$ is 
\begin{multline}\label{eq:VbA}
\gamma_1 \cdots \gamma_s \mathscr{T} +\softO( \rc n^3)  \\  = \gamma_1 \cdots \gamma_s\sum_{\substack{\bi=(i_1,\dots,i_\kappa)\\ \kappa\le\min(n-s-1,p)}}
\alpha_{i_1} \cdots \alpha_{i_\kappa} T_{\bP,{\rm row}}((\alpha_{i_1},\dots,\alpha_{i_\kappa}), n-s-1)
+\softO( \rc n^3).
\end{multline}

\paragraph*{An upper bound $e'$ on the degree of the homotopy curve.}
Next, we need to determine an upper bound $\re$ on the degree of the
homotopy curve $V(J')$, where $J'$ is the union of the one-dimensional
irreducible components of $V(\bB) \subset \KKbar{}^{n+1}$ whose
projection on the $T$-axis is dense. Proceeding as in
Section~\ref{sec:columndegree}, we can take for $\re$ the integer
$(\gamma_1+1)\cdots(\gamma_s+1) S_{n-s}(\alpha_1
+1,\dots,\alpha_p+1)$.

\medskip



\begin{algorithm}[!t]
\caption{$\mathsf{RowDegree\_simple}(\Gamma)$}
{\bf Input}:  a straight-line program $\Gamma$ of length $\sigma$ that computes
 $F \in \KK[X_1, \ldots, X_n]^{p \times q}$ with $\deg(f_{i,j}) \leq \alpha_i$
and $G = (g_1, \ldots, g_s)$ in $\KK[X_1, \ldots, X_n]$ with $p \leq q$, $n = q-p+s+1$\\
{\bf Output}: a zero-dimensional parametrization of the isolated points of $\VpFG{p}{\mF}{G}$
\begin{enumerate}
\item construct a straight-line program $\Delta$ that computes
  $\bL \in \KK[X_1,\dots,X_n]^{p \times q}$ as
  in~\eqref{eqdef:type1aff}
  
\hfill length of $\Delta$ is $O(n^2 p \alpha)$
\item for any sequence $\bu=(u_1,\dots,u_s)$, with $u_j \in \{1,\dots,\gamma_j\}$ for all $j$
\begin{enumerate}
\item apply Gaussian elimination to the system 
  $\mu_{1,u_1}=\cdots=\mu_{s,u_s}=0$ from~\eqref{eqdef:ai} to
  rewrite $(X_{n-s+1}, \ldots, X_n)$ as linear forms
  $(\phi_{k,\bu})_{n-s+1 \le k \le n}$ in $(X_1,\dots,X_{n-s})$
  
\hfill  $\text{\sf{cost:~}} O(\gamma_1 \cdots \gamma_s n^3)$

\item construct a straight-line program $\Delta_{\bu}$ that computes the matrix 
  $\bL_\bu  \in \KK[X_1, \dots, X_{n-s}]^{p \times q}$ obtained
  by substituting $(\phi_{k,\bu})_{n-s+1 \le k \le n}$ into $\bL$
  
  \hfill length of $\Delta_{\bu}$ is $O(n^2 p\alpha)$

\item $\scrR'_\bu \gets \mathsf{RowDegreeDiagonal}(\Gamma_\bu)$ (points have coordinates $(X_1,\dots,X_{n-s})$
  
\hfill  $\text{\sf{cost:~}} \gamma_1 \cdots \gamma_s \mathscr{T}$, for $\mathscr{T}$ as in~\eqref{eqdef:T}

\item deduce $\scrR_\bu$ from $\scrR'_\bu$ by adding the expressions for $(X_{n-s+1},\dots,X_n)$
  
\hfill  $\text{\sf{cost:~}} O( \rc n^2)$, with $\rc= \gamma_1 \cdots \gamma_s S_{n-s}(\alpha_1,\dots,\alpha_p)$
\end{enumerate}
\item combine all $\scrR_\bu$ into $\scrR$

  \hfill $\text{\sf{cost:~}} \softO( \rc n)$

\item construct a straight-line program $\Gamma'$ that computes all polynomials $\bB$

  \hfill length of $\Gamma'$ is $\sigma'=O(\sigma + {q \choose p} n^3 + n^2 p \alpha + n^2\gamma)$

\item\label{step:final:rowdegreesimple} return $\mathsf{Homotopy\_simple}(\Gamma',\scrR)$

  \hfill $\text{\sf{cost:~}} \softO( \rc^2 m n^2 + \rc \re n (\sigma'  + n^2))$, 
with $\re=(\gamma_1+1) \cdots (\gamma_s+1) S_{n-s}(\alpha_1+1,\dots,\alpha_p+1)$
\end{enumerate}
\label{RowHom_simple}
\end{algorithm}



\begin{example}
  We show how to solve the example given in the introduction
  (Example~\ref{ex:1}), using this time a row-degree homotopy. Recall
  that in this example, we have $s=0$, $(\alpha_1,\alpha_2)=(1,2)$ and
  $c'=7$.  Thus, we do not need polynomials $\bM$; our start matrix
  $\bL$ is taken as in Example~\ref{ex:N}.

  We already gave in Example~\ref{ex:Npar} a zero-dimensional
  parametrization $\scrR_0$ of the solutions of the start system, that is, of
  the $2$-minors of $\bL$. The upper bound $e'$ on the degree 
  of the homotopy curve is $e'=2^2 + 2\cdot 3 + 3^3=19$.

  Running algorithm $\mathsf{Homotopy\_simple}$, we construct a
  zero-dimensional parametrization $\scrS$ with coefficients in
  $\Q(T)$ that describes the homotopy curve; specialization at $T=1$
  in it does not induce any division by zero. Doing so gives us the
  zero-dimensional parametrization describing the zeros of the
  $2$-minors of $\mF$ already seen in Example~\ref{ex:2}.
\end{example}


Algorithm $\mathsf{RowDegree\_simple}$ that we deduce from the above
discussion is given hereafter. We indicate in the pseudo-code the
arithmetic costs for intermediate steps. All cost estimates were given
just above and are summarized in~\eqref{eq:VbA}, save for that of the
last step. To estimate this cost, we apply
Proposition~\ref{prop:compute_regular}, which gives a runtime of
$\softO(\rc^2 m n^2 + \rc \re n (\sigma' + n^2))$ operations in~$\KK$.
Now, we write $\sigma'+ n^2 = \sigma +O( {q \choose p} n^3 + n^2 p
\alpha + n^2 \gamma)$, for which we use the upper bound ${q \choose p}
n^3(\sigma + p\alpha + \gamma)$ (recall $\alpha=\max(\alpha_1, \ldots,
\alpha_p)$ and $\gamma=\max(\gamma_1, \ldots, \gamma_s)$).  This gives
the upper bound
\[
\softO\left (\rc^2 m n^2 + \rc\re n {q \choose p} n^3(\sigma + p\alpha + \gamma )\right ).
\]
Using the inequalities $\rc \le \re$ and $m\leq n + {q \choose p} \le n {q \choose p}$,
we see that the second term in the sum is dominant.
Thus, the bound for the cost of Algorithm {\sf Homotopy\_simple} becomes
\[
\softO\left ( \rc\re {q \choose p} n^4(\sigma + p\alpha + \gamma) \right ).
\]
Hence, the total cost of the algorithm is
\begin{align*}
  T_{\rm row}(\sigma,\bgamma,\balpha,q)= \gamma_1 \cdots\gamma_s \mathscr{T} +
  \softO\left (  {q \choose p} n^4 \rc  \re ( \sigma + p \alpha +\gamma )  \right ),  
\end{align*}
with $\mathscr{T}$ as in~\eqref{eqdef:T}. Since $\re \ge 2^n$ (because
$\alpha_i\geq 1$ and $\gamma_i\geq 1$ by assumption), this becomes
\begin{align}\label{eq:recT}
  T_{\rm row}(\sigma,\bgamma,\balpha,q)= \gamma_1 \cdots\gamma_s \mathscr{T} +
  \softO\left (  {q \choose p}  \rc  \re ( \sigma + p \alpha +\gamma )  \right ).
\end{align}

This will now allow us to give an estimate on $T_{\bP,{\rm row}}$ by
solving a few recurrence relations. Recall that $T_{\bP,{\rm row}}$ describes
the case where $s=0$, so that $\gamma_1 \cdots \gamma_s=1$, and $\bP$
is a $p \times q$ input matrix as in~\eqref{eqdef:type2aff}. In this
case, we can take $\sigma=O((q-p) q(\alpha_1 + \cdots + \alpha_p)) \in
O((q-p)pq\alpha)$; following our convention in the previous section,
the runtime $T_{\rm row}(\sigma,(),(\alpha_1,\dots,\alpha_p),q)$ is
then written $T_{\bP,{\rm row}}((\alpha_1,\dots,\alpha_p),q)$.

\begin{lemma}\label{lemma:TMrow}
  One can take
  $$T_{\bP,{\rm row}}((\alpha_1,\dots,\alpha_p),q)= \softO\left ({q
    \choose p} S_{q-p+1}(\alpha_1,\dots,\alpha_p) S_{q-p+1}(\alpha_1+1,\dots,\alpha_p+1) pq \alpha \right), $$ with
 $\alpha =\max(\alpha_1,\dots,\alpha_p)$.
\end{lemma}
\begin{proof}
  Taking into account that $\gamma=1$, Equation~\eqref{eq:recT},
  combined with the definition of $\mathscr{T}$ in~\eqref{eqdef:T}, gives the
  recursion
  \begin{multline}
  T_{\bP,{\rm row}}((\alpha_1,\dots,\alpha_p),q)=
\sum_{\substack{\bi=(i_1,\dots,i_\kappa)\\ \kappa\le\min(q-p,p)}}
\alpha_{i_1} \cdots \alpha_{i_\kappa} T_{\bP,{\rm row}}((\alpha_{i_1},\dots,\alpha_{i_\kappa}), q-p)
\\ + \softO\Big(
 {q \choose p} S_{q-p+1}(\alpha_1,\dots,\alpha_p) S_{q-p+1}(\alpha_1+1,\dots,\alpha_p+1) p q  \alpha )
\Big);
  \end{multline}
notice that 
a factor $(q-p)$ disappeared from the last term, since
it can be absorbed in the logarithmic factors in the $\softO(\ )$.
Let us rewrite the second summand as 
$S_{q-p+1}(\alpha_1,\dots,\alpha_p) C((\alpha_1,\dots,\alpha_p),q)$,
with
\[
C((\alpha_1,\dots,\alpha_p),q)= \softO\Big(
 {q \choose p} S_{q-p+1}(\alpha_1+1,\dots,\alpha_p+1) p q  \alpha 
\Big).
\]
This term is at its maximum at
  the root of the recursion tree.  Thus, we can find an upper bound on
  $T_{\bP,{\rm row}}$ by finding a solution to the recurrence
  \begin{align}\label{rec:TrowC}
  T_{\bP,{\rm row}}((\alpha_1,\dots,\alpha_p),q)&=
\sum_{\substack{\bi=(i_1,\dots,i_\kappa)\\ \kappa\le\min(q-p,p)}}
\alpha_{i_1} \cdots \alpha_{i_\kappa} T_{\bP,{\rm row}}((\alpha_{i_1},\dots,\alpha_{i_\kappa}), q-p)\\
&+ S_{q-p+1}(\alpha_1,\dots,\alpha_p) K,
  \end{align}
  for some constant $K$, and replacing $K$ by $
 \softO\Big(
 {q \choose p} S_{q-p+1}(\alpha_1+1,\dots,\alpha_p+1) p q  \alpha 
\Big). $  Now, a quick induction shows
that the   solution of~\eqref{rec:TrowC} satisfies 
$$T_{\bP,{\rm row}} \le   (q-p+1) S_{q-p+1} 
(\alpha_1,\dots,\alpha_p)K,$$  and the conclusion follows.
\end{proof}

We can then take the expression given in this lemma, and combine it
with the definition of $\mathscr{T}$ given in~\eqref{eqdef:T}. Using  the fact that
$n-s-1 = q-p$, we have
\begin{align*}
\mathscr{T}&=\sum_{\substack{\bi=(i_1,\dots,i_\kappa)\\ \kappa\le\min(q-p,p)}}
\alpha_{i_1} \cdots \alpha_{i_\kappa} T_{\bP,{\rm row}}((\alpha_{i_1},\dots,\alpha_{i_\kappa}), q-p)
+
\softO( S_{q-p+1}(\alpha_1,\dots,\alpha_p) n^3).
\end{align*}
Using the previous lemma,
we obtain that a term such as $T_{\bP,{\rm row}}((\alpha_{i_1},\dots,\alpha_{i_\kappa}),q-p)$
is
$$
 \softO\left ({q-p \choose \kappa} 
S_{q-p+1-\kappa}(\alpha_{i_1},\dots,\alpha_{i_\kappa}) 
S_{q-p+1-\kappa}(\alpha_{i_1}+1,\dots,\alpha_{i_\kappa}+1) \kappa (q-p) \alpha \right ).$$
As in the proof of the previous lemma,  we rewrite this expression by
factoring out the first complete function, as
$S_{q-p+1-\kappa}(\alpha_{i_1},\dots,\alpha_{i_\kappa}) D(\alpha_{i_1},\dots,\alpha_{i_\kappa},p,q),$ with
$$
D(\alpha_{i_1},\dots,\alpha_{i_\kappa},p,q)= \softO\left ({q-p \choose \kappa} 
 S_{q-p+1-\kappa}(\alpha_{i_1}+1,\dots,\alpha_{i_\kappa}+1) \kappa (q-p) \alpha \right ).$$
Now, we use the fact that (for the values of $\kappa$ that show up
in the sum), we have
\begin{align*}
{q-p \choose \kappa} & \le {q \choose p}\\
S_{q-p+1-\kappa}(\alpha_{i_1}+1,\dots,\alpha_{i_\kappa}+1) & \le S_{q-p+1}(\alpha_{1}+1,\dots,\alpha_{p}+1).
\end{align*}
Thus, 
$$
D(\alpha_{i_1},\dots,\alpha_{i_\kappa},p,q)= \softO\left ( 
{q \choose p}  S_{q-p+1}(\alpha_{1}+1,\dots,\alpha_{p}+1) p(q-p)\alpha\right ),$$
independently of the choice of $\alpha_{i_1},\dots,\alpha_{i_\kappa}$.
The sum in the definition of $\mathscr{T}$ becomes
\[
\left(\sum_{\substack{\bi=(i_1,\dots,i_\kappa)\\ \kappa\le\min(q-p,p)}}
\alpha_{i_1} \cdots \alpha_{i_\kappa} 
S_{q-p+1-\kappa}(\alpha_{i_1},\dots,\alpha_{i_\kappa})\right)
\softO\left ( 
{q \choose p}  S_{q-p+1}(\alpha_{1}+1,\dots,\alpha_{p}+1) p(q-p)\alpha\right ),
\]
or equivalently
\[
\softO\left ( 
{q \choose p}  S_{q-p+1}(\alpha_{1},\dots,\alpha_{p}) S_{q-p+1}(\alpha_{1}+1,\dots,\alpha_{p}+1)p(q-p)\alpha \right ).
\]
The value of $\mathscr{T}$ we infer from this is
\[
\softO\left ( 
{q \choose p}  S_{q-p+1}(\alpha_{1},\dots,\alpha_{p}) S_{q-p+1}(\alpha_{1}+1,\dots,\alpha_{p}+1)p(q-p)\alpha 
+  S_{q-p+1}(\alpha_{1},\dots,\alpha_{p}) n^3\right ).
\]
We inject this value in the runtime analysis~\eqref{eq:recT}. Terms
such as $(q-p)$ or $n^3$ are poly-logarithmic in $\re$; removing them,
the first-hand term $\gamma_1 \cdots \gamma_s \mathscr{T}$ in~\eqref{eq:recT} is
then bounded above by the second one, so that the runtime is simply
\begin{align}\label{eq:recTT}
  T_{\rm row}(\sigma,\bgamma,\balpha,q)=  \softO\left (  {q \choose p}  \rc  \re ( \sigma + p \alpha +\gamma )  \right ).
\end{align}
This establishes Proposition~\ref{prop:rowdegree_simple}.

Algorithm $\mathsf{RowDegree}$ is similar to
$\mathsf{RowDegree\_simple}$: the only difference consists in calling
Algorithm $\mathsf{Homotopy}$ from
Proposition~\ref{prop:compute_isolated} at the last step
\eqref{step:final:rowdegreesimple}, instead of
$\mathsf{Homotopy\_simple}$. The cost of Algorithm
$\mathsf{Homotopy}$ is $\softO(\rc^5 m n^2 + \rc(\re+\rc^5) n(\sigma'
+ n^3))$.  Using the facts that $\sigma'= \sigma +O( {q \choose p} n^3
+ n^2 p \alpha+ n^2\gamma)$, and that $n$ is in $\softO(\re)$, we
rewrite this as $\softO(\rc^5 m n^2 + \rc(\re+\rc^5 ) {q \choose
  p}(\sigma + p \alpha+ \gamma))$.  Then, we use the inequality $m \le
{q \choose p} n$, which gives $\rc^5 m n^2 \le (\re+\rc^5) {q \choose
  p} n^3$; hence the first term can be neglected, and the runtime of
$\mathsf{Homotopy}$ is thus
\[
\softO\left( \rc(\re+\rc^5 ) {q \choose p}(\sigma  +  p \alpha+ \gamma)\right).
\]
The costs of all other steps are the same as those in
$\mathsf{RowDegree\_simple}$, and the analysis above shows that can be
neglected. As a result, the bound given above holds for the whole
algorithm, and Proposition~\ref{prop:rowdegree} is proved.

\paragraph*{Acknowledgements.} J.D. Hauenstein is supported by Sloan Research
Fellowship BR2014-110 TR14 and NSF grant ACI-1460032. \'E. Schost is supported
by an NSERC Discovery Grant. Mohab Safey El Din and Huu Phuoc Le are supported
by the ANR grants ANR-18-CE33-0011 \textsc{Sesame}, and ANR-19-CE40-0018
\textsc{De Rerum Natura}, the joint ANR-FWF ANR-19-CE48-0015 \textsc{ECARP}
project, the PGMO grant \textsc{CAMiSAdo} and the European Union’s Horizon 2020
research and innovation programme under the Marie Skłodowska-Curie grant
agreement N° 813211 (POEMA). T.X. Vu was supported by a labex CalsimLab
fellowship/scholarship. The labex CalsimLab, reference ANR-11-LABX-0037-01, is
funded by the program ''Investissements d'avenir'' of the Agence Nationale de la
Recherche, reference ANR-11-IDEX-0004-02.

\vspace{-0.5cm}


%%%%%%%%%%%%%%%%%%%%%%%%%%%%%%%%%%%%%%%%%%%%%%%%%%%%%%%%%%%%
%%%%%%%%%%%%%%%%%%%%%%%%%%%%%%%%%%%%%%%%%%%%%%%%%%%%%%%%%%%%
%%%%%%%%%%%%%%%%%%%%%%%%%%%%%%%%%%%%%%%%%%%%%%%%%%%%%%%%%%%%

\section*{Appendix}\label{appendix}

We finally prove properties that were stated without proof in
Section~\ref{sec:prel-row}.  Recall that we work with two families of
matrices of size $p \times q$, namely
\begin{align}\label{eqdef:type2b}
\bP^H= \left( \begin{matrix}
\lambda^H_{1,1} & \lambda^H_{1,2} & \cdots & \lambda^H_{1, q}\\
 \lambda^H_{2,1} &  \lambda^H_{2,2} & \cdots & \lambda^H_{2, q}\\
 \vdots & & & \vdots\\
 \lambda^H_{p,1} &  \lambda^H_{p,2}& \cdots & \lambda^H_{p, q}
\end{matrix} \right)
\end{align}
and
\begin{align}\label{eqdef:type1b}
\bL^H= \left( \begin{matrix}
\lambda^H_{1,1} & 0 & \cdots & 0 & \lambda^H_{1,p+1} & \cdots & \lambda^H_{1, q}\\
0 & \lambda^H_{2,2} & \cdots & 0 & \lambda^H_{2,p+1} & \cdots & \lambda^H_{2, q}\\
\vdots & \vdots & \ddots & \vdots & \vdots & \ddots & \vdots\\
0 & 0 & \cdots & \lambda^H_{p,p} & \lambda^H_{p,p+1} & \cdots & \lambda^H_{p, q}
\end{matrix} \right),
\end{align}
each $\lambda^H_{i,j}$ being the product of $\alpha_i$ homogeneous
linear forms in $n+1$ variables $X_0,\dots,X_n$, that is,
$\lambda^H_{i,j}=\prod_{k=1}^{\alpha_i} \lambda^H_{i,j,k}$.

We are interested in describing the projective algebraic sets
$V_p(\bP^H)$ and $V_p(\bL^H)$ defined in $\P^n(\KKbar)$ by the
$p$-minors of $\bP^H$ and $\bL^H$. More generally, in the rest of this
section, if $\bA^H$ is a matrix with polynomial entries that are
homogeneous in $X_0,\dots,X_n$, we use the notation $V_t(\bA^H)$ to
denote the projective set defined by its $t$-minors in $\P^n(\KKbar)$,
for any $t\ge 1$. Then, the results we have to prove are the following
(see Proposition~\ref{lemma:appendix}).
\begin{propositionnonumber}\label{lemma:appendix2}
  For generic choices of the coefficients of the linear forms
  $\lambda^H_{i,j,k}$, the following holds:
  \begin{itemize}
  \item the projective algebraic sets $V_p(\bP^H)$ and $V_p(\bL^H)$
    have no solution at infinity (that is, with $X_0=0$);
  \item the Jacobian matrices of the sets of $p$-minors 
    of $\bP^H$, resp.\ of $\bL^H$, has rank $n$ at every point 
    in $V_p(\bP^H)$, resp.\ $V_p(\bL^H)$.
\end{itemize}
\end{propositionnonumber}

Our strategy is to work all along with linear forms with indeterminate
coefficients, and establish the properties we want in this context.
Let thus ${\cal A}=q(n+1)(\alpha_1 + \cdots + \alpha_p)$; this is the
number of coefficients needed to define homogeneous linear forms
$\lambda^H_{i,j,k}$ in $X_0,\dots,X_n$, for $i=1,\dots,p$,
$j=1,\dots,q$ and $k=1,\dots,\alpha_i$. If needed, we will write
${\cal A}={\cal A}(\balpha,q)$ to make the dependency in $\balpha$ and
$q$ explicit.  Let then $\mathfrak{Q}$ be the sequence of ${\cal A}$
indeterminates $\mathfrak{Q}=(\mathfrak{l}_{i,j,k,r})$, for $i,j,k$ as
above and $r=0,\dots,n$, and define
$$\mathfrak{l}^H_{i,j,k} = \mathfrak{l}_{i,j,k,0}X_0 + \mathfrak{l}_{i,j,k,1} X_1 +\cdots + \mathfrak{l}_{i,j,k,n} X_n,$$
as well as 
$$\mathfrak{l}^H_{i,j} = \mathfrak{l}^H_{i,j,1} \cdots \mathfrak{l}^H_{i,j,\alpha_i} \in \KK[\mathfrak{Q}][\tilde\bX],$$
with $\tilde\bX=(X_0,X_1,\dots,X_n)$. We can then define the
matrix
\begin{align}\label{eq:matM}
\mathfrak{P}^H_{\balpha,q}=\left [\begin{matrix}
\mathfrak{l}^H_{1,1} & \cdots & \mathfrak{l}^H_{1,q}\\
 \vdots & & \vdots\\
\mathfrak{l}^H_{p,1} & \cdots & \mathfrak{l}^H_{p,q}
  \end{matrix}\right ]\in \KK[\mathfrak{Q}][\tilde\bX]^{p\times q}.  
\end{align}
Remark that for all $i,j$, the $(i,j)$-th entry of
$\mathfrak{P}^H_{\balpha,q}$ has degree $\alpha_i$ in $\tilde \bX$;
this matrix is thus the ``generic'' model of the matrix $\bP^H$ seen previously.

Given $\Lambda=(\lambda_{i,j,k,r})\in \KKbar{}^{\cal A}$, for any polynomial
$\mathfrak{F}$ in $\KK(\mathfrak{Q})[\tilde \bX]$, we write
$\mathfrak{F}(\Lambda,\tilde\bX)$ for the polynomial obtained by
evaluating $\mathfrak{l}_{i,j,k,r}$ at $\lambda_{i,j,k,r}$, for all
indices $i,j,k,r$ as above, as long as no denominator vanishes through
this evaluation; the notation extends to polynomial matrices. More
generally, for a field $\LL$ containing $\KK$, and $\Lambda$ in $\LL^{\cal A}$, the
notation $\mathfrak{F}(\Lambda,\tilde\bX)$ is defined similarly.

Let next ${\cal A}'=n(n+1)(\alpha_1+\cdots+\alpha_p)$; as above, 
we will write ${\cal A}'={\cal A}'(\balpha,q)$ when needed. Let
$\mathfrak{Q}'\subset \mathfrak{Q}$ be the sequence of ${\cal A}'$
indeterminates $\mathfrak{Q}'=(\mathfrak{l}_{i,j,k,r})$, for indices
$i,j,k,r$ as follows: $i$ is in $\{1,\dots,p\}$, $j$ is in
$\{i,p+1,\dots,q\}$, and as previously, $k$ is in
$\{1,\dots,\alpha_i\}$ and $r$ is in $\{0,\dots,n\}$. Remark that the
polynomials $\mathfrak{l}^H_{i,j}$, for $i,j$ as above, are in
$\KK[\mathfrak{Q}'][\tilde\bX] \subset \KK[\mathfrak{Q}][\tilde \bX]$,
and allow us to define
\begin{align}\label{eq:matMprime}
\mathfrak{L}^H_{\balpha,q}=\left [\begin{matrix} \mathfrak{l}^H_{1,1} & 0 & 0
    &\mathfrak{l}^H_{1,p+1} & \cdots & \mathfrak{l}^H_{1,q}\\ \vdots & \ddots &
    \vdots & \vdots & & \vdots\\ 0&0& \mathfrak{l}^H_{p,p}
    &\mathfrak{l}^H_{p,p+1} & \cdots & \mathfrak{l}^H_{p,q}
  \end{matrix}\right ]\in \KK[\mathfrak{Q}'][\tilde\bX]^{p\times q}.
\end{align}
For $\Lambda' \in \KKbar{}^{{\cal A}'}$ and $\mathfrak{F}$ in 
$\KK(\mathfrak{Q}')[\tilde\bX]$, the notation
$\mathfrak{F}(\Lambda',\tilde\bX)$ is defined as in the case of
polynomials over $\KK(\mathfrak{Q})$ described previously.


\subsection{Setting up the recurrences}
The basic idea behind the proofs below is the following: to prove that
a property such as rank-deficiency holds for a matrix
$\mathfrak{P}^H_{\balpha,q}$, we prove that it holds for a matrix of
the form $\mathfrak{L}^H_{\balpha,q}$, and use an openness
property. To prove that property for the latter matrices, we proceed
by induction, relying on the presence of the left-hand diagonal
block. Indeed, for a matrix such as $\mathfrak{L}^H_{\balpha,q}$ to be
rank-deficient at $\tilde\bx \in \P^n(\KKGpbar)$, at least one of
$\mathfrak{l}^H_{1,1},\dots,\mathfrak{l}^H_{p,p}$ must vanish at
$\tilde\bx$.

Suppose for instance that
$\mathfrak{l}^H_{1,1}(\tilde\bx)=\mathfrak{l}^H_{2,2}(\tilde\bx)=0$,
while all other terms are non-zero. Then, the
$((1,2),(p+1,\dots,q))$-submatrix of
${\mathfrak{L}^H_{\balpha,q}}(\tilde\bx)$ itself must be
rank-deficient.  The constraints
$\mathfrak{l}^H_{1,1}(\tilde\bx)=\mathfrak{l}^H_{2,2}(\tilde\bx)=0$
give us two linear equations, which allow us to eliminate two
coordinates of $\tilde\bx$, say $X_{n-1}$ and $X_n$. We can perform
the corresponding substitution in the above submatrix, and we are left
with a matrix of size $2 \times (n-1)$ that is of the form
$\mathfrak{P}^H_{(\alpha_1,\alpha_2),n-1}(\mathfrak{H},(X_0,\dots,X_{n-2}))$,
with entries depending on $X_0,\dots,X_{n-2}$, for some vector of
coefficients $\mathfrak{H}$ obtained through the elimination of
$X_{n-1}$ and $X_n$. We can then invoke our induction assumption on
the latter matrix.

To formalize this process, recall that for a subsequence
$\bi=(i_1,\dots,i_\kappa)$ of $(1,\dots,p)$, we call the
$(\bi,(p+1,\dots,q))$-submatrix of $\mathfrak{L}^H_{\balpha,q}$ the submatrix of
$\mathfrak{L}^H_{\balpha,q}$ {\em associated} to $\bi$; it consists of the rows of
$\mathfrak{L}^H_{\balpha,q}$ indexed by $\bi$ and columns $p+1,\dots,q$.
For such an $\bi$, we let $R_\bi$ be the set of
all tuples $\br=(r_1,\dots,r_\kappa)$, with $r_1$ in
$\{1,\dots,\alpha_{i_1}\}$, \dots, $r_\kappa$ in
$\{1,\dots,\alpha_{i_\kappa}\}$; for any $k$ in $\{1,\dots,\kappa\}$,
$r_k$ will be the index of the factor $\mathfrak{l}^H_{i_k,i_k,r_k}$
of $\mathfrak{l}^H_{i_k,i_k}$ we cancel. For given $\bi$ and $\br$,
we will let $\mathfrak{Q}'_{\bi,\br} \subset \mathfrak{Q}'$ be the
indeterminates corresponding to the coefficients of
$\mathfrak{l}^H_{i_1,i_1,r_1},\dots,\mathfrak{l}^H_{i_\kappa,i_\kappa,r_\kappa}$, and
of all entries $\mathfrak{l}^H_{i_1,p+1},\dots,\mathfrak{l}^H_{i_\kappa,q}$
of the submatrix associated to $\bi$ in $\mathfrak{L}^H_{\balpha,q}$.

By Gaussian elimination, we can rewrite the homogeneous linear
equations
$\mathfrak{l}^H_{i_1,i_1,r_1}=\dots=\mathfrak{l}^H_{i_\kappa,i_\kappa,r_\kappa}=0$ as
\begin{align}\label{eq:f_sr}
X_{n-\kappa+1}=\mathfrak{f}_{n-\kappa+1,\bi,\br}(X_0,\dots,X_{n-\kappa}),\dots,X_{n}=\mathfrak{f}_{n,\bi,\br}(X_0,\dots,X_{n-\kappa}),  
\end{align}
for some homogeneous linear forms
$\mathfrak{f}_{n-\kappa+1,\bi,\br},\dots,\mathfrak{f}_{n,\bi,\br}$ of
$(X_0,\dots,X_{n-\kappa})$ with coefficients in
$\KK(\mathfrak{Q}'_{\bi,\br})$. Applying this substitution in the
entries of the submatrix of ${\mathfrak{L}^H_{\balpha,q}}$ associated
to $\bi$ gives us the $\kappa \times (n-1)$ matrix
$\mathfrak{P}^H_{\balpha_\bi,n-1}(\mathfrak{H}_{\bi,\br},\tilde\bX')$,
with $\balpha_\bi=(\alpha_{i_1},\dots,\alpha_{i_\kappa})$, whose
entries are products of homogeneous linear forms in
$\tilde\bX'=(X_0,\dots,X_{n-\kappa})$, and where $\mathfrak{H}_{\bi,\br}$ is a
vector of ${\cal A}(\balpha_\bi,n-1)$ elements in $\KK(\mathfrak{Q}'_{\bi,\br})$.

The main result we will use in this section is the following lemma,
which summarizes how the above process allows us to describe the
projective zero-set of $t$-minors of $\mathfrak{L}^H_{\balpha,q}$, for
any $t \le p$. This will be the basis of several recursions.
\begin{lemma}\label{lemma:union}
  For $t$ in $\{1,\dots,p\}$, $V_t(\mathfrak{L}^H_{\balpha,q}) \subset \P^{n}(\KKGpbar)$ is the
  union of the sets
 \begin{align}\label{eq:union}
 \left \{(\tilde\bx',\mathfrak{f}_{n-\kappa+1,\bi,\br}(\tilde\bx'),\dots,\mathfrak{f}_{n,\bi,\br}(\tilde\bx')) \mid \tilde\bx' \in
  V_{\kappa-(p-t)}(\mathfrak{P}^H_{\balpha_\bi,n-1}(\mathfrak{H}_{\bi,\br},\tilde\bX')) \subset \P^{n-\kappa}(\KKGpbar)\right \},   
 \end{align}
 for $\bi=(i_1,\dots,i_\kappa)$ of length $\kappa \in \{p-t+1,\dots,\min(p,n-1)\}$ and $\br$ in $R_\bi$,
 and with $\tilde\bX'=(X_0,\dots,X_{n-\kappa})$, together with
 $$\left \{
 (1,\mathfrak{f}_{1,\bi,\br}(1),\dots,\mathfrak{f}_{n,\bi,\br}(1))\right
 \}$$ if $t=p$ and $n \le p$, with $\bi=(i_1,\dots,i_n)$ and $\br$ in $R_\bi$.
\end{lemma}
\noindent We have to write a special case for $t=p$ and $n \le p$ in the last part of the lemma,
since taking $\bi=(i_1,\dots,i_n)$ of length $\kappa=n$
in~\eqref{eq:union} would lead us to consider points in $\P^0(\KKGpbar)$.
\begin{proof}
  A point $\tilde\bx \in \P^n(\KKGpbar)$ belongs to
  $V_t(\mathfrak{L}^H_{\balpha,q})$ if and only if some diagonal terms
  of $\mathfrak{L}^H_{\balpha,q}$ vanish at $\tilde\bx$, say
  $\mathfrak{l}^H_{i_k,i_k}(\tilde\bx)=0$ for $k=1,\dots,\kappa$ (all
  other $\mathfrak{l}^H_{i,i}(\tilde\bx)$ being non-zero), and
  if the submatrix of $\mathfrak{L}^H_{\balpha,q}$ associated to
  $\bi=(i_1,\dots,i_\kappa)$ has rank less than $\kappa-(p-t)$ at $\tilde
  \bx$.  In particular, we must have
  $\kappa-(p-t) > 0$, that is, $\kappa \ge p-t+1$.

  For $k=1,\dots,\kappa$, $\mathfrak{l}^H_{i_k,i_k}(\tilde\bx)=0$ if
  and only if there exists $r_k$ in $\{1,\dots,\alpha_{i_k}\}$ such
  that $\mathfrak{l}^H_{i_k,i_k,r_k}(\tilde\bx)=0$. Thus, $\tilde\bx$
  is in $V_t(\mathfrak{L}^H_{\balpha,q})$ if and only if there exists
  a subsequence $\bi=(i_1,\dots,i_\kappa)$  of $(1,\dots,p)$, with
  $\kappa \ge p-t+1$, and $\br=(r_1,\dots,r_\kappa)$ in $R_\bi$ such that
  $\mathfrak{l}^H_{i_1,i_1,r_1}(\tilde\bx)=\cdots=\mathfrak{l}^H_{i_\kappa,i_\kappa,r_\kappa}(\tilde\bx)=0$
  and the submatrix of $\mathfrak{L}^H_{\balpha,q}$ associated to $\bi$
  has rank less than $\kappa-(p-t)$ at $\tilde\bx$.  

  Applying~\eqref{eq:f_sr}, we deduce that the coordinates $(x_0,\dots,x_n)$ 
  of $\tilde \bx$ satisfy
  \begin{align*}
    x_{n-\kappa+1}=\mathfrak{f}_{n-\kappa+1,\bi,\br}(\tilde\bx'),\dots,x_{n}=\mathfrak{f}_{n,\bi,\br}(\tilde\bx'),
  \end{align*}
  with $\tilde\bx'=(x_0,\dots,x_{n-\kappa})$.  In particular, $\kappa
  \le n$, since otherwise this linear system would have no solution
  (recall that the coefficients are algebraically independent
  indeterminates). Remark also that $\tilde\bx'$ is a well-defined
  element of $\P^{n-\kappa}(\KKGpbar)$, that is, it is not identically
  zero, since otherwise $\tilde\bx$ would vanish as well.

  For $\bi=(i_1,\dots,i_\kappa)$ with $\kappa \le n-1$, applying the above
  substitution in the submatrix of $\mathfrak{L}^H_{\balpha,q}$
  associated to $\bi$ (which has size $\kappa \times (n-1)$), the rank
  condition above becomes that
  $\mathfrak{P}^H_{\balpha_\bi,n-1}(\mathfrak{H}_{\bi,\br},\tilde\bX')$
  has rank less than $\kappa-(p-t)$ at $\tilde \bx'$, that is, $\tilde
  \bx'$ is in
  $V_{\kappa-(p-t)}(\mathfrak{P}^H_{\balpha_\bi,n-1}(\mathfrak{H}_{\bi,\br},\tilde\bX'))$.
  In this case, we are done.

  When $\kappa=n$, that is, $\bi=(i_1,\dots,i_n)$ (this can happen
  only if $n \le p$), the linear equations above determine $\tilde\bx$
  entirely; setting $x_0=1$, we obtain
  $x_{1}=\mathfrak{f}_{1,\bi,\br}(1),\dots,x_{n}=\mathfrak{f}_{n,\bi,\br}(1).$
  In this case, the submatrix of $\mathfrak{L}^H_{\balpha,q}$
  associated to $\bi$ has size $n \times (n-1)$. Using the
  specialization of the coefficients that sets the off-diagonal entry
  to $0$ and the $i$th diagonal entries to $X_0^{\alpha_i}$,
  $i=1,\dots,n-1$, we see that its evaluation at $\tilde\bx$ has rank
  $n-1$; as a result $\mathfrak{L}^H_{\balpha,q}$ has rank $p-1$ at
  $\tilde \bx$. Thus, we need to take $\kappa=n$ into account only if
  $t=p$, that is, if we are interested in the maximal minors; in this
  case, we have to take into account the point $\left \{
  (1,\mathfrak{f}_{1,\bi,\br}(1),\dots,\mathfrak{f}_{n,\bi,\br}(1))\right
  \}$.
\end{proof}

%%%%%%%%%%%%%%%%%%%%%%%%%%%%%%%%%%%%%%%%%%%%%%%%%%%%%%%%%%%%

\subsection{Solutions with higher rank defect} 

We discuss here the case $t=p-1$.  We take parameters
$\balpha=(\alpha_1,\dots,\alpha_p)$ and $q$, with $2 \le p \le q$, and
we write ${\cal A}={\cal A}(\balpha,q)$ and ${\cal A}'={\cal
  A}'(\balpha,q)$.  The rest of this subsection is dedicated to
proving the following claims.

\begin{lemma} The following holds:
\begin{description}[leftmargin=*]
\item[$\assI_1(\balpha,q).$] The projective algebraic set
  $V_{p-1}(\mathfrak{P}^H_{\balpha,q})\subset\P^n(\KKGbar)$ is empty.
\item[$\assJ_1(\balpha,q).$] The projective algebraic set
  $V_{p-1}(\mathfrak{L}^H_{\balpha,q})\subset\P^n(\KKGpbar)$ is empty.
\end{description}
\end{lemma}


The first step of the proof is to establish that for $\balpha$ and $q$
as above, $\assJ_1(\balpha,q)$ implies $\assI_1(\balpha,q)$. Consider
the ideal generated by the $(p-1)$-minors of
$\mathfrak{P}^H_{\balpha,q}$ in the polynomial ring
$\KK[\mathfrak{Q},\tilde\bX]$ in ${\cal A}+n+1$ variables. This ideal
defines an algebraic set $Z_{\balpha,q}$ in $\KKbar{}^{\cal A} \times
\P^n(\KKbar)$, and we let $\Delta_{\balpha,q} \subset \KKbar{}^{\cal
  A}$ be its projection on the first factor: this is the set of all
$\Lambda$ such that
$V_{p-1}(\mathfrak{P}^H_{\balpha,q}(\Lambda,\tilde\bX))$ is not
empty. Because the source is a projective space, $\Delta_{\balpha,q}$
is closed (so its complement is open), and we just have to verify that
it is not equal to the whole $\KKbar{}^{\cal A}$. This follows readily
from property $\assJ_1(\balpha,q)$, which proves that generic matrices
of the form $\mathfrak{L}^H_{\balpha,q}(\Lambda',\tilde\bX)$ do not
belong to $\Delta_{\balpha,q}$. Thus, if $\assJ_1(\balpha,q)$ holds, 
$\assI_1(\balpha,q)$ does.

We finish the proof by induction. We first take $p=q$ and consider
$\assJ_1(\balpha,q)$.  In this case, $n=1$ and
$\mathfrak{L}^H_{\balpha,q}$ is a diagonal matrix, whose diagonal
entries are products of linear forms in $(X_0,X_1)$ with indeterminate
coefficients. Hence, no pair of entries $\mathfrak{L}^H_{\balpha,q}$
have any common solution in $\P^1(\KKGpbar)$, so the rank of
$\mathfrak{L}^H_{\balpha,q}$ is at least $p-1$ at any $\tilde\bx \in
\P^1(\KKGpbar)$. As a result, $\assJ_1(\balpha,p)$ holds, and so does
$\assI_1(\balpha,p)$, by the claim in the previous paragraph.

Consider next a pair $(\balpha,q)$, with
$\balpha=(\alpha_1,\dots,\alpha_p)$ and $2 \le p < q$, and suppose
that $\assI_1(\balpha',q')$ holds for all $(\balpha',q')$ with
$\balpha'=(\alpha'_1,\dots,\alpha'_{p'})$, $2 \le p' \le q'$, $p' \le p$ and $q'
< q$; we prove that $\assJ_1(\balpha,q)$ holds (as above, this will
also imply $\assI_1(\balpha,q)$).

Take $t=p-1$ in Lemma~\ref{lemma:union}. Then, the parameters
$(\kappa-(p-t),\balpha_\bi,n-1)$ used in each
expression~\eqref{eq:union} are of the form
$(\kappa-1,\balpha_\bi,n-1)$, with $2 \le \kappa \le \min(p,n-1)$.
Since the ${\cal A}(\balpha_\bi,n-1)$ entries of $\mathfrak{H}_{\bi,\br}$ are algebraically
independent over $\KK$, $\KK(\mathfrak{H}_{\bi,\br})$ is isomorphic to 
$\KK(\lambda_{u,j,k,r})$, for $u=1,\dots,\kappa$, 
$j=1,\dots,n-1$, $k=1,\dots,\alpha_{i_u}$ 
and $r=0,\dots,n-\kappa$, so that $V_{\kappa-1}(\mathfrak{P}^H_{\balpha_\bi,n-1}(\mathfrak{H}_{\bi,\br},\tilde\bX'))$
has the same cardinality as 
$V_{\kappa-1}(\mathfrak{P}^H_{\balpha_\bi,n-1})$.
As a result, since $\balpha_\bi$ has length $\kappa\ge 2$, and since we also
have $\kappa \le n-1$, $\kappa \le p$ and $n-1 < q$, we can apply the
induction hypothesis and deduce that all
$V_{\kappa-1}(\mathfrak{P}^H_{\balpha_\bi,n-1}(\mathfrak{H}_{\bi,\br},\tilde\bX'))$ appearing
in Lemma~\ref{lemma:union} are empty. This in turn implies that
$V_{p-1}(\mathfrak{L}^H_{\balpha,q})$ is empty, as claimed.


\subsection{Solutions at infinity} Next, we focus on the case $t=p$.
We take parameters $\balpha=(\alpha_1,\dots,\alpha_p)$ and $q$, with
$1 \le p \le q$, and we write ${\cal A}={\cal A}(\balpha,q)$ and
${\cal A}'={\cal A}'(\balpha,q)$.  The rest of this subsection is
dedicated to proving the following properties; note that they imply
the first item in Proposition~\ref{lemma:appendix}.

\begin{lemma}The following holds:
\begin{description}[leftmargin=*]
\item[$\assI_2(\balpha,q).$] The projective algebraic set
  $V_p(\mathfrak{P}^H_{\balpha,q}) \subset \P^n(\KKGbar)$ has no point
  satisfying $X_0=0$.
\item[$\assJ_2(\balpha,q).$] The projective algebraic set
  $V_p(\mathfrak{L}^H_{\balpha,q}) \subset \P^n(\KKGpbar)$ has no point
  satisfying $X_0=0$.
\end{description}
\end{lemma}
We will prove these properties as we did in the previous paragraph;
the first step is thus to establish that for $\balpha$ and $q$ as
above, $\assJ_2(\balpha,q)$ implies $\assI_2(\balpha,q)$.

Let us thus fix $\balpha$ and $q$, and assume that
$\assJ_2(\balpha,q)$ holds. We prove that
$V_p(\mathfrak{P}^H_{\balpha,q}(\Lambda,\tilde\bX))$ has no point at
infinity for a generic $\Lambda$ in $\KKbar{}^{\cal A}$; this will
imply $\assI_2(\balpha,q)$. Consider the ideal generated by the
$p$-minors or $\mathfrak{P}^H_{\balpha,q}$ and $X_0$ in the polynomial
ring $\KK[\mathfrak{Q},\tilde\bX]$ in ${\cal A}+n+1$ variables. This
ideal defines an algebraic set $Z'_{\balpha,q}$ in $\KKbar{}^{\cal A}
\times \P^n(\KKbar)$, and we let $\Delta'_{\balpha,q} \subset
\KKbar{}^{\cal A}$ be its projection on the first factor: this is thus
the set of all $\Lambda$ in $\KKbar{}^{\cal A}$ such that
$V_p(\mathfrak{P}^H_{\balpha,q}(\Lambda,\tilde\bX))$ has a point at
infinity. Because the source is a projective space,
$\Delta'_{\balpha,q}$ is closed (so its complement is open), and we
just have to verify that it is not equal to the whole $\KKbar{}^{\cal
  A}$. This follows from property $\assJ_2(\balpha,q)$, which implies
that matrices of the form
$\mathfrak{L}^H_{\balpha,q}(\Lambda',\tilde\bX)$, for generic
$\Lambda'$ in $\KKbar{}^{{\cal A}'}$, do not belong to
$\Delta'_{\balpha,q}$.

Again, we finish the proof by induction. We first take $p=q$, and we
prove that $\assJ_2(\balpha,q)$ holds ($\assI_2(\balpha,q)$ will follow,
by the previous paragraph). In this case, $n=1$ and
$\mathfrak{L}^H_{\balpha,q}$ is a diagonal matrix, whose diagonal entries
are products of homogeneous linear forms in $(X_0,X_1)$ with indeterminate
coefficients. Then, $\mathfrak{L}^H_{\balpha,q}$ has rank less than $p$ at
$\tilde\bx\in\P^1(\KKGpbar)$ if and only if one of the linear factors
of some diagonal term vanishes at $\tilde \bx$. None of these linear
forms has a projective root at infinity, so we are done.

Consider next a pair $(\balpha,q)$, with
$\balpha=(\alpha_1,\dots,\alpha_p)$ and $1 \le p \le q$ and suppose
that $\assI_2(\balpha',q')$ holds for all $(\balpha',q')$ with
$\balpha'=(\alpha'_1,\dots,\alpha'_{p'})$, $1 \le p' \le q'$, $p' \le p$ and $q'
< q$; we prove that $\assJ_2(\balpha,q)$ holds; as above, this will
imply $\assI_2(\balpha,q)$.

Take $t=p$ in Lemma~\ref{lemma:union}. We first deal with the last
contribution, corresponding to $\bi=(i_1,\dots,i_n)$, and thus
$\kappa=n$: by design, the corresponding point is not at infinity. For
the other contributions, the parameters $(\kappa-(p-t),\balpha_\bi,n-1)$
used in~\eqref{eq:union} are of the form $(\kappa,\balpha_\bi,n-1)$, with
$\balpha_\bi$ of length $\kappa \in \{1,\dots, \min(p,n-1)\}$; since all
conditions $1 \le \kappa \le n-1$, $\kappa \le p$ and $n-1 < q$ are satisfied,
we can invoke the induction assumption. Since the coefficients
$\mathfrak{H}_{\bi,\br}$ are algebraically independent, we deduce that
none of the projective sets
$V_\kappa(\mathfrak{P}^H_{\balpha_\bi,n-1}(\mathfrak{H}_{\bi,\br},\tilde\bX'))$ appearing in
Lemma~\ref{lemma:union} has any point with $X_0=0$. As a consequence,
$V_p(\mathfrak{L}^H_{\balpha,q})$ has no point at infinity either, as
claimed.

%%%%%%%%%%%%%%%%%%%%%%%%%%%%%%%%%%%%%%%%%%%%%%%%%%%%%%%%%%%%

\subsection{Refining $\assI_1$} 

%% The following is a strengthening of property $\assI_1$ above. That
%% property asserts that for any $\tilde\bx$ in $\P^n(\KKGbar)$, the $p
%% \times q$ matrix $\mathfrak{P}^H_{\balpha,q}(\tilde\bx)$ has rank at
%% least $p-1$, so that there exists a non-zero $(p-1)$-minor in this
%% matrix.  We claim that the following stronger property holds.
%% \begin{lemma}
%% All $(p-1)\times q$ submatrices of
%% $\mathfrak{P}^H_{\balpha,q}(\tilde\bx)$ have rank $p-1$.
%% \end{lemma}
%% The proof occupies the rest of this subsection.
%% To rephrase our claim, 

Consider $\balpha=(\alpha_1,\dots,\alpha_p)$ and
$q$, with $1 \le p \le q$, together with a matrix
$\mathfrak{p}^H_{\balpha,q}$, built as $\mathfrak{P}^H_{\balpha,q}$
before, but using products of homogeneous linear forms in $(n-1)+1=q-p+1$
variables $X_0,\dots,X_{n-1}$, instead of $n+1$ variables
$X_0,\dots,X_n$. Such a matrix takes the form
\begin{align}\label{eq:matM2}
\mathfrak{p}^H_{\balpha,q}=\left [\begin{matrix}
\mathfrak{g}^H_{1,1} & \cdots & \mathfrak{g}^H_{1,q}\\
 \vdots & & \vdots\\
\mathfrak{g}^H_{p,1} & \cdots & \mathfrak{g}^H_{p,q}
  \end{matrix}\right ]\in \KK[\mathfrak{G}][X_0,\dots,X_{n-1}]^{p\times q},
\end{align}
with 
$$\mathfrak{g}^H_{i,j,k} = \frak{g}_{i,j,k,0}X_0 + \frak{g}_{i,j,k,1} X_1 +\cdots + \frak{g}_{i,j,k,n-1} X_{n-1},$$
and
$$\mathfrak{g}^H_{i,j} = \mathfrak{g}^H_{i,j,1} \cdots
\mathfrak{g}^H_{i,j,\alpha_i} \in
\KK[\mathfrak{G}][X_0,\dots,X_{n-1}],$$ where
$\mathfrak{G}=(\mathfrak{g}_{i,j,k,\ell})$ are indeterminates, for
$i=1,\dots,p$, $j=1,\dots,q$, $k=1,\dots,\alpha_i$ and
$\ell=0,\dots,n-1$; we let ${\cal B}=q n (\alpha_1+\cdots +\alpha_p)$
be the total number of coefficients $\mathfrak{g}_{i,j,k,\ell}$
involved. The rest of this subsection establishes the following property.

\begin{lemma}
The following holds:
\begin{description}[leftmargin=*]
\item[$\assI_3(\balpha,q).$] The projective algebraic set
  $V_p(\mathfrak{p}^H_{\balpha,q}) \subset \P^{n-1}(\KKCbar)$ is empty.
\end{description}
\end{lemma}
To prove this property, take $\balpha=(\alpha_1,\dots,\alpha_p)$ and
$q$ as above. If $q=p$, we have $n=1$, so the $(i,j)$ entry of
$\mathfrak{p}^h_{\balpha,q}$ has the form
$\mathfrak{g}_{i,j,1,0}\cdots\mathfrak{g}_{i,j,\alpha_i,0}
X_0^{\alpha_i}$; hence, the determinant of this matrix is non-zero,
and the claim follows.

We can thus suppose $q > p$, so that $q-1 \ge p$.  Then, the
$((1,\dots,p),(1,\dots,q-1))$-submatrix of $\mathfrak{p}^H_{\balpha,q}$ is
of the form $\mathfrak{P}^H_{\balpha,q-1}$, with entries depending on
${\cal A}(\balpha,q-1)$ parameters.  Let $(c_i)_{i \in I}$ be the $p$-minors
of $\mathfrak{p}^H_{\balpha,q}$ built by taking $p-1$ of the first $q-1$
columns of $\mathfrak{p}^H_{\balpha,q}$, together with its last column.
Any such minor can be expanded along the last column as $c_i =
\mathfrak{g}^H_{1,q} c_{i,1} + \cdots + \mathfrak{g}^H_{p,q}
c_{i,p}$, where $\mathfrak{g}^H_{1,q},\dots,\mathfrak{g}^H_{p,q}$ are
the entries of the last column, and $c_{i,1},\dots,c_{i,p}$
are $(p-1)$-minors from $\mathfrak{P}^H_{\balpha,q-1}$. Remark that
$(c_{i,j})_{i \in I, 1 \le j \le p}$ are {\em all} $(p-1)$-minors
of $\mathfrak{P}^H_{\balpha,q-1}$ (if $p=1$, we have $I=\{1\}$ and
$c_1=\mathfrak{g}^H_{1,q}$, with $c_{1,1}=1$).

By $\assI_2(\balpha,q-1)$, we deduce that
$V_p(\mathfrak{P}^H_{\balpha,q-1}) \subset \P^{n-1}(\KKCbar)$ is
finite. For all other points $\tilde\bx$ in $\P^{n-1}(\KKCbar)$,
$\mathfrak{P}^H_{\balpha,q-1}$ has full rank $p$ at $\tilde\bx$, and
thus so does $\mathfrak{p}^H_{\balpha,q}$. Hence, we can focus on the
points in $V_p(\mathfrak{P}^H_{\balpha,q-1})$.  Consider a point
$\tilde\bx$ in this set; in particular, by $\assI_2(\balpha,q-1)$, we
can take its first coordinate $x_0$ equal to $1$. Using
$\assI_1(\balpha,q-1)$, together with our remark on the $(p-1)$-minors
of $\mathfrak{P}^H_{\balpha,q-1}$, we deduce that not all minors
$(c_{i,j})_{i \in I, 1 \le j \le p}$ vanish at $\tilde\bx$. Suppose
thus that $c_{i_0,j_0}(\tilde\bx) \ne 0$; we prove that
$c_{i_0}(\tilde\bx) \ne 0$, which is enough to conclude.

Let us split the ${\cal B}$ indeterminates $\mathfrak{G}$ into
$\mathfrak{G}_1$ and $\mathfrak{G}_2$, where $\mathfrak{G}_1$ has
cardinality ${\cal B}_1={\cal A}(\balpha,q-1)$ and corresponds to the coefficients
used in the entries
$\mathfrak{g}^H_{1,1},\dots,\mathfrak{g}^H_{p,q-1}$ in
$\mathfrak{P}^H_{\balpha,q}$, and $\mathfrak{G}_2$ of cardinality
${\cal B}_2={\cal B}-{\cal B}_1$ stands for the coefficients of the entries
$\mathfrak{g}^H_{1,q},\dots,\mathfrak{g}^H_{p,q}$ in the last column
of $\mathfrak{p}^H_{\balpha,q}$.  Let us further
write $$c_{i_0}(\tilde\bx)= \mathfrak{g}^H_{1,q}(\tilde\bx)
c_{i_0,1}(\tilde\bx) + \cdots + \mathfrak{g}^H_{p,q}(\tilde\bx)
c_{i_0,p}(\tilde\bx).$$ Since $V_p(\mathfrak{P}^H_{\balpha,q-1})$ is
finite, the coordinates of $\tilde\bx$ are algebraic over
$\KK(\mathfrak{G}_1)$.  Thus, since $x_0=1$, the polynomial
$\mathfrak{g}^H_{j_0,q}(\tilde\bx)\in
\overline{\KK(\mathfrak{G}_1)}[\mathfrak{G}_2]$ admits
$\mathfrak{g}_{j_0,q,1,0}\cdots \mathfrak{g}_{j_0,q,\alpha_{j_0},0}$ as a
specialization, by setting to zero all coefficients
$\mathfrak{g}_{j_0,q,k,\ell}$, for $k=1,\dots,\alpha_{j_0}$ and
$\ell=1,\dots,n-1$ (remark that these coefficients belong to $\mathfrak{G}_2$).  For $j \ne
j_0$, $\mathfrak{g}^H_{j,q}(\tilde\bx)\in
\overline{\KK(\mathfrak{G}_1)}[\mathfrak{G}_2]$ admits $0$ as a
specialization, by setting to zero all coefficients
$\mathfrak{g}_{j,q,k,\ell}$, for $k=1,\dots,\alpha_j$ and
$\ell=0,\dots,n-1$ (again, these coefficients belong to $\mathfrak{G}_2$).

The coefficients $c_{i_0,j}(\tilde\bx)$ are algebraic over
$\KK(\mathfrak{G}_1)$, so that $c_{i_0}(\tilde\bx)$ is in
$\overline{\KK(\mathfrak{G}_1)}[\mathfrak{G}_2]$. By the previous 
discussion, it admits
$$ \mathfrak{g}_{j_0,q,1,0}\cdots \mathfrak{g}_{j_0,q,\alpha_{j_0},0} c_{i_0,j_0}(\tilde\bx)$$ as a
specialization, which is non-zero. Thus,  $c_{i_0}(\tilde\bx)$ 
is non-zero, as claimed.

\subsection{Multiplicity of the solutions} 
The following is the last property we prove for matrices 
$\mathfrak{P}^H_{\balpha,q}$ and $\mathfrak{L}^H_{\balpha,q}$.
Again, we take parameters $\balpha=(\alpha_1,\dots,\alpha_p)$ and $q$,
with $1 \le p \le q$, and we write ${\cal A}={\cal A}(\balpha,q)$ and
${\cal A}'={\cal A}'(\balpha,q)$; we will establish the following
(note that this finishes the proof of Proposition~\ref{lemma:appendix}).
\begin{lemma}The following holds:
\begin{description}[leftmargin=*]
\item[$\assI_4(\balpha,q).$] The Jacobian matrix of
  the $p$-minors of $\mathfrak{P}^H_{\balpha,q}$ with respect to
  $\tilde\bX=(X_0,\dots,X_n)$ has rank $n$ at all points in
  $V_p(\mathfrak{P}^H_{\balpha,q})$.
\item[$\assJ_4(\balpha,q).$] The Jacobian matrix of
  the $p$-minors of $\mathfrak{L}^H_{\balpha,q}$ with respect to
  $\tilde\bX=(X_0,\dots,X_n)$ has rank $n$ at all points in
  $V_p(\mathfrak{L}^H_{\balpha,q})$.
\end{description}
\end{lemma}
As for other proofs involving both $\mathfrak{P}^H_{\balpha,q}$ and
$\mathfrak{L}^H_{\balpha,q}$, we first show that $\assJ_4(\balpha,q)$
implies $\assI_4(\balpha,q)$.

We fix $\balpha$ and $q$, and we assume that $\assJ_4(\balpha,q)$
holds. Consider the ideal of the polynomial ring
$\KK[\mathfrak{Q},\tilde\bX]$ in ${\cal A}+n+1$ variables generated by the
$p$-minors of $\mathfrak{P}^H_{\balpha,q}$, together with the
$n$-minors of the Jacobian matrix of these equations with respect to
$(X_0,\dots,X_n)$. This ideal defines an algebraic set
$Z''_{\balpha,q}$ in $\KKbar{}^{\cal A} \times \P^n(\KKbar)$, and we let
$\Delta''_{\balpha,q} \subset \KKbar{}^{\cal A}$ be its projection on the
first factor. By construction, for $\Lambda$ in
$\KKbar{}^{\cal A}-\Delta''_{\balpha,q}$, the Jacobian matrix of
$M_p(\mathfrak{P}^H_{\balpha,q}(\Lambda,\tilde\bX))$ has rank $n$ at
any $\tilde\bx$ in
$V_p(\mathfrak{P}^H_{\balpha,q}(\Lambda,\tilde\bX))$. As before,
because the source is a projective space, $\Delta''_{\balpha,q}$ is closed
(so its complement is open), and we just have to verify that it is not
equal to the whole $\KKbar{}^{\cal A}$. This follows from property
$\assJ_4(\balpha,q)$, which proves that generic matrices of the form
$\mathfrak{L}^H_{\balpha,q}$ do not belong to $\Delta''_{\balpha,q}$.

Again, we finish the proof by induction. We first take $p=q$, and we
prove that $\assJ_4(\balpha,q)$ holds ($\assI_4(\balpha,q)$ will
follow, by the previous paragraph). In this case, $n=1$ and
$\mathfrak{L}^H_{\balpha,q}$ is a diagonal matrix, whose diagonal
entries are products of homogeneous linear forms
$\mathfrak{l}^H_{i,i}$ depending on $(X_0,X_1)$ and with indeterminate
coefficients. The ideal of $p$-minors of $\mathfrak{L}^H_{\balpha,q}$
is generated by the product of the terms $\mathfrak{l}^H_{i,i}$, which
admits no repeated factors; the conclusion follows.

Consider next a pair $(\balpha,q)$, with
$\balpha=(\alpha_1,\dots,\alpha_p)$ and $1 \le p \le q$ and suppose
that $\assI_4(\balpha',q')$ holds for all $(\balpha',q')$ with
$\balpha'=(\alpha'_1,\dots,\alpha'_{p'})$, $1 \le p' \le q'$, $p' \le
p$ and $q' < q$; we prove that $\assJ_4(\balpha,q)$ holds; this will
imply $\assI_4(\balpha,q)$.

We take $t=p$ in the formula of Lemma~\ref{lemma:union}, and we first
deal with the terms in~\eqref{eq:union}.  Thus, we choose a
subsequence $\bi=(i_1,\dots,i_\kappa)$ of $(1,\dots,p)$, with $1 \le \kappa\le
\min(p,n-1)$, and indices $\br=(r_1,\dots,r_\kappa)$, with $ 1\le r_k \le
\alpha_{i_k}$ for all $k$. We prove that the Jacobian matrix of the $p$-minors of $\mathfrak{L}^H_{\balpha,q}$
with respect to $\tilde\bX=(X_0,\dots,X_n)$ has rank $n$ at all points $\tilde\bx=(x_0,\dots,x_n)$ of
$V_p(\mathfrak{L}^H_{\balpha,q})$  such that
$\tilde\bx'=(x_0,\dots,x_{n-\kappa})$ is in
$V_\kappa(\mathfrak{P}^H_{\balpha_\bi,n-1}(\mathfrak{H}_{\bi,\br},\tilde\bX')) \subset
\P^{n-\kappa}(\KKGpbar)$, and such that
\begin{align}\label{eq:subsX}
  x_{n-\kappa+1}=\mathfrak{f}_{n-\kappa+1,\bi,\br}(\tilde\bx'),\dots,x_{n}=\mathfrak{f}_{n,\bi,\br}(\tilde\bx').
\end{align}
By Lemma~\ref{lemma:union}, taking all such $\tilde\bx$ into account,
for all $\bi$ and $\br$, will cover all points in
$V_p(\mathfrak{L}^H_{\balpha,q})$, up to the exception of those points
obtained from $\kappa=n$, which will admit a simpler treatment.
For simplicity, we continue the proof with $\bi=(1,\dots,\kappa)$, so
that we have $\balpha_\bi=(\alpha_1,\dots,\alpha_\kappa)$.  

We are going to exhibit some polynomials that belong to
$I_p(\mathfrak{L}^H_{\balpha,q})$, for which we can control the rank
of the Jacobian at $\tilde\bx$. First, we prove that for $i$ in
$\{1,\dots,\kappa\}$ and $r$ in $\{1,\dots,\alpha_i\}-\{r_i\}$, as
well as $i$ in $\{\kappa+1,\dots,p\}$ and $r$ in
$\{1,\dots,\alpha_i\}$, the value $\mathfrak{l}^H_{i,i,r}(\tilde\bx)$
is non-zero.  We subdivide the indeterminates $\mathfrak{Q}'$ into
$\mathfrak{Q}'_{\bi,\br}$ and $\mathfrak{Q}''_{\bi,\br}$, where
$\mathfrak{Q}'_{\bi,\br}$ corresponds to the coefficients involved in
$\mathfrak{l}^H_{i,i,r_i}$, for $i=1,\dots,\kappa$, and in the
submatrix of $\mathfrak{L}^H_{\balpha,q}$ associated to $\bi$, and
$\mathfrak{Q}''_{\bi,\br}$ are the other coordinates.  By
$\assI_2(\balpha_s,n-1)$,
$V_\kappa(\mathfrak{P}^H_{\balpha_\bi,n-1}(\mathfrak{H}_{\bi,\br},\tilde\bX'))$
is finite; as a result, since all entries of $\mathfrak{H}_{\bi,\br}$ 
are in $\KK(\mathfrak{Q}'_{\bi,\br})$,
all coordinates of $\tilde\bx$ are algebraic
over $\KK(\mathfrak{Q}'_{\bi,\br})$. For $i,r$ as above, the
coefficients of the equation
$$\mathfrak{l}^H_{i,i,r} = \frak{l}_{i,i,r,0} X_0+ \frak{l}_{i,i,r,1}
X_1 +\cdots + \frak{l}_{i,i,r,n} X_n$$ are
in  $\KK(\mathfrak{Q}''_{\bi,\br})$, thus algebraically independent
over the field of definition of $\tilde\bx$, so that $\mathfrak{l}^H_{i,i,r}(\tilde\bx)$
is non-zero.

\begin{remark}\label{remark:disjoint}
  This implies in particular that the union in Lemma~\ref{lemma:union} is disjoint,
as claimed in Section~\ref{sec:prel-row}.
\end{remark}

In the following two paragraphs, assume $\kappa \ge 2$
and take $i$ in $\{1,\dots,\kappa\}$. We can then define
$\bi^*=(1,\dots,i-1,i+1,\dots,\kappa)$,
$\balpha^*=(\alpha_1,\dots,\alpha_{i-1},\alpha_{i+1},\dots,\alpha_\kappa)$,
and we call $\mathfrak{L}^H_i$ the submatrix of
$\mathfrak{L}^H_{\balpha,q}$ associated to $\bi^*$; this is a matrix
with $\kappa-1$ rows (indexed by $\bi^*$ in
$\mathfrak{L}^H_{\balpha,q}$) and $n-1$ columns (of indices $p+1,\dots,q$ 
in $\mathfrak{L}^H_{\balpha,q}$).

We prove that there exists a $(\kappa-1)$-minor $c_i$ of
$\mathfrak{L}^H_i$ such that $c_i(\tilde\bx)\ne 0$.  Let indeed
$\mathfrak{p}^H_i$ be the matrix obtained by applying the
substitution~\eqref{eq:subsX} in $\mathfrak{L}^H_i$. This matrix 
has $\kappa-1$ rows and $n-1$ columns; its entries are products of
linear forms in $(n-\kappa)+1$ variables $X_0,\dots,X_{n-\kappa}$,
with coefficients that are algebraically
independent over $\KK$. We can thus apply $\assI_3(\balpha^*,n-1)$ to
$\mathfrak{p}_{i}^H$, and deduce that this matrix has full rank $\kappa-1$
at $\tilde\bx'$.  Thus, $\mathfrak{L}^H_i$ has rank $\kappa-1$ at
$\tilde\bx$, from which the existence of the minor $c_i$ follows.
If $\kappa=1$, we define $c_1=1$.

We next deduce that for $i$ in $\{1,\dots,\kappa\}$, there exists a
polynomial of the form $b_{i} \mathfrak{l}^H_{i,i,r_i}$ in the ideal
of $p$-minors of $\mathfrak{L}^H_{\balpha,q}$, with
$b_{i}(\tilde\bx)\ne 0$. Indeed, we consider the $p$-minor of
$\mathfrak{L}^H_{\balpha,q}$ obtained by taking the columns
$i$,$\kappa+1,\dots,p$, and all $\kappa-1$ columns in the
$(\kappa-1)$-minor $c_i$ (if $\kappa=1$, there is no need to consider
such columns). Using the factorization
$$\mathfrak{l}^H_{i,i} = \beta_i \mathfrak{l}^H_{i,i,r_i},\quad\text{with}\quad
\beta_i=\mathfrak{l}^H_{i,i,1}\cdots \mathfrak{l}^H_{i,i,r_i-1}\mathfrak{l}^H_{i,i,r_i+1}\cdots \mathfrak{l}^H_{i,i,\alpha_i},$$
that minor evaluates to 
$$b_i \mathfrak{l}^H_{i,i,r_i}\quad\text{with}\quad b_i = \beta_i
\mathfrak{l}^H_{\kappa+1,\kappa+1}\cdots \mathfrak{l}^H_{p,p}c_i.$$
Hence, $b_i\, \mathfrak{l}^H_{i,i,r_i}$ belongs to the ideal of
$p$-minors of $\mathfrak{L}^H_{\balpha,q}$, and by the discussion of
the three previous paragraphs, $b_i(\tilde\bx)\ne 0$, as claimed. In
what follows, we write $b=b_1 \cdots b_\kappa$, so that $b(\tilde\bx)
\ne 0$ and $b\, \mathfrak{l}^H_{i,i,r_i}$ is in
the ideal of $p$-minors of 
$\mathfrak{L}^H_{\balpha,q}$.  This in turn implies that all
polynomials
$$
b(X_{n-\kappa+1}-\mathfrak{f}_{n-\kappa+1,\bi,\br}(\tilde\bX')),\dots,b(X_{n}-\mathfrak{f}_{n,\bi,\br}(\tilde\bX'))
$$ are in this ideal as well.

Similarly, for every $\kappa$-minor $\eta$ of the submatrix of
$\mathfrak{L}^H_{\balpha,q}$ associated to $\bi$, the polynomial
$\mathfrak{l}^H_{\kappa+1,\kappa+1}\cdots \mathfrak{l}^H_{p,p}\, \eta$
belongs to ideal of $p$-minors of $\mathfrak{L}^H_{\balpha,q}$. Thus,
$b \,\eta$ is in this ideal as well.

As a result, the polynomial $b\,
\eta(\tilde\bX',\mathfrak{f}_{n-\kappa+1,\bi,\br}(\tilde\bX'),\dots,\mathfrak{f}_{n,\bi,\br}(\tilde\bX'))$
belongs to that same ideal. Now,
$\gamma=\eta(\tilde\bX',\mathfrak{f}_{n-\kappa+1,\bi,\br}(\tilde\bX'),\dots,\mathfrak{f}_{n,\bi,\br}(\tilde\bX'))$
is one of the $\kappa$-minors of
$\mathfrak{P}^H_{\balpha_\bi,n-1}(\mathfrak{H}_{\bi,\br},\tilde\bX')$,
and all $\kappa$-minors of this matrix are obtained this way.  To
summarize, we have proved that
$$b\, \mathfrak{l}^H_{1,1,r_1},\dots,b\,
\mathfrak{l}^H_{\kappa,\kappa,r_\kappa} \quad\text{and}\quad b\,
\gamma, \text{~for all $\kappa$-minors $\gamma$ of
  $\mathfrak{P}^H_{\balpha_\bi,n-1}(\mathfrak{H}_{\bi,\br},\tilde\bX')$}$$
are in the $p$-minor ideal of 
 $\mathfrak{L}^H_{\balpha,q}$, with $b(\tilde\bx) \ne
0$. The Jacobian matrix of these polynomials at $\tilde\bx$ is, up to
the non-zero constant $b(\tilde\bx)$, equal to that of
$\mathfrak{l}^H_{1,1,r_1},\dots,
\mathfrak{l}^H_{\kappa,\kappa,r_\kappa}$ (which is simply a matrix of
constants), and of all $\kappa$-minors $\gamma$. Using our induction
assumption, we know that the Jacobian matrix of the ideal of
$\kappa$-minors $\gamma$ with respect to $\tilde\bX'$ has rank
$n-\kappa$ at $\tilde\bx'$. As a result, the larger Jacobian matrix of
all equations above has rank $n$ at $\tilde\bx$, as claimed.

It remains to deal with the case $\kappa=n$, for $n \le p$; as above,
we may simplify the discussion by assuming that $\bi=(1,\dots,n)$. In
this case, the discussion is simpler: proceeding as above, but dealing
only with the polynomials
$\mathfrak{l}^H_{1,1},\dots,\mathfrak{l}^H_{n,n}$, we obtain the fact
that equations of the form $b\, \mathfrak{l}^H_{1,1,r_1},\dots,b\,
\mathfrak{l}^H_{n,n,r_n}$ belong to the $p$-minor ideal of
$\mathfrak{L}^H_{\balpha,q}$, with $b(\tilde\bx) \ne 0$. The
conclusion follows directly.


\bibliographystyle{plain} \bibliography{roadmap}

\end{document}
