\documentclass[12pt]{article}
\usepackage{bbm,fullpage}
\usepackage{bm}
\usepackage{amsmath,amssymb,amsthm}
\usepackage{algorithm, pseudocode}
\usepackage{mathrsfs}
\usepackage{enumitem}
\usepackage[titles]{tocloft}
\usepackage{yfonts}
\usepackage{xcolor}
\usepackage[pdftex,                %
%    pagebackref,                %
    bookmarks         = true,%     % Signets
    bookmarksnumbered = true,%     % Signets numérotés
    pdfpagemode       = None,%     % Signets/vignettes fermé à l'ouverture
    pdfstartview      = FitH,%     % La page prend toute la largeur
    pdfpagelayout     = SinglePage,% Vue par page
    colorlinks        = true,%     % Liens en couleur
%    linkcolor= monvert, %    % couleur des liens internes
%    anchorcolor= blue, %    % couleur des liens internes
   citecolor         =blue,
    urlcolor          = magenta,%  % Couleur des liens externes
    pdfborder         = {0 0 0}%   % Style de bordure : ici, pas de bordure
    ]{hyperref}%                   % Utilisation de HyperTeX

\usepackage{amsmath}
\usepackage{easybmat}
\usepackage{multirow,bigdelim}

\newcommand*\hexbrace[2]{%
  \underset{#2}{\underbrace{\rule{#1}{0pt}}}}

\usepackage[indexonlyfirst,ucmark,toc]{glossaries}
\renewcommand*{\glstextformat}[1]{\textcolor{black}{#1}}
%\glsdisablehyper %pour enlever les liens hypertexte



\def\softO{\ensuremath{{O}{\,\tilde{ }\,}}}


\def\Vc{\ensuremath{\mathcal{V}}}
\def\A {\ensuremath{\mathbb{A}}}
\def\B {\ensuremath{\mathbb{B}}}
\def\Co {\ensuremath{\mathbb{C}}}
\def\CC {\ensuremath{\overline{\mathbf{K}}}}
\def\PP {\ensuremath{\mathsf{P}}}
\def\BB {\ensuremath{H}}
\def\P {\ensuremath{\mathbb{P}}}
\def\Q {\ensuremath{\mathbb{Q}}}
\def\N {\ensuremath{\mathbb{N}}}
\def\Re {\ensuremath{\mathbb{R}}}
\def\RP {\ensuremath{\tilde{\mathbf{R}}}}
\def\CP {\ensuremath{\tilde{\mathbf{C}}}}
\def\Z {\ensuremath{\mathbb{Z}}}
\def\RR {\ensuremath{\mathbb{K}}}
\def\k{\ensuremath{\mathbb{k}}}
\def\kbar{\ensuremath{\bar{\mathbb{k}}}}

\def\K {\ensuremath{\mathbb{K}}}
\def\KKbar {\ensuremath{\overline{\mathbf{K}}}}
\def\KKCbar {\ensuremath{\overline{\mathbf{K}(\mathfrak{G})}}}
\def\KKGbar {\ensuremath{\overline{\mathbf{K}(\mathfrak{L})}}}
\def\KKGpbar {\ensuremath{\overline{\mathbf{K}(\mathfrak{L}')}}}
\def\BS {\ensuremath{{B}}}
\def\CS {\ensuremath{{C}}}
\def\RS {\ensuremath{\mathsf{R}}}
\def\LL {\ensuremath{\mathbf{L}}}
\def\LLbar {\ensuremath{\overline{\mathbf{L}}}}
\def\GG {\ensuremath{\mathbf{G}}}
\def\KK {\ensuremath{\mathbf{K}}}
\def\D {\ensuremath{\mathbf{D}}}
\def\n {\ensuremath{\mathbf{n}}}
\def\d {\ensuremath{\mathbf{d}}}
\def\p {\ensuremath{\mathbf{p}}}
\def\I {\ensuremath{\mathbf{I}}}
\def\mI {\ensuremath{\mathbf{I}}}
\def\mS {\ensuremath{\mathbf{S}}}
\def\mzero {\ensuremath{\mathbf{0}}}
\def\mT {\ensuremath{\mathbf{T}}}
\def\mJ {\ensuremath{\mathbf{J}}}
\def\JJ {\ensuremath{\mathbf{J}}}
\def\sfB {\ensuremath{\sf B}}
\def\sfA {\ensuremath{\sf A}}


\def\MM {\ensuremath{\mathbb{M}}}
\def\mm {\ensuremath{\mathbf{m}}}
\def\v {\ensuremath{\mathbf{v}}}
\def\vc {\ensuremath{\mathbf{c}}}
\def\vd {\ensuremath{\mathbf{d}}}

\def\mA{\ensuremath{{A}}}
\def\mU{\ensuremath{{U}}}
\def\mL{\ensuremath{{L}}}
\def\mF{\ensuremath{{F}}}
\def\mG{\ensuremath{{G}}}
\def\mH{\ensuremath{{H}}}
\def\mB{\ensuremath{{B}}}
\def\mM{\ensuremath{{M}}}
\def\mN{\ensuremath{{N}}}
\def\mD{\ensuremath{\mathbf{D}}}
\def\mR{\ensuremath{\mathbf{R}}}
\def\G {\ensuremath{\mathrm{GL}}}
\def\Mat{\ensuremath{{\rm M}}}
\def\grad{\ensuremath{{\rm grad}}}
\def\ker{\ensuremath{{\rm ker}}}
\def\rank{\ensuremath{{\rm rank}}}
\def\jac{\ensuremath{{\rm jac}}}
\def\ext{\ensuremath{{\rm ext}}}

\def\scrS{\ensuremath{\mathscr{S}}}
\def\scrG{\ensuremath{\mathscr{G}}}
\def\scrZ{\ensuremath{\mathscr{Z}}}
\def\scrA{\ensuremath{\mathcal{A}}}
\def\scrT{\ensuremath{\mathscr{T}}}
\def\scrL{\ensuremath{\mathscr{L}}}
\def\scrU{\ensuremath{\mathcal{U}}}
\def\scrZ{\ensuremath{\mathscr{Z}}}
\def\scrR{\ensuremath{\mathscr{R}}}
\def\scrY{\ensuremath{\mathscr{Y}}}
\def\scrB{\ensuremath{\mathscr{B}}}

\def\y {\ensuremath{\mathbf{y}}}
\def\e {\ensuremath{\mathbf{e}}}
\def\a {\ensuremath{\mathbf{a}}}
\def\b {\ensuremath{\mathbf{b}}}
\def\z {\ensuremath{\mathbf{z}}}
\def\w {\ensuremath{\mathbf{w}}}
\def\f {\ensuremath{\mathbf{f}}}
\def\r {\ensuremath{\mathbf{r}}}
\def\s {\ensuremath{\mathbf{s}}}
\def\L {\ensuremath{\mathbf{L}}}
\def\F {\ensuremath{\mathbf{F}}}
\def\G {\ensuremath{\mathbf{G}}}
\def\E {\ensuremath{\mathbf{E}}}
\def\X {\ensuremath{\mathbf{X}}}
\def\Y {\ensuremath{\mathbf{Y}}}
\def\H {\ensuremath{\mathbf{H}}}
\def\m {\ensuremath{\mathfrak{m}}}
\def\v {\ensuremath{\mathbf{v}}}
\def\u {\ensuremath{\mathbf{u}}}
\def\q {\ensuremath{\mathbf{q}}}
\def\U {\ensuremath{\mathbf{U}}}
\def\V {\ensuremath{\mathbf{V}}}
\def\t {\ensuremath{\mathbf{t}}}


\def\blambda{\mbox{\boldmath$\lambda$}}
\def\bdelta{\mbox{\boldmath$\delta$}}
\def\bzeta{\mbox{\boldmath$\zeta$}}
\def\bkappa{\mbox{\boldmath$\kappa$}}
\def\bmu{\mbox{\boldmath$\mu$}}
\def\bsigma{\mbox{\boldmath$\sigma$}}
\def\bpsi{\mbox{\boldmath$\psi$}}
\def\bphi{\mbox{\boldmath$\phi$}}
\def\bh{\mbox{\boldmath$h$}}
\def\bg{\mbox{\boldmath$g$}}
\def\bbf{\mbox{\boldmath$f$}}
\def\brho{\mbox{\boldmath$\rho$}}
\def\bell{\mbox{\boldmath$\ell$}}
\def\sbdelta{\mbox{\scriptsize \boldmath$\delta$}}
\def\sbell{\mbox{\scriptsize \boldmath$\ell$}}

\def\bm{{\mathbf{m}}}

\def\xxx{\textbf{(xxx)}}
\def\why{\textbf{(why?)}}
\def\todo#1{(\textbf{todo:} #1)}
\def\check{\textbf{(check!)}}

\def\KKbar {\ensuremath{\overline{\mathbf{K}}}}
 
\DeclareBoldMathCommand{\bM}{M}
\DeclareBoldMathCommand{\bD}{D}
\DeclareBoldMathCommand{\bB}{B}
\DeclareBoldMathCommand{\bA}{A}
\DeclareBoldMathCommand{\bC}{C}
\DeclareBoldMathCommand{\bX}{X}
\DeclareBoldMathCommand{\bi}{i}
\DeclareBoldMathCommand{\bk}{k}
\DeclareBoldMathCommand{\bY}{Y}
\DeclareBoldMathCommand{\bn}{n}
\DeclareBoldMathCommand{\bbb}{b}
\DeclareBoldMathCommand{\d}{d}
\DeclareBoldMathCommand{\bc}{c}
\DeclareBoldMathCommand{\bj}{j}
\DeclareBoldMathCommand{\be}{e}
\DeclareBoldMathCommand{\bh}{h}
\DeclareBoldMathCommand{\br}{r}
\DeclareBoldMathCommand{\bu}{u}
\DeclareBoldMathCommand{\bv}{v}
\DeclareBoldMathCommand{\bx}{x}
\DeclareBoldMathCommand{\by}{y}
\DeclareBoldMathCommand{\btheta}{\vartheta}
\DeclareBoldMathCommand{\c}{c}
\DeclareBoldMathCommand{\f}{f}
\DeclareBoldMathCommand{\g}{g}
\DeclareBoldMathCommand{\h}{h}
\DeclareBoldMathCommand{\x}{x}
\DeclareBoldMathCommand{\bell}{\ell}
\DeclareBoldMathCommand{\bbeta}{\beta}
\DeclareBoldMathCommand{\balpha}{\alpha}
\DeclareBoldMathCommand{\bgamma}{\gamma}

\def\mult{\ensuremath{\mathrm{mult}}}
\def\rdeg{\ensuremath{\mathrm{rdeg}}}
\def\cdeg{\ensuremath{\mathrm{cdeg}}}

\def\assH{\ensuremath{\mathsf{A}}}
\def\assA{\ensuremath{\mathsf{B}}}
\def\assG{\ensuremath{\mathsf{C}}}
\def\assD{\ensuremath{\mathsf{D}}}
\def\assI{\ensuremath{\mathsf{J}}}
\def\assJ{\ensuremath{\mathsf{K}}}

\def\sfG {\ensuremath{\sf G}}

\newcommand{\VpF}[2]{V_{#1}(#2)}
\newcommand{\VpFG}[3]{V_{#1}(#2,#3)}

\def\rc{\ensuremath{{c'}{}}}
\def\re{\ensuremath{{e'}{}}}




\def\NOTE#1#2{{\begin{quote}\marginpar[\hfill{#1}]{{#1}}{{\textsf{[\![{#2}]\!]}}}\end{quote}}}
\def\respond#1{\NOTE{\textcircled{\textsc{a}}}{Note:~{#1}}}



\newtheorem{pbm}{Problem}
\newtheorem{definition}{Definition}
\newtheorem{theorem}[definition]{Theorem}
\newtheorem{corollary}[definition]{Corollary}
\newtheorem{proposition}[definition]{Proposition}
\newtheorem{lemma}[definition]{Lemma}
\newtheorem{remark}[definition]{Remark}

\begin{document}

This is a short note to study the maximum degree of intermediate
algebraic sets studied during the incremental solving process of
systems defining polar varieties through the geometric resolution
algorithm due to \cite{GiLeSa01}. This is not so restrictive ; all
what is written below does not really depend on the fact that the
entries of the rank defective matrix are partial derivatives of some
input polynomials which also vanish on the algebraic set under study.

We start by fixing notations. We let $\K$ be a field and polynomials
$\f=(f_1, \ldots, f_p)\subset \K[X_1, \ldots, X_n]$ with $D_i=\deg(f_i)$.

We want to solve the system of equations
$f_1=\cdots=f_p=M_1=\cdots=M_N=0$ where the $M_i$'s are the maximal
minors of the matrix $\jac(\f, 1)$ (the truncated jacobian matrix
associated to $\f$ obtained by removing the first column).

We assume in the sequel that
\begin{itemize}
\item the solution set of this system is
finite; 
\item the sequence $\f$ is in generic coordinates; 
\item the sequence $\f$ is reduced and regular.
\end{itemize}


We focus on the solutions of that system at which $\jac(\f, 1)$ have
rank $p-1$. Hence at these solutions at least one of the $(p-1, p-1)$
minors of $\jac(\f, 1)$ does not vanish; let us take w.l.o.g. the
upper left such minor and let $M_1, \ldots, M_{n-p}$ be the
$p\times p$ minors of $\jac(\f, 1)$ obtained by adding the missing row
and the successive missing $n-p$ columns to the upper left submatrix
of size $p(-1,p-1)$ of $\jac(\f, 1)$.

We look at the intermediate degrees of the solution sets defined as
the Zariski closures of the sets defined by
$f_1=\cdots=f_p=0, M_1=\cdots=M_i=0, m\neq 0$.  These sets are the
union of irreducible components of the polar varieties $W(\pi_i, V)$
(associated to the restriction of the projection
$(x_1, \ldots, x_n)\to (x_1, \ldots, x_i)$ to $V$). This statement
comes from the equidimensionality of these polar varieties (which are
consequences of the assumption on the genericity of our system of
coordinates).

Since we are in generic coordinates, some result in \cite{SaSc03}
establishes that $W(\pi_i, V)$ is in Noether position w.r.t.
$(X_1, \ldots, X_{i-1})$. Using some other result in [Krick, Pardo,
Sombra and maybe other colleagues] one concludes that for any
$x\in \bar{\K}^{i-1}$, the degree of the fiber
$\pi_{i-1}^{-1}(x)\cap W(\pi_i, V)$ (which is what is computed by
\cite{GiLeSa01}) equals that of $W(\pi_i, V)$.

Now, observe that $W(\pi_{i}, V\cap \pi_{i-1}^{-1}(x))$ coincides with
$W(\phi_i, V\cap \pi_{i-1}^{-1}(x))$ where $\phi_i$ is the projection
$(x_1, \ldots, x_n)\to x_i$. This statement is not immediate but can
be smoothly established by writing the systems of equations defining
both sets ; up to substitution of the variables $X_1, \ldots, X_{i-1}$
by the coordinates in $x$, in some polynomials, these systems
coincide.

Now, one can apply the result bounding the degrees of critical loci in
the current homotopy paper to bound the degree of the varieties
$W(\varphi_i, V\cap \pi_{i-1}^{-1}(x))$ (we already established that
this degree is also the one of $W(\pi_i, V)$). Immediate computations
show that the intermediate bounds are less than the final one.





 \bibliographystyle{plain} \bibliography{roadmap}


\end{document}