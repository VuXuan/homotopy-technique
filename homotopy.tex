\documentclass[12pt]{article}
\usepackage{bbm,fullpage}
\usepackage{bm}
\usepackage{amsmath,amssymb,amsthm}
\usepackage{mathrsfs}
\usepackage[titles]{tocloft}
\usepackage{xcolor}
\usepackage[pdftex,                %
%    pagebackref,                %
    bookmarks         = true,%     % Signets
    bookmarksnumbered = true,%     % Signets numérotés
    pdfpagemode       = None,%     % Signets/vignettes fermé à l'ouverture
    pdfstartview      = FitH,%     % La page prend toute la largeur
    pdfpagelayout     = SinglePage,% Vue par page
    colorlinks        = true,%     % Liens en couleur
%    linkcolor= monvert, %    % couleur des liens internes
%    anchorcolor= blue, %    % couleur des liens internes
   citecolor         =blue,
    urlcolor          = magenta,%  % Couleur des liens externes
    pdfborder         = {0 0 0}%   % Style de bordure : ici, pas de bordure
    ]{hyperref}%                   % Utilisation de HyperTeX

\usepackage{amsmath}
\usepackage{easybmat}
\usepackage{multirow,bigdelim}

\newcommand*\hexbrace[2]{%
  \underset{#2}{\underbrace{\rule{#1}{0pt}}}}

\usepackage[indexonlyfirst,ucmark,toc]{glossaries}
\renewcommand*{\glstextformat}[1]{\textcolor{black}{#1}}
%\glsdisablehyper %pour enlever les liens hypertexte



\def\softO{\ensuremath{{O}{\,\tilde{ }\,}}}


\def\Vc{\ensuremath{\mathcal{V}}}
\def\A {\ensuremath{\mathbb{A}}}
\def\B {\ensuremath{\mathbb{B}}}
\def\Co {\ensuremath{\mathbb{C}}}
\def\CC {\ensuremath{\overline{\mathbf{K}}}}
\def\PP {\ensuremath{\mathsf{P}}}
\def\BB {\ensuremath{H}}
\def\P {\ensuremath{\mathbb{P}}}
\def\Q {\ensuremath{\mathbb{Q}}}
\def\N {\ensuremath{\mathbb{N}}}
\def\Re {\ensuremath{\mathbb{R}}}
\def\RP {\ensuremath{\tilde{\mathbf{R}}}}
\def\CP {\ensuremath{\tilde{\mathbf{C}}}}
\def\Z {\ensuremath{\mathbb{Z}}}
\def\RR {\ensuremath{\mathbb{K}}}
\def\k{\ensuremath{\mathbb{k}}}
\def\kbar{\ensuremath{\bar{\mathbb{k}}}}

\def\K {\ensuremath{\mathbb{K}}}
\def\KKbar {\ensuremath{\overline{\mathbf{K}}}}
\def\KKCbar {\ensuremath{\overline{\mathbf{K}(\mathfrak{G})}}}
\def\KKGbar {\ensuremath{\overline{\mathbf{K}(\mathfrak{L})}}}
\def\KKGpbar {\ensuremath{\overline{\mathbf{K}(\mathfrak{L}')}}}
\def\BS {\ensuremath{{B}}}
\def\CS {\ensuremath{{C}}}
\def\RS {\ensuremath{\mathsf{R}}}
\def\LL {\ensuremath{\mathbf{L}}}
\def\LLbar {\ensuremath{\overline{\mathbf{L}}}}
\def\GG {\ensuremath{\mathbf{G}}}
\def\KK {\ensuremath{\mathbf{K}}}
\def\D {\ensuremath{\mathbf{D}}}
\def\n {\ensuremath{\mathbf{n}}}
\def\d {\ensuremath{\mathbf{d}}}
\def\p {\ensuremath{\mathbf{p}}}
\def\I {\ensuremath{\mathbf{I}}}
\def\mI {\ensuremath{\mathbf{I}}}
\def\mS {\ensuremath{\mathbf{S}}}
\def\mzero {\ensuremath{\mathbf{0}}}
\def\mT {\ensuremath{\mathbf{T}}}
\def\mJ {\ensuremath{\mathbf{J}}}
\def\JJ {\ensuremath{\mathbf{J}}}
\def\sfB {\ensuremath{\sf B}}
\def\sfA {\ensuremath{\sf A}}


\def\MM {\ensuremath{\mathbb{M}}}
\def\mm {\ensuremath{\mathbf{m}}}
\def\v {\ensuremath{\mathbf{v}}}
\def\vc {\ensuremath{\mathbf{c}}}
\def\vd {\ensuremath{\mathbf{d}}}

\def\mA{\ensuremath{{A}}}
\def\mU{\ensuremath{{U}}}
\def\mL{\ensuremath{{L}}}
\def\mF{\ensuremath{{F}}}
\def\mG{\ensuremath{{G}}}
\def\mH{\ensuremath{{H}}}
\def\mB{\ensuremath{{B}}}
\def\mM{\ensuremath{{M}}}
\def\mN{\ensuremath{{N}}}
\def\mD{\ensuremath{\mathbf{D}}}
\def\mR{\ensuremath{\mathbf{R}}}
\def\G {\ensuremath{\mathrm{GL}}}
\def\Mat{\ensuremath{{\rm M}}}
\def\grad{\ensuremath{{\rm grad}}}
\def\ker{\ensuremath{{\rm ker}}}
\def\rank{\ensuremath{{\rm rank}}}
\def\jac{\ensuremath{{\rm jac}}}
\def\ext{\ensuremath{{\rm ext}}}

\def\scrS{\ensuremath{\mathscr{S}}}
\def\scrG{\ensuremath{\mathscr{G}}}
\def\scrZ{\ensuremath{\mathscr{Z}}}
\def\scrA{\ensuremath{\mathcal{A}}}
\def\scrT{\ensuremath{\mathscr{T}}}
\def\scrL{\ensuremath{\mathscr{L}}}
\def\scrU{\ensuremath{\mathcal{U}}}
\def\scrZ{\ensuremath{\mathscr{Z}}}
\def\scrR{\ensuremath{\mathscr{R}}}
\def\scrY{\ensuremath{\mathscr{Y}}}
\def\scrB{\ensuremath{\mathscr{B}}}

\def\y {\ensuremath{\mathbf{y}}}
\def\e {\ensuremath{\mathbf{e}}}
\def\a {\ensuremath{\mathbf{a}}}
\def\b {\ensuremath{\mathbf{b}}}
\def\z {\ensuremath{\mathbf{z}}}
\def\w {\ensuremath{\mathbf{w}}}
\def\f {\ensuremath{\mathbf{f}}}
\def\r {\ensuremath{\mathbf{r}}}
\def\s {\ensuremath{\mathbf{s}}}
\def\L {\ensuremath{\mathbf{L}}}
\def\F {\ensuremath{\mathbf{F}}}
\def\G {\ensuremath{\mathbf{G}}}
\def\E {\ensuremath{\mathbf{E}}}
\def\X {\ensuremath{\mathbf{X}}}
\def\Y {\ensuremath{\mathbf{Y}}}
\def\H {\ensuremath{\mathbf{H}}}
\def\m {\ensuremath{\mathfrak{m}}}
\def\v {\ensuremath{\mathbf{v}}}
\def\u {\ensuremath{\mathbf{u}}}
\def\q {\ensuremath{\mathbf{q}}}
\def\U {\ensuremath{\mathbf{U}}}
\def\V {\ensuremath{\mathbf{V}}}
\def\t {\ensuremath{\mathbf{t}}}


\def\blambda{\mbox{\boldmath$\lambda$}}
\def\bdelta{\mbox{\boldmath$\delta$}}
\def\bzeta{\mbox{\boldmath$\zeta$}}
\def\bkappa{\mbox{\boldmath$\kappa$}}
\def\bmu{\mbox{\boldmath$\mu$}}
\def\bsigma{\mbox{\boldmath$\sigma$}}
\def\bpsi{\mbox{\boldmath$\psi$}}
\def\bphi{\mbox{\boldmath$\phi$}}
\def\bh{\mbox{\boldmath$h$}}
\def\bg{\mbox{\boldmath$g$}}
\def\bbf{\mbox{\boldmath$f$}}
\def\brho{\mbox{\boldmath$\rho$}}
\def\bell{\mbox{\boldmath$\ell$}}
\def\sbdelta{\mbox{\scriptsize \boldmath$\delta$}}
\def\sbell{\mbox{\scriptsize \boldmath$\ell$}}

\def\bm{{\mathbf{m}}}

\def\xxx{\textbf{(xxx)}}
\def\why{\textbf{(why?)}}
\def\todo#1{(\textbf{todo:} #1)}
\def\check{\textbf{(check!)}}

\def\KKbar {\ensuremath{\overline{\mathbf{K}}}}
 
\DeclareBoldMathCommand{\bM}{M}
\DeclareBoldMathCommand{\bD}{D}
\DeclareBoldMathCommand{\bB}{B}
\DeclareBoldMathCommand{\bA}{A}
\DeclareBoldMathCommand{\bC}{C}
\DeclareBoldMathCommand{\bX}{X}
\DeclareBoldMathCommand{\bi}{i}
\DeclareBoldMathCommand{\bk}{k}
\DeclareBoldMathCommand{\bY}{Y}
\DeclareBoldMathCommand{\bn}{n}
\DeclareBoldMathCommand{\bbb}{b}
\DeclareBoldMathCommand{\d}{d}
\DeclareBoldMathCommand{\bc}{c}
\DeclareBoldMathCommand{\bj}{j}
\DeclareBoldMathCommand{\be}{e}
\DeclareBoldMathCommand{\bh}{h}
\DeclareBoldMathCommand{\br}{r}
\DeclareBoldMathCommand{\bu}{u}
\DeclareBoldMathCommand{\bv}{v}
\DeclareBoldMathCommand{\bx}{x}
\DeclareBoldMathCommand{\by}{y}
\DeclareBoldMathCommand{\btheta}{\vartheta}
\DeclareBoldMathCommand{\c}{c}
\DeclareBoldMathCommand{\f}{f}
\DeclareBoldMathCommand{\g}{g}
\DeclareBoldMathCommand{\h}{h}
\DeclareBoldMathCommand{\x}{x}
\DeclareBoldMathCommand{\bell}{\ell}
\DeclareBoldMathCommand{\bbeta}{\beta}
\DeclareBoldMathCommand{\balpha}{\alpha}
\DeclareBoldMathCommand{\bgamma}{\gamma}

\def\mult{\ensuremath{\mathrm{mult}}}
\def\rdeg{\ensuremath{\mathrm{rdeg}}}
\def\cdeg{\ensuremath{\mathrm{cdeg}}}

\def\assH{\ensuremath{\mathsf{A}}}
\def\assA{\ensuremath{\mathsf{B}}}
\def\assG{\ensuremath{\mathsf{C}}}
\def\assD{\ensuremath{\mathsf{D}}}
\def\assI{\ensuremath{\mathsf{J}}}
\def\assJ{\ensuremath{\mathsf{K}}}

\def\sfG {\ensuremath{\sf G}}

\newcommand{\VpF}[2]{V_{#1}(#2)}
\newcommand{\VpFG}[3]{V_{#1}(#2,#3)}

\def\rc{\ensuremath{{c'}{}}}
\def\re{\ensuremath{{e'}{}}}




\def\NOTE#1#2{{\begin{quote}\marginpar[\hfill{#1}]{{#1}}{{\textsf{[\![{#2}]\!]}}}\end{quote}}}
\def\respond#1{\NOTE{\textcircled{\textsc{a}}}{Note:~{#1}}}

\newtheorem{definition}{Definition}
\newtheorem{theorem}[definition]{Theorem}
\newtheorem{corollary}[definition]{Corollary}
\newtheorem{proposition}[definition]{Proposition}
\newtheorem{lemma}[definition]{Lemma}

\begin{document}


\section{Testing if a point is isolated} \label{sec:isolated}

Let $\f=(f_1,\dots,f_M)$ be polynomials in $\KK[\Y]$, with
$\Y=(Y_1,\dots,Y_N)$, for a field $\KK$. Given a point $\by$ in
$V(\f)$, we discuss here how to decide whether $\by$ is an isolated
point in $V(\f)$. We make the following assumption (denoted by $\sfB$
below): we are given as input an integer $\mu$ such that
\begin{itemize}
\item either $\by$ belongs to a positive-dimensional component of $V(\f)$,
\item or $\by$ is isolated in $V(\f)$, with multiplicity at most $\mu$
  with respect to the ideal $\langle \f\rangle$.
\end{itemize}

\begin{proposition}\label{prop:testisolated}
  Suppose that $\f$ is given by a straight-line program of length $E$.
  If assumption $\sfB$ is satisfied, we can decide whether $\by$ is an
  isolated root of $V(\f)$ using $(\mu\,E\,M)^{O(1)}$ operations in~$\KK$.
\end{proposition}
Reference~\cite{BaHaPeSo09} gives an algorithm to compute the
dimension of $V(\f)$ at $\by$, but its complexity is not known to us, as
it relies on linear algebra with matrices of potentially large size
(not necessarily polynomial in $\mu,E,M$).
Instead, we use an adaptation of a prior result by
Mourrain~\cite{Mourrain97}, which allows us to control the size of the
matrices we handle. We only give detailed proofs for new ingredients
that are specific to our context, a key difference being the cost
analysis in the straight-line program model: Mourrain's original
result depends on the number of monomials appearing when we expand
$\f$, which would be too high for the applications we will make of
this result.

Without loss of generality, we assume that $M\ge N$ (otherwise, $\by$
cannot be an isolated solution). We also assume that $\by=0$; this is
done by replacing $\f$ by the polynomials $\f(\Y+\by)$, which have
complexity of evaluation $E'=E+N$.  The basis of our algorithm is the
following remark.

\begin{lemma}
  Let $I$ be the zero-dimensional ideal $\langle \f \rangle +
  \m^{\mu+1}$, where $\m=\langle Y_1,\dots,Y_N\rangle$ is the maximal
  ideal at the origin. Then, if $0$ is isolated in $V(\f)$ if and only
  if the multiplicity $m$ of $I$ at the origin is at most $\mu$.
\end{lemma}
\begin{proof}
  This follows from the following
  result~\cite[Theorem~A.1]{BaHaPeSo09}.  For $k \ge 1$, let $I_k$ be
  the zero-dimensional ideal $\langle \f \rangle + \m^{k}$, and let
  $\nu_k$ be multiplicity of the origin with respect to this
  ideal. Then, the reference above proves that the sequence
  $(\nu_k)_{k \ge 1}$ is non-decreasing, and that $0$ is isolated in
  $V(\f)$ if and only if there exists $k\ge 1$ such that
  $\nu_k=\nu_{k+1}$.
  \begin{itemize}
  \item If $0$ is isolated in $V(\f)$, then by assumption $\sfB$ 
    its multiplicity with respect to $\langle \f\rangle$ is at most $\mu$,
    and its multiplicity $m$ with respect to $I$ cannot be larger.
  \item Otherwise, by the result above, $\nu_{k+1} > \nu_k$ holds for
    all $k \ge 1$, so that $\nu_k \ge k$ holds for all such $k$ (since
    $\nu_1=1$). In particular, the multiplicity $m$ of 
 $I$ at the origin, which is $\nu_{\mu+1}$, is at least $\mu+1$.
    \qedhere
  \end{itemize}
\end{proof}

Hence, we are left with deciding whether the multiplicity of the
$\m$-primary ideal $I$ is at most $\mu$. We do this by following and
slightly modifying Mourrain's algorithm for the computation of the
orthogonal $I^{\perp}$, that is, the set of $\KK$-linear forms
$\KK[\Y] \to \KK$ that vanish on $I$; this is a $\KK$-vector space
naturally identified with the dual of $\KK[\Y]/I$, so it has dimension
$m$, the multiplicity of $I$ at the origin (note that $I$ is $\m$-primary).

We do not need to give all details of the algorithm, let alone proof
of correctness; we just mention the key ingredients for the cost
analysis in our setting. A linear form $\beta: \KK[\Y] \to \KK$ that
vanishes on $I$ must vanish on all monomials, except a finite number
(since all monomials, except a finite number, belong to $I$); a
natural way to represent such a linear form would then be as the
finite generating series $\sum_{\balpha \in \N^N}
\beta(Y_1^{\alpha_1}\cdots Y_N^{\alpha_N}) d_1^{\alpha_1}\cdots
d_N^{\alpha_N}$, for some new variables $d_1,\dots,d_N$; however the
number of non-zero coefficients in such a sum cannot be bounded
polynomially in terms of $N,\mu$.

Hence, the algorithm uses another way to represent the elements in
$I^{\perp}$, by means of {\em multiplication matrices}. An important
feature of $I^{\perp}$ is that it admits the structure of a
$\KK[\Y]$-module: for $k$ in $\{1,\dots,N\}$ and $\beta$ in
$I^{\perp}$, the $\KK$-linear form $Y_k \cdot \beta: f \mapsto
\beta(Y_k f)$ is easily seen to still be in $I^{\perp}$.  In
particular, if $\bbeta=(\beta_1,\dots,\beta_m)$ is a $\KK$-basis of
$I^{\perp}$, then for all $k$ as above, and all $i$ in
$\{1,\dots,m\}$, $Y_k \cdot \beta_i$ is a linear combination of
$\beta_1,\dots,\beta_m$. Mourrain's algorithm computes a
basis $\bbeta=(\beta_1,\dots,\beta_m)$ with the following features:
\begin{itemize}
\item for $i$ in $\{1,\dots,m\}$ and $k$ in $\{1,\dots,N\}$, we have
  $Y_k \cdot \beta_i=\sum_{0 \le j < i} \lambda^{(k)}_{i,j} \beta_j$
  (hence $\lambda^{(k)}_{i,j}$ may be non-zero 
  only for $j<i$)
\item $\beta_1$ is the evaluation at $0$, $f \mapsto f(0)$
\item for $i$ in $\{2,\dots,m\}$, $\beta_i(1)=0$.
\end{itemize}
The following lemma shows that the coefficients $(\lambda^{(k)}_{i,j})$
are sufficient to evaluate  the linear forms $\beta_i$ at any $f$ in
$\KK[\Y]$. More precisely, knowing only their values for $j < i \le s$,
for any $s \le m$, allows us to evaluate $\beta_1,\dots,\beta_s$ at such an $f$.
The following lemma follows~\cite{Mourrain97} in its description
of the matrices $\bM_{k,s}$; the (rather straightforward) complexity analysis 
in the straight-line program model is new.
\begin{lemma}\label{lemma:evalbeta}
   Let $s$ be in $1,\dots,m$, and suppose that the coefficients
  $\lambda^{(k)}_{i,j}$ are known for $i=1,\dots,s$, $j=0,\dots,i-1$
  and $k=1,\dots,N$. Given a straight-line program $\Gamma$ of length
  $L$ that computes $\h=(h_1,\dots,h_R)$, one can compute
  $\beta_i(h_r)$, for all $i=1,\dots,s$ and $r=1,\dots,R$, using
  $(s\,L)^{O(1)}$ operations.
\end{lemma}
\begin{proof}
  By definition, for $h$ in $\KK[\Y]$ and $k=1,\dots,N$, the following equality
  holds:
$$
  \begin{bmatrix}
    \beta_1(Y_k h)\\
    \vdots\\
    \beta_s(Y_k h)
  \end{bmatrix}=
\bM_{k,s}
  \begin{bmatrix}
    \beta_1(h)\\
    \vdots\\
    \beta_s(h)
  \end{bmatrix},
\quad\text{with}\quad
\bM_{k,s}= \begin{bmatrix}
    \lambda^{(k)}_{1,1} & \cdots & \lambda^{(k)}_{s,1}\\
    \vdots && \vdots \\
    \lambda^{(k)}_{1,s} & \cdots & \lambda^{(k)}_{s,s}
  \end{bmatrix}.
$$ 
 Remark that the matrices $\bM_{k,s}$ all commute. Indeed, 
for any $k,k'$ in $\{1,\dots,N\}$, and $h$ as above, the relation above implies
that 
$$
\Delta_{k,k',s}
  \begin{bmatrix}
    \beta_1(h)\\
    \vdots\\
    \beta_s(h)
  \end{bmatrix} =
  \begin{bmatrix}
0\\ \vdots \\ 0 
  \end{bmatrix},
$$
where $\Delta_{k,k',s} = \bM_{k,s}\bM_{k',s}-\bM_{k',s}\bM_{k,s}.$ Because 
the linear forms $\beta_1,\dots,\beta_s$ are linearly independent, this implies
that all rows of $\Delta_{k,k',s}$ must be zero, as claimed.
We then deduce that for any polynomial $h$ in $\KK[\Y]$, we have
the equality
$$  \begin{bmatrix}
    \beta_1(h)\\
    \vdots\\
    \beta_s(h)
  \end{bmatrix} =
h(\bM_{1,s},\dots,\bM_{N,s})   \begin{bmatrix}
    \beta_1(1)\\
    \vdots\\
    \beta_s(1)
  \end{bmatrix} $$ On the other hand, our assumptions imply that the
  sequence $(\beta_1(1),\dots,\beta_s(1))$ is simply $(1,0,\dots,0)$.
  To prove the proposition, it is enough to note that the evaluations \sloppy
  $h_1(\bM_{1,s},\dots,\bM_{N,s}),\dots,h_R(\bM_{1,s},\dots,\bM_{N,s})$
  can be computed using the straight-line program doing
  $(s\,L)^{O(1)}$ operations.
\end{proof}

Mourrain's alorithm proceeds in an iterative manner, starting from
$\bbeta^{(1)}=(\beta_{1})$ (and setting $e_1=1$), and computing
successively $\bbeta^{(2)}=(\beta_{e_1+1},\dots,\beta_{e_2})$,
$\bbeta^{(3)}=(\beta_{e_2+1},\dots,\beta_{e_3})$, \dots for some
integers $e_1 \le e_2 \le e_3 \dots$ Mourrain's algorithm stops when
$e_{\ell+1}=e_{\ell}$, in which case $\beta_1,\dots,\beta_{e_\ell}$ is
a $\KK$-basis of $I^\perp$, and $e_\ell=m$. In our case, we
are not interested in computing this multiplicity, but only in
deciding whether it is less than or equal to the parameter $\mu$. We do it as follows: assume that we have
computed $\bbeta^{(1)},\bbeta^{(2)},\dots,\bbeta^{(\ell)}$, together
with the corresponding integers $e_1,e_2,\dots,e_\ell$, with $e_1 <
\cdots < e_\ell \le \mu$. We compute $\bbeta^{(\ell+1)}$ and $e_{\ell+1}$,
and continue according to the following:
\begin{itemize}
\item if $e_{\ell+1}=e_{\ell}$, we conclude that the multiplicity $\mult(0,I)$
  is $e_\ell \le \mu$; we stop the algorithm;
\item if $e_{\ell+1} > \mu$, we conclude that the multiplicity is greater 
  than $\mu$; we stop the algorithm;
\item else, when $e_\ell < e_{\ell+1} \le \mu$, we do another loop.
\end{itemize}
Because the $e_\ell$'s are an increasing sequence of integers, they
satisfy $e_\ell \ge \ell$; hence, every time we enter the loop above we
have $\ell \le \mu$. To finish the analysis of the algorithm, it
remains to explain how to compute $\bbeta^{(\ell+1)}$ from
$(\bbeta^{(1)},\bbeta^{(2)},\dots,\bbeta^{(\ell)})=(\beta_{1},\dots,\beta_{e_\ell})$.

As per our description above, at any step of the algorithm,
$\beta_{1},\dots,\beta_{e_\ell}$ are represented by means of the
coefficients $\lambda^{(k)}_{i,j}$, for $0 \le j < i \le e_{\ell}$ and
$1 \le k \le N$.  At step $\ell$, Mourrain's algorithm solves a homogeneous linear system
$S_\ell$ with $N(N-1) e_\ell/2+M'$ equations and $N e_\ell$ unknowns,
where $M'$ is the number of generators of the ideal $I= \langle \f
\rangle + \m^{\mu+1}$. Remark that $M'$ is not polynomial in $\mu$ 
and $N$, so the size of $S_\ell$ is {\em a priori} too large to 
fit our cost bound; we will explain below how to resolve this issue.

The nullspace dimension of this linear system gives us the cardinality
$e_{\ell+1}-e_{\ell}$ of $\bbeta^{(\ell+1)}$. Similarly, the coordinates of
the $e_{\ell+1}-e_{\ell}$ vectors in a nullspace basis are precisely
the coefficients $\lambda^{(k)}_{i,j}$ for
$i=e_{\ell}+1,\dots,e_{\ell+1}$, $j=1,\dots,e_\ell$ and $k=1,\dots,N$
(we have $\lambda^{(k)}_{i,j}=0$ for $j=e_{\ell}+1,\dots,i-1$). For
all $\ell \ge 2$, all linear forms $\beta$ in $\bbeta^{(\ell)}$ are
such that for all $k$ in $\{1,\dots,N\}$, $Y_k \cdot \beta$ belongs to
the span of $\bbeta^{(1)},\dots,\bbeta^{(\ell-1)}$; in particular, a
quick induction shows that all linear forms in
$\bbeta^{(1)},\dots,\bbeta^{(\ell)}$ vanish on all monomials of degree
at least $\ell$.

There remains the question of setting up the system $S_\ell$. For $k$
in $\{1,\dots,N\}$ and a $\KK$-linear form $\beta$, we denote by
$Y_k^{-1} \cdot \beta$ the $\KK$-linear form defined as follows:
\begin{itemize}
\item $(Y_k^{-1} \cdot \beta)(Y_k f) = \beta(f)$ for all $f$ in $\KK[\Y]$,
\item $(Y_k^{-1} \cdot \beta)(f)=0$ if $f\in \KK[\Y]$ does not depend on $Y_k$.
\end{itemize}
In other words, $(Y_k^{-1} \cdot \beta)(f)=\beta(\delta_k(f))$ holds
for all $f$, where $\delta_k:\KK[\Y] \to \KK[\Y]$ is the $k$th divided
difference operator
$$f\mapsto \frac
{f(Y_1,\dots,Y_N)-f(Y_1,\dots,Y_{k-1},0,Y_{k+1},\dots,Y_N)}{Y_k}.$$
One verifies that, as the notation suggests, $Y_k \cdot (Y_k^{-1}
\cdot \beta)$ is equal to $\beta$. This being said, we can then
describe what the entries of $S_\ell$ are:
\begin{itemize}
\item the first $N(N-1) e_\ell/2$ equations involve only the coefficients 
  $\lambda^{(k)}_{i,j}$ previously computed (we refer to~\cite[Section~4.4]{Mourrain97} for details of how exactly 
these entries are distributed in $S_\ell$, as we do not need such details here).
\item each of the other $M'$ equations has coefficient vector
$$c_f = \big (\
 (Y_k^{-1} \cdot \beta_1)(f(Y_1,\dots,Y_k,0,\dots,0)),\dots,\ (Y_k^{-1} \cdot \beta_{e_\ell})(f(Y_1,\dots,Y_k,0,\dots,0))\
\big )_{1 \le k \le N},$$
where $f$ is a generator of $I=\langle \f \rangle +\m^{\mu+1}$.
\end{itemize}
We claim that only those equations corresponding to generators
$f_1,\dots,f_M$ of the input system $\f$ are useful, as all others are identically
zero.

We pointed out above that any linear form $\beta_i$ in
$\beta_1,\dots,\beta_{e_\ell}$ vanishes on all monomials of degree at
least $\ell$. Since we saw that we must have $\ell \le \mu$, all
$\beta_i$ as above vanish on monomials of degree $\mu$; this implies
that $Y_k^{-1}\cdot \beta_i$ vanishes on all monomials of degree
$\mu+1$. The generators $f$ of $\m^{\mu+1}$ have degree $\mu+1$, and
for any such $f$, $f(Y_1,\dots,Y_k,0,\dots,0)$ is either zero, or of
degree $\mu+1$ as well. Hence, for any $k$, $\beta_i$ in
$\beta_1,\dots,\beta_{e_\ell}$ and $f$ as above, $(Y_k^{-1} \cdot
\beta_i)(f(Y_1,\dots,Y_k,0,\dots,0))$ vanishes. This implies that the
vector $c_f$ is identically zero for such an $f$, and that the
corresponding equation can be discarded.

Altogether, as claimed above, we see that we have to compute the
values
$$(Y_k^{-1} \cdot \beta_i)(f_j(Y_1,\dots,Y_k,0,\dots,0)),$$ for
$k=1,\dots,N$, $i=1,\dots,e_\ell$ and $j=1,\dots,M$.  Fixing $k$, we
let $\f_k = (f_{j,k})_{1 \le j \le M}$, where $f_{j,k}$ is the
polynomial $f_j(Y_1,\dots,Y_k,0,\dots,0)$; note that the system $\f_k$
can be computed by a straight-line program of length $E'=E+N$. Then,
applying the following lemma with $s=e_\ell \le \mu$ and $\h = \f_k$,
we deduce that the values $(Y_k^{-1} \cdot
\beta_i)(f_j(Y_1,\dots,Y_k,0,\dots,0))$, for $k$ fixed, can be
computed in time $(\mu\,E\,N)^{O(1)}$.


\begin{lemma}
  Let $s$ be in $1,\dots,m$, and suppose that the coefficients
  $\lambda^{(k)}_{i,j}$ are known for $i=1,\dots,s$, $j=0,\dots,i-1$
  and $k=1,\dots,N$. Given a straight-line program $\Gamma$ of length
  $L$ that computes $\h=(h_1,\dots,h_R)$ and given $k$ in
  $\{1,\dots,N\}$, one can compute $(Y_k^{-1}\cdot \beta_i)(h_r)$, for
  all $i=1,\dots,s$ and $r=1,\dots,R$, using $(s\,L\,N)^{O(1)}$
  operations in $\KK$.
\end{lemma}
\begin{proof}
  In view of the formula $(Y_k^{-1} \cdot
  \beta)(f)=\beta(\delta_k(f))$, and of Lemma~\ref{lemma:evalbeta}, it is
  enough to prove the existence of a straight-line program of length
  $O(L+N)$ that computes $(\delta_k(h_1),\dots,\delta_k(h_R))$.

  To do this, we replace all polynomials
  $\gamma_{-N+1},\dots,\gamma_L$ computed by $\Gamma$ by terms
  $\lambda_{-N+1},\dots,\lambda_L$ and $\mu_{-N+1},\dots,\mu_L$, with
  $\lambda_\ell=\gamma_\ell(Y_1,\dots,Y_{k-1},0,Y_{k+1},\dots,Y_N)$
  and $\mu_\ell$ in $\KK[\Y]$ such that $\gamma_\ell= \lambda_\ell+Y_k
  \mu_\ell$ holds for all $\ell$, so that in particular
  $\mu_\ell=\delta_k(\gamma_\ell)$.  To compute $\lambda_\ell$ and
  $\mu_\ell$, assuming all previous $\lambda_{\ell'}$ and
  $\mu_{\ell'}$ are known, we proceed as follows:
  \begin{itemize}
  \item if $\gamma_\ell=Y_k$, we set $\lambda_\ell=0$ and $\mu_\ell=1$;
  \item if $\gamma_\ell=Y_{k'}$, with $k' \ne k$, we set $\lambda_\ell=Y_{k'}$ and $\mu_\ell=0$;
  \item if $\gamma_\ell =c_\ell$, with $c_\ell \in \KK$,
    then we set $\lambda_\ell=c_\ell$ and  $\mu_\ell=0$;
  \item if $\gamma_\ell = \gamma_{a_\ell} \pm \gamma_{b_\ell}$,
    for some indices $a_\ell,b_\ell < \ell$, 
    then we set $\lambda_\ell=\lambda_{a_\ell}\pm\lambda_{b_\ell}$
    and $\mu_\ell=\mu_{a_\ell}\pm\mu_{b_\ell}$;
\item if $\gamma_\ell = \gamma_{a_\ell} \gamma_{b_\ell}$,
      for some indices $a_\ell,b_\ell < \ell$,
    then we set $\lambda_\ell=\lambda_{a_\ell} \lambda_{b_\ell}$
    and $$\mu_\ell=
\lambda_{a_\ell} \mu_{b_\ell}
+
\mu_{a_\ell} \lambda_{b_\ell}
+
Y_k\mu_{a_\ell} \mu_{b_\ell}.$$
\end{itemize}
One verifies that in all cases, the relation $\gamma_\ell=
\lambda_\ell+Y_k \mu_\ell$ still holds. Since the previous
construction allows us to compute $\lambda_\ell$ and $\mu_\ell$ in
$O(1)$ operations from the knowledge of all previous $\lambda_{\ell'}$
and $\mu_{\ell'}$, we deduce that all $\lambda_\ell$ and $\mu_\ell$,
for $\ell=-N+1,\dots,L$, can be computed by a straight-line program of
length $O(L+N)$.
\end{proof}

Taking all values of $k$ into account, we see that we can compute all
entries we need to set up the linear system $S_\ell$ using
$(\mu\,E\,N)^{O(1)}$ operations in $\KK$. After discarding the useless
equations described above, the number of equations and unknowns in the
system $S_\ell$ is polynomial in $N$, $M$ and $e_\ell$; since we saw
that $N \le M$ and $e_\ell \le \mu$, this implies that we can find a
nullspace basis of it in time $(\mu\,M)^{O(1)}$. 

Altogether, the time spent to find $\bbeta^{(\ell+1)}$ from
$(\bbeta^{(1)},\bbeta^{(2)},\dots,\bbeta^{(\ell)})=(\beta_{1},\dots,\beta_{e_\ell})$
is polynomial in $\mu\,E\,M$. Since we saw that we do at most $\mu$
such loops, the cumulated time remains polynomial in $\mu\,E\,M$, and
Proposition~\ref{prop:testisolated} is proved.

%%%%%%%%%%%%%%%%%%%%%%%%%%%%%%%%%%%%%%%%%%%%%%%%%%%%%%%%%%%%
%%%%%%%%%%%%%%%%%%%%%%%%%%%%%%%%%%%%%%%%%%%%%%%%%%%%%%%%%%%%
%%%%%%%%%%%%%%%%%%%%%%%%%%%%%%%%%%%%%%%%%%%%%%%%%%%%%%%%%%%%

\section{Symbolic homotopies}

In this section, we work over a field $\KK$ of characteristic zero
\todo{make it more general?}, using $N$ variables $\bY=(Y_1,\dots,Y_N)$. Given
polynomials $\f=(f_1,\dots,f_M)$ in $\KK[\bY]^M$, we give an algorithm
to compute a zero-dimensional parametrization of the isolated points
of $V(\f)$. Of course, we assume $M\ge N$, otherwise no such points
exist.

%%%%%%%%%%%%%%%%%%%%%%%%%%%%%%%%%%%%%%%%%%%%%%%%%%%%%%%%%%%%

\subsection{Properties of some parametric systems of equations}

Let $\bY=(Y_1,\dots,Y_N)$ be indeterminates over $\KK$, as above, and
let $T$ be a new variable. We consider polynomials
$\h=(h_1,\dots,h_M)$ in $\KK[T,\bY]$, with $M \ge N$, and the ideal $J=\langle \h
\rangle \subset \KKbar[T,\bY]$; for $\tau$ in $\KKbar$, we write
$\h_\tau=(h_{\tau,1},\dots,h_{\tau,M})=\h(\tau,\bY)\subset
\KKbar[\bY]$. In this subsection, we give some general properties of
the system $\h$, that hold under a few assumptions.  First, consider
the following properties related to the system $\h$ itself.
\begin{description}
\item[${\sf H}_1.$] Any irreducible component of $V(J) \subset
  \KKbar{}^{N+1}$ has dimension at least one (equivalently, $J$ has
  height at most $N$).
\item[${\sf H}_2.$] For any prime $P \subset\KKbar[T,\bY]$, if the
  localization $J_P \subset \KKbar[T,\bY]_P$ has height $N$, then it is
  unmixed (that is, all associated primes have height $N$).
\end{description}
An obvious example where such properties hold is when $M=N$. Then,
${\sf H}_1$ is Krull's theorem, and ${\sf H}_2$ is Macaulay's
unmixedness theorem in the Cohen-Macaulay ring
$\KKbar[T,\bY]_P$~\cite[Corollary~18.14]{Eisenbud95}. More generally,
these properties hold when $\h$ is the sequence of $p$-minors of 
a $p \times q$ matrix with entries in $\KK[\bY]$, with $N=q-p+1$ 
(the case $M=N$ corresponds to $p=1$); see~\cite[Section~6]{EN62}.

For $\tau$ in $\KKbar$, we further denote by ${\sf G}(\tau)$ the
following three properties.
\begin{description}
\item[${\sf G}_1(\tau).$] The ideal $\langle \h_\tau \rangle$ is
  radical in $\KKbar[\bY]$.
\item[${\sf G}_2(\tau).$] For $k=1,\dots,M$,
  $\deg_\bY(h_k)=\deg_\bY(h_{\tau,k})$.
\item[${\sf G}_3(\tau).$] The only common solution to
  $h_{\tau,1}^H(0,\bY)=\cdots=h_{\tau,M}^H(0,\bY)=0$ is
  $(0,\dots,0)\in\KKbar{}^N$, where for $k=1,\dots,M$, $h_{\tau,k}^H$ is
  the polynomial in $\KKbar[Y_0,\bY]$ obtained by homogenizing
  $h_{\tau,k}$ using a new variable $Y_0$. In particular, $V(\h_\tau)
  \subset \KKbar{}^N$ is finite.
\end{description}

The main result in this subsection is the following.
\begin{proposition}\label{prop:degree_fiber}
  Suppose that ${\sf H}_1$ and ${\sf H}_2$ hold. Then, there exists an
  integer $c$ such that for all $\tau$ in $\KKbar$, the sum of the
  multiplicities of the isolated solutions of $\h_\tau$ is at most
  $c$, and is equal to $c$ if ${\sf G}(\tau)$ holds.
\end{proposition}

The rest of this subsection is devoted to prove this proposition. In
the course of the proof, we will give a precise characterization
of the integer $c$, although the statement given in the proposition
will actually be enough for our further purposes. {\em In all the rest
  of this subsection, we assume that ${\sf H}_1$ and ${\sf H}_2$
  hold.}

\medskip

Consider an irredundant primary decomposition of $J$ in
$\KKbar[T,\bY]$, of the form $J=Q_1 \cap \cdots
\cap Q_r$, and let $P_1,\dots,P_r$ be the associated primes, that is,
the respective radicals of $Q_1,\dots,Q_r$. We assume that
$P_1,\dots,P_s$ are the minimal primes, for some $s \le r$, so that
$V(P_1),\dots,V(P_s)$ are the (absolutely) irreducible components of
$V(J)\subset \KKbar{}^{N+1}$. By ${\sf H}_1$, these irreducible
components all have dimension at least one. Refining further, we
assume that $t \le s$ is such that $V(P_1),\dots,V(P_t)$ are the
 irreducible components of $V(J)$ of dimension one whose
image by $\pi_T: (\tau,y_1,\dots,y_N) \mapsto \tau$ is dense in
$\KKbar$.

\begin{lemma}\label{lemma:vPi}
  Let $\tau$ be in $\KKbar$ and let $\y \in \KKbar{}^N$ be an isolated
  solution of the system $\h_\tau$. Then, $(\tau,\y)$ belongs to $V(P_i)$
  for at least one index $i$ in $\{1,\dots,t\}$, and does not belong
  to $V(P_i)$ for any index $i$ in $\{t+1,\dots,r\}$.
\end{lemma}
\begin{proof}
  Because $(\tau,\y)$ cancels $\h$, it belongs at least to one of
  $V(P_1),\dots,V(P_r)$. It remains to rule out the possibility that
  $(\tau,\y)$ belongs to $V(P_i)$ for some index $i$ in
  $\{t+1,\dots,r\}$.

  We first deal with indices $i$ in $\{t+1,\dots,s\}$. These are those
  primary components with minimal associated primes $P_i$ that either
  have dimension at least two, or have dimension one but whose image
  by $\pi$ is a single point. In both cases, all irreducible
  components of the intersection $V(P_i)\cap V(T-\tau)$ have dimension
  at least one. Since $\y$ is isolated in $V(\h_\tau)$, $(\tau,\y)$ is
  isolated in $V(\h)\cap V(T-\tau)$, so it cannot belong to
  $V(P_i)\cap V(T-\tau)$ for any $i$ in $\{t+1,\dots,s\}$.
  
  We conclude by proving that $(\tau,\y)$ does not belong to $V(P_i)$,
  for any of the embedded primes $P_{s+1},\dots,P_r$. We proceed by
  contradiction, assuming for definiteness that $(\tau,\y)$ belongs to
  $V(P_{s+1})$. Because $P_{s+1}$ is an embedded prime, $V(P_{s+1})$
  is contained in (at least) one of $V(P_1),\dots,V(P_s)$. In view of
  the previous paragraph, it cannot be one of
  $V(P_{t+1}),\dots,V(P_s)$.  Now, all of $V(P_1),\dots,V(P_t)$ have
  dimension one, so $V(P_{s+1})$ has dimension zero (so it is the point $\{(\tau,\y)\}$). For the same
  reason, if $(\tau,\y)$ belonged to another $V(P_i)$, for some $i >
  s+1$, $V(P_i)$ would also be zero-dimensional, and thus equal to $\{(\tau,\y)\}$; as a result, $V(P_i)$
  would be equal to $V(P_{s+1})$, and this would contradict the
  irredundancy of our decomposition.
  
  To summarize, $(\tau,\y)$ belongs to $V(P_{s+1})$, together with
  $V(P_i)$ for some indices $P_i$ in $\{1,\dots,t\}$ (say
  $P_1,\dots,P_u$, up to reordering, for some $u \ge 1$), and avoids
  all other associated primes.  Let us localize the decomposition
  $J=Q_1 \cap \cdots \cap Q_r$ at
  $P_{s+1}$. By~\cite[Proposition~4.9]{AtMc},
  $J_{P_{s+1}}={Q_1}_{P_{s+1}} \cap \cdots \cap {Q_u}_{P_{s+1}}\cap
  {Q_{s+1}}_{P_{s+1}}$ is an irredundant primary decomposition of
  $J_{P_{s+1}}$ in $\KKbar[T,\bY]_{P_{s+1}}$; the minimal primes are
  ${P_1}_{P_{s+1}},\dots,{P_u}_{P_{s+1}}$.

  By Corollary~4 p.24 in~\cite{Matsumura86}, for any prime
  ${P_i}_{P_{s+1}}$, $i=1,\dots,u$ or $i=s+1$, the localization of
  $\KKbar[T,\bY]_{P_{s+1}}$ at ${P_i}_{P_{s+1}}$ is equal to
  $\KKbar[T,\bY]_{P_{i}}$. In particular, the height of ${P_i}_{P_{s+1}}$
  in $\KKbar[T,\bY]_{P_{s+1}}$ is equal to that of $P_i$ in
  $\KKbar[T,\bY]_{P_{i}}$, that is, $N$ if $i=1,\dots,u$, since then
  $V(P_i)$ has dimension $1$, or $N+1$ if $i=s+1$. Since $u \ge 1$,
  this proves that $J_{P_{s+1}}$ has height $N$. As a result, ${\sf
    H}_2$ implies that $J_{P_{s+1}}$ is unmixed, a contradiction.
  %% The ring $\KK[T,\bY]_{P_{s+1}}$ is
  %% Noetherian~\cite[Lemma~10.30.1]{stacks-project} and
  %% Cohen-Macaulay~\cite[Proposition~18.8]{Eisenbud95}. Hence, the
  %% unmixedness theorem~\cite[Theorem~17.6]{Matsumura86} holds in
  %% $\KK[T,\bY]_{P_{s+1}}$. Since $J$ is generated by the $N$ polynomials
  %% $\h$, this is still the case for
  %% $J_{P_{s+1}}$~\cite[Corollary~3.4]{AtMc}; using this, and the fact
  %% that $J_{P_{s+1}}$ has height $N$, the unmixedness theorem proves
  %% that $J_{P_{s+1}}$ is unmixed, a contradiction.  
\end{proof}

Let us write $J=J' \cap J''$, with $J'=Q_1 \cap \cdots \cap Q_t$ and
$J''=Q_{t+1} \cap \cdots \cap Q_r$. For $\tau$ in $\KKbar$, we denote
by $J_\tau \subset \KKbar[T,\bY]$ the ideal $J + \langle T-\tau \rangle$,
and similarly for $J'_\tau$ and $ J''_\tau$.

\begin{lemma}\label{lemma:JJprime}
  Let $\tau$ and $\y$ be as in Lemma~\ref{lemma:vPi}. Then, the
  multiplicities of the ideals $J_\tau$ and $J'_\tau$ at $(\tau,\y)$
  are the same.
\end{lemma}
\begin{proof}
  Without loss of generality, assume that $\tau=0 \in \KKbar$ and
  $\y=0 \in \KKbar{}^N$. We start from the equality $J=J' \cap J''$,
  which holds in $\KKbar[T,\bY]$, and we see it in $\KKbar[[T,\bY]]$.  The
  previous lemma implies that there exists a polynomial in $J''$ that
  does not vanish at $(\tau,\y)=0 \in \KKbar{}^{N+1}$.  This polynomial
  is a unit in $\KKbar[[T,\bY]]$, which implies that the extension of
  $J''$ in $\KKbar[[T,\bY]]$ is the trivial ideal $\langle 1 \rangle$, and
  finally that the equality of extended ideals $J=J'$ holds in
  $\KKbar[[T,\bY]]$. This implies the equality $J+\langle T \rangle
  =J'+\langle T \rangle $ in $\KKbar[[T,\bY]]$, and the conclusion
  follows.
\end{proof}

%% \begin{lemma}
%%   For $\tau$ in $\KKbar$, the ideal $J_\tau$ has dimension zero.
%% \end{lemma}
%% \begin{proof}
%%   The solutions $(\tau,\y)$ of $J_\tau$ are the unions of the
%%   solutions of the ideal $P_i + \langle T-\tau \rangle$, for
%%   $i=1,\dots,t$. Each such variety $V(P_i)$ is irreducible of
%%   dimension one, and its image by $\pi_T$ is dense in $\KKbar$.
%%   Hence, $V(P_i) \cap V(T-\tau)$ is a strict algebraic subset of
%%   $V(P_i)$, and by $P_i$'s irreducibility, it must have dimension
%%   zero.
%% \end{proof}

Our goal is now to give a bound on the sum of the multiplicites of
$\h_\tau$ at all its isolated roots, for any $\tau$ in $\KKbar$.  To
achieve this, we introduce $\frak{J}$, the extension of $J$ in
$\KKbar(T)[\bY]$, and similarly $\frak{J}'$ and ${\frak J}''$.

\begin{lemma}
  The ideal $\frak{J}'$ has dimension zero and $V(\frak{J}') \subset
  \overline{\KK(T)}{}^N$ is the set of isolated solutions of
  $V(\frak{J}) \subset \overline{\KK(T)}{}^N$.
\end{lemma}
\begin{proof}
 From the equality $J=J' \cap J''$ and Corollary~3.4 in~\cite{AtMc},
 we deduce that $\frak{J}=\frak{J'} \cap \frak{J''}$; the properties
 of $J'$ (the irreducible components of $V(J')$ are precisely those
 irreducible components of $V(J)$ that have dimension one and with a
 dense image by $\pi_T$) imply our claim.
\end{proof}


%% By the previous lemma, we have the equality
Let us write $c=\dim_{\KKbar(T)}(\KKbar(T)[\bY]/{\frak J}')$.  The
following lemma relates this quantity to the multiplicities of the
solutions in any fiber $\h_\tau$. This proves the first statement
in Proposition~\ref{prop:degree_fiber}.

\begin{lemma}\label{lemma:19}
  Let $\tau$ be in $\KKbar$. The sum of the multiplicities of the
  isolated solutions of $\h_\tau$ is at most equal to $c$.
\end{lemma}
\begin{proof}
  The sum in the lemma is also the sum of the multiplicities of the
  ideal $J_\tau$ at all $(\tau,\y)$, for $\y$ an isolated solution of
  $\h_\tau$.  By Lemma~\ref{lemma:JJprime}, this is also the sum of
  the multiplicities of $J'_\tau$ at all $(\tau,\y)$, for $\y$ an
  isolated solution of $\h_\tau$. We prove below that the sum of the
  multiplicities of $J'_\tau$ at all $(\tau,\y)$, for $\y$ such that
  $(\tau,\y)$ cancels $J'_\tau$, is at most $c$; this will be enough
  to conclude (for any isolated solution $\y$ of $\h_\tau$,
  $(\tau,\y)$ is a root of $J'_\tau$, though the converse may not be
  true).
  
  Let $m_1,\dots,m_\mu$ be monomials that form a $\KKbar$-basis of
  $\KKbar[T,\bY]/J'_\tau$; since $T-\tau$ is in $J'_\tau$, these
  monomials can be assumed not to involve $T$.  We will prove that
  they are still $\KKbar(T)$-linearly independent in
  $\KKbar(T)[\bY]/{\frak J}'$; this will imply that $\mu \le c$,
  and finish the proof of the first statement.
  
  Suppose that there exists a linear combination $A_1 m_1 + \cdots +
  A_\mu m_\mu$ in ${\frak J}'$, with all $A_i$'s in $\KKbar(T)$, not
  all of them zero. Thus, we have an equality $a_1/d_1\, m_1 + \cdots
  + a_\mu/d_\mu\, m_\mu = a/d$, with $a_1,\dots,a_\mu$ and
  $d,d_1,\dots,d_\mu$ in $\KKbar[T]$ and $a$ in the ideal
  $J'$. Clearing denominators, we obtain a relation of the form $b_1
  m_1 +\cdots+ b_\mu m_\mu \in J'$, with not all $b_i$'s zero. Let
  $(T-\tau)^e$ be the highest power of $T-\tau$ that divides all
  $b_i$'s (this is well-defined, since not all $b_i$'s vanish) so that
  we can rewrite the above as $(T-\tau)^e (c_1 m_1 +\cdots+ c_\mu
  m_\mu) \in J'$, with $c_i=b_i/(T-\tau)^e \in \KKbar[T]$ for all $i$.
  In particular, our definition of $e$ implies that the values
  $c_i(\tau)$ are not all zero.

  %% Recall that the ideal $J'$ has the form $J'=Q_1 \cap \cdots \cap
  %% Q_t$.  We claim that all $Q_i$'s are actually prime. Indeed, their
  %% extensions in $\KKbar(T)[\bY]$ are (because $\mathfrak{J}'$ is radical),
  %% so that for all $i \le t$ we have $Q_i \cdot \KKbar(T)[\bY] = P_i \cdot \KKbar(T)[\bY]$ (this
  %% is~\cite[Proposition~3.11.v]{AtMc}).  Now, neither $Q_i$ nor $P_i$
  %% contains any non-zero polynomial in $\KKbar[T]$, and these ideals are
  %% respectively primary and prime; as a
  %% result,~\cite[Proposition~2.2.b]{Eisenbud95} implies that $Q_i$ and
  %% $P_i$ are the contractions of respectively $Q_i \cdot \KKbar(T)[\bY]$
  %% and $P_i \cdot \KKbar(T)[\bY]$, which are equal. Hence, $Q_i=P_i$, and
  %% we will write $J'=P_1 \cap \cdots \cap P_t$.


  Recall that the ideal $J'$ has the form $J'=Q_1 \cap \cdots \cap
  Q_t$. For $i=1,\dots,t$, since $Q_i$ is primary, the membership
  equality $(T-\tau)^e (c_1 m_1 +\cdots +c_\mu m_\mu) \in J'$ implies
  that either $c_1 m_1 +\cdots +c_\mu m_\mu$ or some power
  $(T-\tau)^{ef}$, for some $f > 0$, is in $Q_i$. Since $Q_i$ does not
  contain non-zero polynomials in $\KKbar[T]$, $c_1 m_1 +\cdots+ c_\mu
  m_\mu$ belongs to all $Q_i$'s, that is, to $J'$. We can then
  evaluate this relation at $T=\tau$. We saw that the values
  $c_i(\tau)$ do not all vanish on the left, which is a contradiction
  with the independence of the monomials $m_1,\dots,m_\mu$ modulo
  $J'_\tau$.
\end{proof}


We now take $\tau$ in $\KKbar$ and we discuss the geometry of $V(J)$
near $\tau$; without loss of generality, we suppose that $\tau=0$.
The field of Puiseux series $\KKbar\langle\langle T \rangle\rangle$
contains an algebraic closure of $\KKbar(T)$; we thus let
$\Phi_1,\dots,\Phi_{c'}$ be the points of $V(\mathfrak{J}')$, with
coordinates taken in $\KKbar\langle\langle T \rangle\rangle$. In
particular, we see that $c' \le c$; we prove below that we actually
have $c'=c$.

Any Puiseux series $\varphi$ in $\KKbar\langle\langle T
\rangle\rangle$ admits a well-defined {\em valuation} $\nu(\varphi)$,
which is the smallest exponent that appears in its expansion; the
valuation $\nu(\Phi)$, for a vector $\Phi=(\varphi_1,\dots,\varphi_s)$
with entries in $\KKbar\langle\langle T \rangle\rangle$, is the
minimum of the valuations of its exponents. We say that $\Phi$ is {\em
  bounded} if it has non-negative valuation; in this case,
$\lim_0(\Phi)$ is defined as the vector
$(\lim_0(\varphi_1),\dots,\lim_0(\varphi_s))$, with
$\lim_0(\varphi_i)={\rm coeff}(\varphi_i,T^0)$ for all $i$.

Without loss of generality, we assume that
$\Phi_1,\dots,\Phi_\kappa$ are bounded, and
$\Phi_{\kappa+1},\dots,\Phi_{c'}$ are not, for some $\kappa$ in
$\{0,\dots,c'\}$, and we define $\varphi_1,\dots,\varphi_\kappa$ by
$\varphi_i=\lim_0(\Phi_i)\in\KKbar{}^N$ for
$i=1,\dots,\kappa$.




\begin{lemma}\label{lemma:Z1}
  The equality $V(J +\langle T \rangle)=\{\varphi_i \mid i=1,\dots,\kappa\}$ holds.
\end{lemma}
\begin{proof}
  Let $B_1,\dots,B_R$ be generators of the ideal $J$ in
  $\KK[T,\bY]$, so that these polynomials also generate
  $\mathfrak{J}$ in $\KK(T)[\bY]$; then, the polynomials
  $b_i=B_i(0,\bY) \in \KK[\bY]$, for $i=1,\dots,R$, are such that
  $J+\langle T\rangle = \langle T,h_1,\dots,h_R \rangle$.  Consider
  $i \le \kappa$, and the corresponding vector of Puiseux series
  $\Phi_i$. We know that for $j=1,\dots,R$, we have $B_j(\Phi_i)=0$.
  Since all elements involved have non-negative valuation, we can take
  the coefficient of degree $0$ in $T$ in this equality and deduce
  $b_j(\varphi_i)=0$, as claimed. Hence, each $\varphi_i$, for $i \le
  \kappa$, is in $V(J + \langle T \rangle)$.

  Conversely, take indeterminates $T_1,\dots,T_N$, and let $\LL$ be
  the algebraic closure of the field $\KKbar(T_1,\dots,T_N)$; let
  $W\subset{\LL}{}^{N+1}$ be the zero-set of the ideal $J\cdot
  \LL[T,\bY]$ and consider the projection $W \to {\LL}{}^2$ defined
  by $(\tau,y_1,\dots,y_N)\mapsto (\tau,T_1 y_1 + \cdots + T_N y_N)$. The
  Zariski closure $S$ of the image of this mapping is a
  hypersurface. %% On the other hand, because $J$ is radical
  %% \todo{where?}  in $\KK[T,\bY]$, it remains so in $\KK[T,\bY]$
  %% \todo{why?}.
  Since
  the ideal $J$ is generated by polynomials with coefficients in $\KK$,
  one deduces that $S$ admits a squarefree defining equation in
  $\KK(T_1,\dots,T_N)[T,T_0]$.

  Consider such a polynomial, say $C$, and assume without loss of
  generality that $C$ belongs to 
  $\KK[T_1,\dots,T_n][T,T_0]$. Because $J$ admits no irreducible
  component lying above $T=\tau$, for any $\tau$ in $\KKbar$, $C$
  admits no factor in $\KK[T]$; thus, $C(0,T_0)$ is non-zero.

  Let $\ell \in \KK[T_1,\dots,T_N,T]$ be the leading coefficient of
  $C$ with respect to $T_0$. Proposition~1 in~\cite{Schost03} proves
  that $C/\ell$, seen in $\KK(T_1,\dots,T_N,T)[T_0] \subset
  \LL(T)[T_0]$, is the minimal polynomial of $T_1 Y_1 + \cdots +
  T_N Y_N$ in $\LL(T)[\bY]/J\cdot \LL(T)[\bY]$. The latter ideal
  is also the extension of $\mathfrak{J}$ to $\LL(T)[\bY]$, 
  so $C/\ell$ factors as
  $$\frac C\ell = \prod_{1\le i \le c'}(T_0-T_1 \Phi_{i,1} - \cdots - T_N \Phi_{i,N})$$
  in $\LL\langle\langle T \rangle\rangle[T_0]$.
  This gives the equality 
  $$C =\ell \prod_{1\le i \le  c'}(T_0-T_1 \Phi_{i,1} - \cdots - T_N
  \Phi_{i,N})$$ over $\KKbar\langle\langle T \rangle\rangle[T_1,\dots,T_N,T_0]$. 

  Let us extend the valuation $\nu$ on $\KKbar\langle\langle T\rangle\rangle$
to $\KKbar\langle\langle T \rangle\rangle[T_1,\dots,T_N,T_0]$ in the
  direct manner, by setting $\nu(\sum_\alpha f_\alpha T_0^{\alpha_0}
  \cdots T_N^{\alpha_N}) = \min_\alpha \nu(f_\alpha)$. The fact that
  $C$ has no factor in $\KK[T]$ implies that
  $\nu(C)=0$. Using Gauss' Lemma, we see that the valuation of the
  right-hand side is $\nu(\ell) + \sum_{\kappa < i \le c}\mu_i$, with $\mu_i= \nu(\Phi_i)$ for all $i$;
  note that $\mu_i < 0$ for $i > \kappa$. Thus, we can
  rewrite
  $$C =\left ({T}^{-\nu(\ell)} \ell\right ) 
  \prod_{1 \le i \le \kappa}(T_0-T_1 \Phi_{i,1} - \cdots - T_N  \Phi_{i,N} )
  \prod_{\kappa < i \le c'} ({T}^{-\mu_i}T_0-{T}^{-\mu_i}T_1 \Phi_{i,1} - \cdots - {T}^{-\mu_i}T_N  \Phi_{i,N} ),$$
  where all terms appearing above have non-negative valuation.
  As a result, we can take the coefficient of ${T}^0$ term-wise,
  and obtain
  $$C(0,T_0) = s \prod_{1 \le i \le \kappa}(T_0-T_1 \varphi_{i,1} -
  \cdots - T_N \varphi_{i,N} ),$$ where $s$ is in $\KKbar[T_1,\dots,T_N]$;
  note that $s \ne 0$, since $C(0,T_0)$ is non-zero.
 By construction of $C$, for any
  $\by=(y_1,\dots,y_N)$ in $V(J+\langle T \rangle)$, $T_1 y_1 + \cdots + T_N y_N$
  cancels $C(0,T_0)$, so $\by$ must be one of
  $\varphi_1,\dots,\varphi_{\kappa}$.
\end{proof}

To conclude the proof of Proposition~\ref{prop:degree_fiber}, 
we now assume that property ${\sf G}(0)$ holds.

\begin{lemma}
   $\Phi_1,\dots,\Phi_{c'}$ are bounded.
\end{lemma}
\begin{proof}
  For $i=1,\dots,c'$, write $\Phi_i=1/T^{e_i}
  (\Psi_{i,1},\dots,\Psi_{i,N})$, for a vector
  $(\Psi_{i,1},\dots,\Psi_{i,N})$ of Puiseux series of valuation
  zero, that is, such that all $\Psi_{i,j}$ are bounded and
  $(\psi_{i,1},\dots,\psi_{i,N})=\lim_0(\Psi_{i,1},\dots,\Psi_{i,N})$
  is nonzero. Hence,
  $e_i=-\nu(\Phi_i)$, and we have to prove that $e_i \le 0$.  By way
  of contradiction, we assume that $e_i > 0$.

  The Puiseux series $\Phi_i$ cancels $h_1,\dots,h_M$. For
  $k=1,\dots,M$, let $h_k^H \in \KKbar[T][Y_0,\bY]$ be the homogenization
  of $h_k$ with respect to $\bY$. From the equality
  $h_k^H(T^{e_i},\Psi_{i,1},\dots,\Psi_{i,N})=
  T^{e_i}h_k(\Phi_i)$, we deduce that
  $h_k^H(T^{e_i},\Psi_{i,1},\dots,\Psi_{i,N})=0$ for all $k$. We
  can write $h_k = h_{0,k} + T \tilde h_k$, for some polynomial
  $\tilde h_k$ in $\KKbar[T,\bY]$, and ${\sf G}_2(0)$ implies that
  $\deg_\bY(\tilde h_k) \le \deg_\bY(h_{0,k})$. As a result, the
  homogenizations (with respect to $\bY$) of $h_{k},h_{0,k}$ and $\tilde
  h_k$ satisfy a relation of the form $h^H_k = h_{0,k}^H +
  Y_0^{\delta_k} T \tilde h^H_k$, for some $\delta_k \ge 0$. This
  implies the equality
  $$h_{0,k}^H(T^{e_i},\Psi_{i,1},\dots,\Psi_{i,N}) + T^{\delta_k
    e_i+1}\tilde h_k^H(T^{e_i},\Psi_{i,1},\dots,\Psi_{i,N})=0.$$
  The second term has positive valuation, so that
  $h_{0,k}^H(T^{e_i},\Psi_{i,1},\dots,\Psi_{i,N})$ has positive
  valuation as well. Taking the coefficient of $T^0$, this means 
  that $h_{0,k}^H(0,\psi_{i,1},\dots,\psi_{i,N})=0$ (since $e_i > 0$), which implies 
  that $(\psi_{i,1},\dots,\psi_{i,N})=(0,\dots,0)$, in view of ${\sf G}_3(0)$.
  This however contradicts the definition of $(\psi_{i,1},\dots,\psi_{i,N})$.
\end{proof}

%% For $i=1,\dots,c'$, we define $\varphi_i =
%% (\varphi_{i,1},\dots,\varphi_{i,N})$ as $\varphi_i=\lim_0(\Phi_i)\in
%% \KKbar{}^N$. In particular, all $\varphi_i$, $i=1,\dots,c'$, are roots
%% of $\h_0$.

\begin{lemma}\label{lemma:Jprimerad}
  The ideal $\frak{J}'$ is radical; equivalently, $c'=c$.
\end{lemma}
\begin{proof}
 We know that $\frak{J}'$ has dimension zero, so it is enough to prove
 that for $i=1,\dots,c'$, the localization of $\KKbar\langle\langle T
 \rangle\rangle[\bY]/\frak{J}'$ at the maximal ideal
 $\mathfrak{m}_{\Phi_i}$ is a field, or equivalently that the
 localization of $\KKbar\langle\langle T \rangle\rangle[\bY]/\frak{J}$
 at $\mathfrak{m}_{\Phi_i}$ is a field.  By the Jacobian
 criterion~\cite[Theorem~16.19.b]{Eisenbud95}, this is the case if and
 only if the Jacobian matrix of $\h$ with respect to $\bY$ has full
 rank $N$ at $\Phi_i$. We know that $\varphi_i=\lim_0(\Phi_i)$ is a
 root of $\h_0$, and the Jacobian criterion conversely implies that
 since the ideal $\langle \h_0 \rangle$ is radical (by assumption
 ${\sf G}_1(0)$) and zero-dimensional (by assumption ${\sf G}_3(0)$),
 the Jacobian matrix of $\h_0(\bY)=\h(0,\bY)$ has full rank $N$. Since
 this matrix is the limit at zero of the Jacobian matrix of $\h$ with
 respect to $\bY$, taken at $\Phi_i$, the latter must have full rank
 $N$, and our claim that $\frak{J}'$ is radical is proved.
\end{proof}

To finish the proof of Proposition~\ref{prop:degree_fiber}, we have to
establish that $V(\h_0)$ consists of exactly $c$ solutions.  Let thus
$d$ be the number of points in $V(\h_0)$.  Since $\langle \h_0
\rangle$ is radical (this is ${\sf G}_1(0)$), Lemma~\ref{lemma:19}
implies that $d \le c$, so we only have to prove that $c \le d$. To
prove this, we prove that for $i,i'$ in $\{1,\dots,c\}$, with $i \ne
i'$, we have $\varphi_i \ne \varphi_{i'}$.

Suppose to the contrary that $\varphi_i = \varphi_{i'}$. We know that
the Jacobian matrix of $\h_0$ has full rank $N$ at $\varphi_i$; up to
reindexing, we assume that rows $1,\dots,N$ correspond to a maximal
nonzero minor. Let $\h'=(h_1,\dots,h_N)$.

Let $m=\nu(\Phi_i-\Phi_{i'})$; since  $\varphi_i = \varphi_{i'}$, we have
$m > 0$. We can thus write $\Phi_i=f + T^m
\delta_i$ and $\Phi_{i'}=f + T^m \delta_{i'}$, for some vectors of
bounded Puiseux series $f, \delta_i, \delta_{i'}$ such that all terms
in $f$ have valuation less than~$m$; in addition, $\lim_0(\delta_i)
\ne \lim_0(\delta_{i'})$. Write the Taylor expansion of $\h'$ at $f$ as
$$\h'(\Phi_i) = \h'(f) + \jac_f(\h',\bY) T^m \delta_i + T^{2m} r_i =0$$
and
$$\h'(\Phi_{i'}) = \h'(f) + \jac_f(\h',\bY) T^m \delta_{i'} + T^{2m}
r_{i'} =0,$$ for some vectors of bounded Puiseux series $r_i,r_{i'}$.
By subtraction and division by $T^m$, we obtain $\jac_f(\h',\bY)
(\delta_i-\delta_{i'}) = T^m r$, for some vector of bounded Puiseux
series $r$.  Since $\jac_f(\h',\bY)$ is invertible, this further gives
$\delta_i-\delta_{i'} = T^m r'$, where again $r'$ is a vector of
bounded Puiseux series.  However, by construction the left-hand side
has valuation zero, while the right-hand side has positive valuation
(since $m > 0$). Hence, we derived a contradiction to our assumption
that $\varphi_i = \varphi_{i'}$. The proof of Proposition~\ref{prop:degree_fiber} is
complete. (Although we do not need it now, the linearization
used above also implies that all $\Phi_i$ are actually power series.)

%%%%%%%%%%%%%%%%%%%%%%%%%%%%%%%%%%%%%%%%%%%%%%%%%%%%%%%%%%%%

\subsection{Computing the isolated solutions of $\f$} 

We now give our algorithm to compute the isolated solutions of a
polynomial system $\f=(f_1,\dots,f_M)$. We suppose that there exists
$\h=(h_1,\dots,h_M)$ in $\KK[T,\Y]^M$ such that:
\begin{itemize}
\item $\f=\h_1=(h_i(1,\bY))_{1 \le i \le M}$.
\item $\h$ satisfies assumptions ${\sf H}_1$ and ${\sf H}_2$ of the previous subsection.
\item We are given $\tau$ in $\KK$ such that ${\sf G}(\tau)$ holds;
  without loss of generality, we assume that $\tau=0$. We also 
  suppose that we know a description of $V(\h_{0})$ by means of a
  zero-dimensional parametrization
  $\scrR_{0}=((q_{0},v_{0,1},\dots,v_{0,N}),\lambda)$ with
  coefficients in $\KK$. The linear form $\lambda$ needs to satisfy 
  some genericity requirements, that are described below.
\end{itemize}
All the notation of the previous subsection is still in use. In
particular, by Proposition~\ref{prop:degree_fiber}, the degree of
$\scrR_{0}$ is the integer $c$ introduced there. In order to control
the cost of the algorithm, we introduce the following assumptions:
\begin{description}
\item[${\sf A}_1$.] Given $\by \in V(\h_0)$ having coordinates in a field
  extension $\LL$ of $\KK$, we can find in time $(T
  [\LL:\KK])^{O(1)}$ a sequence $\bi_\by=(i_1,\dots,i_N)$, with
$1 \le i_1 < \dots < i_N \le M$, such that the
  Jacobian matrix of $(h_{0,i_1},\dots,h_{0,{i_N}})$ has full
  rank $N$ at $\by$, for some $T$ independent of $\by$.
\item[${\sf A}_2$.] We know an integer $e$ such that the curve $V(J')$
  has degree at most $e$. Lemma~\ref{lemma:vPi} implies that 
  $c \le e$.
\item[${\sf A}_3$.] For any $\bi=(i_1,\dots,i_N)$, with $1 \le i_1 <
  \dots < i_N \le M$, we can compute $(h_{i_1},\dots,h_{{i_N}})$ using
  a straight-line program of length $E$.
\item[${\sf A}_4$.] We can compute $\h$ using a straight-line program
  of length $E' \ge E$.
\end{description}

\begin{proposition}\label{prop:compute_isolated}
  Using the previous assumptions, we can compute a zero-dimensional
  parametrization of the isolated points of 
 of $V(\f)$ using $(T c + e\,E'\,M)^{O(1)}$ operations in~$\KK$.
\end{proposition}


\paragraph{Decomposing $\scrR_0$.}
We start by decomposing $\scrR_0$ into finitely many zero-dimensional
parametrizations
$\scrR_{0,j}=((q_{0,j},v_{0,j,1},\dots,v_{0,j,N}),\lambda)_{1\le j\le
  t}$, all with coefficients in $\KK$, such that for $j$ in
$\{1,\dots,t\}$, there exist $\bi_j=(i_{j,1},\dots,i_{j,N})$ such that
$\bi_\by=\bi_j$ for all $\y$ in $Z(\scrR_{0,j})$. In other words, for
all points described by $\scrR_{0,j}$, the jacobian of $(h_{0,i})_{i
  \in \bi_j}$ has full rank $N$ at $\y$, for all $\y$ in
$Z(\scrR_{0,j})$. This is done by applying the algorithm of assumption
${\sf A}_1$ to the point $(v_{0,1}/q_0',\dots,v_{0,N}/q_0')$ that has
coordinates in the ring $\LL=\KK[T]/\langle q_0 \rangle$.

If the algorithm goes through, we have obtained our answer. However,
$\LL$ may not be a field.  One workaround would be to factor $q_0$,
but we do not want our runtime to depend on the cost of factoring
polynomials. Hence, we will use {\em dynamic evaluation techniques},
as in~\cite{D5}. Indeed, the only issue that may arise is that we
attempt to invert a zero-divisor. If this is the case, it means we
have found a non-trivial factor $r_0$ of $q_0$: we can then replace
$\scrR_0$ by two new zero-dimensional parametrizations,
$\scrR'_0=((r_0,(v_1 s_0) \bmod r_0,\dots,(v_N s_0)\bmod
r_0),\lambda)$ and $\scrR''_0=((s_0,(v_1 r_0) \bmod s_0,\dots,(v_N
r_0)\bmod s_0),\lambda)$, with $s_0=q_0/r_0$ \todo{check the
  formulas!}, that define a partition of $\Zeroes(\scrR_0)$ into the
subsets where $r_0$ vanishes, resp.\ is non-zero. 

We can then start over again, from $\scrR'_0$ and $\scrR''_0$
independently. The runtime assumption in ${\sf A}_1$ implies that the
overall time spent in this step is $(T c)^{O(1)}$.

\paragraph{Lifting power series and rational reconstruction.}
For $j=1,\dots,t$, we can then apply Newton iteration to the system
$(h_i)_{i \in \bi_j}$ to lift
$\scrR_{0,j}=((q_{0,j},v_{0,j,1},\dots,v_{0,j,N}),\lambda)$ into a
zero-dimensional parametrization
$\scrR_{j}=((q_{j},v_{j,1},\dots,v_{j,N}),\lambda)$ with coefficients
in $\KK[[T]]/\langle T^{2e}\rangle$, for $e$ as in ${\sf A}_2$.

As explained in~\cite[Section~2.2]{SaSc16}, using the algorithm
of~\cite{GiLeSa01}, this can be done using $(e\,E\,N)^{O(1)}$
operations in $\KK$. Using the Chinese Remainder Theorem, we can
combine all $\scrR_{j}$ into a single zero-dimensional parametrization
$\scrR$ with coefficients in $\KK[[T]]/\langle T^{2d}\rangle$,
since for $j\ne j'$ $q_{0,j}$ and $q_{0,j'}$ generate the unit ideal
in $\KK[[T]]/\langle T^{2e}\rangle$; this takes time
$(e\,N)^{O(1)}$.

Using the notation of the previous subsection, the zeros of $\scrR$ in
$\KKbar[[T]]/\langle T^{2d}\rangle$ are the truncations of the power
series roots $\Phi_1,\dots,\Phi_c$ of $\mathfrak{J}'$. Since $J'$ is
supposed to have degree at most $e$, knowing $\scrR$ at precision $2e$
allows us to reconstruct a zero-dimensional parametrization $\scrS$
with coefficients in $\KK(\tau)$ such that
$Z(\scrS)=V(\mathfrak{J}')$, with all coefficients having numerator
and denominator of degree at most $e$~\cite[Theorem~1]{Schost03}.
This is done by applying rational function reconstruction to all
coefficients of $\scrR$, as in~\cite{Schost03}, and takes time
$(e\,N)^{O(1)}$.


\paragraph{A finite set containing the isolated points of $V(\f)$.}
As we did in the previous subsection for $T=0$, we let
$\Phi'_1,\dots,\Phi'_c$ be the roots of $\mathfrak{J}'$ in the field
of Puiseux series $\KKbar\langle\langle T'\rangle\rangle$ at $T=1$,
with $T'=T-1$. Without loss of generality, we assume that
$\Phi'_1,\dots,\Phi'_{\kappa'}$ are bounded, and
$\Phi'_{\kappa'+1},\dots,\Phi'_c$ are not, for some $\kappa'$ in
$\{0,\dots,c\}$, and we let $\varphi'_1,\dots,\varphi'_\kappa$ by
$\varphi'_i=\lim_0(\Phi'_i)\in\KKbar{}^N$ for $i=1,\dots,\kappa'$.  By
Lemma~\ref{lemma:Z1}, $V(J' + \langle T-1\rangle) = \{ \varphi'_i \mid
i=1,\dots,k\}$.

We can now specify our requirements on the linear form $\lambda$.
Following~\cite{RRS} and~\cite{SaSc16}, we ask that $\lambda$ is a {\rm well-separating
  element},
\begin{enumerate}
\item $\lambda$ is separating for $V(\mathfrak{J}')=\{\Phi'_1,\dots,\Phi'_c\}$;
\item $\lambda$ is separating for $V(J' + \langle T-1\rangle) = \{ \varphi'_1,\dots,\varphi'_{\kappa'}\}$.
\item $\nu(\lambda(\Phi_i)) = \mu_i$ for all $i=1,\dots,c$, where $\nu$ denotes
 the $T'$-adic valuation.
\end{enumerate}
These conditions are satisfied for a generic choice of $\lambda$, as
showed in~\cite{SaSc16}. When this is the case, Lemma~4.4
in~\cite{RRS} shows how to recover a zero-dimensional parametrization
$\scrR_1=((q_1,v_{1,1},\dots,v_{1,N}),\lambda)$ with coefficients in
$\KK$ for the limit set $V(J' + \langle T-1\rangle) =\{\varphi'_i \mid i=1,\dots,\kappa\}$
starting from $\scrS$, in time $(e\,N)^{O(1)}$.

\paragraph{Cleaning.}
Lemma~\ref{lemma:vPi} implies that for any isolated solution $\y$ of
$\f$, $(1,\y)$ is in $V(J' + \langle T-1\rangle)$, so in a second
time, we discard from $V(J'_1)$ those points that do not correspond to
isolated points of $V(\f)$. All such points belong to a
positive-dimensional component of $V(\f)$, so we can use the algorithm
of Section~\ref{sec:isolated}. By Proposition~\ref{prop:degree_fiber},
we can take $c$ as an upper bound on the multiplicity of isolated
solutions of $\f$ (in the cost analysis below, we simply use $e$, 
since $e \ge c$).

Using the same dynamic evaluation techniques as in the first paragraph
above, we can use the algorithm of Section~\ref{sec:isolated} as if
$Z(\scrR_1)$ were an irreducible variety, with an overhead polynomial
in $c$. The runtime deduced from Proposition~\ref{prop:testisolated}
is $(e\,E'\,M)^{O(1)}$ operations in~$\KK$, where $E'$ is from ${\sf
  A}_4$. Adding all costs seen so far, we prove
Proposition~\ref{prop:compute_isolated}.

\bibliographystyle{plain} \bibliography{roadmap}

\end{document}
