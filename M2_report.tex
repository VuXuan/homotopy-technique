\documentclass[11pt]{article}
\usepackage[utf8]{inputenc}
\usepackage{amsmath,amsxtra,amssymb,latexsym, amscd,amsthm,eufrak, amsfonts, mathrsfs, bm, algorithm, pseudocode}
% \usepackage[pdftex]{color,graphicx}
\usepackage{pgf,tikz}
\usepackage[T5,T1]{fontenc}
\usepackage[T5]{fontenc}
\usepackage{cleveref}
\usetikzlibrary{arrows}
\usepackage{graphicx}
%\usepackage[]{algorithm2e}
%\usepackage{eucal}
\usepackage[left=3cm,right=2.3cm,top=2.5cm,bottom=2.5cm]{geometry}
%\usepackage{program}
\usepackage[all]{xy}
\UseComputerModernTips

 \usepackage[nottoc]{tocbibind}
\newcommand*\DNA{\textsc{dna}}



\newcommand{\be}{\begin{equation}}
\newcommand{\ee}{\end{equation}}
\newcommand{\beq}{\begin{eqnarray}}
\newcommand{\eeq}{\end{eqnarray}}
\newcommand{\beqn}{\begin{eqnarray*}}
\newcommand{\eeqn}{\end{eqnarray*}}
\newcommand{\parag}{\bigskip\noindent}

\newtheorem{pbm}{Problem}
\newtheorem{Property}{Property}
\numberwithin{Property}{section}
\newtheorem{Theorem}{Theorem}%[section]
\numberwithin{Theorem}{section}
\newtheorem{Proposition}{Proposition}%[section]
\numberwithin{Proposition}{section}
\newtheorem{Lemma}{Lemma}%[section]
\numberwithin{Lemma}{section}
\newtheorem{Corollary}{Corollary}%[section]
\numberwithin{Corollary}{section}
\newtheorem{Definition}{Definition}%[section]
\numberwithin{Definition}{section}
\newtheorem{Remark}{Remark}%[section]
\numberwithin{Remark}{section}
\newtheorem{Conjecture}{Conjecture}%[section]
\numberwithin{Conjecture}{section}
\newtheorem{Problem}{Problem}%[section]
\numberwithin{Problem}{section}
%\newtheorem{Example}{Example}%[section]
%\numberwithin{Example}{section}
\newtheorem{Claim}{Claim}
\numberwithin{Claim}{section}
\newtheorem{Question}{Question}%[section]
\newtheorem{Subsec}[Theorem]{}
\newtheorem{Note}[Theorem]{}
\renewcommand{\theTheorem}{\arabic{section}.\arabic{Theorem}}

\theoremstyle{definition}
\newtheorem{Example}{Example}%[section]
\numberwithin{Example}{section}

\def\bell{\mbox{\boldmath$\ell$}}
\def\mult{\ensuremath{\mathrm{mult}}}
\def\y {\ensuremath{\mathbf{y}}}
\def\e {\ensuremath{\mathbf{e}}}
\def\a {\ensuremath{\mathbf{a}}}
\def\b {\ensuremath{\mathbf{b}}}
\def\z {\ensuremath{\mathbf{z}}}
\def\w {\ensuremath{\mathbf{w}}}
\def\f {\ensuremath{\mathbf{f}}}
\def\r {\ensuremath{\mathbf{r}}}
\def\s {\ensuremath{\mathbf{s}}}
\def\L {\ensuremath{\mathbf{L}}}
%\def\F {\ensuremath{\mathbf{F}}}
\def\G {\ensuremath{\mathbf{G}}}
\def\E {\ensuremath{\mathbf{E}}}
\def\X {\ensuremath{\mathbf{X}}}
\def\Y {\ensuremath{\mathbf{Y}}}
\def\H {\ensuremath{\mathbf{H}}}
\def\bfA {\ensuremath{\mathbf{A}}}
\def\m {\ensuremath{\mathfrak{m}}}
\def\v {\ensuremath{\mathbf{v}}}
\def\u {\ensuremath{\mathbf{u}}}
\def\q {\ensuremath{\mathbf{q}}}
\def\U {\ensuremath{\mathbf{U}}}
%\def\V {\ensuremath{\mathbf{V}}}
\def\t {\ensuremath{\mathbf{t}}}
\def\h {\ensuremath{\mathbf{h}}}
\def\g {\ensuremath{\mathbf{g}}}
\newcommand{\algoname}[1]{{\normalfont\textsc{#1}}}

\newcommand{\problemname}[1]{{\normalfont\textsc{#1}}}

\newcommand{\algoword}[1]{\emph{\textsf{#1}}}

\newcommand{\assign}{\leftarrow}

\newcommand{\inlcomment}[1]{\texttt{\small/* #1 */}}

\newcommand{\eolcomment}[1]{\hfill\texttt{\small// #1}}

\renewcommand{\leq}{\leqslant} 
\renewcommand{\geq}{\geqslant} 
\renewcommand{\le}{\leqslant} 
\renewcommand{\ge}{\geqslant} 
\newcommand{\R}{{\Bbb R}}
\newcommand{\Z}{{\Bbb Z}}
\newcommand{\F}{{\Bbb F}}
\newcommand{\Fd}{\F_2}
\newcommand{\cala}{{\cal A}}
\newcommand{\V}{{\Bbb V}}
\newcommand{\glk}{GL_k}
\newcommand{\glki}{GL_{k+i}}
\newcommand{\glone}{GL_1}
\newcommand{\glf}{GL_4}
\newcommand{\glfive}{GL_5}
\newcommand{\pk}{P_k}
\newcommand{\otiglk}{\rlap{$\,\,\otimes$}\lower 7pt\hbox{$_{\glk}$}}
\newcommand{\otiglone}{\rlap{$\,\,\otimes$}\lower 7pt\hbox{$_{\glone}$}}
\newcommand{\otiglf}{\rlap{$\,\,\otimes$}\lower 7pt\hbox{$_{\glf}$}}
\newcommand{\otiglfive}{\rlap{$\,\,\otimes$}\lower 7pt\hbox{$_{\glfive}
$}}
\newcommand{\otiglki}{\rlap{$\,\,\otimes$}\lower 7pt\hbox{$_{\glki}$}}
\newcommand{\otigl}{\rlap{$\,\,\otimes$}\lower 7pt\hbox{$\,_{GL}$}}
\newcommand{\otigln}[1]{\rlap{$\,\,\otimes$}\lower 7pt\hbox{$\,_{GL_
{#1}}$}}
\newcommand{\otiTk}{\rlap{$\otimes$}\lower 7pt\hbox{$_{T_k}$}}
\newcommand{\oticala}{\,\rlap{$\otimes$}\lower 7pt\hbox{$_{\cala}$}\,}
\newcommand{\ov}{\overline}
%-------------------------


%vsize=21.1 truecm
\parskip 1pt

\font\inh=cmr10 \font\cto=cmr10
\font\tit=cmbx12 \font\tmd=cmbx10 \font\ab=cmti9 \font\cn=cmr9
\def\ker{\text{Ker}}
\def\bar{\overline}
\def\a.s{\text{\;a.s.\;}}
\def\supp{\text{supp\,}}
\pagestyle{plain}
%\font\fhead= vncentb at 10pt %font for page heading
%\font\fpart=vncoop at 30pt %font for part mark
\newlength{\tdtrai}
\newlength{\tdphai}
\newlength{\kctdtrai}
\newlength{\kctdphai}
\newcommand\automakeheading [2]{%
\def\trai{Chapter \thechapter. #1}
\def\phai{\thesection. #2}
\markboth{\trai}{\phai}}

\newcommand\manualmakeheading [2]{%
\def\trai{#1}
\def\phai{#2}
\markboth{\trai}{\phai}}
%End For heading

\setcounter{secnumdepth}{2}
\setcounter{tocdepth}{2}

%-----------------------------------------------
\def\eex{{\accent"5E e}\kern-.470em\raise.3ex\hbox{\char'176}}
\def\uw{u\kern-.44em\raise.82ex\hbox{ \vrule width .12em height .0ex depth .075ex \kern-0.16em \char'56}\kern-.05em}
\def\EEX{{\accent"5E E}\kern-.60em\raise.9ex\hbox{\char'176}\kern.1em}
\def\UW{U\kern-.42em\raise1.36ex\hbox{
\vrule width .13em height .0ex depth .075ex \kern-0.16em
\char'56}\kern-.07em}
\def\aah{{\accent"5E a}\kern-.62em\raise.2ex\hbox{\char'22}\kern.12em}

%% --------------------------------------
%% -------------- MISC ------------------
%% --------------------------------------
\newcommand{\storeArg}{} % aux, not to be used in document!!
\newcounter{notationCounter}

%% --------------------------------------
%% ----------- COST BOUNDS --------------
%% --------------------------------------
\newcommand{\bigO}[1]{\mathcal{O}(#1)} % big O for complexity
\newcommand{\softO}[1]{\mathcal{O}\tilde{~}(#1)} % soft O for complexity
\newcommand{\polmultime}[1]{\mathsf{M}(#1)}
\newcommand{\polmatmultime}[1]{\mathsf{MM}(#1)}
\newcommand{\expmatmul}{\omega} % exponent for the cost of matrix multiplication
%% sub item
\renewcommand{\labelenumii}{\theenumii}
\renewcommand{\theenumii}{\theenumi.\arabic{enumii}.}
%% --------------------------------------
%% ------------- INTEGERS ---------------
%% --------------------------------------
\renewcommand{\ge}{\geqslant} % greater or equal
\renewcommand{\le}{\leqslant} % lesser or equal
\newcommand{\ZZ}{\mathbb{Z}} % relative integers
\newcommand{\NN}{\mathbb{N}} % integers
\newcommand{\ZZp}{\mathbb{Z}_{> 0}} % positive integers
\newcommand{\NNp}{\mathbb{N}_{> 0}} % positive integers
\newcommand{\tuple}[1]{\mathbf{#1}} % tuples (mainly used for integers I think)

%% --------------------------------------
%% -------------- SPACES ----------------
%% --------------------------------------
\newcommand{\var}{X} % default variable for univariate polynomials
\newcommand{\field}{\mathbb{K}} % base field
\newcommand{\polRing}{\field[\var]} % polynomial ring
\newcommand{\module}{\mathcal{M}} % some module
\newcommand{\rdim}{m} % default row dimension
\newcommand{\cdim}{n} % default column dimension
\newcommand{\matSpace}[1][\rdim]{\renewcommand\storeArg{#1}\matSpaceAux} % scalar matrix space, 2 opt args
\newcommand{\polMatSpace}[1][\rdim]{\renewcommand\storeArg{#1}\polMatSpaceAux} % polynomial matrix space, 2 opt args
\newcommand{\matSpaceAux}[1][\storeArg]{\field^{\storeArg \times #1}} % not to be used in document
\newcommand{\polMatSpaceAux}[1][\storeArg]{\polRing^{\storeArg \times #1}} % not to be used in document

%% --------------------------------------
%% ----------- SETS,ELEMENTS ------------
%% --------------------------------------
\newcommand{\row}[1]{\mathbf{\MakeLowercase{#1}}} % for a row of a matrix
\newcommand{\rowgrk}[1]{\boldsymbol{#1}} % for a row of a matrix, greek letters
\newcommand{\col}[1]{\mathbf{\MakeLowercase{#1}}} % for a column of a matrix
\newcommand{\colgrk}[1]{\boldsymbol{#1}} % for a column of a matrix, greek letters
\newcommand{\mat}[1]{\mathbf{\MakeUppercase{#1}}} % for a matrix
\newcommand{\matCoeff}[1]{\MakeLowercase{#1}} % for a coefficient in a matrix
\newcommand{\vecc}[1]{\mathbf{#1}} % for a vector
\newcommand{\sumVec}[1]{|#1|} % sum of entries in a tuple
\newcommand{\card}[1]{\mathrm{Card}(#1)}

%% --------------------------------------
%% ------------- MATRICES ---------------
%% --------------------------------------
\newcommand{\trsp}[1]{#1^\mathsf{T}} %transpose
\newcommand{\matrow}[2]{{#1}_{#2,*}} % \mathrm{row}(#1,#2)}
\newcommand{\matcol}[2]{{#1}_{*,#2}} % {\mathrm{col}(#1,#2)}
\newcommand{\diag}[1]{\,\mathrm{diag}(#1)} % diagonal matrix with diagonal entries #1
\newcommand{\idMat}[1][\rdim]{\mat{I}_{#1}} % identity matrix of size mxm
\newcommand{\any}{\ast} % to put a star (indicating some element in a matrix)
\newcommand{\anyMat}{\boldsymbol{\ast}}% to put a bold star (indicating some matrix in a block matrix)

%% --------------------------------------
%% -------- POLYNOMIAL MATRICES ---------
%% --------------------------------------
\newcommand{\rdeg}[2][]{\mathrm{rdeg}_{{#1}}(#2)} % shifted row degree
\newcommand{\cdeg}[2][]{\mathrm{cdeg}_{{#1}}(#2)} % shifted column degree
\newcommand{\leadingMat}[2][\unishift]{\mathrm{lm}_{#1}(#2)} % leading matrix of polynomial matrix, default shifts = 0 (uniform)
\newcommand{\shiftMat}[1]{\mat{\var}^{#1\,}} % shift matrix, diagonal of powers of X with exponents given by #1
\newcommand{\shiftSpace}[1][\rdim]{\ZZ^{#1}} % space for shifts: tuple of rdim integers
\newcommand{\unishift}{\mathbf{0}} % notation for uniform shifts ~ [0,..,0]
\newcommand{\amp}[1][\shifts]{\mathrm{amp}(#1)} % amplitude of shift
\newcommand{\shift}[2][s]{#1_{#2}} % shifts letter, default: s
\newcommand{\shifts}[1][s]{\mathbf{#1}} % shifts vector
\newcommand{\sshifts}[1][\shifts]{|#1|} % sum of entries in shifts vector

%% --------------------------------------
%% ----------- REDUCED FORMS ------------
%% --------------------------------------
\newcommand{\popov}{\mat{P}} % (shifted) Popov form
\newcommand{\hermite}{\mat{H}} % Hermite form
\newcommand{\smith}{\mat{S}} % Smith form
\newcommand{\reduced}{\mat{R}} % (shifted) reduced form


%%%SECTION introduction, notation, problem
\newcommand{\order}{\sigma} % interpolation order
\newcommand{\mulmat}[1]{\mat{M}_{#1}}
\newcommand{\minDeg}{\delta}
\newcommand{\minDegs}{\boldsymbol{\delta}}

%%% KERNEL BASIS
\newcommand{\sys}{\mat{F}} % input matrix for kernel basis (sys = system, because computing kernel is close to solving linear system)
\newcommand{\sysSpace}[1][1]{\polMatSpace[\rdim][#1]} % space for input matrix
\newcommand{\sol}{\row{p}} % one element (solution)
\newcommand{\solSpace}{\polMatSpace[1][\rdim]} % space for all solutions
\newcommand{\mkb}{\mat{P}} % kernel basis
\newcommand{\mkbSpace}{\polMatSpace[\rdim]} % space for kernel bases
\newcommand{\piv}{\pi} % (non-)pivot index
\newcommand{\modulus}[1][m]{\mathfrak{#1}}
\newcommand{\Modulus}{\mathfrak{M}}
\newcommand{\subVec}[3]{#1_{[#2:#3]}} % notation for subvector i:j
\newcommand{\subMat}[5]{#1_{[#2:#3,#4:#5]}} % notation for submatrix i:j,k:l


%% ------------------------------------------------
%% --- Debugging. Should be eventually removed. ---
%% ------------------------------------------------

\newcommand{\todo}[1]{\textcolor{red}{#1}} % TODO remove
\newcommand{\improve}[1]{\textcolor{blue}{#1}} % TODO remove
\title{M2 Internship Report: \\
\textsc{Solving determinantal systems using homotopy techniques}}
\author{\textsc{Vu} Thi Xuan\\
Symbolic Computation Group \\
David R. Cheriton School of Computer Science\\ University of Waterloo, Canada} 
\date{
Supervised by:\\
\vspace*{0.3cm}
Éric \textsc{Schost}\\
David R. Cheriton School of Computer Science\\ University of Waterloo, Canada\\
\vspace*{0.3cm}
and \\
\vspace*{0.3cm}
Mohab \textsc{Safey El Din}\\ 
Sorbonne Universités, UPMC Univ. Paris 6\\
CNRS, INRIA Paris Center, LIP6, PolSys Team, France\\
\vspace*{1cm}
February 1 -- June 16, 2017
}
\begin{document}
\maketitle
\begin{abstract}
Solving determinantal polynomial systems (that is, systems whose equations are obtained as minors of polynomial matrices) is a recurring question in domains such as optimization or real algebraic geometry. Results known as of now are not entirely satisfying; for instance, there is no known algorithm that would solve such systems with a complexity depending on the expected number of solutions. 

Homotopy continuation techniques rely on following a deformation between the system one has to solve and another system (called start system), with a similar combinatorial structure, but whose solutions are easy to describe. In the context of determinantal systems, we define the start system by designing a \emph{start matrix} in such a way that points, at which the rank of the matrix decreases, are easy to identify. 
\end{abstract}
\newpage
\section{Introduction}
\label{sec:intro}
\subsection{Problem and \todo{motivations}} In what follows, $\field$ is a field, $\mat{X} = (X_1, \ldots, X_n)$ is the set of $n$ variables and $\field[\mat{X}]$ is the multivariate polynomial ring with coefficients in $\field$. Let $F \in \field[\mat{X}]^{p \times q}$ be a polynomial matrix, without loss of generity, we assume that $p \leq q$. For several reasons, one is interested in computing the set of points at which the evaluation of the matrix has rank $p-1$, called \emph{maximal rank} problem. We restrict on the case $n = q-p+1$, that is, the zero-dimensional case when the problem has finitely many solutions. Hereafter, given a polynomial matrix $F \in \field[\mat{X}]^{p \times q}$ and a point $\mathbf{x} \in \bar{\field}^n$, the notation $F({\mathbf{x}})$ means the evaluation of the matrix $F$ at the point $\mathbf{x}$. 

\begin{pbm}[Maximal rank problem] \label{problem} Given a field $\field$ , a matrix $F \in \field[\mat{X}]^{p \times q}$ with $p \leq q$ and $n = q-p+1$, compute the set of points at which the evaluation of $F$ has rank at most $p-1$, that is compute the set
\[S := \{\mathbf{x} \in \bar{\field}^n : \mathrm{rank}(F({\mathbf{x}})) \leq p - 1 \}.\]
\end{pbm}

A polynomial is called a $p \times p$ \emph{minor} of $F$ if it is the determinant of a $p \times p$ submatrix of $F$. To study Problem \ref{problem}, we consider the system of all $p \times p$ minors of $F$, which is called  the \emph{determinantal system} of $F$. Indeed, these $p \times p$ minors simultaneously vanish at all the points which are we interested in, and then give rise to a study of the \emph{determinantal ideal} which is generated by all $p \times p$ minors of $F$. Therefore, in order to find the set $S$, we give an algorithm to compute the isolated solutions (see \improve{\cref{sec:not}}) of the determinantal system of $F$. 
\subsection{Contributions, related works, and outline}
Henceforth, we use the notation below for the input matrix
\[ F = 
\left( \begin{matrix}
f_{1,1} & \cdots & f_{1,q}\\
\vdots & \ddots & \vdots \\
f_{p,1} & \cdots & f_{p,q}
\end{matrix} \right), \ \mathrm{where} \ f_{i,j} \in \field[\mat{X}] \ \mathrm{for} \ 1 \leq i \leq p, 1 \leq j \leq q.
\]
 We will present it by means of \emph{straight-line program}, that is, a sequence of elementary operations $+, -, \times$ that computes all of polynomials $f_{i,j}$ from the input variables $\mat{X}$. The \emph{length $\mathcal{E}$} of the input is the number of operations which we need to perform.

In order to solve Problem \ref{problem}, we study two cases of the input matrix. The first case which is called \emph{column degrees} is when the entries in the column $j$ have degrees at most $D_j$, that is, $\deg(f_{i,j}) \leq D_j$ for all $1 \leq i \leq p$; and the second case that is called \emph{row degrees} when the entries in the row $i$ have degrees at most $D_i$,  that is, $\deg(f_{i,j}) \leq D_i$ for all $1 \leq j \leq q$. The elementary symmetric polynomials and the complete homogeneous symmetric polynomials (see \improve{\cref{sec:not}} for more details) play significant role in our algorithm. Let $E_{q-p+1}(D_1, \ldots, D_q)$ and $S_{q-p+1}(D_1, \ldots, D_p)$ be the elementary symmetric polynomial of degree $q-p+1$ in $q$ variables $D_1, \ldots, D_q$ and the complete homogeneous symmetric polynomial of degree $q-p+1$ in $p$ variables $D_1, \ldots, D_p$, respectively. 
We first prove that the sum of the multiplicities of the isolated points of the corresponding determinantal ideal is at most $\mathcal{T}$, where $\mathcal{T}$ is either $E_{q-p+1}(D_1, \ldots, D_q)$ in the column degrees case or $S_{q-p+1}(D_1, \ldots, D_p)$ in the row degrees case.

The main idea behind is to use the \emph{homotopy} 
\[H = (1-T).G + T.F \in \field[T, \mat{X}]^{p \times q}\]
that connects a \emph{start matrix} $G$ to the target matrix $F$, where $T$ is a new variable. However, we need some properties on the start matrix such that we can compute the isolated solutions for the determinantal system of $F$. Assume that we know the solutions for the determinatal system of $G$, by using the Newton interation to the determinantal system of $H$, we can lift into the isolated solutions for that of $F$. Therefore, we need create a start matrix $G$ such that
\begin{itemize}
\item its solutions of the determinantal system can be found effectively,
\item we can extract a square subsystem with full-rank Jacobian,
\item there is no solution of its determinantal system is at infinity. 
\end{itemize}
\begin{Theorem} There is an algorithm which solves Problem \ref{problem} in $\big({{q}\choose{p}} \,\mathcal{E}\,\mathcal{T}\big)^{\bigO{1}}$ operations in $\field$.
\end{Theorem}

Previous works related to Problem \ref{problem} are given in \cite{FauSafSpa13, Miller04, Spa14, SaSc16}. In \cite{FauSafSpa13}, the authors work on the case the input matrix $F$ is a homogeneous polynomial matrix of degree $D$, that is $\deg(f_{i,j}) = D$ for all $1 \leq i \leq p, 1 \leq j \leq q$, under genericity assumptions on the input matrix. By generic, they mean that there exists a non-identically null multivariate polynomial $h$ such that the complexity results hold when this polynomial does not vanish on the coefficients of the polynomials in the matrix \cite{FauSafSpa13}. The authors use Gröbner bases algorithms when the input is the deterninantal system of $F$. By using this algorithm in  \cite{FauSafSpa13}, when the input matrix $F$ is homogeneous of degree $D$, Problem \ref{problem} can be solved in $\bigO{{q \choose p}{{Dm+1} \choose {q-p+1}}^{\expmatmul}}$ operations in $\field$ under genericity assumptions on the input matrix, where $\expmatmul$ the exponent of matrix multiplication with the best known bound being $\expmatmul < 2.38$ \cite{CopWin90, LeGall14}. Notice that in \cite{FauSafSpa13}, the authors provide an algorithm for more general problem: given a matrix $F$ whose entries are polynomials of degree $D$ in $\field[\mat{X}]$ and an integer $r < \min(p,q)$, compute the set of points at which the evaluation $F$ has rank at most $r$ with generic assumptions on the input matrix. In this report, we work on general input matrix $F$ without generic assumptions. We remark that when the input matrix is homogeneous, we can use our algorithm in the case of column degrees. 

The report is \todo{organized} as follows.
\section{Notations and preliminaries}
\label{sec:not}
Give a polynomial matrix $F \in \field[\mat{X}]^{p \times q}$, we write $F_{l:k \mathbf{;} e:f}$ for the submatrix of $F$ contains the rows $l, \ldots, k$ and the columns $e, \ldots, f$. We also write $F_{l:k \mathbf{;} *}$ for the submatrix of $F$ contains the rows $l, \ldots, k$ and all $q$ columns. The similar meaning is applied for the matrix $F_{* \mathbf{;} e:f}$. 

We denote $M_{pq}$ for the vector space of matrices with $p$ rows and $q$ columns over the field $\field$. Given a multivariate polynomial $g \in \field[\mat{X}]$, if we do not give a specific mention, we will consider the degree of $g$ is the total degree.  %\displaystyle 

The elementary symmetric polynomial of degree $r$ in $m$ variables $t_1, \ldots, t_m$ written $E_{r}(t_1, \ldots, t_m)$ is defined as the coefficient of $x^r$ in $(1+t_1x)(1+t_2x)\cdots(1+t_mx)$. That is
\[E_{r}(t_1, \ldots, t_m) = \sum_{(i_1,\ldots,i_{m}) \subset \{1, \ldots, r\}^{m}}\prod_{j =1}^{m}D_{i_j}.\]

The complete homogeneous symmetric polynomial of degree $r$ in $m$ variables $t_1, \ldots, t_m$ written $S_{r}(t_1, \ldots, t_m)$ is defined as the coefficient of $x^{r}$ in $$\frac{1}{(1-t_1x)(1-t_2x)\cdots(1-t_mx)} = (1+t_1x + t_1^2x^2 + \cdots)\cdots(1+t_mx + t_m^2x^2 + \cdots).$$
That is 
\[
S_r(t_1, \ldots, t_m) = \sum_{i_1 + \cdots + i_m = r} t_1^{i_1}\ldots t_m^{i_m}. 
\]
There are some formulas which are obtained directly from the definition of the complete homogeneous symmetric polynomial, such as 
\begin{itemize}
\item[•] $S_{r}(t_1, \ldots, t_m) =  \sum\limits_{k=1}^r\sum\limits_{(i_1, \ldots, i_k) \in \{1, \ldots, m\}^k}S_{r-k}(t_1, \ldots, t_k)$,
\item[•] $S_{r}(t_1, \ldots, t_{m}) = \sum\limits_{i=0}^{r}t_1^{r-i}S_i(t_2, \ldots, t_m)$,
%$S_{q-m}(t_1, \ldots, t_{m+1}) = t_1^{q-m}+t_1^{q-m-1}S_1(t_2, \ldots, t_{m+1}) + t_1^{q-m-2}S_2(t_2, \ldots, t_{m+1}) + \cdots + S_{q-m}(t_2, \ldots, t_{m+1})$, that is, 
\item[•] $S_r(t_1, \ldots, t_{m-1}, t_m) - S_r(t_1, \ldots, t_{m-1}, t_{m+1}) = (t_m - t_{m+1})S_{r-1}(t_1, \ldots, t_{m-1}, t_m, t_{m+1})$. 
\end{itemize}
\paragraph{Ideals and varieties.} Let $\field[\mat{X}]$ be a polynomial ring of $n$ variables. 
\begin{Definition} Let $I \subset \field[\mat{X}]$ be an ideal. The $\mathrm{radical}$ of $I$, denoted $\sqrt{I}$ is the set 
\[
\{f \ : \ f^m \in I \ \mathrm{for \ some} \ m \in \mathbb{N}, \ m \geq 1\}.
\]
\end{Definition}

An ideal $I \subset \field[\mat{X}]$ is \emph{prime} if whenever $f,g \in \field[\mat{X}]$ and $fg \in I$, then either $f \in I$ or $g \in I$. An ideal $I \subset \field[\mat{X}]$ is called a \emph{primary} ideal if $fg \in I$ implies $f \in I$ or $g^k \in I$ for some $k \in \mathbb{N}$. Remark that if $I$ is primary over $\field[\mat{X}]$, then $\sqrt{I}$ is a prime ideal. 
\begin{Definition}
Let $I$ be an ideal in $\field[\mat{X}]$. A \emph{primary decomposition} of $I$ is an expression 
\[
I = Q_1 \cap \cdots \cap Q_s,
\]
where each $Q_i$ is primary.
\end{Definition}
The decomposition is \emph{irredundant} or \emph{minimal} if $\sqrt{Q_i}$ are all distinct $Q_i \not \supset \cap_{j \ne i}Q_j$ for all $1 \leq i \leq s$. After pruning redundant,  we obtain $P_i = \sqrt{Q}_i$ for $1 \leq i \leq r$. The radical ideals $P_i$ are then called the \emph{associated primes} of $I$. 
\begin{Theorem}$[\todo{cite}]$ Let $I \subset \field[\mat{X}]$ be an ideal. Then, there exist unique $\mathcal{P}_1, \ldots,, \mathcal{P}_r$ prime ideal in $\field[\mat{X}]$ such that 
\[
\sqrt{I} = \mathcal{P}_1 \cap \mathcal{P}_2 \cdots \cap \mathcal{P}_r,
\]
and $P_1, \ldots, P_r$ are called $\mathrm{prime \ components \ of} \ \sqrt{I}$. 
\end{Theorem} 

\begin{Example}\label{Ex1} Let $\mathbb{C}[X,Y]$ be the the polynomial ring of two variables with complex coefficients. Let $I = \langle X^2 - Y\rangle \subset \mathbb{C}[X,Y]$ be the ideal generated by $X^2 - Y$; and $J = \langle X^2, Y^3\rangle$ be the ideal generated by $X^2$ and $Y^3$. Then 
\begin{itemize}
\item $I$ is prime and $J$ is not prime;
\item the radical ideal $\sqrt{I} = \langle X^2 - Y\rangle = I$ and the radical ideal of $J$ is $\sqrt{J} = \langle X,Y \rangle$. 
\end{itemize}
\end{Example}
\begin{Example}
Let $I = \langle X^2, XY \rangle \subset \mathbb{C}[X,Y]$. A primary decomposition of $I$ is 
\[
I = \langle X^2, XY \rangle = \langle X \rangle \cap \langle X^2Y \rangle, 
\] where $P_1 = \langle X \rangle$ and $P_2 =  \langle X,Y \rangle$ are the associated primes of $I$. 
\end{Example}
\begin{Definition} $\mathrm{[Dimension \ of \ an \ ideal]}$ Let $\mathcal{P} \subset \field[\mat{X}]$ be a prime ideal. We say that d is $\mathrm{Krull \ dimension }$ of $\mathcal{P}$ if and only if there exists prime ideals $\mathcal{P}_0, \mathcal{P}_1, \ldots, \mathcal{P}_d$ in $\field[\mat{X}]$ such that 
\[
\mathcal{P}_0 \subsetneq \mathcal{P}_1 \subsetneq \cdots \subsetneq \mathcal{P}_d = \mathcal{P}.
\] 
The $\mathrm{Krull \ dimension}$ of an ideal $I \subset \field[\mat{X}]$ is the maximum Krull dimension of the prime components of $\sqrt{I}$. 
\end{Definition} 
\begin{Example} Let $I$ be the ideal defined as in the Example \ref{Ex1}. Then, the dimension of $I$ equals one. 
\end{Example}
The geometrical objects corresponding to ideals of $\field[\mat{X}]$ are \emph{affine \ varieties} (or \emph{algebraic sets}) of $\bar{\field}^n$. An affine variety in $\bar{\field}^n$ is the set of all solutions of a system of equations. Hereafter, we will use variety for the short of affine variety. Given $\f = (f_1, \ldots, f_M)$ in $\field[\mat{X}]^M$, we denote $V(\f) \in \bar{\field}^n$ for the variety defined by $\f$. That is 
\[
V(\f) = \{{\bm \alpha} = (\alpha_1, \ldots, \alpha_n) \in \bar{\field}^n \ | \ f_i({\bm \alpha}) = 0 \ \mathrm{for \ all} \ 1 \leq i \leq M\}. 
\]

One of the important properties of varieties of $\bar{\field}^n$ is they define a topology on $\bar{\field}^n$, namely \emph{Zariski topology}, with the closed sets of the topology are the varieties. The \emph{Zariski closure} of a set $S$ in $\bar{\field}^n$ is the smallest (for the inclusion ordering) Zariski closed set that contains $S$. A set $V$ is \emph{Zariski dense} in a set $W$ if the Zariski closure of V contains in the Zariski closure of W. Zariski topology will be useful for defining an algebraic notion of genericity for structured system. 
\begin{Definition} A property of a family of systems $\mathfrak{F} \subset \bar{\field}[\mat{X}]$ which is $\bar{\field}$-vector space of finite dimension is said to be $\mathrm{generic}$ if this property is satisfied on a nonempty open subset of $\mathfrak{F}$.
\end{Definition}
\begin{Definition}$\mathrm{[Dimension \ of \ a \ variety]}$ Let $V$ be a variety. The dimension of V is the largest d such that the image of V by $(X_1, \ldots, X_n) \to (X_{i_1}, \ldots, X_{i_d})$ is Zariski dense for some $(i_1, \ldots, i_d) \subset \{1, \ldots, n\}$.
\end{Definition}
It is equivalent to define the dimension of a variety $V$ as the Krull dimension of $\field[\mat{X}]/I(V)$, where \[I(V) := \{f \in \field[\mat{X}] \ | \ f(\alpha) = 0 \ \forall \ \alpha \in V\}\]
 which is called the ideal of $V$. A zero-dimensional variety is a finite set. A variety $V \subset \bar{\field}^n$ is \emph{irreducible} if whenever $V$ is written in the form $V = V_1 \cup V_2$, where $V_1, V_2$ are varieties, then either $V = V_1$ or $V = V_2$. The next result is given in \cite[Theorem~4 -- section~6 -- chapter~4]{Cox07}.
\begin{Theorem} Let $V \subset \bar{\field}^{n}$ is a variety. Then V has a minimal decomposition 
\[
V = V_1 \cup \cdots \cup V_m,
\] where each $V_i$ is an irreducible variety and $V_i \not\subset V_j$ for $i \ne j$. Furthermore, this minimal decomposition is unique up to the order in which $V_1, \ldots, V_m$ are written; and $V_1, \ldots, V_m$ are called $\mathrm{irreducible \ components}$ of V.
\end{Theorem}

\begin{Definition} A variety is said to be $\mathrm{equidimensional}$ if and only if there exists $d \in \mathbb{N}$ such that all its irreducible components have dimension d. 
\end{Definition}

When $\field$ is algebraically closed, there is one-to-one correspondence between irreducible varieties in $\field^n$ and prime ideals in $\field[\mat{X}]$. 

\begin{Example}Let $f = X^2-Y \in \mathbb{C}[X,Y]$ and $g = X-Y$; then the variety $V(f) = \{(t^2,t) \ | \ t \in \mathbb{C}\}$ has dimension one while the variety $V(f,g) = \{(0,0), (1,-1)\}$ has zero-dimensional. Moreover, $V(f)$ is irreducible (since $\langle f \rangle$ is prime) while $V(f,g)$ is not an irreducible variety (since $\langle f,g \rangle$ is not a prime ideal). 
\end{Example}
A \emph{hyperplane} $H$ is the vanishing set of a linear polynomial in $\field[\mat{X}]$, that is $H = V(\sum a_iX_i + b)$. 
\begin{Definition}$\mathrm{[Degree \ of \ a \ variety]}$ Let V be an equidimensional variety of dimension d. The $\mathrm{degree}$ of $V$ is the unique integer $D$ such that $V \cap H_1 \cap \cdots \cap H_d$ consists of D points for generic hyperplanes $H_1, \ldots, H_p$. 

For any variety V, the degree of V is the sum of degree of all irreducible components of V. 
\end{Definition}
\begin{Example} Let $V$ be the variety which is defined by $X^2 - YZ = XZ-X = 0$ in $\mathbb{C}^3$, then $V = V_1 \cup V_2 \cup V_3$, where $V_1 = V(X,Y)$, $V_2 = V(X,Z)$ and $V_3 = V(Z-1,X^2 - Y)$. In other words, $V_1$ is the $Z$-axis, $V_2$ is the $Y$-axis and $V_3$ is a parabola.  Since $\langle X,Y\rangle$, $\langle X,Z\rangle$ and $\langle Z-1,X^2 - Y \rangle$ are prime ideals in $\mathbb{C}[X,Y,Z]$ then $V_1, V_2, V_3$ are irreducible. Therefore, $V_1, V_2, V_3$ are irreducible components of $V$. Moreover, all of these components has dimension one, so $V$ is equidimensional of dimension one. 
\end{Example}
Given a set of polynomials $\mathbf{f} = (f_1, \ldots, f_M) \subset \field[\mat{X}]^M$, the \emph{Jacobian matrix} of $\mathbf{f}$ is an $M \times n$ matrix defined as follows 
\[
\mathrm{Jac}(\mathbf{f}) = \left[ \begin{matrix}
\frac{\partial f_1}{X_1}  & \cdots & \frac{\partial f_1}{X_n}\\
\vdots & \ddots & \vdots\\
\frac{\partial f_M}{X_1}  & \cdots & \frac{\partial f_M}{X_n}
\end{matrix} \right].
\]
%Jacobian criterion is an useful tool to check wheather the variety of an ideal is equidimensional. 
%\begin{Theorem}[Jacobian Criterion] $[\todo{cite}]$  \
%Let $\mathbf{f} = (F_1, ..., F_k) \subset \field[\mat{X}]^k$ and $V(\mathbf{f}) \subset \bar{\field}^n $. Assume %that $V(\mathbf{f})$ is nonempty and
%\[
%\{\mathbf{x} \in V(\mathbf{f}) \ | \ \mathrm{rank}(\mathrm{Jac}(\mathbf{f})({\mathbf{x}})) < k\} = \emptyset .
%\] Then, V is equidimensional and has dimension $n - k$.
%\end{Theorem}

\paragraph{Solutions of polynomial systems.} Given a polynomial system of $N$ equations in $n$ unknowns, $A(\mat{X}) = \mat{0}$, we are interested in $\mathbf{x}^*$, an \emph{isolated} solution of $A(\mat{X}) = \mat{0}$:
\[
\mathrm{for \ small \ enough} \ \epsilon > 0 : \{\mathbf{y} \in \bar{\field}^n : ||\mathbf{y} - \mathbf{x}^*|| < \epsilon\} \cap A^{-1}(\mat{0}) = \{\mathbf{x}^*\}, 
\] where $A^{-1}(\mat{0}) := \{\mathbf{y} \in \bar{\field}^n | A(\mathbf{y}) = \mat{0}\}$. We call $\mathbf{x}^*$ a \emph{singular} solution of $A(\mat{X}) = \mat{0}$ if and only if $\mathrm{rank}(\mathrm{Jac}(A)(\mathbf{x}^*)) < n$.  Let us denote $mul(\mathbf{x}^*)$ for the multiplicity of the solution $\mathbf{x}^*$ of $A(\mat{X}) = \mat{0}$.   A solution $\mathbf{x}^*$ is called a \emph{simple} root of $A(\mat{X}) = \mat{0}$ if  $\mathrm{rank}(\mathrm{Jac}(A)(\mathbf{x}^*)) = n$.
\paragraph{Zero-dimentional parametrization.} Let $V \subset \bar{\field}^n$ be a zero-dimensional variety which is defined over $\field$. A \emph{zero-dimensional parametrization} $\mathscr{R} = ((q,v_1, \ldots, v_n), \lambda)$ of $V$ consists in polynomials $(q,v_1, \ldots, v_n)$ such that $q \in \field[T]$ is monic and squarefree, all $v_i \in \field[T]$ and $\deg(v_i) < \deg(q)$, and $\lambda$ is a $\field$-linear form in $n$ variables, such that 
\begin{itemize}
\item $\lambda(v_1, \ldots, v_n) = Tq'$ mod $q$
\item we have $V = \{(\frac{v_1(\tau)}{q'(\tau)}, \ldots, \frac{v_n(\tau)}{q'(\tau)}) \ | \ q(\tau) = 0\}$;
\end{itemize}
the constraint on $\lambda$ says that the root of $q$ are the values taken by $\lambda$ on $V$. 

\section{Start matrix and a bound on the degree of the input}
We study two cases for the input matrix, those are column degrees and row degrees. We recall here that the notation $\mathcal{T}$ is either $E_{q-p+1}(D_1, \ldots, D_q)$ in the case of column degrees or $S_{q-p+1}(D_1, \ldots, D_p)$ in the row degrees case. Notice that in this report, we are working on the case when $n = q-p+1$. Before going to details of the start matrix, we need some genericity assumptions for the case of row degrees. Hereafter, give a polymomial matrix $G$, we use the notation ${\bf g} = (g_1, \ldots, g_M)$ for the determinanatal system of $G$. 
\subsection{Genericity assumptions}
\label{subsec:genericity}
Let $G = [g_{i,j}]_{1 \leq i \leq p, 1 \leq j \leq q}$ be a matrix with $g_{i,j}$ is a product of $D_i$ linear forms with each linear form has generic coefficients. We say that the matrix $G$ satisfies assumption $\mathcal{A}$ if 
\begin{enumerate}
\item rank($G(\mathbf{x}^*)$) $= p-1$ for all $\mathbf{x}^* \in V({\bf g})$.
\item ${\bf g}$ has exactly $S_{n}(D_1, \ldots, D_p)$ distinct solutions.  
\item ${\bf g}$ is a radical ideal (this genericity condition is equivalent to the property that $\mathrm{Jac}({\bf g})(\mathbf{x}^*)$ has full rank for any $\mathbf{x}^* \in V({\bf g}))$. 
\end{enumerate}

Let us denote $\mathcal{G} = \{\gamma_{i,j}^{(t,l_i)} \ | \ 1 \leq j \leq q, 0 \leq t \leq n \ \mathrm{and} \ \mathrm{for} \ i : 1 \leq i \leq p, 1 \leq l_i \leq D_i\}$ for the set of new indeterminantes for these generic coefficients. For $1 \leq i \leq p, 1 \leq j \leq q$ and denote by $\mathfrak{g}_{i,j} = \prod_{l=1}^{D_i}(\gamma_{i,j}^{(n,l_i)}X_n + \cdots + \gamma_{i,j}^{(1,l_i)}X_1 + \gamma_{i,j}^{(0,l_i)}) \in \field[\mathcal{G}, \bf{X}]$. We will prove that assumption $\mathcal{A}$ is generic in the following sense. 
\begin{Proposition} \label{generic}For any $k \in \{1,2,3\}$, there exists a nonempty Zariski open set $\mathcal{O}_k$ such that $I_G$ satisfies $\mathcal{A}(k)$ for $\mathfrak{g}_{i,j} \in \mathcal{O}_k$. 
\end{Proposition}
\begin{proof}
Let us define the matrices $G_1$ and $G_2$ as follow
\[G_1 = \left( \begin{matrix}
\mathfrak{g}_{1,1} & \mathfrak{g}_{1,2} & \cdots  & \mathfrak{g}_{1, q}\\
\mathfrak{g}_{2,1} & \mathfrak{g}_{2,2} & \cdots  & \mathfrak{g}_{2, q}\\
\vdots & \vdots & \ddots & \vdots \\
\mathfrak{g}_{p,1} & \mathfrak{g}_{p,2} & \cdots  & \mathfrak{g}_{p, q}
\end{matrix} \right) \ \mathrm{and} \ 
 G_2 = \left( \begin{matrix}
\mathfrak{g}_{1,1} & 0 & \cdots & 0 & \mathfrak{g}_{1,p+1} & \cdots & \mathfrak{g}_{1, q}\\
0 & \mathfrak{g}_{2,2} & \cdots & 0 & \mathfrak{g}_{2,p+1} & \cdots & \mathfrak{g}_{2, q}\\
\vdots & \vdots & \ddots & \vdots & \vdots & \ddots & \vdots\\
0 & 0 & \cdots & \mathfrak{g}_{p,p} & \mathfrak{g}_{p,p+1} & \cdots & \mathfrak{g}_{p, q}
\end{matrix} \right). \] 
The idea to prove $\mathcal{A}(k)$, for any $k \in \{1,2,3\}$, is by using induction on the number of variables, $n$, for the matrix of form $G_1$; and, after that we show that the property $\mathcal{A}(k)$ is true for the matrix $G_2$. Finally, by using this property of form $G_2$, we prove  $\mathcal{A}(k)$ holds for any matrix of form $G_1$.

The complete proof is given in \improve{\cref{sec:proofgeneric}}. 
\end{proof}
\subsection{Start matrix}
In order to use the symbolic homotopy teachniques, we need a start matrix $G$ that connects to the target matrix $F$ with some properties which we have mentioned in the \improve{\cref{sec:intro}}. 
\subsubsection{Column degrees}
\label{subsec:cd}
We first consider the case when $F = [f_{i,j}] \in \field[\mat{X}]^{p\times q}$ with $\deg({f_{i,j}}) = D_j$ for all $1 \leq i \leq p$. We will construct a polynomial matrix $G \in \field[\mat{X}]^{p \times q}$ such that the determinantal system of $G$ has $E_{n}(D_1, \ldots, D_q)$ solutions and we can find those solutions in effectively way. 

For any $1 \leq i \leq p, 1 \leq j \leq q$, let us define $\lambda_{i,j}^{(k)}  = (i+j)^k \in \field$ and $L_{i,j} = \sum_{k = 1}^{n}\lambda_{i,j}^{(k)}X_k + \lambda_{i,j}^{(0)}$. Then, we define the matrix $G$ as
\[G = 
\left( \begin{matrix}
g_1 & 2g_2 & \cdots & qg_{q}\\
g_1 & 2^2g_2 & \cdots & q^2g_q\\
\vdots & \vdots & \ddots & \vdots \\
g_1 & 2^pg_2 & \cdots & q^pg_q
\end{matrix} \right),
\]
where $g_{i}$ is the product of $D_i$ linear forms, i.e., $g_i = \prod_{j = 1}^{D_i}L_{i,j}$. 
\begin{Lemma} \label{G} Let G be the matrix as we define above and $\g$ be the ideal generated by $p \times p$ minors of $G$. Then, $\g$ has exactly $E_{n}(D_1, \ldots, D_q)$ solutions and there is an algorithm, namely Algorithm \ref{StartMatCol}, to compute these solutions in $\bigO{nE_{n}(D_1, \ldots, D_q)}$ operations in $\field$.
\end{Lemma}
\begin{proof}
Any $p \times p$ minors of $G$ has the form 
\[
f_G = \lambda g_{i_1}\ldots g_{i_p}, \ \mathrm{where} \ (i_1, \ldots, i_p) \in \{1, \ldots,q\}^{p} \ \mathrm{and} \ \lambda \in \field.
\] Then, for any solution $\mathbf{x} \in V({\bf g})$, $\mathbf{x}$ is a solution of a $n$ polynomials system which is taken from $\{g_j\}_{1 \leq j \leq q}$. This implies 
\begin{eqnarray*}
\#\{\mathrm{solutions \ of \ } {\bf g} \} &=& \sum_{(i_1, \ldots, i_{n}) \subset \{1, \ldots,q\}^{n}} \#
\{\mathrm{solutions \ of \ }  g_{i_1} = \cdots = g_{i_{n}} = 0  \} \\
&=& \sum_{(i_1, \ldots, i_{n}) \subset \{1, \ldots,q\}^{n}}\prod_{j =1}^{n}D_{i_j} = E_{n}(D_1, \ldots, D_q). 
\end{eqnarray*}

For the complexity, since each polynomial $L_{i,j}$ is linear, so the time to solve the system in the step $2.i$ is $\bigO{n}$; and there are $E_{n}(D_1, \ldots, D_p)$ systems which is need to solve. Therefore, the complexity of the Algorithm $\ref{StartMatCol}$ is $\bigO{nE_{n}(D_1, \ldots, D_p)}$.
\end{proof}

\begin{algorithm}
\caption{$\mathsf{Start Matrix Column Degrees}$}
\label{StartMatCol}
{\bf Input}: a matrix $G \in \field[\mat{X}]^{p \times q}$ which is defined as above.\\
{\bf Output}: $E_{n}(D_1, \ldots, D_q)$ solutions of $\g$. 

\begin{enumerate}
\item $S \gets \emptyset$
\item for any $(i_1, \ldots, i_{n}) \in \{1, \ldots, q \}^{n}$: 
\begin{itemize}
\item for any $(j_1, \ldots, j_n) \in \{1, \ldots, D_{i_1}\} \times \cdots \times \{1, \ldots, D_{i_n}\} $: 
\begin{enumerate}
\item $\mathbf{x} \gets$ solve the linear system $L_{i_1,j_1} = \cdots = L_{i_n,j_n} = 0$
\item $S = S \cup \{\mathbf{x}\}$
\end{enumerate}
\end{itemize}
\item return $S$
\end{enumerate}
\end{algorithm}
%Furthermore, by the construction of $G$ as above, we can see that there are no solution of ${\bf g}$ at infinity. 
\subsubsection{Row degrees}
We here consider the case when $F = [f_{i,j}] \in \field[\mat{X}]^{p\times q}$ with $\deg({f_{i,j}}) = D_i$ for all $1 \leq i \leq p$. We will construct a polynomial matrix $G \in \field[\mat{X}]^{p \times q}$ such that the determinantal system of $G$ has $S_{n}(D_1, \ldots, D_p)$ solutions. 

As in the proof of Proposition \ref{generic}, we can use a matrix $G$ as in the form of $G_2$ with the generic coefficients of the linear forms. So, let us define a start matrix $G$ as follows 

\[ G = \left( \begin{matrix}
g_{1} & 0 & \cdots & 0 & g_{1,p+1} & \cdots & g_{1, q}\\
0 & g_{2} & \cdots & 0 & g_{2,p+1} & \cdots & g_{2, q}\\
\vdots & \vdots & \ddots & \vdots & \vdots & \ddots & \vdots\\
0 & 0 & \cdots & g_{p} & g_{p,p+1} & \cdots & g_{p, q}
\end{matrix} \right), \] where all of $\{g_i, g_{i,j}\}$ is a product of $D_i$ linear form with generic coefficients. That is $g_i = \prod_{k=1}^{D_i}L_i^{(k)}$ and $g_{i,j} = \prod_{k=1}^{D_i}L_{i,j}^{(k)}$, where $L_i^{(k)}$ and $L_{i,j}^{(k)}$ are in liner form with generic coefficents. 

Let $\mathbf{x}$ be a solution of $\g$, the next property helps us reduce our problem to the problem with smaller dimension. 
\begin{Proposition} \label{r2}
If $g_{i_1}(\mathbf{x}) = \cdots = g_{i_k}(\mathbf{x}) = 0$, then $\mathrm{rank}(G_{i_1:i_k\mathbf{;}p+1:q}({\mathbf{x}})) \leq k-1$.
\end{Proposition}
\begin{proof} Without of loss of generality, we prove this result for $(i_1, \ldots, i_k) = (1, \ldots, k)$. This means we have $g_{1}(\mathbf{x}) = \cdots = g_{k}(\mathbf{x}) = 0$ and $g_{j}(\mathbf{x}) \ne 0$ for any $j \in \{k+1, \ldots, p\}$. Let $G_{1:p;*} \in \field[\mat{X}]^{p \times p}$ be the submatrix of $G$ consisting $p$ rows and the columns $k+1, \ldots, p, j_1, \ldots, j_k$ of $G$, where $(j_1, \ldots, j_k) \in \{p+1, \ldots, q\}^{k}$. Let $f_{1:p;*}$ is the determinant of $G_{1:p;*}$. Then, for any solution $\mathbf{x}$ of $\g$, $g_{1:p;*}(\mathbf{x}) = 0$. Moreover, $g_{1:p;*} = \det(G_{1:p;*}) = g_{k+1,k+1} \ldots g_{p,p} \det(G_{1:k;j_1:j_k})$. Therefore, $\det(G_{1:k;j_1:j_k})(\mathbf{x}) = 0$. This holds for any $(j_1, \ldots, j_k) \in \{p+1, \ldots, q\}^{k}$. This implies that $\mathrm{rank}(G_{i_1:i_k\mathbf{;}p+1:q}({\mathbf{x}})) \leq k-1$. 
\end{proof}

\begin{Lemma}\label{P3} Let G be the matrix as we define above and $\g$ be the ideal generated by $p \times p$ minors of $G$. Then, $\g$ has exactly $S_{n}(D_1, \ldots, D_p)$ solutions. Moreover, there is an algorithm, namely Algorithm \ref{StartMatRow}, to compute these solutions in \todo{Complexity}. 
\end{Lemma}
\begin{proof}
The number of solutions for $I_G$ is proved as before, so we need give here the algorithm complexity \todo{finish with complexity}.  
\end{proof}

\begin{algorithm}[]
\caption{$\mathsf{Start Matrix Row Degrees}$}
\label{StartMatRow}
{\bf Input}: a matrix $G \in \field[\mat{X}]^{p \times q}$ which is defined as above.\\
{\bf Output}: $S_{n}(D_1, \ldots, D_p)$ solutions of ${\bf g}$.
\begin{enumerate}
\item $m = \min \{n,p\}$
\item $S := \emptyset$
\item For $k = 1$ to $m$: 
\begin{enumerate}
\item For any  $(i_1, \ldots, i_k) \subset \{1, \ldots, p\}^k$:
\begin{itemize}
\item[\emph{i}.] From $g_{i_1} = \cdots = g_{i_k} = 0$, rewrite $\{X_i\}_{i = 1}^k$ in the linear form of $\{X_{i}\}_{i=k+1}^n$
\item[\emph{ii}.] Substitute $\{X_i\}_{i = 1}^k$ into $G_{i_1:i_k\mathbf{;}p+1:q}$
\item[\emph{iii}.] $p \gets k; q \gets q-p; n \gets n - k$
\begin{itemize}
\item[•] If $p \leq q$:\\ \hspace{1cm} $\mathbf{x} \gets \mathsf{RowDeterminantal System}(G_{i_1:i_k\mathbf{;}p+1:q}, G_{i_1:i_k;i_1:i_k \cup i_{n}:i_{n+k-1}})$
\item[•] Else :\\ \quad $\mathbf{x} \gets \mathsf{ColumnDeterminantal System}(G_{i_1:i_k\mathbf{;}p+1:q},\bar{G})$, where 
\[\bar{G} \in \field[X_{k+1}, \ldots, X_n]^{p \times q} \ \mathrm{is \ defined \ as \ in \ \improve{\cref{subsec:cd}}} \ \] 
\end{itemize}
\item[\emph{iv}.] $S \gets S \cup \{\mathbf{x}\}$
\end{itemize}
\end{enumerate}
\item Return $S$
\end{enumerate} 
\end{algorithm}
\subsection{A bound for the number isolated solutions of determinantal system}
\label{subsec:bounddegree}
Let ${\bf g} = (g_1, \ldots, g_M)$ in $\field[\mat{X}]^M$ be the determinantal system of $G$, and let $\mathcal{T}$ be the number of solutions of ${\bf g}$. That is $\mathcal{T} = E_{n}(D_1, \ldots, D_q)$ in the column degrees case, and $\mathcal{T} = S_{n}(D_1, \ldots, D_p)$ in the row degrees case. We also write ${\bf f} = (f_1, \ldots, f_M)$ in $\field[\mat{X}]^M$ for the determinantal system of $F$. In this section, we are going to prove that the sum of the multiplies of the isolated solutions of ${\bf f}$ is at most $\mathcal{T}$. We first recall here the result in \cite[Proposition~1]{Hen83}. 
\begin{Proposition}\label{hen} Let $V$ be an irreducible affine variety of dimension $m$ and $\phi: V \to \field^m$ a dominanting morphism. The field extension $\mathsf{k}(\field^m) \subset \mathsf{k}(V)$ induced by $\phi$ is finite. Then $\#\phi^{-1}(y) \leq [ \mathsf{k}(V) : \mathsf{k}(\field^m)]$ for any $y \in \field^m$ with fininte fibre $\phi^{-1}(y)$.
\end{Proposition}

Let $H := (1-T).G + T.F \in \field[T, \mat{X}]$ and ${\bf h} = (h_1, \ldots, h_M)$ in $\field[T, \mat{X}]^M$ is the determinantal system of $H$. From \cite[Section~6]{Eagon188}, we have any irreducible component of ${\bf h}$ has dimension at least one. Then, there is a decomposition for the variety of ${\bf h}$ as 
\[
V({\bf h}) = V_1 \cup V_2 \cup \cdots \cup V_d, \] where each $V_i$
is an irreducible variety. Let us rewrite $V_1 = V_{1,1} \cup V_{1,2}$, where $V_{1,1}$ contains all components whose projection in $T$-axis is dense in $\bar{\field}$. Then, all solutions of ${\bf g}$ and all of isolated solutions of ${\bf f}$ are in $V_{1,1}$. 

Let $\pi_1$ be the projection from $V_{1,1}$ on the first coordinate, i.e., $\pi_1: V_{1,1} \to \bar{\field}$. It is obvious that $\bar{\pi_1(V_{1,1})} = \bar{\field}$, that is $\pi_1$ is a dominating morphism; then, by using the Proposition $\ref{hen}$, we have $\#\pi_1^{-1}(1) \leq [ \mathsf{k}(V_{1,1}) : \mathsf{k}(\bar{\field})]$. Moreover, $\#\pi_1^{-1}(1)$ equals the number of isolated solutions of ${\bf f}$; and $V_{1,1}$ has generically $[ \mathsf{k}(V_{1,1}) : \mathsf{k}(\bar{\field})]$ solutions. Thus, to obtain the number of isolated solutions of ${\bf f}$ is at most $\mathcal{T}$, it sufficies to show that $V_{1,1}$ has generically $\mathcal{T}$ solutions. By definition, the generically number of solutions of $V_{1,1}$ equals the number of solutions of $I(V_{1,1})$ in $\field(T)[\mat{X}]$. So, we are going to prove that $I(V_{1,1})$ has $\mathcal{T}$ solutions in $\field(T)[\mat{X}]$. We notice that $I_G$ has exactly $\mathcal{T}$ solutions; and when $T = 0$ we have ${\bf g} \equiv  {\bf h}_0 = (h_{i}(1, \mat{X}))_{1 \leq i \leq M}$. Therefore, to obtain $I(V_{1,1})$ has $\mathcal{T}$ solutions in $\field(T)[\mat{X}]$, it sufficies to prove that 
\begin{itemize}
\item[•] there is no solution of ${\bf h}_0$ at infinity, and 
\item[•] all solutions of ${\bf h}_0$ are nonsingular. 
\end{itemize}

We notice first that $I(V_{1,1}) \subset \field(T)[\mat{X}]$; and $V(I(V_{1,1})) \subset \bar{\field(T)}^n$ has dimension zero but $V(I(V_{1,1})) \subset \bar{\field}^{n+1}$ has dimension one. 

\begin{Proposition}\label{det}
For any $M \in \field[\mat{X}]^{n \times n}$ and $N \in \field[\mat{X}]^{n \times n}$, let T be a new variable and let $Q := (1-T).M + T.N$ in $\field[T, \mat{X}]^{p \times q}$, then 
\[
\det(Q) = (t-T)^n \det(M) + T.f(T,\mat{X}) \ \mathrm{and} \ \det(Q) = (t-T).g(t,\mat{X}) + T^n.\det(N)) 
\] for any $h(T,\mat{X}), g(T,\mat{X}) \in \field[T,\mat{X}]$.
\end{Proposition}
\begin{proof}
Let us write $M = [g_{i,j}]_{1 \leq i,j \leq n}, N = [f_{i,j}]_{1 \leq i,j \leq n}$ and $Q = [h_{i,j}]_{1 \leq i,j \leq n}$; then for any $i,j$,  $h_{i,j} = (1-T)g_{i,j} + Tf_{i,j}$. By using the definition of a matrix we have 
\begin{eqnarray*}
\det(Q) &=& \sum_{\sigma \in S_n}\mathrm{sgn}(\sigma)\prod_{i = 1}^nh_{i,\sigma_i}\\
&=& \sum_{\sigma \in S_n}\mathrm{sgn}(\sigma)\prod_{i = 1}^n\left[(1-T)g_{i,\sigma_i} + Tf_{i,\sigma_i}\right]\\
&=& \sum_{\sigma \in S_n}\mathrm{sgn}(\sigma)\left[(1-T)^n\prod_{i = 1}^ng_{i,\sigma_i} + \bar{f}(T,\mat{X})\right],
\end{eqnarray*}
where $\bar{h}(T,\mat{X}) = \prod_{i = 1}^n\left[(1-T)g_{i,\sigma_i} + Tf_{i,\sigma_i}\right] - (1-T)^n\prod_{i = 1}^ng_{i,\sigma_i}$. Moreover, $\bar{f}(T,\mat{X})$ is a polynomial in $\field[T, \mat{X}]$ with all the terms are multiple of $T$. Therefore, 
\begin{eqnarray*}
\det(Q) &=&  \sum_{\sigma \in S_n}\mathrm{sgn}(\sigma)\left[(1-T)^n\prod_{i = 1}^ng_{i,\sigma_i}\right] + \sum_{\sigma \in S_n}\mathrm{sgn}(\sigma)\bar{f}(T,\mat{X}) \\
&=& (1-T)^n\det(M) + T.f(T,\mat{X})
\end{eqnarray*}
Similarly, we obtain $\det(Q) = (t-T).g(T,\mat{X}) + T^n\det(N)$.
\end{proof}

Let $\mathbf{x}(T) = (T, f_1(T), \ldots, f_n(T)) \in V(I(V_{1,1}))$ with $f_i(T)$ is considered as Puiseux series in $\bar{\field}\langle\langle T \rangle\rangle$. Puiseux series are a generalization of power series that allow for negative and fractional exponents of the indeterminate $T$. Formally, given a field $\field$, the field of \emph{Puiseux series} with coefficients in $\field$ is the set of expressions of the form $f = \sum_{k \ge k_0}c_kT^{k/n}$. The \emph{valuation} $\nu(f)$ of a series $f = \sum_{k \ge k_0}c_kT^{k/n}$ is the smallest rational $k/n$ such that the coefficient $c_k$ of the term with exponent $k/n$ is non-zero. The next proposition says that there is no solution of ${\bf g}$, i.e. ${\bf h}_0$, at infinity. 
\begin{Proposition}\label{valuation}
For any $i \in \{1, \ldots, n\}$, $f_i(T)$ has non-negative valuation. 
\end{Proposition}
\begin{proof}
Without loss of generity, we can write $f_i(T) = \frac{s_i(T)}{T^d}$ with $\nu(s_i(T)) \ge 0$ for all $1 \leq i \leq n$. We will prove $d \leq 0$. Indeed, assume a contradiction that $d > 0$. First, we have $\mathbf{x}(0) \in V({\bf g})$. In addition, since $V(I(V_{1,1})) = V_{1,1}$, then for any $\mathbf{x}(T) \in V(I(V_{1,1}))$ we have $f_H(\mathbf{x}(T)) = 0$ for any $f_H \in \langle {\bf h} \rangle$. 

From Proposition $\ref{det}$, for any $f_H \in \langle {\bf h} \rangle$, 
\begin{equation}\label{e1}
f_H = (1-T)^p f_G + T.h(T, \mat{X}),
\end{equation}
 where $f_G \in \langle {\bf g} \rangle$ and $h(T, \mat{X}) \in \field[T, \mat{X}]$. Let $D$ be the total degree of $f_G$ and $f_G^{\mathrm{H}} \in \field[X_0, \mat{X}]$ be the homogenize polynomial of $f_G$ using a new variable $X_0$, that is, $f_G^{\mathrm{H}}(X_0, \mat{X}) = X_0^Df_G(\mat{X}/X_0)$. Then, 
 \begin{equation}\label{e2}
 f_G(\mathbf{x}(T)) = T^{-Dd}f_G^{\mathrm{H}}(T^d, s_1(T),\ldots, s_n(T)) 
 \end{equation}
From \cref{e1} and \cref{e2}, we have 
\[
(1-T)^pf_G^{\mathrm{H}}(T^d, s_1(T), \ldots, s_n(T))  + T.h(T, f_1(T), \ldots, f_n(T)).T^{Dd} = 0.
\]
On one hand, since the total degree of $h_{\mat{X}}(T, \mat{X})$ is at most $D$, then \[\nu(h(T, f_1(T), \ldots, f_n(T))T^{Dd}) \ge 0\] This means $\nu(T.h(T, f_1(T), \ldots, f_n(T)).T^{Dd}) > 0$. On the other hand, we claim that \[\nu(f_G^{\mathrm{H}}(T^d, s_1(T), \ldots, s_n(T))) = 0.\] Indeed, it is easy to see that $\nu(f_G^{\mathrm{H}}(T^d, s_1(T), \ldots, s_n(T))) \ge 0$. Furthermore, we claim that $f_G^{\mathrm{H}}(0, s_1(0), \ldots, s_n(0)) \ne 0$. Then, as a consequence,  
\[
(1-T)^pf_G^{\mathrm{H}}(T^d, s_1(T), \ldots, s_n(T))  + T.h(T, f_1(T), \ldots, f_n(T))T^{Dd} \ne 0
\]
 which is a constradiction. Thus, the remaing problem we need to prove is $f_G^{\mathrm{H}}(0, s_1(0), \ldots, s_n(0))$ is indeed different zero. Assume a contradiction that $f_G^{\mathrm{H}}(0, s_1(0), \ldots, s_n(0)) = 0$.
%\begin{enumerate}
\paragraph{Column degrees cases:} Since any $f_G \in \langle {\bf g} \rangle$, $f_G$ has the form $f_G = \lambda. g_{i_1} \ldots g_{i_p}$ where $(i_1, \ldots, i_p)\subset \{1, \ldots, q\}^p$ and $\lambda \in \field$, then 
\[
f_G^{\mathrm{H}}(X_0, \mat{X}) = \lambda.g_{i_1}^{\mathrm{H}}(X_0,\mat{X}) \ldots g_{i_p}^{\mathrm{H}}(X_0, \mat{X}).
\] This implies 
\[
f_G^{\mathrm{H}}(0, s_1(0), \ldots, s_n(0)) = \lambda.g_{i_1}^{\mathrm{Hom}}(0, s_1(0), \ldots, s_n(0)) \ldots g_{i_p}^{\mathrm{H}}(0, s_1(0),\ldots, s_n(0)),
\] i.e., 
\[
g_{i_1}^{\mathrm{H}}(0, s_1(0), \ldots, s_n(0)) \ldots g_{i_p}^{\mathrm{H}}(0, s_1(0), \ldots, s_n(0)) = 0.
\]
Moreover, $(0, s_1(0), \ldots, s_n(0))$ is a solution of $n$ polynomial equations $f_G^{\mathrm{H}}(T, \mat{X}) = 0$; and $g_{i_j}$ is defined as a product of linear forms as above. Therefore, 
\[
\left( \begin{matrix}
\beta_{1,1} & \cdots & \beta_{1,n}\\
\vdots &  \ddots & \vdots \\
\beta_{n,1} & \cdots & \beta_{n,n}\\
\end{matrix} \right)\left(\begin{matrix}
s_1(0)\\
\vdots \\
s_n(0)
\end{matrix}\right) = 0,
\]
where $(\beta_{i,1}, \ldots, \beta_{i,n}) \in \{(\lambda_{i,1}^{(1)}, \lambda_{i,1}^{(2)}, \ldots, \lambda_{i,1}^{(n)}), \ldots, (\lambda_{i,D_1}^{(1)}, \lambda_{i,D_1}^{(2)}, \ldots, \lambda_{i,D_1}^{(n)})\}$. By the construction of $G$ as above, we have the matrix $B \in \field^{n \times n}$ is full rank, where $B = [\beta_{i,j}]_{1 \leq i \leq n, 1 \leq j \leq n}$. From which and $B[s_1(0), \ldots, s_n(0)]^T = 0$ we deduce that $(s_1(0), \ldots, s_n(0)) = (0, ..., 0)$. This contradicts with the fact that $(s_1(0), \ldots, s_n(0)) \ne (0, \ldots, 0)$. 

Hence, we obtain $f_G^{\mathrm{H}}(0, s_1(0),\ldots, s_n(0)) \ne 0$. Therefore, $\nu(f_G^{\mathrm{H}}(T^d, s_1(T), \ldots, s_n(T))) = 0$, this implies $\nu((1-t)^pf_G^{\mathrm{H}}(T^d, s_1(T), \ldots, s_n(T))) = 0$. Thus, \[(1-T)^pf_G^{\mathrm{H}}(T^d, s_1(T), \ldots, s_n(T))  + T.h(T, f_1(T), \ldots, f_n(T))T^{Dd} \neq 0\] which is a contradiction. 
\paragraph{Row degrees case:} \todo{todo: finish the proof}

\end{proof}
\begin{Proposition}
There are no singular solution of ${\bf g}$.
\end{Proposition}
\begin{proof}
For the column degrees case, the result can be obained directly from the construction of the start matrix $G$ in the \improve{\cref{subsec:cd}}. For the row degrees case, the genericity assumption $\mathcal{A}(1)$ in \improve{\cref{subsec:genericity}} says that the ideal $\langle {\bf g} \rangle$ is radical, that is $\mathrm{Jac}({\bf g})(\mathbf{x}^*)$ has full rank for any $\mathbf{x}^* \in V({\bf g})$. In other words, any solution of ${\bf g}$ is regular.  
\end{proof}
By using all the results in this section, we obtain the following theorem which is one of our main results.
\begin{Theorem}
Let $F \in \field[\mat{X}]^{p \times q}$ with $p \leq q$ and $\f$ be the ideal generated by the $p \times p$ minors of $F$. Then, the number of isolated solutions of $\f$ is at most $\mathcal{T}$. 
\end{Theorem}
\section{Excerpting square system}
\label{sec:extractsubsystem}
Given $\mathbf{x}^*$ in $V(\g)$, the aim of this section is extracting a square subsystem of determinantal system with full rank Jacobian for Newton iteration. 

Let $\mathbf{x}^*$ be a solution of $\g$ and $\bar{G} \in \field[\mat{X}]^{(p-1) \times (p-1)}$ be a submatrix of $G$ such that $\det(\bar{G})(\mathbf{x}^*) \ne 0$. Let $\bar{F}$ and $\bar{H}$ be the $(p-1) \times (p-1)$ submatrices of $F$ and $H$, respectively with the column indices as the $\bar{G}$ column indices. Then, $\bar{H} = (1-T).\bar{G} + T.\bar{F}$. By using the Proposition $\ref{det}$, we have 
\[
\det(\bar{H}) = (1-T)^{p-1} \det(\bar{G}) + T.h(T,\mat{X}),
\] where $h(T,\mat{X}) \in \field[T,\mat{X}]$. This implies 
\[
\det(\bar{H})(\mathbf{x}^*) = (1-T)^{p-1} \det(\bar{G})(\mathbf{x}^*) + T.h(T,\mat{X})(\mathbf{x}^*). 
\] From which and $\det(\bar{G})(\mathbf{x}^*) \ne 0$, we can deduce $\det(\bar{H})(\mathbf{x}^*)$ has zero-valuation. This means $\det(\bar{H})(\mathbf{x}^*) \ne 0$. We denote $m$ for $\det(\bar{H})$, thanks to $m$ we can construct the system of $n$ equations (see \improve{\cref{subsec:local}}). 

The structure of this section as follows. In \improve{\cref{subsec:local}}, we will present how to obtain this system if we know $m$; after that, in \improve{\cref{subsec:cd2}} and \improve{\cref{subsec:row2}}, we will show how to find the matrix $\bar{G}$ such that $\bar{G}(\mathbf{x}^*) \ne 0$ for fix $\mathbf{x}^* \in V(\g)$ in the case when we work on column degrees and row degrees respectively. As a consequence, we can build the matrix $\bar{H}$. 
\subsection{Local description}
\label{subsec:local}
The main goal of this section is giving a local description of the variety of $\h$. We will show that it sufficies to use $n$ equations to describe the local variety $V(\h)$. We follow the similar way as in \cite[Section ~ 2.2]{Bank2001}. 

Let $H = [h_{i,j}]$ be a $p \times q$ polynomial matrix of $n+1$ variables with $p \leq q$. Let $l$ and $k$ be any natural numbers with $l \leq q$ and $k \leq \min\{p,l\}$. Let $I_k = (i_1, \ldots, i_k) \subset \{1, \ldots, l\}^k$ be an ordered sequence and $M(I_k)$ be the determinant of $H_{1:k;I_k}$. The Exchange Lemma below will be an important tool in this section. 
\begin{Lemma}$\cite{Bank2001}$ Let $H$, $l$ and $k$ be given as above. Let $I_k = (i_1, \ldots, i_k)$ amd $I_{k-1} = (j_1, \ldots, j_{k-1})$ be two index sets such that $I_k \cap I_{k-1} \ne \emptyset$. Then, 
\[
M({I_{k-1}})M(I_k) = \sum_{j \in I_k \backslash I_{k-1}} \epsilon_j \ M(I_k \backslash \{j\}) \ M(I_{k-1} \cup \{j\}), 
\] for suitable number $\epsilon_j \in \{1, -1\}$. \label{exchange}
\end{Lemma}
The following proposition is a similar version of \cite[Proposition ~ 5]{Bank2001}. In \cite{Bank2001}, the authors use the matrix $H$ as the Jacobian matrix of $p$ polynomials $(f_1, \ldots, f_p) \in \field[\mat{X}]^p$ while here, we rewrite this proposition for the general polynomial matrix $H \in \field[T,\mat{X}]^{p \times q}$ where $p \leq q$. Noting that in \cite{Bank2001}, there is a condition that $p \leq n$. 

Let $m \in \field[T,\mat{X}]$ be the $(p-1) \times (p-1)$ minor of $H$ given by the first $(p-1)$ rows and $(p-1)$ columns, i.e., $m = \det(H_{1:p-1;1:p-1})$. Let us define $V(m) := \{\mathbf{x} \in \bar{\field[T]}^n : m(\mathbf{x}) = 0\}$ and $V(\h)_m : = V(\h) \backslash V(m)$, where $V(\h) = \{\mathbf{x} \in \bar{\field}^{n} : f_H(\mathbf{x}) = 0 \ \mathrm{for \ all} \ f_H \in \langle \h \rangle \}$. Hereafter, for any $1 \leq i_1 \leq \cdots \leq i_p \leq n$, let us denote $M(i_1, \ldots, i_p) \in \field[T,\mat{X}]$ for the determinant of the submatrix of $H$ which contains $p$ rows and the columns $i_1, \ldots, i_p$. 
\begin{Proposition} \label{ppp} Let $m, V(\h), V(\h)_m$ and $V(m)$ be defined as above. Then, 
\[
V(\h)_m = \{\mathbf{x} \in \bar{\field[T]}^n \ | \ M(1, \ldots, p-1, s) = 0, m(\mathbf{x}) \ne 0 \ \mathrm{for} \ s \in \{p, \ldots, q\} \}.
\] In other words, the variety $V(\h)$ is locally described by $q - p + 1$ polynomials $(outside \ of \ V(m))$, i.e., $n$ polynomials 
\[
M(1, \ldots, p-1, p), M(1, \ldots, p-1, p+	1), \ldots, M(1, \ldots, p-1, q).
\]
\end{Proposition}
\begin{proof}
$(\subseteq)$ It is obvious that for any $\mathbf{x} \in \bar{\field[T]}^n$ such that $\mathbf{x} \in V(\h)_m$ we have 
\[\mathbf{x} \in \{\mathbf{x} \in \bar{\field[T]}^n \ | \ M(1, \ldots, p-1, s) = 0, m(\mathbf{x}) \ne 0 \ \mathrm{for} \ s \in \{p, \ldots, q\} \}.\]

$(\supseteq)$ Let $\mathbf{x}$ be any point in $\bar{\field[T]}^n$ such that $m(\mathbf{x}) \ne 0$ and $M(1, \ldots, p-1, s)(\mathbf{x}) = 0$ for any $s \in \{p, \ldots, q\}$. We have to prove that $M(i_1, \ldots, i_p)(\mathbf{x}) = 0$ for any $(i_1, \ldots, i_p) \subset \{1, \ldots, q\}^p$. By using Lemma $\ref{exchange}$ for $M(1, \ldots, p-1) = m$, we have 
\[
m.M(i_1, \ldots, i_p) = \sum_{\{j \in \{i_1, \ldots, i_p\} \backslash \{1, \ldots, p-1 \} } \epsilon_j \ M(\{i_1, \ldots, i_p\} \backslash \{j\}) \ M(1, \ldots, p-1,j),
\] for suitable $\epsilon_j \in \{1,-1\}$. Therefore, 
\[
m(\mathbf{x}).M(i_1, \ldots, i_p)(\mathbf{x}) = \sum_{j \in \{i_1, \ldots, i_p\} \backslash \{1, \ldots, p-1 \} } \epsilon_j \ M(\{i_1, \ldots, i_p\} \backslash \{j\})(\mathbf{x}) \ M(1, \ldots, p-1,j)(\mathbf{x}). 
\] Moreover, $M(1, \ldots, p-1, s)(\mathbf{x}) = 0$ for any $s \in \{p, \ldots, q\}$ and $m(\mathbf{x}) \ne 0$, so $M(i_1, \ldots, i_p)(\mathbf{x}) = 0$ for any $(i_1, \ldots, i_p) \subset \{1, \ldots, q\}^p$. 
\end{proof}
\begin{Remark}The result in Proposition $\ref{ppp}$ remains true if we replace $m$ by any $(p-1) \times (p-1)$ minor of $H$. 
\end{Remark}
\subsection{Column degrees}
\label{subsec:cd2}
In this section, we will find a $(p-1)\times (p-1)$ submatrix of $H$, namely $\bar{H}$, such that $m(\mathbf{x}^*) \ne 0$, where $m = \det(\bar{H})$ and $\mathbf{x}^*$ is a solution of $\g$ when we work on the case $\deg(f_{i,j}) = D_j$ for all $1 \leq i \leq p$. Let us recall the starting matrix $G$ in this case is 
\[G = 
\left( \begin{matrix}
g_1 & 2g_2 & \cdots & qg_{q}\\
g_1 & 2^2g_2 & \cdots & q^2g_q\\
\vdots & \vdots & \ddots & \vdots \\
g_1 & 2^pg_2 & \cdots & q^pg_q
\end{matrix} \right),
\]
where $g_{i}$ is the product of $D_i$ linear forms, i.e., $g_i = \prod_{j = 1}^{D_i}L_{i,j}$. Moreover, any $p \times p$ minor of $G$ has the form $\lambda g_{i_1}\ldots g_{i_p}$, where $(i_1, \ldots, i_p) \in \{1, \ldots,q\}^{p}$ and $\lambda \in \field$.

Without loss of generity, we take $\mathbf{x}^*$ is a solution of the system of $q-p+1$ equations, i.e., $n$ equations $g_1 = \cdots = g_{q-p+1} = 0$. Let $\bar{G} \in \field[\mat{X}]^{(p-1) \times (p-1)}$ be a submatrix of $G_{*;q-p+2:q}$, where $G_{*;q-p+2:q} \in \field[\mat{X}]^{p \times (p-1)}$ contains the columns $q-p+2, \ldots, q$ of $G$. So, $\det(\bar{G}) = \lambda g_{q-p+2}\ldots g_{q}$ for $\lambda \in \field$. By the construction of $\{g_i\}_{i=1}^{q}$, we have $g_i(\mathbf{x}^*) \ne 0$ for all $q-p+2 \leq i \leq q$. As a consequence, $\det(\bar{G})(\mathbf{x}^*) \ne 0$. 

Now, we need to check that the Jacobian matrix corresponding to this system has full rank such that we can use the Newton interation. By the construction of $g_{i}$ as a product of $D_i$ linear form, $\mathbf{x}^*$ is the solution of a linear system of $n$ equations. It is obvious to verify that the Jacobian matrix of this system at $\mathbf{x}^*$ has full rank.
\subsection{Row degrees}
\label{subsec:row2}
In this section, we will find a $(p-1)\times (p-1)$ submatrix of $H$, namely $\bar{H}$, such that $m(\mathbf{x}^*) \ne 0$, where $m = \det(\bar{H})$ and $\mathbf{x}^*$ is a solution of $\g$ when we work on the case $\deg(f_{i,j}) = D_i$ for all $1 \leq i \leq p, 1\leq j \leq q$. Let us recall the starting matrix $G$ in this case is 
\[ G = \left( \begin{matrix}
g_1 & 0 & \cdots & 0 & g_{1,p+1} & \cdots & g_{1, q}\\
0 & g_2 & \cdots & 0 & g_{2,p+1} & \cdots & g_{2, q}\\
\vdots & \vdots & \ddots & \vdots & \vdots & \ddots & \vdots\\
0 & 0 & \cdots & g_p & g_{p,p+1} & \cdots & g_{p, q}
\end{matrix} \right), \] where all $\{g_i\}_{1 \leq i \leq p}$ and $\{g_{k,j}\}_{1 \leq k \leq p, p+1\leq j \leq q}$ has generic coefficients. We have seen that for any solution $\mathbf{x}^{*}$ of $\g$, there is at least one $i \in \{1, \ldots , p\}$ such that $g_i(\mathbf{x}^*) = 0$. 

If there is only one $i \in \{1, \ldots, p\}$, without loss of generility we assume that $i=p$, such that $g_{p}(\mathbf{x}^*) = 0$ and $g_{j}(\mathbf{x}^*) \ne 0$ for $j \in \{1, \ldots, p-1\}$; then we define $\bar{G} = \mathrm{diag}(g_1, \ldots, g_{p-1})$. So, $\det(\bar{G})(\mathbf{x}^*) = \prod_{j=1}^{p-1}g_j(\mathbf{x}^*) \ne 0$.

If there are $k$ polynomials $g_{i_1}, \ldots, g_{i_k}$ for $(i_1, \ldots, i_k) \subset \{1, \ldots, p\}^k$ such that $g_{i_1}(\mathbf{x}^*) = \cdots = g_{i_k}(\mathbf{x}^*) = 0$ and $g_{i_j}(\mathbf{x}^*) \ne 0$ for $i_j \in \{1, \ldots, p\} \setminus \{i_1, \ldots, i_k\}$, we define $\bar{G} \in \field[\mat{X}]^{(p-1) \times (p-1)}$ as follows. Without loss of generity, let us work on $(i_1, \ldots, i_k) = (p-k+1, \ldots, p)$. For other tuples $(i_1, \ldots, i_k)$, we can use the similar argument. The idea is using these $g_{i_j}$ such that $g_{i_j}(\mathbf{x}^*) \ne 0$ to build the matrix $\bar{G}$. To sum up, we have $g_{p-k+1}(\mathbf{x}^*) = \cdots = g_{p}(\mathbf{x}^*) = 0$ and $g_j(\mathbf{x}^*) \ne 0$ for $1 \leq j \leq p-k$. Let us denote $A \in \field[\mat{X}]^{k \times (q-p)}$ for $G_{p-k+1:p\mathbf{;}p+1:q}$. We claim that there exists a submatrix ${G}^* \in \field[\mat{X}]^{(k-1) \times (k-1)}$ of $A$ such that $\det({G}^*)(\mathbf{x}^*) \ne 0$. Indeed, we assume a contradiction that for all ${G}^* \in \field[\mat{X}]^{(k-1) \times (k-1)}$ which are submatrices of $A$ we have $\det(\bar{G})(\mathbf{x}^*) = 0$. Therefore, for all $f_{A} \in I_A$ where $I_A = \langle k \times k - \mathrm{minors \ of \ A}\rangle$, we have $f_A(\mathbf{x}^*) = 0$. This contradicts with the fact that $I_G$ has exactly $S_n(D_1, \ldots, D_p)$ distinct solutions under the generic assumption. We have finished the existence of the matrix ${G}^*$, we are going to find where is ${G}^*$ in $A$. 

Since there exists ${G}^* \in \field[\mat{X}]^{(k-1) \times (k-1)}$  of $A$ such that $\det{{G}^*}(\mathbf{x}^*) \ne 0$ and $\mathrm{rank}(A(\mathbf{x}^{*})) \leq k-1$ (from Proposition $\ref{r2}$), then $\mathrm{rank}(A(\mathbf{x}^{*})) = k-1$. In other words, ${G}^*(\mathbf{x}^*) \in \field^{(k-1)\times (k-1)}$ is a submatrix of $A(\mathbf{x}^*) \in \field^{k \times (q-p)}$ such that ${G}^*(\mathbf{x}^*)$ is full rank. Therefore, we can first, evaluate the matrix $A$ at the point $\mathbf{x}^*$ to obtain the matrix $A(\mathbf{x}^*) \in \field^{k \times (q-p)}$; and afterthat, thanks to Gaussian eliminations we can find a submatrix ${G}^*(\mathbf{x}^*)$ of $A(\mathbf{x}^*)$ such that $\mathrm{rank}({G}^*(\mathbf{x}^*)) = k-1$.  We obtain the matrix ${G}^* \in \field[\mat{X}]^{(k-1) \times (k-1)}$ with the indices are the same as those of ${G}^*(\mathbf{x}^*)$. Finally, let us define $\bar{G}$ as 
\[\bar{G} = 
\left[\begin{array}{c |c}%\hline
\mathrm{diag}(g_1, \ldots, g_{p-k}) & G_{1:p-k; \mathcal{J}} \\  \hline
\mat{O}_{k-1,p-k} & G^*\\
%\hline
\end{array}
\right] \in \field[\mat{X}]^{(p-1) \times (p-1)} \ ,\] where $\mathcal{J}$ is the column indices set of $G^*$. Then, $\det(\bar{G}) = \prod_{j = 1}^{p-k}g_j\det(G^*)$, for which we obtain $\det(\bar{G})(\mathbf{x}^*) \ne 0$.

For $\mathbf{x}^* \in V(\g)$, the \todo{remaining problem} in this subsection is to verify that this square subsystem has full rank Jacobian matrix. 
\section{Homotopy techniques for determinantal systems} 
Given a multivariate polynomial matrix $F \in \field[\mat{X}]^{p \times q}$ with the number of variables is $q-p+1$ and $p \leq q$; in this section, we build an algorithm by using the symbolic homotopy techniques to compute the isolated solutions of the determinantal system made by all $p \times p$ minors of $F$. 

Let us denote ${\bf{f}} = (f_1, \ldots, f_M) \in \field[{\mat{X}}]^M$ be the system of $p \times p$ minors of the input matrix $F \in \field[\mat{X}]^{p \times q}$ with $M = {{q}\choose{p}}$. Let $G \in \field[\mat{X}]^{p \times q}$ be a start matrix and $T$ is a new variable. Let us define the homotopy $H = (1-T)G + T.F \in \field[T,\mat{X}]^{p \times q}$. We denote ${\bf g} = (g_1, \ldots, g_M)$ in $\field[\mat{X}]^M$ and ${\bf h} = (h_1, \ldots, h_M)$ in $\field[T,\mat{X}]^M$ for the determiantal system of $G$ and $H$ respectively. From Proposition $\ref{det}$, for any $h_i$, there exist $\bar{f}_i$ and $\bar{g}_i$ in $\field[T, \mat{X}]$ such that 
\begin{equation} \label{deteq}
h_i = (1-T)^pg_i + T.\bar{f}_{i} \ \mathrm{and} \ h_i = (1-T)\bar{g}_i + T^pf_i.
\end{equation}

\subsection{Testing a point is isolated}
\label{subsec:isolated}
Let $\mathbf{f}=(f_1,\dots,f_M)$ be polynomials in $\field[\mat{X}]$, with $\mat{X}=(X_1,\dots,X_n)$, for a field $\field$. Given a point $\mathbf{x}$ in $V(\mathbf{f})$, we discuss here how to decide whether $\mathbf{x}$ is an isolated point in $V(\mathbf{f})$. Without loss of generality, we assume that $M\ge n$; otherwise, $\mathbf{x}$
cannot be an isolated solution. In the determinantal system context, $M$ equals ${q \choose p}$, the number of $p\times p$ minors of $F$. In this report, we work on the case $n = q-p+1$, so we have $M \ge n$. We make the following assumption (denoted by $\mathsf{H}$ below): we are given as input an integer $\mu$ such that
\begin{itemize}
\item either $\mathbf{x}$ belongs to a positive-dimensional component of $V(\mathbf{f})$,
\item or $\mathbf{x}$ is isolated in $V(\mathbf{f})$, with multiplicity at most $\mu$
  with respect to the ideal $\langle \mathbf{f} \rangle$.
\end{itemize}

\begin{Proposition}\label{testisolated} Suppose that $\mathbf{f}$ is given by a straight-line program of length $\mathcal{E}$. If assumption $\mathsf{H}$ is satisfied, we can decide whether $\mathbf{x}$ is an isolated root of $V(\mathbf{f})$ using $(\mu \mathcal{E} M)^{O(1)}$ operations in $\field$.
\end{Proposition}
\begin{proof}
The proof is given in \improve{\cref{sec:prooftestisolated}}. 
\end{proof}
\subsection{Some properties of a parametric system of equations}
Let $\mat{X}=(X_1,\dots,X_N)$ be indeterminates over $\field$, as above, and let $T$ be a new variable. We consider polynomials $\mathbf{h}=(h_1,\dots,h_M)$ in $\field[T,\mat{X}]$, with $M \ge n$, and the ideal $J=\langle \mathbf{h} \rangle \subset \bar{\field}[\mat{X}]$; for $\tau$ in $\bar{K}$, we write $\mathbf{h}_\tau=(h_{\tau,1},\dots,h_{\tau,M})=\mathbf{h}(\tau,\mat{X})\subset \bar{\field}[\mat{X}]$. 

An ideal $I$ of a Noetherian ring $A$ is called \emph{unmixed} if the height of $I$ is equal to the height of every associated prime of $A/I$. The \emph{unmixedness theorem} is said to hold for the ring $A$ if every ideal $I$ generated by a number of elements equal to its height is unmixed. In this subsection, we give some general properties of the system $\mathbf{h}$, that hold under a few assumptions.  First, consider
the following properties related to the system $\mathbf{h}$ itself.
\begin{description}
\item[$\mathsf{H}_1.$] Any irreducible component of $V(J) \subset
  \bar{\field}{}^{n+1}$ has dimension at least one (equivalently, $J$ has
  height at most $N$).
\item[$\mathsf{H}_2.$] For any prime $P \subset\bar{\field},\mat{X}]$, if the
  localization $J_P \subset \bar{\field},\mat{X}]_P$ has height $n$, then it is
  unmixed (that is, all associated primes have height $n$).
%% \item[${\sf H}_3.$] There exists $\tau$ in $\KKbar$ that satisfies  ${\sf G}(\tau)$.
\end{description}
Then, for $\tau$ in $\bar{K}$, we denote by $\mathsf{G}(\tau)$ the following three properties.
\begin{description}
\item[$\mathsf{G}_1(\tau).$] The ideal $\langle \mathbf{h}_\tau \rangle$ is radical in $\field[\mat{X}]$.
\item[$\mathsf{G}_2(\tau).$] For $k=1,\dots,M$,
  $\deg_\mat{X}(h_k)=\deg_\mat{X}(h_{\tau,k})$.
\item[$\mathsf{G}_3(\tau).$] The only common solution to $h_{\tau,1}^H(0,\mat{X})=\cdots=h_{\tau,M}^H(0,\mat{X})=0$ is $(0,\dots,0)\in \bar{\field}{}^n$, where for $k=1,\dots,M$, $h_{\tau,k}^H$ is the polynomial in $\bar{\field}[X_0,\mat{X}]$ obtained by homogenizing $h_{\tau,k}$ using a new variable $X_0$. In particular, $V(\mathbf{h}\tau) \subset \field{}^n$ is finite.
\end{description}

The main result in this subsection is the following.
\begin{Proposition}\label{degree_fiber}
Suppose that $\mathsf{H}_1$ and $\mathsf{H}_2$ hold. Then, there exists an integer $c$ such that for all $\tau$ in $\bar{\field}$, the sum of the multiplicities of the isolated solutions of $\mathbf{h}_\tau$ is at most $c$, and is equal to $c$ if $\mathsf{G}(\tau)$ holds.
\end{Proposition}
\begin{proof}
The proof is given in \improve{\cref{sec:proofdegree}}. 
\end{proof}

Consider an irredundant primary decomposition of $J$ in $\bar{\field}[T, \mat{X}]$, of the form $J=Q_1 \cap \cdots \cap Q_r$, and let $P_1,\dots,P_r$ be the associated primes, that is, the respective radicals of $Q_1,\dots,Q_r$. We assume that $P_1,\dots,P_s$ are the minimal primes, for some $s \le r$, so that $V(P_1), \dots,V(P_s)$ are the irreducible components of $V(J)\subset \bar{\field}{}^{n+1}$. By ${\sf H}_1$, these irreducible components all have dimension at least one. Refining further, we assume that $t \le s$ is such that $V(P_1),\dots,V(P_t)$ are the irreducible components of $V(J)$ of dimension one whose image by $\pi_T: (\tau,x_1,\dots,x_n) \mapsto \tau$ is dense in $\field$. Let us write $J=J' \cap J''$, with $J'=Q_1 \cap \cdots \cap Q_t$ and $J''=Q_{t+1} \cap \cdots \cap Q_r$. We also introduce  $\frak{J}$, the extension of $J$ in $\bar{\field}(T)[\mat{X}]$, and similarly $\frak{J}'$ and ${\frak J}''$. In the proof of Proposition \ref{degree_fiber} which is given in \improve{\cref{sec:proofdegree}}, we can see $c=\dim_{\bar{\field}(T)}(\bar{\field}(T)[\mat{X}]/{\frak J}')$.

%\todo{Todo: Check ${\sf G}(0)$ holds}.
\subsection{Computing the isolted solutions for determinantal systems}
Given polynomials ${\bf{f}} = (f_1, \ldots, f_M) \in \field[\mat{X}]^M$, we give an algorithm to compute a zero-dimensional parametrization of the isolated points of $V(\bf{f})$. In the deternminantal system context, $\bf{f}$ is the system of $p \times p$ minors of the input matrix $F \in \field[\mat{X}]^{p \times q}$ with $M = {{q}\choose{p}}$. Notice that in this report, we are working on $n = q-p+1$, then $M \ge n$. In general, $M$ should be at least $n$, otherwise no such isolated solutions exists.

First, from $\cref{deteq}$ and the discussions in \improve{\cref{subsec:bounddegree}}, we have ${\bf{f}} = {\bf{h}}_1 = (h_i(1,\mat{X}))_{1 \leq i \leq M}$; and any irreducible component of $V(J) \in \bar{\field}^{n+1}$ has dimension at least one, where $J = \langle {\bf h} \rangle$, that is $\bf{h}$ satisfies $\sf{H}_1$. In addtion, \cite[~Corollary 16.44]{Miller04} says that $\h$ is Cohen--Macaulay, and \cite[~Theorem 17.6]{Matsumura86} tells us that being Cohen--Macaulay, the unmixedness theorem holds for $\h$. In other words, $\bf{h}$ satisfies $\sf{H}_2$. We are going to check the assumption ${\sf G}(0)$ in determinantal system context. Since $\g = {\bf{h}}_0 = (h_i(0,\mat{X}))_{1 \leq i \leq M}$ and by using the genericity assumption in \improve{\cref{subsec:genericity}} we have ${\bf{h}}_0 = \g$ is radical in $\bar{\field}[\mat{X}]$. It is easy to verify the ${\sf G}_2(0)$ from the construction of $H$; and ${\sf G}_3(0)$ is an other expression of Proposition~\ref{valuation}. To sum up, we have
\begin{itemize}
\item ${\bf{f}} = {\bf{h}}_1 = (h_i(1,\mat{X}))_{1 \leq i \leq M}$. 
\item $\bf{h}$ satisfies $\sf{H}_1$ and $\sf{H}_2$.
\item ${\sf G}(0)$ holds.
\end{itemize}
All the notation of the previous subsection is still in use. In order to control the cost of the algorithm, we introduce the following assumptions:
\begin{description}
\item[${\sf A}_1$] Given ${\bf x} \in V(\h_0)$ having coordinates in a field extension $\mathbb{L}$ of $\field$, we can find in time $O\tilde{~}(B\, [\mathbb{L} : \field])$ a sequence ${\bm i}_{\mathbf{x}}= (i_1,\dots,i_n)$, with $1 \le i_1 < \dots < i_n \le M$, such that the Jacobian matrix of $(h_{0,i_1},\dots,h_{0,{i_n}})$ has full rank $n$ at ${\bf x}$, for some $B$ independent of $0$ or ${\bf x}$.
\item[${\sf A}_2$] We know an integer $d$ such that the curve $V(J')$ has degree at most $d$. Lemma~\ref{lemma:vPi} implies that  $c \le d$.
\item[${\sf A}_3$] For any ${\bm i}=(i_1,\dots,i_n)$, with $1 \le i_1 <
  \dots < i_n \le M$, we can compute $(h_{i_1},\dots,h_{{i_n}})$ using
  a straight-line program of length $\mathcal{E}$.
\end{description}

In the determinantal systems context, for the assumption ${\sf A}_1$, as we dicussed in the \improve{\cref{sec:extractsubsystem}}, given ${\bf x} \in V(\h_0)$, we can find ${\bm i}_{\bf x}$ in $\bigO{1}$ for the column degrees case; and for the row degrees case, $B$ is in fuction of evaluating point ${\bf x}$ in $\bar{G}$ and the complexity of Gaussian eliminations.  

\todo{Todo: Check these assumptions ${\sf A}_2$ and ${\sf A}_3$}
\paragraph{Starting points.} The algorithm starts by using the set of solution of the determiantal system of a start matrix which is defined as in \improve{\cref{subsec:cd}}. We can use either Algorithm \ref{StartMatCol} for column degrees case or Algorithm \ref{StartMatRow} for the case of row degrees. We denote this set as $V(\g)$; and $|V(\g)| = \mathcal{T}$ with $\mathcal{T} = E_{n}(D_1, \ldots, D_q)$ in column degrees case  or $\mathcal{T} = S_{n}(D_1, \ldots, D_p)$ in row degrees case. For $j  =1, \ldots, \mathcal{T}$, from ${\sf A}_1 $, we can find ${\bm i}_{\bf x}$ such that $(h_{0,i})_{i \in {\bm i}_{\bf x}}$ has full rank $n$. 

\paragraph{Lifting power series and rational reconstruction.} For $j  =1, \ldots, \mathcal{T}$, we apply the Newton interation to the system $(h_{0,i})_{i \in {\bm i}_{\bf x}}$ to lift ${\bf x}$ into a zero-dimensional parametrization $\mathscr{R}_j = ((q_j,v_{j,1}, \ldots, v_{j,n}), \lambda)$ with coefficients in $\field[[T]]/\langle T^{2d}\rangle$, for $d$ as in ${\sf A}_2$. 

As explained in~\cite[Section~2.2]{SaSc16}, using the algorithm of~\cite{GiLeSa01}, this can be done using $(d\,\mathcal{E}\,n)^{\bigO{1}}$ operations in $\field$ (\todo{Todo: Make more precise ???}). Using the Chinese Remainder Theorem, we can combine all $\mathscr{R}_j$ into a single zero-dimensional parametrization $\mathscr{R}$ with coefficients in $\field[[T]]/\langle T^{2d}\rangle$; this can be done in $(dn)^{\bigO{1}}$. 

Using the notation of the previous subsection and let $\Phi_1,\dots,\Phi_{c}$ be the points of $V(\mathfrak{J}')$, with coordinates taken in $\bar{\field}\langle\langle T \rangle\rangle$, then the zeros of $\mathscr{R}$ in $\bar{\field}[[T]]/\langle T^{2d} \rangle$ are truncations of $\Phi_1,\dots,\Phi_{c}$. Moreover, from ${\sf A}_2$, $J'$ is supposed to has degree at most $d$, knowing $\mathscr{R}$  at precision $2d$ allows us to reconstruct a zero-dimensional parametrization $\mathscr{S}$ with coefficients in $\field(T)$ such that the zeros of $\mathscr{S}$ is $V(\mathfrak{J}')$. This is done by applying rational function reconstruction to allcoefficients of $\mathscr{R}$, as in~\cite{Schost03}, and takes time $(d\,n)^{\bigO{1}}$. Without loss of generality, by multiplying all denominators of $\mathscr{R}$ coefficients in $\field(T)$ by the least common multiple of theirs, we can deduce that all polynomials of $\mathscr{R}$, say $q, v_1, \ldots, v_n$, have coefficients in $\field[T]$. The degree bounds in~\cite{Schost03} show that
if we require that $q, v_1, \ldots, v_n$ are in $\field[T][U]$ and without a common factor in $\field[T]$,
their total degrees are at most $d$, so this normalization can be
computed using $(d\,n)^{\bigO{1}}$ operations in $\field$. 
\paragraph{A finite set containing the isolated points of $V(\f)$.} Let us denote $J_1' = J + \langle T -1 \rangle$, then it can be seen in the proof of Lemma~\ref{lemma:19} that $V(J_1')$ is finite. In addition, Lemma~\ref{lemma:vPi} implies that for any isolated solution ${\bf x}$ of $\f$, $(1,{\bf x})$ is in $V(J'_1)$. So, we first deduce from $\mathscr{S}$ a zero-dimensional parametrization $\mathscr{R}_1$ with coefficients in $\field$
of $V(J'_1)$; and then by using the algorithm of \improve{\cref{subsec:isolated}}, we discard from $V(J'_1)$ those points that do not correspond to isolated points of $V(\f)$. 

Let $\Phi'_1,\dots,\Phi'_c$ be the roots of $\mathfrak{J}'$ in the field of Puiseux series $\bar{\field}\langle\langle T'\rangle\rangle$ at $T=1$, where $T' = T-1$. Without loss of generality, we assume that
$\Phi'_1,\dots,\Phi'_\kappa$ are bounded, and
$\Phi'_{\kappa+1},\dots,\Phi'_c$ are not, for some $\kappa$ in
$\{0,\dots,c\}$, and we let $\varphi'_1,\dots,\varphi'_\kappa$ by
$\varphi'_i=\lim_0(\Phi'_i)\in\bar{\field}{}^n$ for
$i=1,\dots,\kappa$, where $\lim_0(\Phi'_i)$ is the coefficient of $T^0$ in $\varphi'_i$. Lemma~4.4 in~\cite{RRS} then shows how to recover a zero-dimensional parametrization
$\mathscr{R}_1=((q_1,v_{1,1},\dots,v_{1,n}),\lambda)$ with coefficients in
$\field$ for the limit set $\{\varphi'_i \mid i=1,\dots,\kappa\}$
starting from $\mathscr{S}$, under some conditions on the linear form
$\lambda$. As showed in~\cite{SaSc16}, these conditions are satisfied for a generic choice
of $\lambda$. Overall, $\mathscr{R}_1$ can be computed in time $(d\,n)^{\mathcal{O}(1)}$.

The following lemma shows that the parametrization $\mathscr{R}_1$ we just obtained 
describes $V(J' + \langle T' \rangle)=V(J' + \langle T-1 \rangle)$.

\begin{Lemma}\label{lemma:Z1}
  The equality $V(J' +\langle T' \rangle)=\{\varphi'_i \mid i=1,\dots,\kappa\}$ holds.
\end{Lemma}
\begin{proof}
The proof is given in \improve{\cref{sec:proofZ1}}. 
\end{proof}

\subsection{Algorithms.} We first give an algorithm called Algorithm~\ref{DetSys} to compute the isolated solutions set, $S$,  for the determinanal system $\f$ when knowing the solution set $V(\g)$ of the determinantal system of a start matrix. We denote $\mathscr{R}_{1}$ for a zero-dimensional parametrization with coefficients in $\field$ such that $S \subset \mathscr{R}_{1}$. Let $H \in \field[T, \mat{X}]^{p \times q}$ be defined as before, that is $H = (1 - T).G + T.F$. For each ${\bf x} \in V(\g)$, let $\h^{(\bf x)}$ in $\field[T,\mat{X}]^n$ a square subsystem such that $\h^{(\bf x)}(0,{\bf x}) = 0$ and the Jacobian matrix of $\h^{(\bf x)}$ has full rank $n$ at ${\bf x}$. Let us denote $d$ is either $E_{n}(D_1+1, \ldots, D_q+1)$ in column degrees case or $S_{n}(D_1+1, \ldots, D_p+1)$ in row degrees case. 

Through all sections of this report, we obtain the result as follows.
\begin{Theorem}
The Algorithm \ref{DetSys} is correct and its complexity is $\big({{q}\choose{p}} \,\mathcal{E}\,\mathcal{T}\big)^{\bigO{1}}$ operations in $\field$.
\end{Theorem}
\begin{algorithm}
\caption{$\mathsf{Determinantal System}$}
\label{DetSys}
{\bf Input}: a matrix $F \in \field[\mat{X}]^{p \times q}$ with $p \leq q$, the set $V(\g)$.\\
{\bf Output}: the isolated solutions set of $\f$. 
\begin{enumerate}
\item for any ${\bf x} \in V(\g)$: 
\begin{enumerate}
\item find a system $\h^{(\bf x)}$ in $\field[T,\mat{X}]^n$
\item compute $\mathscr{R}_{\bf x}$ zero-dimensional parametrization at precision $2d$. \\
$\mathsf{//* \ use \ \todo{algorithm???} \ in~\cite{GiLeSa01} \ }$
\end{enumerate}
\item combine $\{\mathscr{R}_{\bf x}\}_{{\bf x} \in V(\g)}$ into $\mathscr{R}$
\item compute $\mathscr{S}$ with coefficients in $\field(T)$ from  $\mathscr{R}$\\
$\mathsf{//* apply \ \todo{algorithm???} \ in~\cite{Schost03} \ to \ all \ coefficients \ of \ \mathscr{R} \ at \ degree} \ d$
\item clean denominators of $\mathscr{S}$
\item deduce $\mathscr{R}_{1}$ from $\mathscr{S}$
\item  $S \gets$ removing from $\mathscr{R}_{1}$ non-isolated point of $V(\f)$\\
$\mathsf{//* apply \ algorithm \ in~\cref{subsec:isolated}}$
\item return $S$
\end{enumerate}
\end{algorithm}

\begin{algorithm}
\caption{$\mathsf{ColumnDeterminantal System}$}
\label{CDetSys}
{\bf Input}: a matrix $F \in \field[\mat{X}]^{p \times q}$ with $p \leq q$ and $\deg(f_{i,j}) \leq D_j$ for all $1 \leq i \leq p$.\\
{\bf Output}: the isolated solutions set of $\f$. 
\begin{enumerate}
\item define a column start matrix $G \in \field[\mat{X}]^{p \times q}$ as in \cref{subsec:cd}
\item $V(\g) \gets \mathsf{Start Matrix Column Degrees}(G)$
\item return $\mathsf{Determinantal System}(F, V(\g))$
\end{enumerate}
\end{algorithm}

\begin{algorithm}
\caption{$\mathsf{RowDeterminantal System}$}
\label{RDetSys}
{\bf Input}: a matrix $F \in \field[\mat{X}]^{p \times q}$ with $p \leq q$ and $\deg(f_{i,j}) \leq D_i$ for all $1 \leq j \leq q$.\\
{\bf Output}: the isolated solutions set of $\f$.
\begin{enumerate}
\item define a row start matrix $G \in \field[\mat{X}]^{p \times q}$ as in \cref{subsec:cd}
\item $V(\g) \gets \mathsf{Start Matrix Row Degrees}(G)$
\item return $\mathsf{Determinantal System}(F, V(\g))$
\end{enumerate}
\end{algorithm}
In summary, if the input matrix is $F \in \field[\mat{X}]^{p \times q}$ such that $p \leq q$ and $\deg(f_{i,j}) \leq D_j$ for all $1 \leq i \leq p$, we will use the Algorithm~$\ref{CDetSys}$; and if the input matrix is $F \in \field[\mat{X}]^{p \times q}$ such that $p \leq q$ and $\deg(f_{i,j}) \leq D_i$ for all $1 \leq i \leq q$, we can use the Algorithm~\ref{RDetSys} for the Problem~\ref{problem}. 
\subsection{Implementation}
\section{Acknowledgements}
\newpage
\bibliographystyle{plain}
\bibliography{biblio}
\appendix
\section{Proof of Propositon \ref{generic}}
\label{sec:proofgeneric}
Let $N = q(n+1)(\sum_{i=1}^pD_i)$ be the cardinality of the set $\mathcal{G}$. We notice that we are working on $n = q-p+1$, where $n$ is the number of variables. Let us define the matrices $G_1$ and $G_2$ as follow
\[G_1 = \left( \begin{matrix}
\mathfrak{g}_{1,1} & \mathfrak{g}_{1,2} & \cdots  & \mathfrak{g}_{1, q}\\
\mathfrak{g}_{2,1} & \mathfrak{g}_{2,2} & \cdots  & \mathfrak{g}_{2, q}\\
\vdots & \vdots & \ddots & \vdots \\
\mathfrak{g}_{p,1} & \mathfrak{g}_{p,2} & \cdots  & \mathfrak{g}_{p, q}
\end{matrix} \right) \ \mathrm{and} \ 
 G_2 = \left( \begin{matrix}
\mathfrak{g}_{1,1} & 0 & \cdots & 0 & \mathfrak{g}_{1,p+1} & \cdots & \mathfrak{g}_{1, q}\\
0 & \mathfrak{g}_{2,2} & \cdots & 0 & \mathfrak{g}_{2,p+1} & \cdots & \mathfrak{g}_{2, q}\\
\vdots & \vdots & \ddots & \vdots & \vdots & \ddots & \vdots\\
0 & 0 & \cdots & \mathfrak{g}_{p,p} & \mathfrak{g}_{p,p+1} & \cdots & \mathfrak{g}_{p, q}
\end{matrix} \right). \] 

The idea to prove $\mathcal{A}(k)$, for any $k \in \{1,2,3\}$, is by using induction on the number of variables, $n$, for the matrix of form $G_1$; and, after that we show that the property $\mathcal{A}(k)$ is true for the matrix $G_2$. Finally, by using this property of form $G_2$, we prove  $\mathcal{A}(k)$ holds for any matrix of form $G_1$. In this section, given a polynomial matrix $G$ and an integer $k < p$, we denote $I_G$ for the ideal generated by all $p \times p$ minors of $G$ and $I_G^{(k)}$ for the ideal generated by all $k \times k$ minors of $G$. 
\subsection{Genericity of $\mathcal{A}(1)$}
In this section, we use the fact that for any $\mathbf{x}^* \in V(I_G)$, the rank of the matrix $G(\mathbf{x}^*)$ is exactly $p-1$ if $\mathbf{x}^*$ does not belong to the variety of the ideal generated by $(p-1)\times (p-1)$ minors of $G$. Indeed, if $\mathbf{x}^* \in V(I_{G}^{(p-1)})$ where $I_{G}^{(p-1)} = \langle (p-1) \times (p-1) -\mathrm{minors \ of} \ G \rangle$, then $\mathrm{rank}(G(\mathbf{x}^*)) \leq p-2$.


$\bf{Step \ 1:}$ First, we will prove that there exist a nonempty Zariski open set  $\mathcal{O}_1$ such that $I_{G_1}$ satisfies $\mathcal{A}(1)$ for $g_{i,j} \in\mathcal{O}_1$, when $G_1$ is a square matrix as follows
\[
G_1 = \left( \begin{matrix}
\mathfrak{g}_{1,1}  & \cdots  & \mathfrak{g}_{1, p}\\
\vdots & \ddots & \vdots \\
\mathfrak{g}_{p,1} & \cdots  & \mathfrak{g}_{p, p}
\end{matrix} \right).
\] We notice that in this case, the number of variables is $n=1$ and the number of coefficients is $N = 2q.(\sum_{i=1}^pD_i)$. Let us define $I_{G_1}^{(p-1)} := \langle (p-1) \times (p-1) - \mathrm{minors \ of \ } G_1 \rangle \subset  \field[\mathcal{G}, \mat{X}]$ and $\Omega_1 = V(I_{G_1}^{(p-1)}) \subset \bar{\field}^{N} \times \mathbb{P}^{1}(\bar{\field})$. Then, $\Omega_1$ is a Zariski closed in $\bar{\field}^{N} \times \mathbb{P}^{1}(\bar{\field})$. Let $\pi_{\mathcal{G}} : \bar{\field}^{N} \times \mathbb{P}^{1}(\bar{\field}) \to \bar{\field}^{N}$ be the projection on the $\mathcal{G}$ coordinates and $\Delta_1 = \pi_{\mathcal{G}}(\Omega_1)$. This implies $\Delta_1$ is a Zariski closed in $\bar{\field}^{N}$. So, we define the Zariski open set $\mathcal{O}_1$ in $\bar{\field}^{N}$ as $\mathcal{O}_1 := \bar{\field}^{N} \setminus \Delta_1$. We claim that $\mathcal{O}_1$ is a nonempty set. Indeed, let $G_2 = \mathrm{diag}(\mathfrak{g}_{1,1}, \ldots, \mathfrak{g}_{1,1})$ be a diagonal matrix and $I_{G_2}^{(p-1)} := \langle (p-1) \times (p-1) - \mathrm{minors \ of \ } G_2 \rangle \subset  \field[\mathcal{G}, \mat{X}]$. Therefore, for any $f_{G_2} \in I_{G_2}^{(p-1)}$, $f_{G_2}$ has the form $\mathfrak{g}_{i_1, i_1} \ldots \mathfrak{g}_{i_{p-1}, i_{p-1}}$ where $(i_1, \ldots, i_{p-1}) \in \{1, \ldots, p\}^{p-1}$. Moreover, for any $1 \leq i \leq p$, we have $\mathfrak{g}_{i,i} = (\gamma_{i,i}^{(1,1)}X_1 + \gamma_{i,i}^{(0,1)}) \times \cdots \times (\gamma_{i,i}^{(1,D_i)}X_1 + \gamma_{i,i}^{(0,D_i)})$. So, for any $(\mathfrak{g}^*, \mathbf{x}^*) \in V(I_{G_2}^{(p-1)})$, it is the solution of $\gamma_{i,i}^{(1,t)}X_1 + \gamma_{i,i}^{(0,t)}X_0 = 0$ for $0 \leq t \leq D_i$. This equation always has solution in $\bar{\field}^{2} \times \mathbb{P}^{1}(\bar{\field})$. This implies, if we define $\bar{\Omega}_1 := V(I_{G_2}^{(p-1)}) \in \bar{\field}^{N_2} \times \mathbb{P}^{1}(\bar{\field})$,  $\bar{\Omega}_1$ will be a Zariski closed in $\bar{\field}^{N_2} \times \mathbb{P}^{1}(\bar{\field})$, where $N_2 = 2.(\sum_{i = 1}^pD_i)$ is the number of generic coefficients for $G_2$. Finally, let us define $\bar{\Delta}_1$ as $\pi_{\bar{\mathcal{G}}}(\bar{\Omega}_1)$, where $\pi_{\bar{\mathcal{G}}} : \bar{\field}^{N_2} \times \mathbb{P}^{1}(\bar{\field}) \to \bar{\field}^{N_2}$; and $\bar{\mathcal{O}}_1 = \bar{\Delta}_1\times 0^{N_1}$ which is nonempty set, where $N_1 = N - N_2$; and it can be saw that 
$\bar{\mathcal{O}}_1 \subset \mathcal{O}_1$. As a consequence,  $\mathcal{O}_1$ is nonempty. We finished the first step of induction. 

$\bf{Step \ 2:}$ Let us now assume that $\mathcal{A}(1)$ holds for any $G_1 \in \field[{\mathcal{G}, \mat{X}}]^{p' \times q'}$ for any $n' \leq n$, where $n' = q'-p'+1$; we will prove $\mathcal{A}(1)$ is also true for $G_2 \in \field[\mathcal{G}, \mat{X}]^{p \times q}$ with $n = q-p+1$. Noting that $I_{G_2}^{(p-1)}$ contains $g_{i_1, i_1}\ldots g_{i_{p-1}, i_{p-1}}$ for any $(i_1, \ldots, i_{p-1}) \in \{1, \ldots, p\}^{p-1}$ and $i_j \ne i_t$. For any $(\mathfrak{g}^*, \mathbf{x}^*) \in (\bar{\field}^{N_2} \times \mathbb{P}^{n}(\bar{\field})) \cap V(I_{G_2}^{(p-1)})$, there are at least two $1\leq j,t \leq p, j \ne t$ such that $\mathfrak{g}_{j,j}(\mathfrak{g}^*, \mathbf{x}^*) = \mathfrak{g}_{t,t}(\mathfrak{g}^*, \mathbf{x}^*) = 0$. If there are $k$ numbers $i_1, \ldots, i_k$, where $2 \leq k \leq \min(n,p)$, such that $g_{i_1, i_1}(\mathfrak{g}^*, \mathbf{x}^*) = \cdots = g_{i_k, i_k}(\mathfrak{g}^*, \mathbf{x}^*) = 0$, we define $\mathcal{J}_k = (i_1, \ldots, i_k)$ and $G_{\mathcal{J}_k, p+1:q} \in \field[\mathcal{G}, \mat{X}]^{(n-k)\times (q-p)}$ is the submatrix of $G_2$ that contains the rows $\mathcal{J}_k$ and columns $p+1, \ldots, q$. By using the similar argument as in Proposition \ref{r2}, we have $\mathrm{rank}(G_{\mathcal{J}_k, p+1:q}(\mathfrak{g}^*, \mathbf{x}^*)) = k-1$ for which the property $\mathcal{A}(1)$ holds as from induction hypothesis. Moreover, $$V(I_{G_2}^{(p-1)}) = \bigcup\limits_{\mathcal{J}_k \subset\{1, \ldots, p\}^k, \ 2 \leq k \leq \min(n,p)} V(I_{\mathcal{J}_k}),$$ where $I_{\mathcal{J}_k}$ is the ideal generated by $\mathfrak{g}_{i,i}$ for $i \in \mathcal{J}_k$ and all $(k-2)\times(k-2) -$ minors of $G_{\mathcal{J}_k, p+1:q}$. Therefore, we can define a nonempty Zariski open, $\bar{\mathcal{O}}_1$, for $G_2$ as follows. For any $\mathcal{J}_k \subset \{1, \ldots, p\}^k$, we rewrite $(X_1, \ldots, X_k) \in \field[\mathcal{G}, X_{k+1}, \ldots, X_n]^{k}$ and subtitute into $G_{\mathcal{J}_k, p+1:q}$ to obtain $G_{\mathcal{J}_k, p+1:q} \in \field[\mathcal{G}, X_{k+1}, \ldots, X_n]^{k \times {(q-p)}}$. By using the induction hypothesis, for any $\mathcal{J}_k$, there exists a Zariski closed set $\Delta_{\mathcal{J}_k} \subsetneq \bar{\field}^{N_{\mathcal{J}_k}}$, where $N_{\mathcal{J}_k} = n(q-p+1)(\sum_{i \in \mathcal{J}_k}D_i)$ is the number of coefficients of the matrix $[\mathrm{diag}(g_{i_1, i_1}, \ldots, g_{i_k, i_k})|G_{\mathcal{J}_k, p+1:q}]$. %$\Omega_{\mathcal{J}_k} \subset \bar{\field}^{N_{\mathcal{J}_k}} \times \mathbb{P}^{n-k+1}(\bar{\field})$ be the variety of the ideal $\langle (k-1) \times (k-1) - \mathrm{minors \ of \ } G_{\mathcal{J}_k, p+1:q} \rangle$, where $N_{\mathcal{J}_k} = n(q-p+1)(\sum_{i \in \mathcal{J}_k}D_i)$ is the number of coefficients of the matrix $[\mathrm{diag}(g_{i_1, i_1}, \ldots, g_{i_k, i_k})|G_{\mathcal{J}_k, p+1:q}]$. We define $\Delta_{\mathcal{J}_k}$ as $\pi_{\mathcal{G}_{\mathcal{J}_k}}(\Delta_{\mathcal{J}_k})$, where $$\mathcal{G}_{\mathcal{J}_k} :  \bar{\field}^{N_{\mathcal{J}_k}} \times \mathbb{P}^{n-k+1}(\bar{\field}) \to \bar{\field}^{N_{\mathcal{J}_k}}.$$ 
We finally, define \[\bar{\mathcal{O}}_1 = \bar{\field}^{N_2} \setminus \Delta, \ \mathrm{where}\  N_2 = n(q-p+1)(\sum_{i=1}^pD_i) \ \mathrm{and} \ \Delta = \bigcup_{\mathcal{J}_k \subset\{1, \ldots, p\}^k, \ 2 \leq k \leq \min(n,p)}\Delta_{\mathcal{J}_k}.\]
This is a nonempty Zariski open set in $\bar{\field}^{N_2}$ such that $I_{G_2}$ satisfies $\mathcal{A}(1)$ when $g_{i,j} \in \bar{\mathcal{O}}_1$. This means we finished the second step of the proof. 

$\bf{Step \ 3:}$ Finally, let assume that $\mathcal{A}(1)$ holds for any $G_2 \in \field[\mathcal{G}, \mat{X}]^{p \times q}$, we will prove $\mathcal{A}(1)$ is also true for any $G_1 \in \field[\mathcal{G}, \mat{X}]^{p \times q}$ as follows. Let us define $I_{G_1}^{(p-1)} = \langle (p-1) \times (p-1) - \mathrm{minors \ of} \ G_1 \rangle$ and $\Omega_1 = V(I_{G_1}^{(p-1)}) \subset \bar{\field}^{N} \times \mathbb{P}^n(\bar{\field})$ is a Zariski closed. Similarly as in the first step, we define $\Delta_1 = \pi_{\mathcal{G}}(\Omega_1)$ is a Zariski closed in $\bar{\field}^{N}$, where $\pi_{\mathcal{G}} : \bar{\field}^{N} \times \mathbb{P}^n(\bar{\field}) \to \bar{\field}^{N}$ is a projection. So, we define the Zariski open set $\mathcal{O}_1$ in $\bar{\field}^{N}$ as $\mathcal{O}_1 := \bar{\field}^{N} \setminus \Delta_1$. By using the matrix $G_1$ of the form as in the second step and similar argument as the first step, we can deduce that $\mathcal{O}_1$ is nonempty. 

We finished the proof for the property $\mathcal{A}(1)$. 
\subsection{Genericity of $\mathcal{A}(2)$}
$\bf{Step \ 1:}$ We show here that there exists a nonempty Zariski open set $\mathcal{O}_2$ such that $I_{G_1}$ has $S_1(D_1, \ldots, D_p)$ (that is $\sum_{i=1}^pD_i$) distinct solutions for $\mathfrak{g}_{i,j} \in \mathcal{O}_2$, when $G_1$ is a square matrix as follows
\[
G_1 = \left( \begin{matrix}
\mathfrak{g}_{1,1}  & \cdots  & \mathfrak{g}_{1, p}\\
\vdots & \ddots & \vdots \\
\mathfrak{g}_{p,1} & \cdots  & \mathfrak{g}_{p, p}
\end{matrix} \right).
\] We notice that in this case, each $\mathfrak{g}_{i,j}$ has the form $\mathfrak{g}_{i,j} = (\gamma_{i,j}^{(1,1)}X_1 + \gamma_{i,j}^{(0,1)}) \times \cdots \times (\gamma_{i,j}^{(1,D_i)}X_1 + \gamma_{i,j}^{(0,D_i)})$. Let us denote $m \in \field[\mathcal{G}, X_1]$ for the deteminant of $G_1$, then $m$ is homogeneous in $\mathcal{G}$ of degree $p$ and $\deg_X(m) = \sum_{i=1}^pD_i$. Indeed, the coefficient of $X_1^{d}$, where $d = \sum_{i=1}^pD_i$, in $m$ is $\det(A)$, where 
\[
A = \left( \begin{matrix}
\prod_{l=1}^{D_1}\gamma_{1,1}^{(1,l)}  & \cdots  & \prod_{l=1}^{D_1}\gamma_{1,p}^{(1,l)}\\
\vdots & \ddots & \vdots \\
\prod_{l=1}^{D_p}\gamma_{1,1}^{(1,l)}  & \cdots  & \prod_{l=1}^{D_p}\gamma_{1,p}^{(1,l)}\\
\end{matrix} \right) \in \field[\mathcal{G}]^{p \times p}.
\] which has nonzero determinant. Moreover, in order to obtain the condition that $m$ has $\sum_{i=1}^pD_i$ distinct solutions we use $\mathrm{Res}_{X_1}(m, m') \ne 0$, where $\mathrm{Res}_{X_1}(m, m')$ is the resultant between $m$ and the derivative of $m$ with respect to $X_1$. So, we can define $\mathcal{O}_2$ as follows. 

Let $\Omega_2 = V(\mathrm{Res}_{X_1}(m, m')) \subset \bar{\field}^{N} \times \mathbb{P}^1(\bar{\field})$ is a Zariski closed, where here $N = 2p(\sum_{i=1}^pD_i)$ is the number of coefficients for $G_1$. Then, we define $\Delta_2 = \pi_{\mathcal{G}}(\Omega_2) \subset \bar{\field}^{N}$ is a Zariski closed and take $\mathcal{O}_2$ as $\bar{\field}^{N} \setminus \Delta_2$ is a Zariski open. Therefore, for any $\mathfrak{g}_{i,j} \in \mathcal{O}_2$, we have $m$ has $\sum_{i=1}^pD_i$ solutions. The remaining problem is we need to check that $\mathcal{O}_2$ is nonempty. In order to finish this part, we use the matrix $G_2$ as a diagonal matrix $\mathrm{diag}(\mathfrak{g}_{1,1}, \ldots, \mathfrak{g}_{p,p})$ where $\mathfrak{g}_{i,i} = \prod_{l=1}^{D_i}(\gamma_{i,i}^{(1,l)}X_1 + \gamma_{i,i}^{(0,l)})$. Furthermore, the determinat of $G_2$ has the form $\prod_{i=1}^p\mathfrak{g}_{i,i}$, i.e., $\prod_{i=1}^p\prod_{l=1}^{D_i}(\gamma_{i,i}^{(1,l)}X_1 + \gamma_{i,i}^{(0,l)})$. Then, $\det(G_2)$ always has $\sum_{i=1}^pD_i$  distinct solutions. Therefore, $\bar{\field}^{N_2} \times 0^{N_1} \subset \mathcal{O}_2$ which implies that $\mathcal{O}_2$ is nonempty. 

$\bf{Step \ 2:}$ Let us now assume that $\mathcal{A}(2)$ holds for any $G_1 \in \field[{\mathcal{G}, \mat{X}}]^{p' \times q'}$ for any $n' \leq n$, where $n' = q'-p'+1$; we will prove $\mathcal{A}(2)$ is also true for $G_2 \in \field[\mathcal{G}, \mat{X}]^{p \times q}$ with $n = q-p+1$. We follow the similar argument as in the second step of the proof for $\mathcal{A}(1)$. Here, instead of consider the $I_{G_2}^{(p-1)}$, we consider the ideal $I_{G_2}$ which is generated by $p \times p$ minors of $G_2$. We have $I_{G_2}$ contains $\mathfrak{g}_{1,1}\ldots \mathfrak{g}_{p,p}$. Then, for any solution, $(\mathfrak{g}^*,\mathbf{x}^*) \in \bar{\field}^{N_2} \times \mathbb{P}^n(\bar{\field})$ of $I_{G_2}$, there is at least one $i \in \{1, \ldots, p\}$ such that $\mathfrak{g}_{i,i}(\mathfrak{g}^*,\mathbf{x}^*) = 0$. If there are $k$ numbers $i_1, \ldots, i_k$, where $1 \leq k \leq \min(n,p)$, such that $g_{i_1, i_1}(\mathfrak{g}^*, \mathbf{x}^*) = \cdots = g_{i_k, i_k}(\mathfrak{g}^*, \mathbf{x}^*) = 0$, we define $\mathcal{J}_k = (i_1, \ldots, i_k)$ and $G_{\mathcal{J}_k, p+1:q} \in \field[\mathcal{G}, \mat{X}]^{(n-k)\times (q-p)}$ is the submatrix of $G_2$ that contains the rows $\mathcal{J}_k$ and columns $p+1, \ldots, q$. From Proposition \ref{r2}, we have $\mathrm{rank}(G_{\mathcal{J}_k, p+1:q}(\mathfrak{g}^*, \mathbf{x}^*)) \leq k-1$ for which the property $\mathcal{A}(2)$ holds as from induction hypothesis. Moreover, $$V(I_{G_2}) = \bigcup\limits_{\mathcal{J}_k \subset\{1, \ldots, p\}^k, \ 1 \leq k \leq \min(n,p)} V(I_{\mathcal{J}_k}),$$ where $I_{\mathcal{J}_k}$ is the ideal generated by $\mathfrak{g}_{i,i}$ for $i \in \mathcal{J}_k$ and all $(k-1)\times(k-1)$ minors of $G_{\mathcal{J}_k, p+1:q}$. Therefore, we can define a nonempty Zariski open, $\bar{\mathcal{O}}_2$, for $G_2$ as follows. For any $\mathcal{J}_k \subset \{1, \ldots, p\}^k$, we rewrite $(X_1, \ldots, X_k) \in \field[\mathcal{G}, X_{k+1}, \ldots, X_n]^{k}$ and subtitute into $G_{\mathcal{J}_k, p+1:q}$ to obtain $G_{\mathcal{J}_k, p+1:q} \in \field[\mathcal{G}, X_{k+1}, \ldots, X_n]^{k \times {(q-p)}}$. By using the induction hypothesis, for any $\mathcal{J}_k$, there exists a Zariski closed set $\Delta_{\mathcal{J}_k} \subsetneq \bar{\field}^{N_{\mathcal{J}_k}}$ such that $I_{G_{\mathcal{J}_k, p+1:q}}^{(p)}$ has $S_{n-k}(D_{i_1}, \ldots, D_{i_k})$ distinct solutions, where $N_{\mathcal{J}_k} = n(q-p+1)(\sum_{i \in \mathcal{J}_k}D_i)$ is the number of coefficients of the matrix $[\mathrm{diag}(\mathfrak{g}_{i_1, i_1}, \ldots, \mathfrak{g}_{i_k, i_k})|G_{\mathcal{J}_k, p+1:q}]$. We finally, define \[\bar{\mathcal{O}}_2 = \bar{\field}^{N_2} \setminus \Delta, \ \mathrm{where}\  N_2 = n(q-p+1)(\sum_{i=1}^pD_i) \ \mathrm{and} \ \Delta = \bigcup_{\mathcal{J}_k \subset\{1, \ldots, p\}^k, \ 1 \leq k \leq \min(n,p)}\Delta_{\mathcal{J}_k}.\]
This is a nonempty Zariski open set in $\bar{\field}^{N_2}$ such that $I_{G_2}$ satisfies $\mathcal{A}(2)$ when $g_{i,j} \in \bar{\mathcal{O}}_2$. Indeed, by this construction, the number of solutions of $I_{G_2}$ equals

\[\sum_{k=1}^{\min(n,p)} \sum_{\mathcal{J}_k \subset\{1, \ldots, p\}^k}D_{i_1} \ldots D_{i_k} . \# \{\mathrm{solutions \ of \ } I_{G_{\mathcal{J}_k, p+1:q}}\}\] which is 
\[
\sum_{k=1}^{\min(n,p)} \sum_{\mathcal{J}_k \subset\{1, \ldots, p\}^k}D_{i_1} \ldots D_{i_k}S_{n-k}(D_{i_1}, \ldots, D_{i_k}) = S_n(D_1, \ldots, D_p). 
\]

$\bf{Step \ 3:}$ Finally, let assume that $\mathcal{A}(2)$ holds for any $G_2 \in \field[\mathcal{G}, \mat{X}]^{p \times q}$, we will prove $\mathcal{A}(2)$ is also true for any $G_1 \in \field[\mathcal{G}, \mat{X}]^{p \times q}$ as follows.

\todo{to finish}
\subsection{Genericity of $\mathcal{A}(3)$} 
$\bf{Step \ 1:}$ We show here that there exists a nonempty Zariski open set $\mathcal{O}_3$ such that $I_{G_1}$ is radical ideal for $\mathfrak{g}_{i,j} \in \mathcal{O}_3$, where $G_1$ is as square matrix as follows 
\[
G_1 = \left( \begin{matrix}
\mathfrak{g}_{1,1}  & \cdots  & \mathfrak{g}_{1, p}\\
\vdots & \ddots & \vdots \\
\mathfrak{g}_{p,1} & \cdots  & \mathfrak{g}_{p, p}
\end{matrix} \right).
\] Let us denote $m \in \field[\mathcal{G}, X_1]$ for the deteminant of $G_1$, then $I_{G_1} = \langle m \rangle$. Notice that in this case, the number of variable is $n = 1$. We would like to have the property that \[\frac{\partial m}{\partial X_1}(\mathbf{x}^*) \ne 0 \ \mathrm{for \ any} \ \mathbf{x}^* \in V(m). \] 

Let us define $\mathcal{O}_3$ as follows. Let $\Omega_3 = V(\partial m / \partial X_1) \subset \bar{\field}^N \times \mathbb{P}^1(\bar{\field})$ is a Zariski closed, where here $N = 2p(\sum_{i=1}^pD_i)$ is the number of coefficients for $G_1$. Then, we define $\Delta_3 = \pi_{\mathcal{G}}(\Omega_3) \subset \bar{\field}^{N}$ is a Zariski closed and take $\mathcal{O}_3$ as $\bar{\field}^{N} \setminus \Delta_3$ is a Zariski open. Therefore, for any $\mathfrak{g}_{i,j} \in \mathcal{O}_3$, we have $(\partial m / \partial X_1) (\mathbf{x}^*) \ne 0$. The remaining problem is we need to check that $\mathcal{O}_3$ is nonempty. For this, we use the diagonal marix $G_2$ as $\mathrm{diag}(\mathfrak{g}_{1,1}, \ldots, \mathfrak{g}_{p,p})$. So, the determinant of $G_2$ is $\prod_{i=1}^p\mathfrak{g}_{i,i}$, where $\mathfrak{g}_{i,i} = \prod_{l=1}^{D_i}(\gamma_{i,i}^{(1,l)}X_1 + \gamma_{i,i}^{(0,l)})$. Let us denote $u$ for $\det(G_2)$, then 
\[
\frac{\partial u}{\partial X_1} = \sum_{i = 1}^p\frac{\partial \mathfrak{g}_{i,i}}{\partial X_1}\prod_{j \ne i} \mathfrak{g}_{j,j}.
\]
For convernient, let us denote $f(X)$ for ${\partial u}/{\partial X_1}$ and rewrite $\det(G_2) = \prod_{i=1}^d(a_iX + b_i)$ for $a_i$ and $b_i$ are indeterminantes and $d = \sum_{i=1}^pD_i$. Therefore, 
\[
f(X) = \sum_{i = 1}^d a_i \prod_{j=1, j \ne i}^d(a_jX + b_j).
\]
Moreover, for any $\mathbf{x}^* \in V(\det(G_2))$ and $\mathbf{x}^* $ is the solution of $a_i X+ b_i = 0$, we have $a_j \mathbf{x}^* + b_j \ne 0$ for all $i \ne k$. Therefore, \[f(\mathbf{x}^*) = a_i\prod_{j=1, j \ne i}^d(a_j\mathbf{x}^* + b_j) \ne 0. \]
This means for any $\mathbf{x}^* \in V(\det(G_2))$, we always have $f(\mathbf{x}^*) \ne 0$. Thus, $\bar{\field}^{N_2} \times 0^{N_1} \subset \mathcal{O}_3$ which implies that $\mathcal{O}_3$ is nonempty. 

$\bf{Step \ 2:}$ Let us now assume that $\mathcal{A}(3)$ holds for any $G_1 \in \field[{\mathcal{G}, \mat{X}}]^{p' \times q'}$ for any $n' \leq n$, where $n' = q'-p'+1$; we will prove $\mathcal{A}(3)$ is also true for $G_2 \in \field[\mathcal{G}, \mat{X}]^{p \times q}$ with $n = q-p+1$. As we noticed before, the assumption that $I_{G_1}$ is radical is equivalent to the property that $\mathrm{Jac}(I_{G})(\mathbf{x}^*)$ has full rank for any $\mathbf{x}^* \in V(I_{G})$. In other words, $\mathbf{x}^* \in V(I_{G})$ is a simple root in $V(I_G)$. 

Let us now consider any $\mathbf{x}^* \in V(I_{G_2})$, there exist $k$ polynomials $\mathfrak{g}_{i_1,i_1}, \ldots, \mathfrak{g}_{i_k,i_k}$ such that $\mathfrak{g}_{i_1,i_1}(\mathbf{x}^*) = \cdots = \mathfrak{g}_{i_k,i_k}(\mathbf{x}^*) = 0$, where $1 \leq k \leq \min(n,p)$ and $\mathrm{rank}(G_{\mathcal{J}_k; p+1:q}(\mathbf{x}^*)) \leq k-1$. That is $\mathbf{x}^*$ is solution of the system of $k \times k -$ minors of $G_{\mathcal{J}_k; p+1:q}$, where $\mathcal{J}_k = (i_1, \ldots, i_k)$. Therefore, by considering any system of $k$ polynomials $\mathfrak{g}_{i_1,i_1}(\mat{X}) = \cdots = \mathfrak{g}_{i_k,i_k}(\mat{X}) = 0$, we can rewrite the variables $X_1, \ldots, X_k$ in the linear of $X_{k+1}, \ldots, X_n$; then by substitute $\{X_i\}_{i=1}^{k}$ in to $G_{\mathcal{J}_k; p+1:q}$ we obtain a matrix $\bar{G} \in \field[\mathcal{G}, X_k, \ldots, X_n]^{k \times (q-p)}$. Furthermore, this $\bar{G}$ matrix has property that any solution of the its $k \times k -$ minors system is simple by using induction hypothesis. Therefore, we can define the Zariski open, $\bar{\mathcal{O}}_3$, for $G_2$ as follows. 

For any $\mathcal{J}_k \subset \{1, \ldots, p\}^k$, we rewrite $(X_1, \ldots, X_k) \in \field[\mathcal{G}, X_{k+1}, \ldots, X_n]^{k}$ and subtitute into $G_{\mathcal{J}_k, p+1:q}$ to obtain $G_{\mathcal{J}_k, p+1:q} \in \field[\mathcal{G}, X_{k+1}, \ldots, X_n]^{k \times {(q-p)}}$. By using the induction hypothesis, for any $\mathcal{J}_k$, there exists a Zariski closed set $\Delta_{\mathcal{J}_k} \subsetneq \bar{\field}^{N_{\mathcal{J}_k}}$ such that any solution of $k \times k -$ minors of $G_{\mathcal{J}_k, p+1:q}$ is simple, where $N_{\mathcal{J}_k} = n(q-p+1)(\sum_{i \in \mathcal{J}_k}D_i)$ is the number of coefficients of the matrix $[\mathrm{diag}(\mathfrak{g}_{i_1, i_1}, \ldots, \mathfrak{g}_{i_k, i_k})|G_{\mathcal{J}_k, p+1:q}]$. We finally, define \[\bar{\mathcal{O}}_3 = \bar{\field}^{N_2} \setminus \Delta, \ \mathrm{where}\  N_2 = n(q-p+1)(\sum_{i=1}^pD_i) \ \mathrm{and} \ \Delta = \bigcup_{\mathcal{J}_k \subset\{1, \ldots, p\}^k, \ 1 \leq k \leq \min(n,p)}\Delta_{\mathcal{J}_k}.\]
This is a nonempty Zariski open set in $\bar{\field}^{N_2}$ such that $I_{G_2}$ satisfies $\mathcal{A}(3)$ when $g_{i,j} \in \bar{\mathcal{O}}_3$. Indeed, by this construction, from the linear of $X_1, \ldots, X_k$ in $X_{k+1}, \ldots, X_n$ and any solution of $k \times k -$ minors of $G_{\mathcal{J}_k, p+1:q}$ is simple, we can deduce that any solution of $p \times p -$ minors of $G_2$ is simple. It means we finishes the second step.

$\bf{Step \ 3:}$
Let us consider the matrix $G_1 \in \field[\mathcal{G}, \mat{X}]^{p \times q}$, and $I_{G_1} := \langle p \times p - \ \mathrm{minors \ of } \ G_1 \rangle$. Let $m_1, \ldots, m_d$ be the generators for $I_{G_1}$, where $d = {{q}\choose{p}}$ is the number of $p \times p$ minors of $G_1$. We would like to have the property that $\mathrm{Jac}(I_{G_1})(\mathbf{x}^*)$ has full rank for any $\mathbf{x}^* \in V(I_{G_1})$ . We recall here that $$\mathrm{Jac}(I_{G_1}) = \left[\frac{\partial m_j}{\partial X_i}\right]_{1 \leq j \leq d, 1 \leq i \leq n} \in \field[\mathcal{G}, \mat{X}]^{d \times n}. $$
We first notice that for since $n \leq d$, so for any $\mathbf{x}^* \in \bar{\field}^n$, the matrix $\mathrm{Jac}(I_{G_1})(\mathbf{x}^*)$ has rank at most $n$. Therefore, $\mathrm{Jac}(I_{G_1})(\mathbf{x}^*)$ should has rank $n$ if we want the property that $\mathrm{Jac}(I_{G_1})$ has full rank at any point $\mathbf{x}^*$. For this argument, we consider the ideal which is generated by $n \times n \ \mathrm{minors \ of} \ \mathrm{Jac}(I_{G_1})$, that is $\mathcal{I}_{m} = \langle n \times n  \ \mathrm{minors \ of} \ \mathrm{Jac}(I_{G_1}) \rangle$.

Let us define $\Omega_3 = V(\mathcal{I}_m) \subset \bar{\field}^N \times \mathbb{P}^n(\bar{\field})$ is a Zariski closed. Let $\pi_{\mathcal{G}}$ be the projection on $\mathcal{G}$ coordinates, then we define $\Delta_3 = \pi_{\mathcal{G}}(\Omega_3)$ is a Zariski closed in $\bar{\field}^N $. Finally, let us denote  $\mathcal{O}_3$ for $\bar{\field}^N \setminus \Delta_3$ which is a Zariski open in $\bar{\field}^N $. By using this contruction, when $\mathfrak{g}_{i,j} \in \mathcal{O}_3$, we can see that  $\mathrm{Jac}(I_{G_1})(\mathbf{x}^*)$ has full rank $n$ for any $\mathbf{x}^* \in V(I_{G_1})$). The remaining problem is proving $\mathcal{O}_3$ is indeed a nonempty set. In order to finish this step, we use the matrix $G_2 \in \field[\mathcal{G}, \mat{X}]^{p \times q}$ which is defined as above and using the similar argument as in the first step, we can deduce that $\mathcal{O}_3$ is nonempty.

\section{Proof of Proposition \ref{testisolated}}
\label{sec:prooftestisolated}
Reference~\cite{BaHaPeSo09} gives an algorithm to compute the dimension of $V(\mathbf{f})$ at $\mathbf{x}$, but its complexity is unclear to us, as it relies on linear algebra with matrices of potentially large size.
Instead, we use an adaptation of a prior result by Mourrain~\cite{Mourrain97}, which allows us to control the size of the matrices we handle. We only give detailed proofs for new ingredients that are specific to our context, a key difference being the cost analysis in the straight-line program model: Mourrain's original result depends on the number of monomials appearing when we expand $\mathbf{f}$, which would be too high for the applications we will make of this result.

We assume that $\mathbf{x} = 0$; this is done by replacing $\mathbf{f}$ by the polynomials $\mathbf{f}(\mat{X}+\mathbf{x})$, which have complexity of evaluation $\mathcal{E}'= \mathcal{E}+n$.  The basis of our algorithm is the following remark.
\begin{Lemma}
Let $I$ be the zero-dimensional ideal $\langle \mathbf{f} \rangle + \mathfrak{m} ^{\mu+1}$, where $\mathfrak{m} = \langle X_1,\dots,X_n\rangle$ is the maximal ideal at the origin. Then, if $0$ is isolated in $V(\mathbf{f})$ if and only if it has multiplicity at most $\mu$ with respect to $I$.
\end{Lemma}
\begin{proof}
This follows from the following result~\cite[Theorem~A.1]{BaHaPeSo09}.  For $k \ge 1$, let $I_k$ be the zero-dimensional ideal $\langle \mathbf{f} \rangle + \mathfrak{m}^{k}$, and let $\nu_k$ be multiplicity of the origin with respect to this ideal. Then, the reference above proves that the sequence $(\nu_k)_{k \ge 1}$ is non-decreasing, and that $0$ is isolated in $V(\mathbf{f})$ if and only if there exists $k\ge 1$ such that $\nu_k=\nu_{k+1}$.
\begin{itemize}
\item If $0$ is isolated in $V(\mathbf{f})$, then by assumption $\mathsf{H}$  its multiplicity with respect to $\langle \mathbf{f} \rangle$ is at most $\mu$, and its multiplicity with respect to $I$ cannot be larger.
\item Otherwise, by the result above, $\nu_{k+1} > \nu_k$ holds for all $k \ge 1$, so that $\nu_k \ge k$ holds for all such $k$ (since $\nu_1=1$). In particular, the multiplicity of the origin with respect to $I$, which is $\nu_{\mu+1}$, is at least $\mu+1$.
    \qedhere
  \end{itemize}
\end{proof}

Hence, we are left with deciding whether the multiplicity of the $\mathfrak{m}$-primary ideal $I$ is at most $\mu$. We do this by following and slightly modifying Mourrain's algorithm for the computation of the orthogonal $I^{\perp}$, that is, the set of $\field$-linear forms $\field[\mat{X}] \to \field$ that vanish on $I$; this is a $\field$-vector space naturally identified with the dual of $\field[\mat{X}]/I$, so it has dimension $m = \mathrm{mult}(0,I)$. 

We do not need to give all details of the algorithm, let alone proof of correctness; we just mention the key ingredients for the cost analysis in our setting. A linear form $\beta: \field[\mat{X}] \to \field$ that vanishes on $I$ must vanish on all monomials, except a finite number (since all monomials, except a finite number, belong to $I$); a natural way to represent such a linear form would then be as the finite generating series $\sum_{\mathbf{\alpha} \in \mathbb{N}^n}\beta(X_1^{\alpha_1}\cdots X_n^{\alpha_n}) d_1^{\alpha_1}\cdots d_n^{\alpha_n}$, for some new variables $d_1,\dots,d_n$; however the number of non-zero coefficients in such a sum cannot be bounded polynomially in $n,\mu$.

Hence, the algorithm uses another way to represent the elements in $I^{\perp}$, by means of {\em multiplication matrices}. An important feature of $I^{\perp}$ is that it admits the structure of a $\field[\mat{X}]$-module: for $k$ in $\{1,\dots,n\}$ and $\beta$ in $I^{\perp}$, the $\field$-linear form $X_k \cdot \beta: f \mapsto
\beta(X_k f)$ is easily seen to still be in $I^{\perp}$.  In particular, if $\bm{\beta}=(\beta_1,\dots,\beta_m)$ is a $\field$-basis of $I^{\perp}$, then for all $k$ as above, and all $i$ in $\{1,\dots,m\}$, $Y_k \cdot \beta_i$ is a linear combination of $\beta_1,\dots,\beta_m$. Mourrain's algorithm computes a
basis $\bm{\beta} = (\beta_1,\dots,\beta_m)$ with the following features:
\begin{itemize}
\item for $i$ in $\{1,\dots,m\}$ and $k$ in $\{1,\dots,n\}$, we have
  $X_k \cdot \beta_i=\sum_{0 \le j < i} \lambda^{(k)}_{i,j} \beta_j$
  (hence $\lambda^{(k)}_{i,j}$ may be non-zero 
  only for $j<i$)
\item $\beta_1$ is the evaluation at $0$, $f \mapsto f(0)$
\item for $i$ in $\{2,\dots,m\}$, $\beta_i(1)=0$.
\end{itemize}
The following lemma shows that the coefficients $(\lambda^{(k)}_{i,j})$ are sufficient to evaluate  the linear forms $\beta_i$ at $f$ in $\field[\mat{X}]$. More precisely, knowing only their values for $j < i \le s$, for any $s \le m$, allows us to evaluate $\beta_1,\dots,\beta_s$ at such an $f$. The following lemma follows~\cite{Mourrain97} in its description of the matrices $\mat{M}_{k,s}$; the (rather straightforward) complexity analysis in the straight-line program model is new.
\begin{Lemma}\label{lemma:evalbeta} Let $s$ be in $1,\dots,m$, and suppose that the coefficients $\lambda^{(k)}_{i,j}$ are known for $i=1,\dots,s$, $j=0,\dots,i-1$ and $k=1,\dots,n$. Given a straight-line program $\Gamma$ of length $L$ that computes $\mathbf{h}=(h_1,\dots,h_R)$, one can compute $\beta_i(h_r)$, for all $i=1,\dots,s$ and $r=1 \dots,R$, using $(s\,L)^{O(1)}$ operations.
\end{Lemma}
\begin{proof}
By definition, for $h$ in $\field[\mat{X}]$  and $k=1,\dots,n$, the following equality holds:
\[
 \begin{bmatrix}
    \beta_1(X_k h)\\
    \vdots\\
    \beta_s(X_k h)
  \end{bmatrix}=
\mat{M}_{k,s}
  \begin{bmatrix}
    \beta_1(h)\\
    \vdots\\
    \beta_s(h)
  \end{bmatrix},
\quad\text{with}\quad
\mat{M}_{k,s}= \begin{bmatrix}
    \lambda^{(k)}_{1,1} & \cdots & \lambda^{(k)}_{s,1}\\
    \vdots && \vdots \\
    \lambda^{(k)}_{1,s} & \cdots & \lambda^{(k)}_{s,s}
  \end{bmatrix}.
\]
Remark that the matrices $\mat{M}_{k,s}$ all commute. Indeed, for any $k,k'$ in $\{1,\dots,n\}$, and $h$ as above, the relation above implies that 
\[
\Delta_{k,k',s}
  \begin{bmatrix}
    \beta_1(h)\\
    \vdots\\
    \beta_s(h)
  \end{bmatrix} =
  \begin{bmatrix}
0\\ \vdots \\ 0 
  \end{bmatrix},
\] where $\Delta_{k,k',s} = \mat{M}_{k,s}\mat{M}_{k',s}-\mat{M}_{k',s}\mat{M}_{k,s}.$ Because the linear forms $\beta_1,\dots,\beta_s$ are linearly independent, this implies that all rows of $\Delta_{k,k',s}$ must be zero, as claimed. We then deduce that for any polynomial $h$ in $\field[\mat{X}]$, we have the equality
\[ \begin{bmatrix}
    \beta_1(h)\\
    \vdots\\
    \beta_s(h)
  \end{bmatrix} =
h(\mat{M}_{1,s},\dots,\mat{M}_{N,s})   \begin{bmatrix}
    \beta_1(1)\\
    \vdots\\
    \beta_s(1)
  \end{bmatrix} \]
On the other hand, our assumptions imply that the sequence $(\beta_1(1),\dots,\beta_s(1))$ is simply $(1,0,\dots,0)$. To prove the claim, note that the evaluations $h_1(\mat{M}_{1,s},\dots,\mat{M}_{n,s}),\dots,h_R(\mat{M}_{1,s},\dots,\mat{M}_{n,s})$ can be computed using the straight-line program $\Gamma$ in $(s\,L)^{O(1)}$ operations.
\end{proof}
Mourrain's alorithm proceeds in an iterative manner, starting from $\bm{\beta}^{(1)}=(\beta_{1})$ (and setting $e_1=1$), and computing successively $\bm{\beta}^{(2)}=(\beta_{e_1+1},\dots,\beta_{e_2})$, $\bm{\beta}^{(3)}=(\beta_{e_2+1},\dots,\beta_{e_3})$, \dots for some integers $e_1 \le e_2 \le e_3 \dots$. Mourrain's algorithm stops when $e_{\ell+1}=e_{\ell}$, in which case $\beta_1,\dots,\beta_{e_\ell}$ is a $\field$-basis of $I^\perp$, and $e_\ell=\mult(0,I)$. In our case, we are not interested in computing this multiplicity, but only in
deciding whether it is less than or equal to the parameter $\mu$. We do it as follows: assume that we have
computed $\bm{\beta}^{(1)},\bm{\beta}^{(2)},\dots,\bm{\beta}^{(\ell)}$, together with the corresponding integers $e_1,e_2,\dots,e_\ell$, with $e_1 < \cdots < e_\ell \le \mu$. We compute $\bm{\beta}^{(\ell+1)}$ and $e_{\ell+1}$, and continue according to the following:
\begin{itemize}
\item if $e_{\ell+1}=e_{\ell}$, we conclude that the multiplicity $\mult(0,I)$ is $e_\ell \le \mu$; we stop the algorithm;
\item if $e_{\ell+1} > \mu$, we conclude that the multiplicity is greater than $\mu$; we stop the algorithm;
\item else, when $e_\ell < e_{\ell+1} \le \mu$, we do another loop.
\end{itemize}
Because the $e_\ell$'s are an increasing sequence of integers, they satisfy $e_\ell \ge \ell$; hence, every time we enter the loop above we have $\ell \le \mu$. To finish the analysis of the algorithm, it remains to explain how to compute $\bm{\beta}^{(\ell+1)}$ from $(\bm{\beta}^{(1)},\bm{\beta}^{(2)},\dots,\bm{\beta}^{(\ell)})=(\beta_{1},\dots,\beta_{e_\ell})$.

As per our description above, at any step of the algorithm, $\beta_{1},\dots,\beta_{e_\ell}$ are represented by means of the coefficients $\lambda^{(k)}_{i,j}$, for $0 \le j < i \le e_{\ell}$ and $1 \le k \le n$.  At step $\ell$, Mourrain's algorithm solves a homogeneous linear system $S_\ell$ with $n(n-1) e_\ell/2+M'$ equations and $n.e_\ell$ unknowns, where $M'$ is the number of generators of the ideal $I= \langle \mathbf{f} \rangle + \mathfrak{m}^{\mu+1}$. Remark that $M'$ is not polynomial in $\mu$ and $n$, so the size of $S_\ell$ is {\em a priori} too large to fit our cost bound; we will explain below how to resolve this issue.

The nullspace dimension of this linear system gives us the cardinality $e_{\ell+1}-e_{\ell}$ of $\bm{\beta}^{(\ell +1)}$. Similarly, the coordinates of the $e_{\ell+1}-e_{\ell}$ vectors in a nullspace basis are precisely
the coefficients $\lambda^{(k)}_{i,j}$ for $i=e_{\ell}+1,\dots,e_{\ell+1}$, $j=1,\dots,e_\ell$ and $k=1,\dots,n$
(we have $\lambda^{(k)}_{i,j}=0$ for $j=e_{\ell}+1,\dots,i-1$). For all $\ell \ge 2$, all linear forms $\beta$ in $\bm{\beta}^{(\ell)}$ are such that for all $k$ in $\{1,\dots,n\}$, $X_k \cdot \beta$ belongs to the span of $\bm{\beta}^{(1)},\dots,\bm{\beta}^{(\ell-1)}$; in particular, a quick induction shows that all linear forms in $\bm{\beta}^{(1)},\dots,\bm{\beta}^{(\ell)}$ vanish on all monomials of degree at least $\ell$.

There remains the question of setting up the system $S_\ell$. For $k$ in $\{1,\dots,n\}$ and a $\field$-linear form $\beta$, we denote by $X_k^{-1} \cdot \beta$ the $\field$-linear form defined as follows:
\begin{itemize}
\item $(X_k^{-1} \cdot \beta)(X_k f) = \beta(f)$ for all $f$ in $\field[\mat{X}]$,
\item $(X_k^{-1} \cdot \beta)(f)=0$ if $f\in \field[\mat{X}]$ does not depend on $X_k$.
\end{itemize}
In other words, $(X_k^{-1} \cdot \beta)(f)=\beta(\delta_k(f))$ holds for all $f$, where $\delta_k:\field[\mat{X}] \to \field[\mat{X}]$ is the $k$th divided difference operator
\[f \mapsto \frac{f(X_1,\dots,X_n)-f(X_1,\dots,X_{k-1},0,X_{k+1},\dots,X_n)}{X_k}.\]
One verifies that, as the notation suggests, $X_k \cdot (X_k^{-1} \cdot \beta)$ is equal to $\beta$. This being said, we can then describe what the entries of $S_\ell$ are:
\begin{itemize}
\item the first $n(n-1) e_\ell/2$ equations involve only the coefficients $\lambda^{(k)}_{i,j}$ previously computed (we refer to~\cite[Section~4.4]{Mourrain97} for details of how exactly  these entries are distributed in $S_\ell$, as we do not need such details here).
\item each of the other $M'$ equations has coefficient vector
\[c_f = \big (\
 (X_k^{-1} \cdot \beta_1)(f(X_1,\dots,X_k,0,\dots,0)),\dots,\ (X_k^{-1} \cdot \beta_{e_\ell})(f(X_1,\dots,X_k,0,\dots,0))\
\big )_{1 \le k \le n},\]
where $f$ is a generator of $I=\langle \f \rangle +\m^{\mu+1}$.
\end{itemize}
We claim that only those equations corresponding to generators $f_1,\dots,f_M$ of the input system $\f$ are useful, as all others are identically zero 

We pointed out above that any linear form $\beta_i$ in $\beta_1,\dots,\beta_{e_\ell}$ vanishes on all monomials of degree at least $\ell$. Since we saw that we must have $\ell \le \mu$, all $\beta_i$ as above vanish on monomials of degree $\mu$; this implies that $X_k^{-1}\cdot \beta_i$ vanishes on all monomials of degree $\mu+1$. The generators $f$ of $\m^{\mu+1}$ have degree $\mu+1$, and for any such $f$, $f(X_1,\dots,X_k,0,\dots,0)$ is either zero, or of degree $\mu+1$ as well. Hence, for any $k$, $\beta_i$ in $\beta_1,\dots,\beta_{e_\ell}$ and $f$ as above, $(X_k^{-1} \cdot \beta_i)(f(X_1,\dots,X_k,0,\dots,0))$ vanishes. This implies that the vector $c_f$ is identically zero for such an $f$, and that the corresponding equation can be discarded.

Altogether, as claimed above, we see that we have to compute the values
\[(X_k^{-1} \cdot \beta_i)(f_j(X_1,\dots,X_k,0,\dots,0)),\] for $k=1,\dots,n$, $i=1,\dots,e_\ell$ and $j=1,\dots,M$.  Fixing $k$, we let $\f_k = (f_{j,k})_{1 \le j \le M}$, where $f_{j,k}$ is the polynomial $f_j(X_1,\dots,X_k,0,\dots,0)$; note that the system $\f_k$ can be computed by a straight-line program of length $\mathcal{E}'=\mathcal{E}+n$. Then, applying the following lemma with $s=e_\ell \le \mu$ and $\h = \f_k$, we deduce that the values $(X_k^{-1} \cdot \beta_i)(f_j(X_1,\dots,X_k,0,\dots,0))$, for $k$ fixed, can be computed in time $(\mu\,\mathcal{E}\,n)^{\mathcal{O}(1)}$.

\begin{Lemma} Let $s$ be in $1,\dots,m$, and suppose that the coefficients $\lambda^{(k)}_{i,j}$ are known for $i=1,\dots,s$, $j=0,\dots,i-1$ and $k=1,\dots,n$. Given a straight-line program $\Gamma$ of length $L$ that computes $\h=(h_1,\dots,h_R)$ and given $k$ in $\{1,\dots,n\}$, one can compute $(X_k^{-1}\cdot \beta_i)(h_r)$, for all $i=1,\dots,s$ and $r=1,\dots,R$, using $(s\,L\,N)^{O(1)}$ operations in $\field$.
\end{Lemma}
\begin{proof} In view of the formula $(X_k^{-1} \cdot \beta)(f)=\beta(\delta_k(f))$, and of Lemma \ref{lemma:evalbeta}, it is enough to prove the existence of a straight-line program of length $\mathcal{O}(L)$ that computes $(\delta_k(h_1),\dots,\delta_k(h_R))$.

To do this, we replace all polynomials $\gamma_{-n+1},\dots,\gamma_L$ computed by $\Gamma$ by terms $\lambda_{-n+1},\dots,\lambda_L$ and $\mu_{-n+1},\dots,\mu_L$, with $\lambda_\ell=\gamma_\ell(X_1,\dots,X_{k-1},0,X_{k+1},\dots,X_N)$ and $\mu_\ell$ in $\field[\mat{X}]$ such that $\gamma_\ell= \lambda_\ell+X_k \mu_\ell$ holds for all $\ell$, so that in particular
  $\mu_\ell=\delta_k(\gamma_\ell)$.  To compute $\lambda_\ell$ and
  $\mu_\ell$, assuming all previous $\lambda_{\ell'}$ and
  $\mu_{\ell'}$ are known, we proceed as follows:
  \begin{itemize}
  \item if $\gamma_\ell=X_k$, we set $\lambda_\ell=0$ and $\mu_\ell=1$;
  \item if $\gamma_\ell=X_{k'}$, with $k' \ne k$, we set $\lambda_\ell=X_{k'}$ and $\mu_\ell=0$;
  \item if $\gamma_\ell =c_\ell$, with $c_\ell \in \field$,
    then we set $\lambda_\ell=c_\ell$ and  $\mu_\ell=0$;
  \item if $\gamma_\ell = \gamma_{a_\ell} \pm \gamma_{b_\ell}$,
    for some indices $a_\ell,b_\ell < \ell$, 
    then we set $\lambda_\ell=\lambda_{a_\ell}\pm\lambda_{b_\ell}$
    and $\mu_\ell=\mu_{a_\ell}\pm\mu_{b_\ell}$;
\item if $\gamma_\ell = \gamma_{a_\ell} \gamma_{b_\ell}$,
      for some indices $a_\ell,b_\ell < \ell$,
    then we set $\lambda_\ell=\lambda_{a_\ell} \lambda_{b_\ell}$
    and $$\mu_\ell=
\lambda_{a_\ell} \mu_{b_\ell}
+
\mu_{a_\ell} \lambda_{b_\ell}
+
X_k\mu_{a_\ell} \mu_{b_\ell}.$$
\end{itemize}
One verifies that in all cases, the relation $\gamma_\ell= \lambda_\ell+X_k \mu_\ell$ still holds. Since the previous construction allows us to compute $\lambda_\ell$ and $\mu_\ell$ in $\mathcal{O}(1)$ operations from the knowledge of all previous $\lambda_{\ell'}$ and $\mu_{\ell'}$, we deduce that all $\lambda_\ell$ and $\mu_\ell$,
for $\ell=-n+1,\dots,L$, can be computed by a straight-line program of length $\mathcal{O}(L+n)$.
\end{proof}

Taking all values of $k$ into account, we see that we can compute all entries we need to set up the linear system $S_\ell$ using $(\mu\,\mathcal{E}\,n)^{\bigO{1}}$ operations in $\field$. After discarding the useless equations described above, the number of equations and unknowns in the system $S_\ell$ is polynomial in $n$, $M$ and $e_\ell$; since we saw that $n \le M$ and $e_\ell \le \mu$, this implies that we can find a nullspace basis of it in time $(\mu\,M)^{\bigO{1}}$. 

Altogether, the time spend to find $\bm{\beta}^{(\ell+1)}$ from $(\bm{\beta}^{(1)},\bm{\beta}^{(2)},\dots,\bm{\beta}^{(\ell)})=(\beta_{1},\dots,\beta_{e_\ell})$ is polynomial in $\mu\,\mathcal{E}\,M$. Since we saw that we do at most $\mu$ such loops, the cumulated time remains polynomial in $\mu\,\mathcal{E}\,M$, and Proposition~\ref{testisolated} is proved.
\section{Proof of Proposition \ref{degree_fiber}}
\label{sec:proofdegree}
We recall here $J=Q_1 \cap \cdots \cap Q_r$ is an irredundant primary decomposition of $J$ in $\bar{\field}[T,\mat{X}]$ and $P_1,\dots,P_r$ are the associated primes. For some $s \leq r$, let $P_1, \ldots, P_s$ be the minimal primes so that $V(P_1), \ldots, V(P_s)$ are the irreducible components of $V(J)$ in $\bar{\field}^{n+1}$. As from ${\sf H}_1$, these irreducible components all have dimension at least one. Let $t \leq s$ be such that $V(P_1),\dots,V(P_t)$ are the irreducible components of $V(J)$ of dimension one whose image by $\pi_T: (\tau,x_1,\dots,x_N) \mapsto \tau$ is dense in $\field$.
\begin{Lemma}\label{lemma:vPi}
  Let $\tau$ be in $\bar{\field}$ and let ${\bf x} \in \bar{\field}{}^n$ be an isolated solution of the system $\h_\tau$. Then, $(\tau,{\bf x})$ belongs to $V(P_i)$ for at least one index $i$ in $\{1,\dots,t\}$, and does not belong to $V(P_i)$ for any index $i$ in $\{t+1,\dots,r\}$.
\end{Lemma}
\begin{proof}
 Because $(\tau,{\bf x})$ cancels $\h$, it belongs at least to one of
  $V(P_1),\dots,V(P_r)$. It remains to rule out the possibility that
  $(\tau,\y)$ belongs to $V(P_i)$ for some index $i$ in
  $\{t+1,\dots,r\}$.
  
We first deal with indices $i$ in $\{t+1,\dots,s\}$. These are those
  primary components with minimal associated primes $P_i$ that either
  have dimension at least two, or have dimension one but whose image
  by $\pi$ is a single point. In both cases, all irreducible
  components of the intersection $V(P_i)\cap V(T-\tau)$ have dimension
  at least one. Since ${\bf x}$ is isolated in $V(\h_\tau)$, $(\tau,{\bf x})$ is
  isolated in $V(\h)\cap V(T-\tau)$, so it cannot belong to
  $V(P_i)\cap V(T-\tau)$ for any $i$ in $\{t+1,\dots,s\}$.
  
  
  We conclude by proving that $(\tau,{\bf x})$ does not belong to $V(P_i)$,
  for any of the embedded primes $P_{s+1},\dots,P_r$. We proceed by
  contradiction, assuming for definiteness that $(\tau,{\bf x})$ belongs to
  $V(P_{s+1})$. Because $P_{s+1}$ is an embedded prime, $V(P_{s+1})$
  is contained in (at least) one of $V(P_1),\dots,V(P_s)$. In view of
  the previous paragraph, it cannot be one of
  $V(P_{t+1}),\dots,V(P_s)$.  Now, all of $V(P_1),\dots,V(P_t)$ have
  dimension one, so $V(P_{s+1})$ has dimension zero (so it is the point $\{(\tau,{\bf x})\}$). For the same
  reason, if $(\tau,{\bf x})$ belonged to another $V(P_i)$, for some $i >
  s+1$, $V(P_i)$ would also be zero-dimensional, and thus equal to $\{(\tau,{\bf x})\}$; as a result, $V(P_i)$
  would be equal to $V(P_{s+1})$, and this would contradict the
  irredundancy of our decomposition.
  
   To summarize, $(\tau,{\bf x})$ belongs to $V(P_{s+1})$, together with
  $V(P_i)$ for some indices $P_i$ in $\{1,\dots,t\}$ (say
  $P_1,\dots,P_u$, up to reordering, for some $u \ge 1$), and avoids
  all other associated primes.  Let us localize the decomposition
  $J=Q_1 \cap \cdots \cap Q_r$ at
  $P_{s+1}$. By~\cite[Proposition~4.9]{AtMc},
  $J_{P_{s+1}}={Q_1}_{P_{s+1}} \cap \cdots \cap {Q_u}_{P_{s+1}}\cap
  {Q_{s+1}}_{P_{s+1}}$ is an irredundant primary decomposition of
  $J_{P_{s+1}}$ in $\bar{\field},\mat{X}]_{P_{s+1}}$; the minimal primes are
  ${P_1}_{P_{s+1}},\dots,{P_u}_{P_{s+1}}$.
  
  By Corollary~4 p.24 in~\cite{Matsumura86}, for any prime
  ${P_i}_{P_{s+1}}$, $i=1,\dots,u$ or $i=s+1$, the localization of
  $\bar{\field}[T,\mat{X}]_{P_{s+1}}$ at ${P_i}_{P_{s+1}}$ is equal to
  $\bar{\field}[T,\mat{X}]_{P_{i}}$. In particular, the height of ${P_i}_{P_{s+1}}$
  in $\bar{\field}[T,\mat{X}]_{P_{s+1}}$ is equal to that of $P_i$ in
  $\bar{\field}[T,\mat{X}]_{P_{i}}$, that is, $n$ if $i=1,\dots,u$, since then
  $V(P_i)$ has dimension $1$, or $n+1$ if $i=s+1$. Since $u \ge 1$,
  this proves that $J_{P_{s+1}}$ has height $n$. As a result, ${\sf
    H}_2$ implies that $J_{P_{s+1}}$ is unmixed, a contradiction.
\end{proof}

For $\tau$ in $\bar{\field}$, we denote by $J_\tau \subset \bar{\field}[T,\mat{X}]$ the ideal $J + \langle T-\tau \rangle$, and similarly for $J'_\tau$ and $ J''_\tau$.

\begin{Lemma}\label{lemma:JJprime}
  Let $\tau$ and ${\bf x}$ be as in Lemma~\ref{lemma:vPi}. Then, the
  multiplicities of the ideals $J_\tau$ and $J'_\tau$ at $(\tau,{\bf x})$
  are the same.
\end{Lemma}
\begin{proof}
  Without loss of generality, assume that $\tau=0 \in \bar{\field}$ and
  ${\bf x}=0 \in \bar{\field}{}^N$. We start from the equality $J=J' \cap J''$,
  which holds in $\bar{\field}[T, \mat{X}]$, and we see it in $\bar{\field}[T, \mat{X}]$.  The
  previous lemma implies that there exists a polynomial in $J''$ that
  does not vanish at $(\tau,{\bf x})=0 \in \bar{\field}{}^{n+1}$.  This polynomial
  is a unit in $\bar{\field}[T, \mat{X}]$, which implies that the extension of
  $J''$ in $\bar{\field}[T, \mat{X}]$ is the trivial ideal $\langle 1 \rangle$, and
  finally that the equality of extended ideals $J=J'$ holds in
  $\bar{\field}[T, \mat{X}]$. This implies the equality $J+\langle T \rangle
  =J'+\langle T \rangle $ in $\bar{\field}[T, \mat{X}]$, and the conclusion
  follows.
\end{proof}

Our goal is now to give a bound on the sum of the multiplicites of
$\h_\tau$ at all its isolated roots, for any $\tau$ in $\bar{\field}$.

\begin{Lemma}
  The ideal $\frak{J}'$ has dimension zero and $V(\frak{J}') \subset
  \overline{\field(T)}{}^n$ is the set of isolated solutions of
  $V(\frak{J}) \subset \overline{\field(T)}{}^n$.
\end{Lemma}
\begin{proof}
 From the equality $J=J' \cap J''$ and Corollary~3.4 in~\cite{AtMc},
 we deduce that $\frak{J}=\frak{J'} \cap \frak{J''}$; the properties
 of $J'$ (the irreducible components of $V(J')$ are precisely those
 irreducible components of $V(J)$ that have dimension one and with a
 dense image by $\pi_T$) imply our claim.
\end{proof}

Let us write $c=\dim_{\bar{\field}(T)}(\bar{\field}(T)[\mat{X}]/{\frak J}')$.  The
following lemma relates this quantity to the multiplicities of the
solutions in any fiber $\h_\tau$. This proves the first statement
in Proposition~\ref{degree_fiber}.

\begin{Lemma}\label{lemma:19}
  Let $\tau$ be in $\bar{\field}$. The sum of the multiplicities of the
  isolated solutions of $\h_\tau$ is at most equal to $c$.
\end{Lemma}
\begin{proof}
 The sum in the lemma is also the sum of the multiplicities of the
  ideal $J_\tau$ at all $(\tau,{\bf x})$, for ${\bf x}$ an isolated solution of
  $\h_\tau$.  By Lemma~\ref{lemma:JJprime}, this is also the sum of
  the multiplicities of $J'_\tau$ at all $(\tau,{\bf x})$, for ${\bf x}$ an
  isolated solution of $\h_\tau$. We prove below that the sum of the
  multiplicities of $J'_\tau$ at all $(\tau,{\bf x})$, for ${\bf x}$ such that
  $(\tau,{\bf x})$ cancels $J'_\tau$, is at most $c$; this will be enough
  to conclude (for any isolated solution ${\bf x}$ of $\h_\tau$,
  $(\tau, {\bf x})$ is a root of $J'_\tau$, though the converse may not be
  true).
  
  Let $m_1,\dots,m_\mu$ be monomials that form a $\bar{\field}$-basis of
  $\bar{\field}[T, \mat{X}]/J'_\tau$; since $T-\tau$ is in $J'_\tau$, these
  monomials can be assumed not to involve $T$.  We will prove that
  they are still $\bar{\field}(T)$-linearly independent in
  $\bar{\field}(T)[\mat{X}]/{\frak J}'$; this will imply that $\mu \le c$,
  and finish the proof of the first statement.
  
   Suppose that there exists a linear combination $A_1 m_1 + \cdots +
  A_\mu m_\mu$ in ${\frak J}'$, with all $A_i$'s in $\bar{\field}(T)$, not
  all of them zero. Thus, we have an equality $a_1/d_1\, m_1 + \cdots
  + a_\mu/d_\mu\, m_\mu = a/d$, with $a_1,\dots,a_\mu$ and
  $d,d_1,\dots,d_\mu$ in $\bar{\field}[T]$ and $a$ in the ideal
  $J'$. Clearing denominators, we obtain a relation of the form $b_1
  m_1 +\cdots+ b_\mu m_\mu \in J'$, with not all $b_i$'s zero. Let
  $(T-\tau)^e$ be the highest power of $T-\tau$ that divides all
  $b_i$'s (this is well-defined, since not all $b_i$'s vanish) so that
  we can rewrite the above as $(T-\tau)^e (c_1 m_1 +\cdots+ c_\mu
  m_\mu) \in J'$, with $c_i=b_i/(T-\tau)^e \in \bar{\field}[T]$ for all $i$.
  In particular, our definition of $e$ implies that the values
  $c_i(\tau)$ are not all zero.
  
  
  Recall that the ideal $J'$ has the form $J'=Q_1 \cap \cdots \cap
  Q_t$. For $i=1,\dots,t$, since $Q_i$ is primary, the membership
  equality $(T-\tau)^e (c_1 m_1 +\cdots +c_\mu m_\mu) \in J'$ implies
  that either $c_1 m_1 +\cdots +c_\mu m_\mu$ or some power
  $(T-\tau)^{ef}$, for some $f > 0$, is in $Q_i$. Since $Q_i$ does not
  contain non-zero polynomials in $\bar{\field}[T]$, $c_1 m_1 +\cdots+ c_\mu
  m_\mu$ belongs to all $Q_i$'s, that is, to $J'$. We can then
  evaluate this relation at $T=\tau$. We saw that the values
  $c_i(\tau)$ do not all vanish on the left, which is a contradiction
  with the independence of the monomials $m_1,\dots,m_\mu$ modulo
  $J'_\tau$.
\end{proof}

To conclude the proof of Proposition~\ref{degree_fiber}, we
consider $\tau \in \bar{\field}$ such that ${\sf G}(\tau)$ holds. Without
loss of generality, we assume that $\tau=0$.

The field of Puiseux series $\bar{\field}\langle\langle T \rangle\rangle$
contains an algebraic closure of $\bar{\field}(T)$; we thus let
$\Phi_1,\dots,\Phi_{c'}$ be the points of $V(\mathfrak{J}')$, with
coordinates taken in $\bar{\field}\langle\langle T \rangle\rangle$. In
particular, we see that $c' \le c$; we prove below that we actually
have $c'=c$.

For a vector $\Phi=(\varphi_1,\dots,\varphi_s)$
with entries in $\bar{\field}\langle\langle T \rangle\rangle$, the valuation $\nu(\Phi)$ is the minimum of the valuations of its exponents. We say that $\Phi$ is {\em
  bounded} if it has non-negative valuation; in this case,
$\lim_0(\Phi)$ is defined as the vector
$(\lim_0(\varphi_1),\dots,\lim_0(\varphi_s))$, with
$\lim_0(\varphi_i)={\rm coeff}(\varphi_i,T^0)$ for all $i$.
\begin{Lemma}
   $\Phi_1,\dots,\Phi_{c'}$ are bounded.
\end{Lemma}
\begin{proof}
  For $i=1,\dots,c'$, write $\Phi_i=1/T^{e_i}
  (\Psi_{i,1},\dots,\Psi_{n})$, for a vector
  $(\Psi_{i,1},\dots,\Psi_{i,n})$ of Puiseux series of valuation
  zero, that is, such that all $\Psi_{i,j}$ are bounded and
  $(\psi_{i,1},\dots,\psi_{i,n})=\lim_0(\Psi_{i,1},\dots,\Psi_{i,n})$
  is nonzero. Hence,
  $e_i=-\nu(\Phi_i)$, and we have to prove that $e_i \le 0$.  By way
  of contradiction, we assume that $e_i > 0$.

  The Puiseux series $\Phi_i$ cancels $h_1,\dots,h_M$. For
  $k=1,\dots,M$, let $h_k^H \in \bar{\field}[T][X_0,\mat{X}]$ be the homogenization
  of $h_k$ with respect to $\mat{X}$. From the equality
  $h_k^H(T^{e_i},\Psi_{i,1},\dots,\Psi_{i,n})=
  T^{e_i}h_k(\Phi_i)$, we deduce that
  $h_k^H(T^{e_i},\Psi_{i,1},\dots,\Psi_{i,n})=0$ for all $k$. We
  can write $h_k = h_{0,k} + T \tilde h_k$, for some polynomial
  $\tilde h_k$ in $\bar{\field}[T,\mat{X}]$, and ${\sf G}_2(0)$ implies that
  $\deg_\mat{X}(\tilde h_k) \le \deg_\mat{X}(h_{0,k})$. As a result, the
  homogenizations (with respect to $\mat{X}$) of $h_{k},h_{0,k}$ and $\tilde
  h_k$ satisfy a relation of the form $h^H_k = h_{0,k}^H +
  X_0^{\delta_k} T \tilde h^H_k$, for some $\delta_k \ge 0$. This
  implies the equality
  $$h_{0,k}^H(T^{e_i},\Psi_{i,1},\dots,\Psi_{i,n}) + T^{\delta_k
    e_i+1}\tilde h_k^H(T^{e_i},\Psi_{i,1},\dots,\Psi_{i,n})=0.$$
  The second term has positive valuation, so that
  $h_{0,k}^H(T^{e_i},\Psi_{i,1},\dots,\Psi_{i,n})$ has positive
  valuation as well. Taking the coefficient of $T^0$, this means 
  that $h_{0,k}^H(0,\psi_{i,1},\dots,\psi_{i,n})=0$ (since $e_i > 0$), which implies 
  that $(\psi_{i,1},\dots,\psi_{i,n})=(0,\dots,0)$, in view of ${\sf G}_3(0)$.
  This however contradicts the definition of $(\psi_{i,1},\dots,\psi_{i,n})$.
\end{proof}

For $i=1,\dots,c'$, we define $\varphi_i =
(\varphi_{i,1},\dots,\varphi_{i,n})$ as $\varphi_i=\lim_0(\Phi_i)\in
\bar{\field}{}^n$. In particular, all $\varphi_i$, $i=1,\dots,c'$, are roots
of $\h_0$.

\begin{Lemma}\label{lemma:Jprimerad}
  The ideal $\frak{J}'$ is radical; equivalently, $c'=c$.
\end{Lemma}

\begin{proof}
 We know that $\frak{J}'$ has dimension zero, so it is enough to prove
 that for $i=1,\dots,c'$, the localization of $\bar{\field}\langle\langle T
 \rangle\rangle[\mat{X}]/\frak{J}'$ at the maximal ideal
 $\mathfrak{m}_{\Phi_i}$ is a field, or equivalently that the
 localization of $\bar{\field}\langle\langle T \rangle\rangle[\mat{X}]/\frak{J}$
 at $\mathfrak{m}_{\Phi_i}$ is a field.  By the Jacobian
 criterion~\cite[Theorem~16.19.b]{Eisenbud95}, this is the case if and
 only if the Jacobian matrix of $\h$ with respect to $\mat{X}$ has full
 rank $n$ at $\Phi_i$. We know that $\varphi_i=\lim_0(\Phi_i)$ is a
 root of $\h_0$, and the Jacobian criterion conversely implies that
 since the ideal $\langle \h_0 \rangle$ is radical (by assumption
 ${\sf G}_1(0)$) and zero-dimensional (by assumption ${\sf G}_3(0)$),
 the Jacobian matrix of $\h_0(\mat{X})=\h(0,\mat{X})$ has full rank $n$. Since
 this matrix is the limit at zero of the Jacobian matrix of $\h$ with
 respect to $\mat{X}$, taken at $\Phi_i$, the latter must have full rank
 $n$, and our claim that $\frak{J}'$ is radical is proved.
\end{proof}

To finish the proof of Proposition~\ref{degree_fiber}, we have to
establish that $V(\h_0)$ consists of exactly $c$ solutions.  Let thus
$d$ be the number of points in $V(\h_0)$.  Since $\langle \h_0
\rangle$ is radical (this is ${\sf G}_1(0)$), Lemma~\ref{lemma:19}
implies that $d \le c$, so we only have to prove that $c \le d$. To
prove this, we prove that for $i,i'$ in $\{1,\dots,c\}$, with $i \ne
i'$, we have $\varphi_i \ne \varphi_{i'}$.

Suppose to the contrary that $\varphi_i = \varphi_{i'}$. We know that
the Jacobian matrix of $\h_0$ has full rank $N$ at $\varphi_i$; up to
reindexing, we assume that rows $1,\dots,n$ correspond to a maximal
nonzero minor. Let $\h'=(h_1,\dots,h_n)$.

Let $m=\nu(\Phi_i-\Phi_{i'})$; since  $\varphi_i = \varphi_{i'}$, we have
$m > 0$. We can thus write $\Phi_i=f + T^m
\delta_i$ and $\Phi_{i'}=f + T^m \delta_{i'}$, for some vectors of
bounded Puiseux series $f, \delta_i, \delta_{i'}$ such that all terms
in $f$ have valuation less than~$m$; in addition, $\lim_0(\delta_i)
\ne \lim_0(\delta_{i'})$. Write the Taylor expansion of $\h'$ at $f$ as
$$\h'(\Phi_i) = \h'(f) + \mathrm{jac}(\h',\mat{X}) T^m \delta_i + T^{2m} r_i =0$$
and
$$\h'(\Phi_{i'}) = \h'(f) + \mathrm{jac}_f(\h',\mat{X}) T^m \delta_{i'} + T^{2m}
r_{i'} =0,$$ for some vectors of bounded Puiseux series $r_i,r_{i'}$.
By subtraction and division by $T^m$, we obtain $\mathrm{jac}(\h',\mat{X})
(\delta_i-\delta_{i'}) = T^m r$, for some vector of bounded Puiseux
series $r$.  Since $\mathrm{jac}(\h',\mat{X})$ is invertible, this further gives
$\delta_i-\delta_{i'} = T^m r'$, where again $r'$ is a vector of
bounded Puiseux series.  However, by construction the left-hand side
has valuation zero, while the right-hand side has positive valuation
(since $m > 0$). Hence, we derived a contradiction to our assumption
that $\varphi_i = \varphi_{i'}$. The proof of Proposition~\ref{degree_fiber} is
complete.

\section{Proof of Lemma~\ref{lemma:Z1}}
\label{sec:proofZ1}
\end{document}