\documentclass[12pt]{article}
\usepackage{bbm,fullpage}
\usepackage{bm}
\usepackage{amsmath,amssymb,amsthm}
\usepackage{algorithm, pseudocode}
\usepackage{mathrsfs}
\usepackage{enumitem}
\usepackage[titles]{tocloft}
\usepackage{yfonts}
\usepackage{xcolor}
\usepackage[pdftex,                %
%    pagebackref,                %
    bookmarks         = true,%     % Signets
    bookmarksnumbered = true,%     % Signets numérotés
    pdfpagemode       = None,%     % Signets/vignettes fermé à l'ouverture
    pdfstartview      = FitH,%     % La page prend toute la largeur
    pdfpagelayout     = SinglePage,% Vue par page
    colorlinks        = true,%     % Liens en couleur
%    linkcolor= monvert, %    % couleur des liens internes
%    anchorcolor= blue, %    % couleur des liens internes
   citecolor         =blue,
    urlcolor          = magenta,%  % Couleur des liens externes
    pdfborder         = {0 0 0}%   % Style de bordure : ici, pas de bordure
    ]{hyperref}%                   % Utilisation de HyperTeX

\usepackage{amsmath}
\usepackage{easybmat}
\usepackage{multirow,bigdelim}

\newcommand*\hexbrace[2]{%
  \underset{#2}{\underbrace{\rule{#1}{0pt}}}}

\usepackage[indexonlyfirst,ucmark,toc]{glossaries}
\renewcommand*{\glstextformat}[1]{\textcolor{black}{#1}}
%\glsdisablehyper %pour enlever les liens hypertexte



\def\softO{\ensuremath{{O}{\,\tilde{ }\,}}}


\def\Vc{\ensuremath{\mathcal{V}}}
\def\A {\ensuremath{\mathbb{A}}}
\def\B {\ensuremath{\mathbb{B}}}
\def\Co {\ensuremath{\mathbb{C}}}
\def\CC {\ensuremath{\overline{\mathbf{K}}}}
\def\PP {\ensuremath{\mathsf{P}}}
\def\BB {\ensuremath{H}}
\def\P {\ensuremath{\mathbb{P}}}
\def\Q {\ensuremath{\mathbb{Q}}}
\def\N {\ensuremath{\mathbb{N}}}
\def\Re {\ensuremath{\mathbb{R}}}
\def\RP {\ensuremath{\tilde{\mathbf{R}}}}
\def\CP {\ensuremath{\tilde{\mathbf{C}}}}
\def\Z {\ensuremath{\mathbb{Z}}}
\def\RR {\ensuremath{\mathbb{K}}}
\def\k{\ensuremath{\mathbb{k}}}
\def\kbar{\ensuremath{\bar{\mathbb{k}}}}

\def\K {\ensuremath{\mathbb{K}}}
\def\KKbar {\ensuremath{\overline{\mathbf{K}}}}
\def\KKCbar {\ensuremath{\overline{\mathbf{K}(\mathfrak{G})}}}
\def\KKGbar {\ensuremath{\overline{\mathbf{K}(\mathfrak{L})}}}
\def\KKGpbar {\ensuremath{\overline{\mathbf{K}(\mathfrak{L}')}}}
\def\BS {\ensuremath{{B}}}
\def\CS {\ensuremath{{C}}}
\def\RS {\ensuremath{\mathsf{R}}}
\def\LL {\ensuremath{\mathbf{L}}}
\def\LLbar {\ensuremath{\overline{\mathbf{L}}}}
\def\GG {\ensuremath{\mathbf{G}}}
\def\KK {\ensuremath{\mathbf{K}}}
\def\D {\ensuremath{\mathbf{D}}}
\def\n {\ensuremath{\mathbf{n}}}
\def\d {\ensuremath{\mathbf{d}}}
\def\p {\ensuremath{\mathbf{p}}}
\def\I {\ensuremath{\mathbf{I}}}
\def\mI {\ensuremath{\mathbf{I}}}
\def\mS {\ensuremath{\mathbf{S}}}
\def\mzero {\ensuremath{\mathbf{0}}}
\def\mT {\ensuremath{\mathbf{T}}}
\def\mJ {\ensuremath{\mathbf{J}}}
\def\JJ {\ensuremath{\mathbf{J}}}
\def\sfB {\ensuremath{\sf B}}
\def\sfA {\ensuremath{\sf A}}


\def\MM {\ensuremath{\mathbb{M}}}
\def\mm {\ensuremath{\mathbf{m}}}
\def\v {\ensuremath{\mathbf{v}}}
\def\vc {\ensuremath{\mathbf{c}}}
\def\vd {\ensuremath{\mathbf{d}}}

\def\mA{\ensuremath{{A}}}
\def\mU{\ensuremath{{U}}}
\def\mL{\ensuremath{{L}}}
\def\mF{\ensuremath{{F}}}
\def\mG{\ensuremath{{G}}}
\def\mH{\ensuremath{{H}}}
\def\mB{\ensuremath{{B}}}
\def\mM{\ensuremath{{M}}}
\def\mN{\ensuremath{{N}}}
\def\mD{\ensuremath{\mathbf{D}}}
\def\mR{\ensuremath{\mathbf{R}}}
\def\G {\ensuremath{\mathrm{GL}}}
\def\Mat{\ensuremath{{\rm M}}}
\def\grad{\ensuremath{{\rm grad}}}
\def\ker{\ensuremath{{\rm ker}}}
\def\rank{\ensuremath{{\rm rank}}}
\def\jac{\ensuremath{{\rm jac}}}
\def\ext{\ensuremath{{\rm ext}}}

\def\scrS{\ensuremath{\mathscr{S}}}
\def\scrG{\ensuremath{\mathscr{G}}}
\def\scrZ{\ensuremath{\mathscr{Z}}}
\def\scrA{\ensuremath{\mathcal{A}}}
\def\scrT{\ensuremath{\mathscr{T}}}
\def\scrL{\ensuremath{\mathscr{L}}}
\def\scrU{\ensuremath{\mathcal{U}}}
\def\scrZ{\ensuremath{\mathscr{Z}}}
\def\scrR{\ensuremath{\mathscr{R}}}
\def\scrY{\ensuremath{\mathscr{Y}}}
\def\scrB{\ensuremath{\mathscr{B}}}

\def\y {\ensuremath{\mathbf{y}}}
\def\e {\ensuremath{\mathbf{e}}}
\def\a {\ensuremath{\mathbf{a}}}
\def\b {\ensuremath{\mathbf{b}}}
\def\z {\ensuremath{\mathbf{z}}}
\def\w {\ensuremath{\mathbf{w}}}
\def\f {\ensuremath{\mathbf{f}}}
\def\r {\ensuremath{\mathbf{r}}}
\def\s {\ensuremath{\mathbf{s}}}
\def\L {\ensuremath{\mathbf{L}}}
\def\F {\ensuremath{\mathbf{F}}}
\def\G {\ensuremath{\mathbf{G}}}
\def\E {\ensuremath{\mathbf{E}}}
\def\X {\ensuremath{\mathbf{X}}}
\def\Y {\ensuremath{\mathbf{Y}}}
\def\H {\ensuremath{\mathbf{H}}}
\def\m {\ensuremath{\mathfrak{m}}}
\def\v {\ensuremath{\mathbf{v}}}
\def\u {\ensuremath{\mathbf{u}}}
\def\q {\ensuremath{\mathbf{q}}}
\def\U {\ensuremath{\mathbf{U}}}
\def\V {\ensuremath{\mathbf{V}}}
\def\t {\ensuremath{\mathbf{t}}}


\def\blambda{\mbox{\boldmath$\lambda$}}
\def\bdelta{\mbox{\boldmath$\delta$}}
\def\bzeta{\mbox{\boldmath$\zeta$}}
\def\bkappa{\mbox{\boldmath$\kappa$}}
\def\bmu{\mbox{\boldmath$\mu$}}
\def\bsigma{\mbox{\boldmath$\sigma$}}
\def\bpsi{\mbox{\boldmath$\psi$}}
\def\bphi{\mbox{\boldmath$\phi$}}
\def\bh{\mbox{\boldmath$h$}}
\def\bg{\mbox{\boldmath$g$}}
\def\bbf{\mbox{\boldmath$f$}}
\def\brho{\mbox{\boldmath$\rho$}}
\def\bell{\mbox{\boldmath$\ell$}}
\def\sbdelta{\mbox{\scriptsize \boldmath$\delta$}}
\def\sbell{\mbox{\scriptsize \boldmath$\ell$}}

\def\bm{{\mathbf{m}}}

\def\xxx{\textbf{(xxx)}}
\def\why{\textbf{(why?)}}
\def\todo#1{(\textbf{todo:} #1)}
\def\check{\textbf{(check!)}}

\def\KKbar {\ensuremath{\overline{\mathbf{K}}}}
 
\DeclareBoldMathCommand{\bM}{M}
\DeclareBoldMathCommand{\bD}{D}
\DeclareBoldMathCommand{\bB}{B}
\DeclareBoldMathCommand{\bA}{A}
\DeclareBoldMathCommand{\bC}{C}
\DeclareBoldMathCommand{\bX}{X}
\DeclareBoldMathCommand{\bi}{i}
\DeclareBoldMathCommand{\bk}{k}
\DeclareBoldMathCommand{\bY}{Y}
\DeclareBoldMathCommand{\bn}{n}
\DeclareBoldMathCommand{\bbb}{b}
\DeclareBoldMathCommand{\d}{d}
\DeclareBoldMathCommand{\bc}{c}
\DeclareBoldMathCommand{\bj}{j}
\DeclareBoldMathCommand{\be}{e}
\DeclareBoldMathCommand{\bh}{h}
\DeclareBoldMathCommand{\br}{r}
\DeclareBoldMathCommand{\bu}{u}
\DeclareBoldMathCommand{\bv}{v}
\DeclareBoldMathCommand{\bx}{x}
\DeclareBoldMathCommand{\by}{y}
\DeclareBoldMathCommand{\btheta}{\vartheta}
\DeclareBoldMathCommand{\c}{c}
\DeclareBoldMathCommand{\f}{f}
\DeclareBoldMathCommand{\g}{g}
\DeclareBoldMathCommand{\h}{h}
\DeclareBoldMathCommand{\x}{x}
\DeclareBoldMathCommand{\bell}{\ell}
\DeclareBoldMathCommand{\bbeta}{\beta}
\DeclareBoldMathCommand{\balpha}{\alpha}
\DeclareBoldMathCommand{\bgamma}{\gamma}

\def\mult{\ensuremath{\mathrm{mult}}}
\def\rdeg{\ensuremath{\mathrm{rdeg}}}
\def\cdeg{\ensuremath{\mathrm{cdeg}}}

\def\assH{\ensuremath{\mathsf{A}}}
\def\assA{\ensuremath{\mathsf{B}}}
\def\assG{\ensuremath{\mathsf{C}}}
\def\assD{\ensuremath{\mathsf{D}}}
\def\assI{\ensuremath{\mathsf{J}}}
\def\assJ{\ensuremath{\mathsf{K}}}

\def\sfG {\ensuremath{\sf G}}

\newcommand{\VpF}[2]{V_{#1}(#2)}
\newcommand{\VpFG}[3]{V_{#1}(#2,#3)}

\def\rc{\ensuremath{{c'}{}}}
\def\re{\ensuremath{{e'}{}}}




\def\NOTE#1#2{{\begin{quote}\marginpar[\hfill{#1}]{{#1}}{{\textsf{[\![{#2}]\!]}}}\end{quote}}}
\def\respond#1{\NOTE{\textcircled{\textsc{a}}}{Note:~{#1}}}

\title{Solving determinantal systems using homotopy techniques}

\newtheorem{pbm}{Problem}
\newtheorem{definition}{Definition}
\newtheorem{theorem}[definition]{Theorem}
\newtheorem{corollary}[definition]{Corollary}
\newtheorem{proposition}[definition]{Proposition}
\newtheorem{lemma}[definition]{Lemma}

\begin{document}

\maketitle

\begin{abstract}
  Let $\KK$ be a field of characteristic $0$ and $\KKbar$ be an
  algebraic closure of $\KK$. Consider a sequence of polynomials
  $G=(g_1,\dots,g_s)$ in $\KK[X_1,\dots,X_n]$, a polynomial matrix
  $\mF=[f_{i,j}] \in \KK[X_1,\dots,X_n]^{p \times q}$, with $p \leq q$
  and the set $S_{\mF, G}$ of points in $\KKbar$ at which all polynomials in $\mG$
  vanish and the rank of $\mF$ is deficient.

  For $i=1,\dots,p$, we write $\rdeg(\mF,i)$ for the degree of the
  $i$th row of $\mF$ and let
  $\delta_{\rm row}(\mF) = S_{q-p+1}(\rdeg(\mF,1), \ldots,
  \rdeg(\mF,p))$
  where for $k\ge 0$, $S_k(C_1,\dots,C_p)$ is the $k$th complete
  symmetric polynomial in variables $C_1,\dots,C_p$.

  Assume that a straight-line program of length $E$ evaluating $G$ and
  $\mF$ is given.  We describe a probabilistic algorithm for computing
  \emph{all} isolated points of $S_{\mF, G}$. When these solutions
  have multiplicity one, the algorithm runs in time linear in $E$,
  quadratic in $\deg(g_1)\cdots \deg(g_s)\delta_{{\rm row}}(\mF)$.
  This latter quantity is the generic (and maximum) number of
  solutions which somewhat reflects the optimality of our algorithm.

  Variants of this algorithm are given when isolated points have
  positive multiplicity (at the cost of an extra factor
  ${{q}\choose{p}}$). 

  Applying these results to polynomial optimization problems allows to
  compute all local minimizers and maximizers in time polynomial in
  their algebraic degree.
\end{abstract}

\section{Introduction}\label{sec:intro}

In what follows, $\KK$ is a field and $\KKbar$ is an algebraic closure
of $\KK$, $(X_1, \ldots, X_n)$ is a set of $n$ variables and
$\KK[X_1,\dots,X_n]$ is the multivariate polynomial ring with
coefficients in $\KK$. Let
$\mF=[f_{i,j}] \in \KK[X_1,\dots,X_n]^{p \times q}$ be a polynomial
matrix, with $p \leq q$. The first question which will interest us in
this paper is to describe the set $S_\mF$ of points
$\x \in \KKbar{}^n$ at which the evaluation of the matrix $\mF$ has
rank less than $p$. Observe that $S_\mF$ is an algebraic set, since it
is defined by the vanishing of all maximal minors of $\mF$.  In the
particular case $p=1$, this simply means finding all common solutions
of $f_{1,1},\dots,f_{1,q}$. 

We restrict ourselves to the case where $n = q-p+1$.  Indeed, in this
case, $S_\mF$ is expected to be finite: for arbitrary $n$, results due
to Macaulay~\cite{Macaulay16} and Eagon and Northcott~\cite{EN62}
imply that all irreducible components of $S_\mF$ have dimension at
least (and, expectedly, equal to) $n-(q-p+1)$; indeed, in the case
$n = q-p+1$, $S_\mF$ is finite for a generic choice of the entries of
$\mF$ (we will make this statement more precise in the following
sections). However, beyond the equality $n = q-p+1$, we make no
assumption on the input matrix $\mF$.

Because $S_\mF$ is an algebraic set, its {\em isolated points} form a
finite subset of $\KKbar{}^n$, defined over $\KK$; this notion makes
sense for any field $\KK$; when $\KK=\mathbb{R}$, these points are
indeed isolated for the metric topology. The next section recalls the
basic facts we need regarding polynomial systems and algebraic
varieties.  Then, our first question is the following.


\begin{pbm} \label{problem} 
  Given a field $\KK$, a matrix $\mF \in \KK[X_1,\dots,X_n]^{p \times q}$ with $p
  \leq q$ and $n = q-p+1$, compute the isolated points of $S_\mF$.
\end{pbm}

Problem~\ref{problem} is a particular case of a slightly more general
question% , where we solve determinantal systems of an algebraic set
. In addition to matrix $\mF$, we may indeed take into account
algebraic equations of the form $G=(g_1,\dots,g_s)$ in
$\KK[X_1,\dots,X_n]$. In this setting, the natural relation between
the number $n$ of variables, the size of $\mF$ and the number $s$ of
polynomials in $G$ is now $n=q-p+s+1$. Then, we define the algebraic
set
$$S_{\mF,G} =  \{\bx \in \KKbar{}^n \mid
\mathrm{rank}(\mF({\bx})) < p \text{~and~}
g_1(\bx)=\cdots=g_s(\bx)=0   \};$$
for generic $\mF$ and $G$, this is a finite set. Our main problem is then
the following.
\begin{pbm} \label{problem2} 
  Given a field $\KK$, a matrix $\mF \in \KK[X_1,\dots,X_n]^{p \times q}$ and
  polynomials $G=(g_1,\dots,g_s)$ in $\KK[X_1,\dots,X_n]$, with $p \leq q$ and
  $n = q-p+s+1$, compute the isolated points of $S_{\mF,G}$.
\end{pbm}

We will represent the output of our algorithm using univariate
polynomials. Let $V \subset \KKbar{}^n$ be a zero-dimensional algebraic
set defined over $\KK$. A \emph{zero-dimensional parametrization}
$\scrR = ((q,v_1, \ldots, v_n), \lambda)$ of $V$ consists in
polynomials $(q,v_1, \ldots, v_n)$ such that $q \in \KK[X]$ is monic
and squarefree, all $v_i \in \KK[X]$ and $\deg(v_i) < \deg(q)$, and
$\lambda$ is a $\KK$-linear form in $n$ variables, such that
\begin{itemize}
\item $\lambda(v_1, \ldots, v_n) = Xq'$ mod $q$
\item we have $V = Z(\scrR)$, with $$Z(\scrR)= \left\{\left(\frac{v_1(\tau)}{q'(\tau)}, \ldots, \frac{v_n(\tau)}{q'(\tau)}\right) \ | \ q(\tau) = 0\right\};$$
\end{itemize}
the constraint on $\lambda$ says that the root of $q$ are the values
taken by $\lambda$ on $V$. This representation was introduced
in~\cite{Kronecker82,Macaulay16}, and has been used in a variety of
algorithms, such
as those in~\cite{GiMo89,GiHeMoPa95,ABRW,GiHeMoMoPa98,Rouillier99,GiLeSa01}.
The reason we use a rational parametrization, with $q'$ as a
denominator, goes back to~\cite{ABRW, Rouillier99, GiLeSa01}: when
$\KK=\Q$, this allows us to control precisely the bit-size of the
coefficients, using bounds such as those
in~\cite{Schost03,DaSc04}. (The same phenomenom holds with $\K=k(T)$,
in which case we want to control degrees in $T$ of the numerators and
denominators of the coefficients of $\scrR$.)


Henceforth, we write the matrix $\mF$ as 
\[ \mF = 
\left( \begin{matrix}
f_{1,1} & \cdots & f_{1,q}\\
\vdots & \ddots & \vdots \\
f_{p,1} & \cdots & f_{p,q}
\end{matrix} \right), \ \mathrm{where} \ f_{i,j} \in \KK[X_1,\dots,X_n] \ \mathrm{for} \ 1 \leq i \leq p, 1 \leq j \leq q.
\]
We will consider two kinds of degree for matrix $\mF$ (these two
situations have been discussed before for determinantal
ideals~\cite{NieRan09,MiSt04}). For $i=1,\dots,p$, we will write $\rdeg(\mF,i)$ for the
degree of the $i$th row of $\mF$, that is,
$\rdeg(\mF,i)=\max(\deg(f_{i,j}))_{1 \le j \le q}$; similarly, for $j=1,\dots,q$, we write
$\cdeg(\mF,j)$ for the degree of the $j$th column of $\mF$, that is,
$\cdeg(\mF,j)=\max(\deg(f_{i,j}))_{1 \le i \le p}$. Then, we define
$$\delta_{\rm row}(\mF) = S_{q-p+1}(\rdeg(\mF,1), \ldots, \rdeg(\mF,p)),$$
where for $k\ge 0$, $S_k(C_1,\dots,C_p)$ is the $k$th complete symmetric
polynomial in variables $C_1,\dots,C_p$; we also set 
$$\delta_{\rm col}(\mF) = E_{q-p+1}(\cdeg(\mF,1), \ldots, \cdeg(\mF,q)),$$
where for $k \ge 0$, $E_k(\delta_1,\dots,\delta_q)$ is the 
elementary symmetric polynomial of degree $k$ in $\delta_1, \ldots, \delta_q$.

For any matrix $\mF$ over a ring $R$, and for any integer $r$,
$M_r(\mF)$ will denote the set of $r$-minors of $\mF$, and $I_r(\mF)$
will denote the ideal they generate in $R$. If $R$ is a polynomial
ring in $n$ variables over a field $\KK$, $V_r(\mF)$ will denote the
zero-set of $I_r(\mF)$ in $\KKbar{}^n$. Our first result is then the
following.
\begin{theorem}\label{theo:1}
  Let $\mF$ be in $\KK[X_1,\dots,X_n]^{p \times q}$ and let
  $(g_1,\dots,g_s)$ be in $\KK[X_1,\dots,X_n]$, with $p \le q$ and
  $n=q-p+s+1$. Then, sum of the multiplicities of the isolated points
  of $I_p(\mF) + \langle g_1,\dots,g_s \rangle$ is at most $\deg(g_1)
  \cdots \deg(g_s) \min(\delta_{\rm row}(\mF), \delta_{\rm
    col}(\mF))$.
\end{theorem}
Previous work by Miller and Sturmfels~\cite[Chapter~15]{MiSt04} proved
very general results on the multi-degrees of determinantal ideals
built from matrices with indeterminate entries (in which case we have
$s=0$, but the assumption $n=q-p+1$ does not hold); in particular,
they obtain analogues (and generalizations) of the result in
Theorem~\ref{theo:1} in that context.

Nie and Ranestad proved that the bounds in Theorem~\ref{theo:1} are
equalities for two families of polynomials:
\begin{itemize}
\item when the entries of $\mF$ are chosen generic homogeneous and
  such that $\deg(f_{i,j}) = \rdeg(\mF,i)$ for all $i,j$, then the
  ideal $I_{\mF}$ has degree $\delta_{\rm row}(\mF)$;
\item when the entries of $\mF$ are chosen generic homogeneous and
  such that $\deg(f_{i,j}) = \cdeg(\mF,j)$ for all $i,j$, the ideal
  $I_{\mF}$ has degree $\delta_{\rm col}(\mF)$.
\end{itemize}
From this, they deduce that the degree of the ideal $I_{\mF,G}$ is at
most $\deg(g_1) \cdots \deg(g_s) \delta_{\rm row}(\mF)$, in the
particular case when the matrix $\mF$ is the Jacobian matrix of $G$
(together with one extra row, corresponding to the gradient of a
function $q$\footnote{\`a changer car $q$ est utilis\'e dans les
  param\'etrisations et le nombre de colonnes} that we want to
optimize on $V(G)$), and assuming that this ideal has dimension zero.

For generic (homogeneous) polynomials $G$, and $\mF$ as in the
previous paragraph, Spaenlehauer gave in~\cite{Spa14} an explicit
expression for the Hilbert function of the ideal $I_{\mF,G}$.


Our second result gives a bound on the cost of computing a
zero-dimensional parametrization of the isolated solutions of
$I_\mF$. Our algorithms take as input a \emph{straight-line program}
(that is, a sequence of elementary operations $+, -, \times$) that
computes the entries of $\mF$ from the input variables $\X$. The
\emph{length $L$} of the input is the number of operations it
performs. This assumption is not restrictive, since any matrix $\mF$
can be computed by a (trivial) straight-line program.

\begin{theorem}
  Let $\mF$ be in $\KK[\X]^{p \times q}$, with $p \le q$, and assume
  that either~\eqref{eq:rowdeg} or~\eqref{eq:coldeg} holds. Then,
  there exists a randomized algorithm which computes a
  zero-dimensional parametrization for the isolated solutions of
  $I_\mF$, using $({q \choose p} L \delta)^{O(1)}$.
\end{theorem}
The number ${q \choose p}$ is the number of elements in the
determinantal system of $\mF$. In several cases, we can improve the
algorithm and replace this by the number $q$ of columns of $\mF$,
for instance, when all isolated solutions of $I_\mF$ are known
to have multiplicity one.

The main idea of our algorithms is to use the \emph{homotopy}
\[\mH = (1-T)\cdot \mG + T \cdot \mF \in \KK[T, \X]^{p \times q}\]
that connects a \emph{start matrix} $\mG$ to the target matrix $\mF$,
where $T$ is a new variable\footnote{\`a changer car $\mG$ est
  utilis\'e aussi comme une liste de polyn\^omes}. However, we need
some properties on the start matrix such that we can compute the
isolated solutions for the determinantal system of $F$. Assume that we
know the solutions for the determinatal system of $G$, by using the
Newton iteration to the determinantal system of $H$, we can lift into
the isolated solutions for that of $F$. Therefore, we need to create a
start matrix $G$ such that
\begin{itemize}
\item its solutions of the determinantal system can be found effectively,
\item we can extract a square subsystem with full-rank Jacobian,
\item there is no solution of its determinantal system at infinity. 
\end{itemize}

This paper is organized as follows.

\todo{Some background on polynomial systems and stuff}

%%%%%%%%%%%%%%%%%%%%%%%%%%%%%%%%%%%%%%%%%%%%%%%%%%%%%%%%%%%%
%%%%%%%%%%%%%%%%%%%%%%%%%%%%%%%%%%%%%%%%%%%%%%%%%%%%%%%%%%%%
%%%%%%%%%%%%%%%%%%%%%%%%%%%%%%%%%%%%%%%%%%%%%%%%%%%%%%%%%%%%

\section{Preliminary: a local dimension test} \label{sec:isolated}

Let $\bC=(c_1,\dots,c_m)$ be polynomials in $\KK[\bX]$, with
$\bX=(X_1,\dots,X_n)$, for a field $\KK$. Given a point $\bx$ in
$V(\bC)$, we discuss here how to decide whether $\bx$ is an isolated
point in $V(\bC)$. We make the following assumption:
\begin{description}[leftmargin=*]
\item [$\assH.$] We are given as input an integer $\mu$ such that
 either $\bx$ belongs to a positive-dimensional component of $V(\bC)$,
 or $\bx$ is isolated in $V(\bC)$, with multiplicity at most $\mu$
  with respect to the ideal $\langle \bC\rangle$.
\end{description}

\begin{proposition}\label{prop:testisolated}
  Suppose that $\bC$ is given by a straight-line program of length $E$.
  If assumption $\assH$ is satisfied, we can decide whether $\bx$ is an
  isolated root of $V(\bC)$ using 
$$O(n^4 \mu^4 + n^2 m \mu^3 + n E \mu^4) \subset (\mu\,E\,m)^{O(1)}$$ operations in~$\KK$.
\end{proposition}
Reference~\cite{BaHaPeSo09} gives an algorithm to compute the
dimension of $V(\bC)$ at $\bx$, but its complexity is not known to us, as
it relies on linear algebra with matrices of potentially large size
(not necessarily polynomial in $\mu,E,m$).
Instead, we use an adaptation of a prior result by
Mourrain~\cite{Mourrain97}, which allows us to control the size of the
matrices we handle. We only give detailed proofs for new ingredients
that are specific to our context, a key difference being the cost
analysis in the straight-line program model: Mourrain's original
result depends on the number of monomials appearing when we expand
$\bC$, which would be too high for the applications we will make of
this result.

Without loss of generality, we assume that $m\ge n$ (otherwise, $\bx$
cannot be an isolated solution). We also assume that $\bx=0$; this is
done by replacing $\bC$ by the polynomials $\bC(\bX+\bx)$, which have
complexity of evaluation $E'=E+n$.  The basis of our algorithm is the
following remark.

\begin{lemma}
  Let $I$ be the zero-dimensional ideal $\langle \bC \rangle +
  \m^{\mu+1}$, where $\m=\langle X_1,\dots,X_n\rangle$ is the maximal
  ideal at the origin. Then, if $0$ is isolated in $V(\bC)$ if and only
  if the multiplicity $\delta$ of $I$ at the origin is at most $\mu$.
\end{lemma}
\begin{proof}
  This follows from the following
  result~\cite[Theorem~A.1]{BaHaPeSo09}.  For $k \ge 1$, let $I_k$ be
  the zero-dimensional ideal $\langle \bC \rangle + \m^{k}$, and let
  $\nu_k$ be multiplicity of the origin with respect to this
  ideal. Then, the reference above proves that the sequence
  $(\nu_k)_{k \ge 1}$ is non-decreasing, and that $0$ is isolated in
  $V(\bC)$ if and only if there exists $k\ge 1$ such that
  $\nu_k=\nu_{k+1}$.
  \begin{itemize}
  \item If $0$ is isolated in $V(\bC)$, then by assumption $\assH$ 
    its multiplicity with respect to $\langle \bC\rangle$ is at most $\mu$,
    and its multiplicity $\delta$ with respect to $I$ cannot be larger.
  \item Otherwise, by the result above, $\nu_{k+1} > \nu_k$ holds for
    all $k \ge 1$, so that $\nu_k \ge k$ holds for all such $k$ (since
    $\nu_1=1$). In particular, the multiplicity $\delta$ of 
 $I$ at the origin, which is $\nu_{\mu+1}$, is at least $\mu+1$.
    \qedhere
  \end{itemize}
\end{proof}

Hence, we are left with deciding whether the multiplicity of the
$\m$-primary ideal $I$ is at most $\mu$. We do this by following and
slightly modifying Mourrain's algorithm for the computation of the
orthogonal $I^{\perp}$, that is, the set of $\KK$-linear forms
$\KK[\bX] \to \KK$ that vanish on $I$; this is a $\KK$-vector space
naturally identified with the dual of $\KK[\bX]/I$, so it has dimension
$\delta$, the multiplicity of $I$ at the origin (note that $I$ is $\m$-primary).

We do not need to give all details of the algorithm, let alone proof
of correctness; we just mention the key ingredients for the cost
analysis in our setting. A linear form $\beta: \KK[\bX] \to \KK$ that
vanishes on $I$ must vanish on all monomials, except a finite number
(since all monomials, except a finite number, belong to $I$); a
natural way to represent such a linear form would then be as the
finite generating series $\sum_{\balpha \in \N^n}
\beta(X_1^{\alpha_1}\cdots X_n^{\alpha_n}) d_1^{\alpha_1}\cdots
d_n^{\alpha_n}$, for some new variables $d_1,\dots,d_n$; however the
number of non-zero coefficients in such a sum cannot be bounded
polynomially in terms of $n,\mu$.

Hence, the algorithm uses another way to represent the elements in
$I^{\perp}$, by means of {\em multiplication matrices}. An important
feature of $I^{\perp}$ is that it admits the structure of a
$\KK[\bX]$-module: for $k$ in $\{1,\dots,n\}$ and $\beta$ in
$I^{\perp}$, the $\KK$-linear form $X_k \cdot \beta: f \mapsto
\beta(X_k f)$ is easily seen to still be in $I^{\perp}$.  In
particular, if $\bbeta=(\beta_1,\dots,\beta_m)$ is a $\KK$-basis of
$I^{\perp}$, then for all $k$ as above, and all $i$ in
$\{1,\dots,m\}$, $X_k \cdot \beta_i$ is a linear combination of
$\beta_1,\dots,\beta_m$. Mourrain's algorithm computes a
basis $\bbeta=(\beta_1,\dots,\beta_m)$ with the following features:
\begin{itemize}
\item for $i$ in $\{1,\dots,m\}$ and $k$ in $\{1,\dots,n\}$, we have
  $X_k \cdot \beta_i=\sum_{0 \le j < i} \lambda^{(k)}_{i,j} \beta_j$
  (hence $\lambda^{(k)}_{i,j}$ may be non-zero 
  only for $j<i$)
\item $\beta_1$ is the evaluation at $0$, $f \mapsto f(0)$
\item for $i$ in $\{2,\dots,m\}$, $\beta_i(1)=0$.
\end{itemize}
The following lemma shows that the coefficients $(\lambda^{(k)}_{i,j})$
are sufficient to evaluate  the linear forms $\beta_i$ at any $f$ in
$\KK[\bX]$. More precisely, knowing only their values for $j < i \le s$,
for any $s \le m$, allows us to evaluate $\beta_1,\dots,\beta_s$ at such an $f$.
The following lemma follows~\cite{Mourrain97} in its description
of the matrices $\bM_{k,s}$; the (rather straightforward) complexity analysis 
in the straight-line program model is new.
\begin{lemma}\label{lemma:evalbeta}
   Let $s$ be in $1,\dots,\delta$, and suppose that the coefficients
  $\lambda^{(k)}_{i,j}$ are known for $i=1,\dots,s$, $j=0,\dots,i-1$
  and $k=1,\dots,n$. Given a straight-line program $\Gamma$ of length
  $L$ that computes $\h=(h_1,\dots,h_R)$, one can compute
  $\beta_i(h_r)$, for all $i=1,\dots,s$ and $r=1,\dots,R$, using
  $O(s^3\,L)$ operations.
\end{lemma}
\begin{proof}
  By definition, for $h$ in $\KK[\bX]$ and $k=1,\dots,n$, the following equality
  holds:
$$
  \begin{bmatrix}
    \beta_1(X_k h)\\
    \vdots\\
    \beta_s(X_k h)
  \end{bmatrix}=
\bM_{k,s}
  \begin{bmatrix}
    \beta_1(h)\\
    \vdots\\
    \beta_s(h)
  \end{bmatrix},
\quad\text{with}\quad
\bM_{k,s}= \begin{bmatrix}
    \lambda^{(k)}_{1,1} & \cdots & \lambda^{(k)}_{s,1}\\
    \vdots && \vdots \\
    \lambda^{(k)}_{1,s} & \cdots & \lambda^{(k)}_{s,s}
  \end{bmatrix}.
$$ 
 Remark that the matrices $\bM_{k,s}$ all commute. Indeed, 
for any $k,k'$ in $\{1,\dots,n\}$, and $h$ as above, the relation above implies
that 
$$
\Delta_{k,k',s}
  \begin{bmatrix}
    \beta_1(h)\\
    \vdots\\
    \beta_s(h)
  \end{bmatrix} =
  \begin{bmatrix}
0\\ \vdots \\ 0 
  \end{bmatrix},
$$
where $\Delta_{k,k',s} = \bM_{k,s}\bM_{k',s}-\bM_{k',s}\bM_{k,s}.$ Because 
the linear forms $\beta_1,\dots,\beta_s$ are linearly independent, this implies
that all rows of $\Delta_{k,k',s}$ must be zero, as claimed.
We then deduce that for any polynomial $h$ in $\KK[\bX]$, we have
the equality
$$  \begin{bmatrix}
    \beta_1(h)\\
    \vdots\\
    \beta_s(h)
  \end{bmatrix} =
h(\bM_{1,s},\dots,\bM_{n,s})   \begin{bmatrix}
    \beta_1(1)\\
    \vdots\\
    \beta_s(1)
  \end{bmatrix} $$ On the other hand, our assumptions imply that the
  sequence $(\beta_1(1),\dots,\beta_s(1))$ is simply $(1,0,\dots,0)$.
  To prove the proposition, it is then enough to note that the evaluations \sloppy
  $h_1(\bM_{1,s},\dots,\bM_{n,s}),\dots,h_R(\bM_{1,s},\dots,\bM_{n,s})$
  can be computed using the straight-line program doing
  $O(s^3\,L)$ operations.
\end{proof}

Mourrain's alorithm proceeds in an iterative manner, starting from
$\bbeta^{(1)}=(\beta_{1})$ (and setting $e_1=1$), and computing
successively $\bbeta^{(2)}=(\beta_{e_1+1},\dots,\beta_{e_2})$,
$\bbeta^{(3)}=(\beta_{e_2+1},\dots,\beta_{e_3})$, \dots for some
integers $e_1 \le e_2 \le e_3 \dots$ Mourrain's algorithm stops when
$e_{\ell+1}=e_{\ell}$, in which case $\beta_1,\dots,\beta_{e_\ell}$ is
a $\KK$-basis of $I^\perp$, and $e_\ell=\delta$. In our case, we
are not interested in computing this multiplicity, but only in
deciding whether it is less than or equal to the parameter $\mu$. We do it as follows: assume that we have
computed $\bbeta^{(1)},\bbeta^{(2)},\dots,\bbeta^{(\ell)}$, together
with the corresponding integers $e_1,e_2,\dots,e_\ell$, with $e_1 <
\cdots < e_\ell \le \mu$. We compute $\bbeta^{(\ell+1)}$ and $e_{\ell+1}$,
and continue according to the following:
\begin{itemize}
\item if $e_{\ell+1}=e_{\ell}$, we conclude that the multiplicity
  $\delta$ of $I$ at the origin is $e_\ell \le \mu$; we stop the
  algorithm;
\item if $e_{\ell+1} > \mu$, we conclude that this multiplicity is greater 
  than $\mu$; we stop the algorithm;
\item else, when $e_\ell < e_{\ell+1} \le \mu$, we do another loop.
\end{itemize}
Because the $e_\ell$'s are an increasing sequence of integers, they
satisfy $e_\ell \ge \ell$; hence, every time we enter the loop above we
have $\ell \le \mu$. To finish the analysis of the algorithm, it
remains to explain how to compute $\bbeta^{(\ell+1)}$ from
$(\bbeta^{(1)},\bbeta^{(2)},\dots,\bbeta^{(\ell)})=(\beta_{1},\dots,\beta_{e_\ell})$.

As per our description above, at any step of the algorithm,
$\beta_{1},\dots,\beta_{e_\ell}$ are represented by means of the
coefficients $\lambda^{(k)}_{i,j}$, for $0 \le j < i \le e_{\ell}$ and
$1 \le k \le n$.  At step $\ell$, Mourrain's algorithm solves a homogeneous linear system
$S_\ell$ with $n(n-1) e_\ell/2+m'$ equations and $n e_\ell$ unknowns,
where $m'$ is the number of generators of the ideal $I= \langle \bC
\rangle + \m^{\mu+1}$. Remark that $m'$ is not polynomial in $\mu$ 
and $n$, so the size of $S_\ell$ is {\em a priori} too large to 
fit our cost bound; we will explain below how to resolve this issue.

The nullspace dimension of this linear system gives us the cardinality
$e_{\ell+1}-e_{\ell}$ of $\bbeta^{(\ell+1)}$. Similarly, the coordinates of
the $e_{\ell+1}-e_{\ell}$ vectors in a nullspace basis are precisely
the coefficients $\lambda^{(k)}_{i,j}$ for
$i=e_{\ell}+1,\dots,e_{\ell+1}$, $j=1,\dots,e_\ell$ and $k=1,\dots,n$
(we have $\lambda^{(k)}_{i,j}=0$ for $j=e_{\ell}+1,\dots,i-1$). For
all $\ell \ge 2$, all linear forms $\beta$ in $\bbeta^{(\ell)}$ are
such that for all $k$ in $\{1,\dots,n\}$, $X_k \cdot \beta$ belongs to
the span of $\bbeta^{(1)},\dots,\bbeta^{(\ell-1)}$; in particular, a
quick induction shows that all linear forms in
$\bbeta^{(1)},\dots,\bbeta^{(\ell)}$ vanish on all monomials of degree
at least $\ell$.

There remains the question of setting up the system $S_\ell$. For $k$
in $\{1,\dots,n\}$ and a $\KK$-linear form $\beta$, we denote by
$X_k^{-1} \cdot \beta$ the $\KK$-linear form defined as follows:
\begin{itemize}
\item $(X_k^{-1} \cdot \beta)(X_k f) = \beta(f)$ for all $f$ in $\KK[\bX]$,
\item $(X_k^{-1} \cdot \beta)(f)=0$ if $f\in \KK[\bX]$ does not depend on $X_k$.
\end{itemize}
In other words, $(X_k^{-1} \cdot \beta)(f)=\beta(\delta_k(f))$ holds
for all $f$, where $\delta_k:\KK[\bX] \to \KK[\bX]$ is the $k$th divided
difference operator
$$f\mapsto \frac
{f(X_1,\dots,X_n)-f(X_1,\dots,X_{k-1},0,X_{k+1},\dots,X_n)}{X_k}.$$
One verifies that, as the notation suggests, $X_k \cdot (X_k^{-1}
\cdot \beta)$ is equal to $\beta$. This being said, we can then
describe what the entries of $S_\ell$ are:
\begin{itemize}
\item the first $n(n-1) e_\ell/2$ equations involve only the coefficients 
  $\lambda^{(k)}_{i,j}$ previously computed (we refer to~\cite[Section~4.4]{Mourrain97} for details of how exactly 
these entries are distributed in $S_\ell$, as we do not need such details here).
\item each of the other $m'$ equations has coefficient vector
$$v_f = \big (\
 (X_k^{-1} \cdot \beta_1)(f(X_1,\dots,X_k,0,\dots,0)),\dots,\ (X_k^{-1} \cdot \beta_{e_\ell})(f(X_1,\dots,X_k,0,\dots,0))\
\big )_{1 \le k \le n},$$
where $f$ is a generator of $I=\langle \bC \rangle +\m^{\mu+1}$.
\end{itemize}
We claim that only those equations corresponding to generators
$c_1,\dots,c_m$ of the input system $\bC$ are useful, as all others are identically
zero.

We pointed out above that any linear form $\beta_i$ in
$\beta_1,\dots,\beta_{e_\ell}$ vanishes on all monomials of degree at
least $\ell$. Since we saw that we must have $\ell \le \mu$, all
$\beta_i$ as above vanish on monomials of degree $\mu$; this implies
that $X_k^{-1}\cdot \beta_i$ vanishes on all monomials of degree
$\mu+1$. The generators $f$ of $\m^{\mu+1}$ have degree $\mu+1$, and
for any such $f$, $f(X_1,\dots,X_k,0,\dots,0)$ is either zero, or of
degree $\mu+1$ as well. Hence, for any $k$, $\beta_i$ in
$\beta_1,\dots,\beta_{e_\ell}$ and $f$ as above, $(X_k^{-1} \cdot
\beta_i)(f(X_1,\dots,X_k,0,\dots,0))$ vanishes. This implies that the
vector $v_f$ is identically zero for such an $f$, and that the
corresponding equation can be discarded.

Altogether, as claimed above, we see that we have to compute the
values
$$(X_k^{-1} \cdot \beta_i)(c_j(X_1,\dots,X_k,0,\dots,0)),$$ for
$k=1,\dots,n$, $i=1,\dots,e_\ell$ and $j=1,\dots,m$.  Fixing $k$, we
let $\bC_k = (c_{j,k})_{1 \le j \le m}$, where $c_{j,k}$ is the
polynomial $c_j(X_1,\dots,X_k,0,\dots,0)$; note that the system $\bC_k$
can be computed by a straight-line program of length $E'=E+n$. Then,
applying the following lemma with $s=e_\ell \le \mu$ and $\h = \bC_k$,
we deduce that the values $(X_k^{-1} \cdot
\beta_i)(c_j(X_1,\dots,X_k,0,\dots,0))$, for $k$ fixed, can be
computed in time $O(\mu^3 (E+n))$.


\begin{lemma}
  Let $s$ be in $1,\dots,\delta$, and suppose that the coefficients
  $\lambda^{(k)}_{i,j}$ are known for $i=1,\dots,s$, $j=0,\dots,i-1$
  and $k=1,\dots,n$. Given a straight-line program $\Gamma$ of length
  $L$ that computes $\h=(h_1,\dots,h_R)$ and given $k$ in
  $\{1,\dots,n\}$, one can compute $(X_k^{-1}\cdot \beta_i)(h_r)$, for
  all $i=1,\dots,s$ and $r=1,\dots,R$, using $O(s^3 (L+n))$
  operations in $\KK$.
\end{lemma}
\begin{proof}
  In view of the formula $(X_k^{-1} \cdot
  \beta)(f)=\beta(\delta_k(f))$, and of Lemma~\ref{lemma:evalbeta}, it is
  enough to prove the existence of a straight-line program of length
  $O(L+n)$ that computes $(\delta_k(h_1),\dots,\delta_k(h_R))$.

  To do this, we replace all polynomials
  $\gamma_{-n+1},\dots,\gamma_L$ computed by $\Gamma$ by terms
  $\lambda_{-n+1},\dots,\lambda_L$ and $\mu_{-n+1},\dots,\mu_L$, with
  $\lambda_\ell=\gamma_\ell(X_1,\dots,X_{k-1},0,X_{k+1},\dots,X_n)$
  and $\mu_\ell$ in $\KK[\bX]$ such that $\gamma_\ell= \lambda_\ell+X_k
  \mu_\ell$ holds for all $\ell$, so that in particular
  $\mu_\ell=\delta_k(\gamma_\ell)$.  To compute $\lambda_\ell$ and
  $\mu_\ell$, assuming all previous $\lambda_{\ell'}$ and
  $\mu_{\ell'}$ are known, we proceed as follows:
  \begin{itemize}
  \item if $\gamma_\ell=X_k$, we set $\lambda_\ell=0$ and $\mu_\ell=1$;
  \item if $\gamma_\ell=X_{k'}$, with $k' \ne k$, we set $\lambda_\ell=X_{k'}$ and $\mu_\ell=0$;
  \item if $\gamma_\ell =c_\ell$, with $c_\ell \in \KK$,
    then we set $\lambda_\ell=c_\ell$ and  $\mu_\ell=0$;
  \item if $\gamma_\ell = \gamma_{a_\ell} \pm \gamma_{b_\ell}$,
    for some indices $a_\ell,b_\ell < \ell$, 
    then we set $\lambda_\ell=\lambda_{a_\ell}\pm\lambda_{b_\ell}$
    and $\mu_\ell=\mu_{a_\ell}\pm\mu_{b_\ell}$;
\item if $\gamma_\ell = \gamma_{a_\ell} \gamma_{b_\ell}$,
      for some indices $a_\ell,b_\ell < \ell$,
    then we set $\lambda_\ell=\lambda_{a_\ell} \lambda_{b_\ell}$
    and $$\mu_\ell=
\lambda_{a_\ell} \mu_{b_\ell}
+
\mu_{a_\ell} \lambda_{b_\ell}
+
X_k\mu_{a_\ell} \mu_{b_\ell}.$$
\end{itemize}
One verifies that in all cases, the relation $\gamma_\ell=
\lambda_\ell+X_k \mu_\ell$ still holds. Since the previous
construction allows us to compute $\lambda_\ell$ and $\mu_\ell$ in
$O(1)$ operations from the knowledge of all previous $\lambda_{\ell'}$
and $\mu_{\ell'}$, we deduce that all $\lambda_\ell$ and $\mu_\ell$,
for $\ell=-n+1,\dots,L$, can be computed by a straight-line program of
length $O(L+n)$.
\end{proof}

Taking all values of $k$ into account, we see that we can compute all
entries we need to set up the linear system $S_\ell$ using $O(\mu^3
n(E+n))$ operations in $\KK$. After discarding the useless equations
described above, the numbers of equations and unknowns in the system
$S_\ell$ are respectively at most $n^2 \mu+m$ and $n \mu$; this
implies that we can find a nullspace basis of it in time $O(n^4 \mu^3
+ n^2 m \mu^2)$. Altogether, the time spent to find
$\bbeta^{(\ell+1)}$ from
$(\bbeta^{(1)},\bbeta^{(2)},\dots,\bbeta^{(\ell)})=(\beta_{1},\dots,\beta_{e_\ell})$
is $O(n^4 \mu^3 + n^2 m \mu^2 + n E \mu^3)$.

Since we saw that we do at most $\mu$ such loops, the cumulated time
is $O(n^4 \mu^4 + n^2 m \mu^3 + n E \mu^4)$, and
Proposition~\ref{prop:testisolated} is proved.


%%%%%%%%%%%%%%%%%%%%%%%%%%%%%%%%%%%%%%%%%%%%%%%%%%%%%%%%%%%%
%%%%%%%%%%%%%%%%%%%%%%%%%%%%%%%%%%%%%%%%%%%%%%%%%%%%%%%%%%%%
%%%%%%%%%%%%%%%%%%%%%%%%%%%%%%%%%%%%%%%%%%%%%%%%%%%%%%%%%%%%

\section{Symbolic homotopies}\label{sec:homotopy}

In this section, we work over a field $\KK$, using $n$ variables
$\bX=(X_1,\dots,X_n)$. Given polynomials $\bC=(c_1,\dots,c_m)$ in
$\KK[\bX]^m$, we give an algorithm to compute a zero-dimensional
parametrization of the isolated points of $V(\bC)$, assuming the
existing of a suitable {\em homotopy deformation} of $\bC$. We assume
$m\ge n$, otherwise no isolated points exist in $V(\bC)$.

Let $T$ be a new variable and consider polynomials
$\bB=(b_1,\dots,b_m)$ in $\KK[T,\bX]$; for $\tau$ in $\KKbar$,
we write $\bB_\tau=(b_{\tau,1},\dots,b_{\tau,m})=\bB(\tau,\bX)\subset
\KKbar[\bX]$ and we assume that $\bB$ is such that $\bB_1=\bC$.
Define further the ideal $J=\langle \bB \rangle \subset \KKbar[T,\bX]$ and
consider the folllowing assumptions.
\begin{description}[leftmargin=*]
\item[$\assA_1.$] Any irreducible component of $V(J) \subset
  \KKbar{}^{n+1}$ has dimension at least one.
\item[$\assA_2.$] For any maximal ideal $\m \subset\KKbar[T,\bX]$, if the
  localization $J_\m \subset \KKbar[T,\bX]_\m$ has height $n$, then it is
  unmixed (that is, all associated primes have height $n$).
\end{description}
An obvious example where such properties hold is when $m=n$. Then,
$\assA_1$ is Krull's theorem, and $\assA_2$ is Macaulay's unmixedness
theorem in the Cohen-Macaulay ring
$\KKbar[T,\bX]_\m$~\cite[Corollary~18.14]{Eisenbud95}. More generally,
these properties hold when $\bB$ is the sequence of $p$-minors of a $p
\times q$ matrix with entries in $\KK[T,\bX]$, with $n=q-p+1$; we
discuss this, and a slightly more general situation, in Section~\ref{sec:check}.  

For $\tau$ in $\KKbar$, we further denote by $\assG(\tau)$ the
following three properties.
\begin{description}[leftmargin=*]
\item[$\assG_1(\tau).$] For $k=1,\dots,m$,
  $\deg_\bX(b_k)=\deg_\bX(b_{\tau,k})$.
\item[$\assG_2(\tau).$] The only common solution to
  $b_{\tau,1}^H(0,\bX)=\cdots=b_{\tau,m}^H(0,\bX)=0$ is
  $(0,\dots,0)\in\KKbar{}^n$, where for $k=1,\dots,m$, $b_{\tau,k}^H$ is
  the polynomial in $\KKbar[X_0,\bX]$ obtained by homogenizing
  $b_{\tau,k}$ using a new variable $X_0$. In particular, $V(\bB_\tau)
  \subset \KKbar{}^n$ is finite.
\item[$\assG_3(\tau).$] The ideal $\langle \bB_\tau \rangle$ is
  radical in $\KKbar[\bX]$.
\end{description}

The first result in this section is the following.
\begin{proposition}\label{prop:degree_fiber}
  Suppose that $\assA_1$ and $\assA_2$ hold. Then, there exists an
  integer $c$ such that for all $\tau$ in $\KKbar$, the sum of the
  multiplicities of the isolated solutions of $\bB_\tau$ is at most
  $c$, and is equal to $c$ if $\assG(\tau)$ holds.
\end{proposition}

We next give our algorithm to compute the isolated solutions of the
polynomial system $\bC=(c_1,\dots,c_m)$. In order to control the cost
of the algorithm, we introduce the following assumptions.
\begin{description}[leftmargin=*]
\item[${\assD}_1$.] We are given $\tau$ in $\KK$ such that $\assG(\tau)$
  holds; without loss of generality, we assume that $\tau=0$. We also
  suppose that we know a description of $V(\bB_{0})$ by means of a
  zero-dimensional parametrization with
  coefficients in $\KK$. The linear form $\lambda$ needs to satisfy
  some genericity requirements, that are described in
  Subsection~\ref{proof:prop2}.
\item[${\assD}_2$.] We know an integer $e$ such that the union of the
  one-dimensional components of $V(J)$ in $\KKbar{}^{n+1}$ has degree
  at most $e$ (note that $e \ge c$, by Lemma~\ref{lemma:vPi}).
\item[${\assD}_3$.] We can compute $\bB$ using a straight-line program
  of length $E$.
\end{description}
Then, the second main result in this section is the following.
\begin{proposition}\label{prop:compute_isolated}
  Under the previous assumptions, we can compute a zero-dimensional
  parametrization of the isolated points of 
 of $V(\bC)$ using
$$\softO(c^5 m n^2  + c(e+c^5 n)(E + n^3)) \subset (e\,E\,m)^{O(1)}$$ operations in~$\KK$.
\end{proposition}

%%%%%%%%%%%%%%%%%%%%%%%%%%%%%%%%%%%%%%%%%%%%%%%%%%%%%%%%%%%%

\subsection{Proof of Proposition~\ref{prop:degree_fiber}}

This subsection is devoted to prove
Proposition~\ref{prop:degree_fiber}. In the course of the proof, we
will give a precise characterization of the integer $c$ mentioned in
the proposition, although the statement given in the proposition will
actually be enough for our further purposes. {\em In all the rest of
  this subsection, we assume that $\assA_1$ and $\assA_2$ hold.}

Consider an irredundant primary decomposition of $J$ in
$\KKbar[T,\bX]$, of the form $J=Q_1 \cap \cdots
\cap Q_r$, and let $P_1,\dots,P_r$ be the associated primes, that is,
the respective radicals of $Q_1,\dots,Q_r$. We assume that
$P_1,\dots,P_s$ are the minimal primes, for some $s \le r$, so that
$V(P_1),\dots,V(P_s)$ are the (absolutely) irreducible components of
$V(J)\subset \KKbar{}^{n+1}$. By $\assA_1$, these irreducible
components all have dimension at least one. Refining further, we
assume that $t \le s$ is such that $V(P_1),\dots,V(P_t)$ are the
 irreducible components of $V(J)$ of dimension one whose
image by $\pi_T: (\tau,x_1,\dots,x_n) \mapsto \tau$ is dense in
$\KKbar$.

\begin{lemma}\label{lemma:vPi}
  Let $\tau$ be in $\KKbar$ and let $\bx \in \KKbar{}^n$ be an isolated
  solution of the system $\bB_\tau$. Then, $(\tau,\bx)$ belongs to $V(P_i)$
  for at least one index $i$ in $\{1,\dots,t\}$, and does not belong
  to $V(P_i)$ for any index $i$ in $\{t+1,\dots,r\}$.
\end{lemma}
\begin{proof}
  Because $(\tau,\bx)$ cancels $\bB$, it belongs at least to one of
  $V(P_1),\dots,V(P_r)$. It remains to rule out the possibility that
  $(\tau,\bx)$ belongs to $V(P_i)$ for some index $i$ in
  $\{t+1,\dots,r\}$.

  We first deal with indices $i$ in $\{t+1,\dots,s\}$. These are those
  primary components with minimal associated primes $P_i$ that either
  have dimension at least two, or have dimension one but whose image
  by $\pi$ is a single point. In both cases, all irreducible
  components of the intersection $V(P_i)\cap V(T-\tau)$ have dimension
  at least one. Since $\bx$ is isolated in $V(\bB_\tau)$, $(\tau,\bx)$ is
  isolated in $V(\bB)\cap V(T-\tau)$, so it cannot belong to
  $V(P_i)\cap V(T-\tau)$ for any $i$ in $\{t+1,\dots,s\}$.
  
  We conclude by proving that $(\tau,\bx)$ does not belong to $V(P_i)$,
  for any of the embedded primes $P_{s+1},\dots,P_r$. We proceed by
  contradiction, assuming for definiteness that $(\tau,\bx)$ belongs to
  $V(P_{s+1})$. Because $P_{s+1}$ is an embedded prime, $V(P_{s+1})$
  is contained in (at least) one of $V(P_1),\dots,V(P_s)$. In view of
  the previous paragraph, it cannot be one of
  $V(P_{t+1}),\dots,V(P_s)$.  Now, all of $V(P_1),\dots,V(P_t)$ have
  dimension one, so $V(P_{s+1})$ has dimension zero (so it is the point $\{(\tau,\bx)\}$). For the same
  reason, if $(\tau,\bx)$ belonged to another $V(P_i)$, for some $i >
  s+1$, $V(P_i)$ would also be zero-dimensional, and thus equal to $\{(\tau,\bx)\}$; as a result, $V(P_i)$
  would be equal to $V(P_{s+1})$, and this would contradict the
  irredundancy of our decomposition.
  
  To summarize, $(\tau,\bx)$ belongs to $V(P_{s+1})$, together with
  $V(P_i)$ for some indices $P_i$ in $\{1,\dots,t\}$ (say
  $P_1,\dots,P_u$, up to reordering, for some $u \ge 1$), and avoids
  all other associated primes.  Let us localize the decomposition
  $J=Q_1 \cap \cdots \cap Q_r$ at
  $P_{s+1}$. By~\cite[Proposition~4.9]{AtMc},
  $J_{P_{s+1}}={Q_1}_{P_{s+1}} \cap \cdots \cap {Q_u}_{P_{s+1}}\cap
  {Q_{s+1}}_{P_{s+1}}$ is an irredundant primary decomposition of
  $J_{P_{s+1}}$ in $\KKbar[T,\bX]_{P_{s+1}}$; the minimal primes are
  ${P_1}_{P_{s+1}},\dots,{P_u}_{P_{s+1}}$.

  By Corollary~4 p.24 in~\cite{Matsumura86}, for any prime
  ${P_i}_{P_{s+1}}$, $i=1,\dots,u$ or $i=s+1$, the localization of
  $\KKbar[T,\bX]_{P_{s+1}}$ at ${P_i}_{P_{s+1}}$ is equal to
  $\KKbar[T,\bX]_{P_{i}}$. In particular, the height of ${P_i}_{P_{s+1}}$
  in $\KKbar[T,\bX]_{P_{s+1}}$ is equal to that of $P_i$ in
  $\KKbar[T,\bX]_{P_{i}}$, that is, $n$ if $i=1,\dots,u$, since then
  $V(P_i)$ has dimension $1$, or $n+1$ if $i=s+1$. Since $u \ge 1$,
  this proves that $J_{P_{s+1}}$ has height $n$. As a result, $\assA_2$ implies that $J_{P_{s+1}}$ is unmixed, a contradiction.
\end{proof}

Let us write $J=J' \cap J''$, with $J'=Q_1 \cap \cdots \cap Q_t$ and
$J''=Q_{t+1} \cap \cdots \cap Q_r$. For $\tau$ in $\KKbar$, we denote
by $J_\tau \subset \KKbar[T,\bX]$ the ideal $J + \langle T-\tau \rangle$,
and similarly for $J'_\tau$ and $ J''_\tau$.

\begin{lemma}\label{lemma:JJprime}
  Let $\tau$ and $\bx$ be as in Lemma~\ref{lemma:vPi}. Then, the
  multiplicities of the ideals $J_\tau$ and $J'_\tau$ at $(\tau,\bx)$
  are the same.
\end{lemma}
\begin{proof}
  Without loss of generality, assume that $\tau=0 \in \KKbar$ and
  $\bx=0 \in \KKbar{}^n$. We start from the equality $J=J' \cap J''$,
  which holds in $\KKbar[T,\bX]$, and we see it in $\KKbar[[T,\bX]]$.  The
  previous lemma implies that there exists a polynomial in $J''$ that
  does not vanish at $(\tau,\bx)=0 \in \KKbar{}^{n+1}$.  This polynomial
  is a unit in $\KKbar[[T,\bX]]$, which implies that the extension of
  $J''$ in $\KKbar[[T,\bX]]$ is the trivial ideal $\langle 1 \rangle$, and
  finally that the equality of extended ideals $J=J'$ holds in
  $\KKbar[[T,\bX]]$. This implies the equality $J+\langle T \rangle
  =J'+\langle T \rangle $ in $\KKbar[[T,\bX]]$, and the conclusion
  follows.
\end{proof}

Our goal is now to give a bound on the sum of the multiplicites of
$\bB_\tau$ at all its isolated roots, for any $\tau$ in $\KKbar$.  To
achieve this, we introduce $\frak{J}$, the extension of $J$ in
$\KKbar(T)[\bX]$, and similarly $\frak{J}'$ and ${\frak J}''$.

\begin{lemma}
  The ideal $\frak{J}'$ has dimension zero and $V(\frak{J}') \subset
  \overline{\KK(T)}{}^n$ is the set of isolated solutions of
  $V(\frak{J}) \subset \overline{\KK(T)}{}^n$.
\end{lemma}
\begin{proof}
 From the equality $J=J' \cap J''$ and Corollary~3.4 in~\cite{AtMc},
 we deduce that $\frak{J}=\frak{J'} \cap \frak{J''}$; the properties
 of $J'$ (the irreducible components of $V(J')$ are precisely those
 irreducible components of $V(J)$ that have dimension one and with a
 dense image by $\pi_T$) imply our claim.
\end{proof}


Let us write $c=\dim_{\KKbar(T)}(\KKbar(T)[\bX]/{\frak J}')$.  The
following lemma relates this quantity to the multiplicities of the
solutions in any fiber $\bB_\tau$. This proves the first statement
in Proposition~\ref{prop:degree_fiber}.

\begin{lemma}\label{lemma:19}
  Let $\tau$ be in $\KKbar$. The sum of the multiplicities of the
  isolated solutions of $\bB_\tau$ is at most equal to $c$.
\end{lemma}
\begin{proof}
  The sum in the lemma is also the sum of the multiplicities of the
  ideal $J_\tau$ at all $(\tau,\bx)$, for $\bx$ an isolated solution of
  $\bB_\tau$.  By Lemma~\ref{lemma:JJprime}, this is also the sum of
  the multiplicities of $J'_\tau$ at all $(\tau,\bx)$, for $\bx$ an
  isolated solution of $\bB_\tau$. We prove below that the sum of the
  multiplicities of $J'_\tau$ at all $(\tau,\bx)$, for $\bx$ such that
  $(\tau,\bx)$ cancels $J'_\tau$, is at most $c$; this will be enough
  to conclude (for any isolated solution $\bx$ of $\bB_\tau$,
  $(\tau,\bx)$ is a root of $J'_\tau$, though the converse may not be
  true).
  
  Let $m_1,\dots,m_\mu$ be monomials that form a $\KKbar$-basis of
  $\KKbar[T,\bX]/J'_\tau$; since $T-\tau$ is in $J'_\tau$, these
  monomials can be assumed not to involve $T$.  We will prove that
  they are still $\KKbar(T)$-linearly independent in
  $\KKbar(T)[\bX]/{\frak J}'$; this will imply that $\mu \le c$,
  and finish the proof of the first statement.
  
  Suppose that there exists a linear combination $A_1 m_1 + \cdots +
  A_\mu m_\mu$ in ${\frak J}'$, with all $A_i$'s in $\KKbar(T)$, not
  all of them zero. Thus, we have an equality $a_1/d_1\, m_1 + \cdots
  + a_\mu/d_\mu\, m_\mu = a/d$, with $a_1,\dots,a_\mu$ and
  $d,d_1,\dots,d_\mu$ in $\KKbar[T]$ and $a$ in the ideal
  $J'$. Clearing denominators, we obtain a relation of the form $b_1
  m_1 +\cdots+ b_\mu m_\mu \in J'$, with not all $b_i$'s zero. Let
  $(T-\tau)^e$ be the highest power of $T-\tau$ that divides all
  $b_i$'s (this is well-defined, since not all $b_i$'s vanish) so that
  we can rewrite the above as $(T-\tau)^e (c_1 m_1 +\cdots+ c_\mu
  m_\mu) \in J'$, with $c_i=b_i/(T-\tau)^e \in \KKbar[T]$ for all $i$.
  In particular, our definition of $e$ implies that the values
  $c_i(\tau)$ are not all zero.

  Recall that the ideal $J'$ has the form $J'=Q_1 \cap \cdots \cap
  Q_t$. For $i=1,\dots,t$, since $Q_i$ is primary, the membership
  equality $(T-\tau)^e (c_1 m_1 +\cdots +c_\mu m_\mu) \in J'$ implies
  that either $c_1 m_1 +\cdots +c_\mu m_\mu$ or some power
  $(T-\tau)^{ef}$, for some $f > 0$, is in $Q_i$. Since $Q_i$ does not
  contain non-zero polynomials in $\KKbar[T]$, $c_1 m_1 +\cdots+ c_\mu
  m_\mu$ belongs to all $Q_i$'s, that is, to $J'$. We can then
  evaluate this relation at $T=\tau$. We saw that the values
  $c_i(\tau)$ do not all vanish on the left, which is a contradiction
  with the independence of the monomials $m_1,\dots,m_\mu$ modulo
  $J'_\tau$.
\end{proof}


We now take $\tau$ in $\KKbar$ and we discuss the geometry of $V(J)$
near $\tau$; without loss of generality, we suppose that $\tau=0$.
The field of Puiseux series $\KKbar\langle\langle T \rangle\rangle$
contains an algebraic closure of $\KKbar(T)$; we thus let
$\Phi_1,\dots,\Phi_{c'}$ be the points of $V(\mathfrak{J}')$, with
coordinates taken in $\KKbar\langle\langle T \rangle\rangle$. In
particular, we see that $c' \le c$; we prove below that we actually
have $c'=c$.

Any Puiseux series $\varphi$ in $\KKbar\langle\langle T
\rangle\rangle$ admits a well-defined {\em valuation} $\nu(\varphi)$,
which is the smallest exponent that appears in its expansion; the
valuation $\nu(\Phi)$, for a vector $\Phi=(\varphi_1,\dots,\varphi_s)$
with entries in $\KKbar\langle\langle T \rangle\rangle$, is the
minimum of the valuations of its exponents. We say that $\Phi$ is {\em
  bounded} if it has non-negative valuation; in this case,
$\lim_0(\Phi)$ is defined as the vector
$(\lim_0(\varphi_1),\dots,\lim_0(\varphi_s))$, with
$\lim_0(\varphi_i)={\rm coeff}(\varphi_i,T^0)$ for all $i$.

Without loss of generality, we assume that
$\Phi_1,\dots,\Phi_\kappa$ are bounded, and
$\Phi_{\kappa+1},\dots,\Phi_{c'}$ are not, for some $\kappa$ in
$\{0,\dots,c'\}$, and we define $\varphi_1,\dots,\varphi_\kappa$ by
$\varphi_i=\lim_0(\Phi_i)\in\KKbar{}^n$ for
$i=1,\dots,\kappa$.

\begin{lemma}\label{lemma:Z1}
  The equality $V(J +\langle T \rangle)=\{\varphi_i \mid i=1,\dots,\kappa\}$ holds.
\end{lemma}
\begin{proof}
  Let $S_1,\dots,S_R$ be generators of the ideal $J$ in
  $\KK[T,\bX]$, so that these polynomials also generate
  $\mathfrak{J}$ in $\KK(T)[\bX]$; then, the polynomials
  $s_i=S_i(0,\bX) \in \KK[\bX]$, for $i=1,\dots,R$, are such that
  $J+\langle T\rangle = \langle T,S_1,\dots,S_R \rangle$.  Consider
  $i \le \kappa$, and the corresponding vector of Puiseux series
  $\Phi_i$. We know that for $j=1,\dots,R$, we have $S_j(\Phi_i)=0$.
  Since all elements involved have non-negative valuation, we can take
  the coefficient of degree $0$ in $T$ in this equality and deduce
  $s_j(\varphi_i)=0$, as claimed. Hence, each $\varphi_i$, for $i \le
  \kappa$, is in $V(J + \langle T \rangle)$.

  Conversely, take indeterminates $T_1,\dots,T_n$, and let $\LL$ be
  the algebraic closure of the field $\KKbar(T_1,\dots,T_n)$; let
  $W\subset{\LL}{}^{n+1}$ be the zero-set of the ideal $J\cdot
  \LL[T,\bX]$ and consider the projection $W \to {\LL}{}^2$ defined
  by $(\tau,x_1,\dots,x_n)\mapsto (\tau,T_1 x_1 + \cdots + T_n x_n)$. The
  Zariski closure $S$ of the image of this mapping is a
  hypersurface. 
  Since
  the ideal $J$ is generated by polynomials with coefficients in $\KK$,
  one deduces that $S$ admits a squarefree defining equation in
  $\KK(T_1,\dots,T_n)[T,T_0]$.

  Consider such a polynomial, say $C$, and assume without loss of
  generality that $C$ belongs to 
  $\KK[T_1,\dots,T_n][T,T_0]$. Because $J$ admits no irreducible
  component lying above $T=\tau$, for any $\tau$ in $\KKbar$, $C$
  admits no factor in $\KK[T]$; thus, $C(0,T_0)$ is non-zero.

  Let $\ell \in \KK[T_1,\dots,T_n,T]$ be the leading coefficient of
  $C$ with respect to $T_0$. Proposition~1 in~\cite{Schost03} proves
  that $C/\ell$, seen in $\KK(T_1,\dots,T_n,T)[T_0] \subset
  \LL(T)[T_0]$, is the minimal polynomial of $T_1 X_1 + \cdots +
  T_n X_n$ in $\LL(T)[\bX]/J\cdot \LL(T)[\bX]$. The latter ideal
  is also the extension of $\mathfrak{J}$ to $\LL(T)[\bX]$, 
  so $C/\ell$ factors as
  $$\frac C\ell = \prod_{1\le i \le c'}(T_0-T_1 \Phi_{i,1} - \cdots - T_n \Phi_{i,n})$$
  in $\LL\langle\langle T \rangle\rangle[T_0]$.
  This gives the equality 
  $$C =\ell \prod_{1\le i \le  c'}(T_0-T_1 \Phi_{i,1} - \cdots - T_n
  \Phi_{i,n})$$ over $\KKbar\langle\langle T \rangle\rangle[T_1,\dots,T_n,T_0]$. 

  Let us extend the valuation $\nu$ on $\KKbar\langle\langle T\rangle\rangle$
to $\KKbar\langle\langle T \rangle\rangle[T_1,\dots,T_n,T_0]$ in the
  direct manner, by setting $\nu(\sum_\alpha f_\alpha T_0^{\alpha_0}
  \cdots T_n^{\alpha_n}) = \min_\alpha \nu(f_\alpha)$. The fact that
  $C$ has no factor in $\KK[T]$ implies that
  $\nu(C)=0$. Using Gauss' Lemma, we see that the valuation of the
  right-hand side is $\nu(\ell) + \sum_{\kappa < i \le c}\mu_i$, with $\mu_i= \nu(\Phi_i)$ for all $i$;
  note that $\mu_i < 0$ for $i > \kappa$. Thus, we can
  rewrite
  $$C =\left ({T}^{-\nu(\ell)} \ell\right ) 
  \prod_{1 \le i \le \kappa}(T_0-T_1 \Phi_{i,1} - \cdots - T_n  \Phi_{i,n} )
  \prod_{\kappa < i \le c'} ({T}^{-\mu_i}T_0-{T}^{-\mu_i}T_1 \Phi_{i,1} - \cdots - {T}^{-\mu_i}T_n  \Phi_{i,n} ),$$
  where all terms appearing above have non-negative valuation.
  As a result, we can take the coefficient of ${T}^0$ term-wise,
  and obtain
  $$C(0,T_0) = s \prod_{1 \le i \le \kappa}(T_0-T_1 \varphi_{i,1} -
  \cdots - T_n \varphi_{i,n} ),$$ where $s$ is in $\KKbar[T_1,\dots,T_n]$;
  note that $s \ne 0$, since $C(0,T_0)$ is non-zero.
 By construction of $C$, for any
  $\bx=(x_1,\dots,x_n)$ in $V(J+\langle T \rangle)$, $T_1 x_1 + \cdots + T_n x_n$
  cancels $C(0,T_0)$, so $\bx$ must be one of
  $\varphi_1,\dots,\varphi_{\kappa}$.
\end{proof}

To conclude the proof of Proposition~\ref{prop:degree_fiber}, 
we now assume that property $\assG(0)$ holds.

\begin{lemma}
   $\Phi_1,\dots,\Phi_{c'}$ are bounded.
\end{lemma}
\begin{proof}
  For $i=1,\dots,c'$, write $\Phi_i=1/T^{e_i}
  (\Psi_{i,1},\dots,\Psi_{i,n})$, for a vector
  $(\Psi_{i,1},\dots,\Psi_{i,n})$ of Puiseux series of valuation
  zero, that is, such that all $\Psi_{i,j}$ are bounded and
  $(\psi_{i,1},\dots,\psi_{i,n})=\lim_0(\Psi_{i,1},\dots,\Psi_{i,n})$
  is non-zero. Hence,
  $e_i=-\nu(\Phi_i)$, and we have to prove that $e_i \le 0$.  By way
  of contradiction, we assume that $e_i > 0$.

  The Puiseux series $\Phi_i$ cancels $b_1,\dots,b_m$. For
  $k=1,\dots,m$, let $b_k^H \in \KKbar[T][X_0,\bX]$ be the homogenization
  of $b_k$ with respect to $\bX$. From the equality
  $b_k^H(T^{e_i},\Psi_{i,1},\dots,\Psi_{i,n})=
  T^{e_i}b_k(\Phi_i)$, we deduce that
  $b_k^H(T^{e_i},\Psi_{i,1},\dots,\Psi_{i,n})=0$ for all $k$. We
  can write $b_k = b_{0,k} + T \tilde b_k$, for some polynomial
  $\tilde b_k$ in $\KKbar[T,\bX]$, and $\assG_1(0)$ implies that
  $\deg_\bX(\tilde b_k) \le \deg_\bX(b_{0,k})$. As a result, the
  homogenizations (with respect to $\bX$) of $b_{k},b_{0,k}$ and $\tilde
  b_k$ satisfy a relation of the form $b^H_k = b_{0,k}^H +
  X_0^{\delta_k} T \tilde b^H_k$, for some $\delta_k \ge 0$. This
  implies the equality
  $$b_{0,k}^H(T^{e_i},\Psi_{i,1},\dots,\Psi_{i,n}) + T^{\delta_k
    e_i+1}\tilde b_k^H(T^{e_i},\Psi_{i,1},\dots,\Psi_{i,n})=0.$$
  The second term has positive valuation, so that
  $b_{0,k}^H(T^{e_i},\Psi_{i,1},\dots,\Psi_{i,n})$ has positive
  valuation as well. Taking the coefficient of $T^0$, this means 
  that $b_{0,k}^H(0,\psi_{i,1},\dots,\psi_{i,n})=0$ (since $e_i > 0$), which implies 
  that $(\psi_{i,1},\dots,\psi_{i,n})=(0,\dots,0)$, in view of $\assG_2(0)$.
  This however contradicts the definition of $(\psi_{i,1},\dots,\psi_{i,n})$.
\end{proof}

\begin{lemma}\label{lemma:Jprimerad}
  The ideal $\frak{J}'$ is radical; equivalently, $c'=c$.
\end{lemma}
\begin{proof}
 We know that $\frak{J}'$ has dimension zero, so it is enough to prove
 that for $i=1,\dots,c'$, the localization of $\KKbar\langle\langle T
 \rangle\rangle[\bX]/\frak{J}'$ at the maximal ideal
 $\mathfrak{m}_{\Phi_i}$ is a field, or equivalently that the
 localization of $\KKbar\langle\langle T \rangle\rangle[\bX]/\frak{J}$
 at $\mathfrak{m}_{\Phi_i}$ is a field.  By the Jacobian
 criterion~\cite[Theorem~16.19.b]{Eisenbud95}, this is the case if and
 only if the Jacobian matrix of $\bB$ with respect to $\bX$ has full
 rank $n$ at $\Phi_i$. We know that $\varphi_i=\lim_0(\Phi_i)$ is a
 root of $\bB_0$, and the Jacobian criterion conversely implies that
 since the ideal $\langle \bB_0 \rangle$ is radical (by assumption
 $\assG_3(0)$) and zero-dimensional (by assumption $\assG_2(0)$),
 the Jacobian matrix of $\bB_0(\bX)=\bB(0,\bX)$ has full rank $n$. Since
 this matrix is the limit at zero of the Jacobian matrix of $\bB$ with
 respect to $\bX$, taken at $\Phi_i$, the latter must have full rank
 $n$, and our claim that $\frak{J}'$ is radical is proved.
\end{proof}

To finish the proof of Proposition~\ref{prop:degree_fiber}, we have to
establish that $V(\bB_0)$ consists of exactly $c$ solutions.  Let thus
$d$ be the number of points in $V(\bB_0)$.  Since $\langle \bB_0
\rangle$ is radical (this is $\assG_3(0)$), Lemma~\ref{lemma:19}
implies that $d \le c$, so we only have to prove that $c \le d$. To
prove this, we prove that for $i,i'$ in $\{1,\dots,c\}$, with $i \ne
i'$, we have $\varphi_i \ne \varphi_{i'}$.

Suppose to the contrary that $\varphi_i = \varphi_{i'}$. We know that
the Jacobian matrix of $\bB_0$ has full rank $n$ at $\varphi_i$; up to
reindexing, we assume that rows $1,\dots,n$ correspond to a maximal
non-zero minor. Let $\bB'=(b_1,\dots,b_n)$.

Let $z=\nu(\Phi_i-\Phi_{i'})$; since  $\varphi_i = \varphi_{i'}$, we have
$z > 0$. We can thus write $\Phi_i=f + T^z
\delta_i$ and $\Phi_{i'}=f + T^z \delta_{i'}$, for some vectors of
bounded Puiseux series $f, \delta_i, \delta_{i'}$ such that all terms
in $f$ have valuation less than~$z$; in addition, $\lim_0(\delta_i)
\ne \lim_0(\delta_{i'})$. Write the Taylor expansion of $\bB'$ at $f$ as
$$\bB'(\Phi_i) = \bB'(f) + \jac_f(\bB',\bX) T^z \delta_i + T^{2z} r_i =0$$
and
$$\bB'(\Phi_{i'}) = \bB'(f) + \jac_f(\bB',\bX) T^z \delta_{i'} + T^{2z}
r_{i'} =0,$$ for some vectors of bounded Puiseux series $r_i,r_{i'}$.
By subtraction and division by $T^z$, we obtain $\jac_f(\bB',\bX)
(\delta_i-\delta_{i'}) = T^z r$, for some vector of bounded Puiseux
series $r$.  Since $\jac_f(\bB',\bX)$ is invertible, this further gives
$\delta_i-\delta_{i'} = T^z r'$, where again $r'$ is a vector of
bounded Puiseux series.  However, by construction the left-hand side
has valuation zero, while the right-hand side has positive valuation
(since $z > 0$). Hence, we derived a contradiction to our assumption
that $\varphi_i = \varphi_{i'}$. The proof of Proposition~\ref{prop:degree_fiber} is
complete. (Although we do not need it now, the linearization
used above also implies that all $\Phi_i$ are actually power series.)

%%%%%%%%%%%%%%%%%%%%%%%%%%%%%%%%%%%%%%%%%%%%%%%%%%%%%%%%%%%%

\subsection{Proof of Proposition~\ref{prop:compute_isolated}} \label{proof:prop2}

Let $\scrR_0 =((q_{0},v_{0,1},\dots,v_{0,n}),\lambda)$ be a
zero-dimensional parametrization of $V(\bB_{0})$ obtained by means of
assumption $\assD_1$, with $q_0$ and all $v_{0,j}$ in $\KK[Y]$. Note
that the degree of $q_0$ is the integer $c$.

\paragraph{Decomposing $\scrR_0$.}
We start by decomposing $\scrR_0$ into finitely many zero-dimensional
parametrizations
$\scrR_{0,j}=((q_{0,j},v_{0,j,1},\dots,v_{0,j,n}),\lambda)_{1\le j\le
  t}$, all with coefficients in $\KK$, such that for $j$ in
$\{1,\dots,t\}$, there exist $\bi_j=(i_{j,1},\dots,i_{j,n})$ such that
the Jacobian matrix of $(b_{0,i})_{i \in \bi_j}$ has full rank $n$ at
$\bx$, for all $\bx$ in $Z(\scrR_{0,j})$.

If $q_0$ were irreducible, we would simply evaluate the Jacobian
matrix of $\bB_0$ at the point $(v_{0,1}/q_0',\dots,v_{0,n}/q_0')$,
which has coordinates in the field $\LL=\KK[Y]/\langle q_0 \rangle$,
and find a non-zero minor of size $n$ in this matrix. It takes $O(n
E)$ oeprations in $\LL$ to compute this Jacobian matrix, and $O(mn^2)$
operations in $\LL$ to find an invertible minor. The total time, under
the assumption that $q_0$ is invertible, is thus $O(mn^2 + nE)$ operations
in $\LL$, that is, $\softO( (mn^2 + nE) c)$ operations in $\KK$.

When $q_0$ is not invertible, $\LL=\KK[Y]/\langle q_0 \rangle$ is a
product of fields. We can still apply the same process as in the
irreducible case; if the algorithm goes through, we have obtained our
answer. In general, one workaround would be to factor $q_0$, but we do
not want our runtime to depend on the cost of factoring
polynomials. Hence, we will use {\em dynamic evaluation techniques},
as in~\cite{D5}. Indeed, the only issue that may arise is that we
attempt to invert a zero-divisor. If this is the case, it means we
have found a non-trivial factor $r_0$ of $q_0$: we can then replace
$\scrR_0$ by two new zero-dimensional parametrizations,
$\scrR'_0=((r_0,(v_{0,1}/s_0) \bmod r_0,\dots,(v_{0,n}/s_0)\bmod
r_0),\lambda)$ and $\scrR''_0=((s_0,(v_{0,1}/r_0) \bmod
s_0,\dots,(v_{0,n}/ r_0)\bmod s_0),\lambda)$, with $s_0=q_0/r_0$, that
define a partition of $Z(\scrR_0)$ into the subsets
$Z(\scrR'_0)$ and $Z(\scrR''_0)$ where $r_0$
vanishes, resp.\ is non-zero.

We can then start over again, from $\scrR'_0$ and $\scrR''_0$
independently. Overall, in the worst case, this splitting process 
induces a extra factor $O(c)$ in the runtime, 
for a total of $\softO( (mn^2 + nE) c^2)$ operations in $\KK$.

\paragraph{Lifting power series and rational reconstruction.}
For $j=1,\dots,t$, we can then apply Newton iteration to the system
$(b_i)_{i \in \bi_j}$ to lift
$\scrR_{0,j}=((q_{0,j},v_{0,j,1},\dots,v_{0,j,n}),\lambda)$ into a
zero-dimensional parametrization
$\scrR_{j}=((q_{j},v_{j,1},\dots,v_{j,n}),\lambda)$ with coefficients
in $\KK[[T]]/\langle T^{2e}\rangle$, for $e$ as in ${\assD}_2$.
(Note that Lemma~\ref{lemma:vPi} implies that $c \le e$).

As explained in~\cite[Section~2.2]{SaSc16}, using the algorithm
of~\cite{GiLeSa01}, this can be done using $\softO(c\,e (E+n^2)n)$
operations in $\KK$.  Using the Chinese Remainder Theorem, we can
combine all $\scrR_{j}$ into a single zero-dimensional parametrization
$\scrR$ with coefficients in $\KK[[T]]/\langle T^{2e}\rangle$, since
for $j\ne j'$ $q_{0,j}$ and $q_{0,j'}$ generate the unit ideal in
$\KK[[T]]/\langle T^{2e}\rangle$; this takes time 
$\softO(c\,e\,n)$.

Using the notation of the previous subsection, the zeros of $\scrR$ in
$\KKbar[[T]]/\langle T^{2e}\rangle$ are the truncations of the power
series roots $\Phi_1,\dots,\Phi_c$ of $\mathfrak{J}'$. Since $J'$ is
supposed to have degree at most $e$, knowing $\scrR$ at precision $2e$
allows us to reconstruct a zero-dimensional parametrization $\scrS$
with coefficients in $\KK(\tau)$ such that
$Z(\scrS)=V(\mathfrak{J}')$, with all coefficients having numerator
and denominator of degree at most $e$~\cite[Theorem~1]{Schost03}.
This is done by applying rational function reconstruction to all
coefficients of $\scrR$, as in~\cite{Schost03}, and takes time
$\softO(c\,e\,n)$.


\paragraph{A finite set containing the isolated points of $V(\bC)$.}
As we did in the previous subsection for $T=0$, we let
$\Phi'_1,\dots,\Phi'_c$ be the roots of $\mathfrak{J}'$ in the field
of Puiseux series $\KKbar\langle\langle T'\rangle\rangle$ at $T=1$,
with $T'=T-1$. Without loss of generality, we assume that
$\Phi'_1,\dots,\Phi'_{\kappa'}$ are bounded, and
$\Phi'_{\kappa'+1},\dots,\Phi'_c$ are not, for some $\kappa'$ in
$\{0,\dots,c\}$, and we let $\varphi'_1,\dots,\varphi'_\kappa$ by
$\varphi'_i=\lim_0(\Phi'_i)\in\KKbar{}^n$ for $i=1,\dots,\kappa'$.  By
Lemma~\ref{lemma:Z1}, $V(J' + \langle T-1\rangle) = \{ \varphi'_i \mid
i=1,\dots,k\}$.

We can now specify our requirements on the linear form $\lambda$.
Following~\cite{RRS} and~\cite{SaSc16}, we ask that $\lambda$ is a {\rm well-separating
  element},
\begin{enumerate}
\item $\lambda$ is separating for $V(\mathfrak{J}')=\{\Phi'_1,\dots,\Phi'_c\}$;
\item $\lambda$ is separating for $V(J' + \langle T-1\rangle) = \{ \varphi'_1,\dots,\varphi'_{\kappa'}\}$.
\item $\nu(\lambda(\Phi_i)) = \mu_i$ for all $i=1,\dots,c$, where $\nu$ denotes
 the $T'$-adic valuation.
\end{enumerate}
These conditions are satisfied for a generic choice of $\lambda$, as
showed in~\cite{SaSc16}. When this is the case, Lemma~4.4
in~\cite{RRS} shows how to recover a zero-dimensional parametrization
$\scrR_1=((q_1,v_{1,1},\dots,v_{1,n}),\lambda)$ with coefficients in
$\KK$ for the limit set $V(J' + \langle T-1\rangle) =\{\varphi'_i \mid i=1,\dots,\kappa\}$
starting from $\scrS$, in time 
$\softO(c\,e\,n)$.

\paragraph{Cleaning.}
Lemma~\ref{lemma:vPi} implies that for any isolated solution $\bx$ of
$\bC$, $(1,\bx)$ is in $V(J' + \langle T-1\rangle)$, so in a second
time, we discard from $V(J'_1)$ those points that do not correspond to
isolated points of $V(\bC)$. All such points belong to a
positive-dimensional component of $V(\bC)$, so we can use the algorithm
of Section~\ref{sec:isolated}. By Proposition~\ref{prop:degree_fiber},
we can take $c$ as an upper bound on the multiplicity of isolated
solutions of $\bC$.

Using the same dynamic evaluation techniques as in the first paragraph
above, we can use the algorithm of Section~\ref{sec:isolated} as if
$Z(\scrR_1)$ were an irreducible variety, with an overhead 
$O(c)$. The runtime deduced from Proposition~\ref{prop:testisolated} is
$$O(n^4 c^6 + n^2 m c^5 + n E c^6)$$ operations in~$\KK$. Adding all
costs seen so far, we prove Proposition~\ref{prop:compute_isolated}.



%%%%%%%%%%%%%%%%%%%%%%%%%%%%%%%%%%%%%%%%%%%%%%%%%%%%%%%%%%%%
%%%%%%%%%%%%%%%%%%%%%%%%%%%%%%%%%%%%%%%%%%%%%%%%%%%%%%%%%%%%
%%%%%%%%%%%%%%%%%%%%%%%%%%%%%%%%%%%%%%%%%%%%%%%%%%%%%%%%%%%%

\section{Properties of determinantal ideals}\label{sec:check}

The following two sections will show how to apply the algorithm of the
previous section to Problems~\ref{problem} and~\ref{problem2}, by
applying Proposition~\ref{prop:compute_isolated} to suitable
deformations of our input systems. This proposition requires several
assumptions to hold: some (noted $\assA_1$ and $\assA_2$) are related
to the deformed system as a whole, while the other ($\assG_3$ to
$\assG_2$) involve properties at the starting point of the homotopy
($T=0$). In this section, we prove that a large variety of systems
satisfy $\assA_1$ and $\assA_2$.

Let $T$ and $\bX=(X_1,\dots,X_n)$ be variables, let $J$ be an ideal in
$\KKbar[T,\bX]$, and consider the following properties:
\begin{description}[leftmargin=*]
\item[$\assA_1.$] Any irreducible component of $V(J) \subset
  \KKbar{}^{n+1}$ has dimension at least one.
\item[$\assA_2.$] For any maximal ideal $\m \subset\KKbar[T,\bX]$,
  if the localization $J_\m \subset \KKbar[T,\bX]_\m$ has height $n$,
  then it is unmixed (that is, all associated primes have height $n$).
\end{description}

We pointed out in the previous section that when $J$ is generated by
$n$ polynomials, these results are well-known. To study the case of
maximal minors of a polynomial matrix, we will use the following
results, taken from~\cite[Section~6]{EN62}. Let $R$ be a
Cohen-Macaulay ring and let $I$ the ideal generated by all $p$-minors
of a $p\times q$ matrix $\mF \in R^{p\times q}$, with $p \le q$. Then:
\begin{itemize}
\item if $I \ne R$, then the height of $I$ is at most $q-p+1$;
\item if $I$ has height $q-p+1$, then $I$ is unmixed (all associated
  primes have height $q-p+1$).
\end{itemize}
Let us transcribe these properties in the case of a matrix $\mF$ in
$\KK[T,\bX]^{p \times q}$, $p \le q$, by letting $I=I_p(\mF)$; we do not make any
assumption relating $n$ and $q-p+1$ yet, since we will need the extra
generality further down. Let thus, and assume that $V(I)\ne
\emptyset$, so that $I \ne \KK[T,\bX]$. Let $V_1,\dots,V_s$ be the
$\KKbar$-irreducible components of $V(I)\subset \KKbar{}^n$.  Then, we
claim that $\dim(V_i) \ge (n+1) -(q-p+1)$ holds for all $i$.

Take a point $\bx$ in $V(I)\subset \KKbar{}^{n+1}$, and let $\m \subset
\KKbar[T,\bX]$ be the maximal ideal at $\bx$. The height of
$I_\m$ in $\KKbar[T,\bX]_\m$ is equal to $(n+1)-\max\{ \dim(V_i)
\mid 1 \le i \le s, \bx \in V_i\}$. For $i=1,\dots,t$, let $\bx_i$
be a point in $V_i$ that does not belong to any other $V_{i'}$, $i'
\ne i$ and let $\m_i$ be the corresponding maximal ideal; then, the
previous equality becomes ${\rm height}(I_{\m_i})=(n+1)-\dim(V_i)$.
Applying the first item above in $\KKbar[T,\bX]_{\m_i}$ (which
is Cohen-Macaulay), we deduce that $(n+1)-\dim(V_i) \le q-p+1$, that is,
$\dim(V_i) \ge (n+1) -(q-p+1)$. The following proposition
follows easily.

\begin{proposition}
  If $n=q-p+1$, the ideal $I$ satisfies $\assA_1$ and $\assA_2$.
\end{proposition}
\begin{proof}
  Taking $n=q-p+1$, the previous inequality becomes $\dim(V_i) \ge 1$;
  this proves $\assA_1$.  To prove $\assA_2$, we take maximal
  ideal $\m$ in $\KKbar[T,\bX]$, and we apply the second item above
  over the ring $R=\KKbar[T,\bX]_\m$; this gives the conclusion
  immediately.
\end{proof}

%% (bonus) Indeed, write the primary decomposition of $I$ as $I=Q_1 \cap \cdots
%% Q_t \cap Q_{t+1}\cdots \cap Q_u$, where $\x \in V(Q_i)$ if and
%% only if $i \le t$. Then, the localization of $I$ at $\m$ admits the
%% primary decomposition
%% $I_\m={Q_1}_\m \cap \cdots {Q_t}_\m$ in $\KKbar[X_1,\dots,X_n]_\m$,
%% so that its height is the minimum of the heights of ${P_i}_\m$,
%% for $i=1,\dots,t$, where $P_i=\sqrt{Q_i}$ for all $i$. Now, 
%% that height is simply the height of $P_i$. The minimum is reached for one of
%% the minimal primes.

%% (bonus): If $I=Q_1 \cots Q_t$, then $h(I)=\min(h(P_i))$, with
%% $P_i=\sqrt{Q_i}$. Indeed, $I \subset P_i$ for all $i$.  On the other
%% hand, take a prime $J$ containing $I$. Then, there is $i$ such that
%% $Q_i \subset J$ (otherwise, for all $i$ there is $q_i$ in $Q_i$, $q_i$
%% not in $J$; then $q_1\cdots q_t$ is in $I$, but not in $J$). Then,
%% $P_i$ is in $J$ too, so $h(P_i) \le h(J)$.

Finally, we prove a generalization of this claim, which will be the
most general form we will need for such statements. Let
$G=(g_1,\dots,g_s)$ be polynomials in $\KK[T,\bX]$, with $s \le
n$. Let also $\mF$ be a polynomial matrix in $\KK[T,\bX]^{p \times
  q}$, with $p \le q$; we denote by $K$ the ideal generated by all
$p$-minors of $\mF$, together with the polynomials $G$ in
$\KKbar[T,\bX]$.

\begin{proposition}\label{prop:KH1H2}
  If $n=q-p+s+1$, the ideal $K$ satisfies $\assA_1$ and $\assA_2$.
\end{proposition}

The proof occupies the rest of this section. Let $\bB=M_p(\mF)$, the
set of all $p$-minors of~$\mF$. For $\assA_1$, we saw in the previous
paragraphs that all irreducible components of $V(\bB)$ have dimension
at least $(n+1)-(q-p+1)=s+1$. Since $G$ consists of $s$ polynomials,
all irreducible components of $V(K)$ must have dimension at least $1$,
as claimed.

We next prove $\assA_2$. Let $K_\m=Q_1 \cap \cdots \cap Q_t$ be an
irredundant primary decomposition of $K_\m$ in $\KKbar[T,\bX]_\m$, and
let $P_1,\dots,P_t$ be the corresponding primes; we assume that the
height of $K_\m$ is $n$, and our goal is to prove that all $P_i$'s
have height~$n$.

Of course, we can restrict to an ideal $\m$ containing $K$; $\m$ is
then the maximal ideal at a point $\x \in \KKbar{}^{n+1}$ that belongs
to $V(K)$. The height of the localization $K_\m \subset
\KKbar[T,\bX]_\m$ can be rewritten as $(n+1)-\dim(V_\x)$, where $V_\x$
is the union of the irreducible components of $V(K)$ passing through
$\x$. Our assumption is that the height of $K_\m$ is $n$, that is,
that $\dim(V_\x)=1$. Thus, every irreducible component of $V(K)$
containing $\x$ has dimension~$1$.

Let $W$ be an irreducible component of $V(\bB)$ containing $\x$.  We
claim that $\dim(W)=s+1$. Indeed, we mentioned in the first paragraph
that $\dim(W) \ge s+1$. If $\dim(W) > s+1$, then by Krull's theorem,
every irreducible component of $W \cap V(G)$ has dimension greater
than $1$; since $W \cap V(G)$ is a subset of $V(K)$ and contains $\x$,
we have reached a contradiction. Now, the fact that $\dim(W)=s+1$ for
any irreducible component of $V(\bB)$ containing $\x$ means that
$\langle \bB \rangle_\m$ has height $n-s=q-p+1$.  As a
result,~\cite[Theorem~18.18]{Eisenbud95} shows that
$\KKbar[T,\bX]_\m/\langle \bB \rangle_\m$ is Cohen-Macaulay.

By the remarks following~\cite[Theorem~IV.5.9]{ZaSa58}, $\bar Q_1 \cap
\cdots \cap \bar Q_t$ is an irredundant primary decomposition of $\bar
K_\m$ in $\KK[T,\bX]_\m/\langle \bB \rangle_\m$, with associated
primes $\bar P_1,\dots,\bar P_t$, where for an ideal $I \subset
\KKbar[T,\bX]_\m$, $\bar I$ denotes its image modulo $\langle \bB
\rangle_\m$. In addition, if we let $P_1,\dots,P_u$ be the minimal
primes of $K_\m$, for some $s \le t$, $\bar P_1,\dots,\bar P_u$ are
the minimal primes of $\bar K_\m$.

Our assumption says that $P_1,\dots,P_u$ have height $n$. Because
$\KKbar[T,\bX]_\m/\langle \bB \rangle_\m$ is Cohen-Macaulay, for any
$i \le t$, we have 
$$\dim(\KKbar[T,\bX]_\m/\langle \bB \rangle_\m)=\dim((\KKbar[T,\bX]_\m/\langle \bB \rangle_\m) / \bar P_i) + {\rm height}(\bar P_i).$$
The factor ring $(\KKbar[T,\bX]_\m/\langle \bB \rangle_\m) / \bar P_i$ is simply
$\KKbar[T,\bX]_\m/P_i$, so this can be rewritten as
$$s+1 = \dim(\KKbar[T,\bX]_\m/P_i) + {\rm height}(\bar P_i).$$ For $i\le
u$, we have $\dim(\KKbar[T,\bX]_\m/P_i)=1$, so that ${\rm height}(\bar
P_i)=s$; for $i > u$, the height of $\bar P_i$ is necessarily
$s+1$. Because $\bar P_1,\dots,\bar P_u$ are the minimal primes of
$\bar K_\m$, the height of $\bar K_\m$ is thus $s$ as well.

The ideal $\bar K_\m$ is generated in $\KKbar[T,\bX]_\m/\langle \bB
\rangle_\m$ by $G=(g_1,\dots,g_s)$. Since $\KKbar[T,\bX]_\m/\langle
\bB \rangle_\m$ is Cohen-Macaulay, $\bar K_\m$ is unmixed, that is,
$u=t$.  As a result, $Q_1 \cap \cdots \cap Q_u$ is an irredundant
primary decomposition of $K_\m$, and $K_\m$ is unmixed.

%%%%%%%%%%%%%%%%%%%%%%%%%%%%%%%%%%%%%%%%%%%%%%%%%%%%%%%%%%%%
%%%%%%%%%%%%%%%%%%%%%%%%%%%%%%%%%%%%%%%%%%%%%%%%%%%%%%%%%%%%
%%%%%%%%%%%%%%%%%%%%%%%%%%%%%%%%%%%%%%%%%%%%%%%%%%%%%%%%%%%%

\section{The column-degree homotopy}

Consider an input to Problem~\ref{problem2}: we are given a matrix
$\mF =[f_{i,j}]\in \KK[X_1,\dots,X_n]^{p \times q}$ and polynomials
$G=(g_1,\dots,g_s)$ in $\KK[X_1,\dots,X_n]$, with $p \leq q$ and $n =
q-p+s+1$; we want to compute the isolated points of $S_{\mF,G}$,
with
$$S_{\mF,G} = \{\bx \in \KKbar{}^n \mid  \mathrm{rank}(\mF({\bx})) < p
\text{~and~} g_1(\bx)=\cdots=g_s(\bx)=0\}.$$ In this section, we
design an algorithm for this task whose costs depends on the column
degrees $\delta_1=\cdeg(\mF,1),\dots,\delta_q=\cdeg(\mF,q)$; note in particular
that with this notation, $\deg(f_{i,j}) \leq \delta_j$ holds for all $i,j$.
We will also write $\gamma_1=\deg(g_1),\dots,\gamma_s=\deg(g_s)$.

\begin{proposition}
  There exists a randomized algorithm that computes the isolated
  points of $S_{\mF,G}$ using $\softO\left (c(e+c^5 )(E + c)\right)$
  operations in $\KK$, with $c=\gamma_1\cdots\gamma_s \delta_{\rm
    row}(\mF)$ and $e=(\gamma_1+1)\cdots(\gamma_s+1) \tilde\delta_{\rm
    row}(\mF)$.
\end{proposition}
Remark that if all $\gamma_i$'s and $\delta_j$'s are at least equal to
$2$, we have the inequality $e \le c^2$, and the runtime becomes
$\softO\left (c^6 (E + c)\right)$.

\medskip

We use the algorithm of Section~\ref{sec:homotopy}. To match the
notation of that section, we let $\bC=(c_1,\dots,c_{s},\dots,c_m)$ be
polynomials defined as follows: $(c_1,\dots,c_{s})=(g_1,\dots,g_s)$,
and $(c_{s+1},\dots,c_{m})$ are the $p$-minors of $\mF$, so that 
$m=s+{q \choose p}$. Thus, $S_{\mF,G}$ is the zero-set of~$\bC$.

Using the degrees $\gamma_1,\dots,\gamma_s$ and  $\delta_1,\dots,\delta_q$, we construct a polynomial matrix $\mL \in
\KK[\bX]^{p \times q}$, and polynomials $M=(m_1,\dots,m_s)$ in
$\KK[\bX]$, to use as a starting point for the homotopy
algorithm. For any $1 \leq j \leq q$ and $1 \leq k \leq \delta_j$, let us
define $$\lambda_{j,k} = \lambda_{j,k,0} + \sum_{\ell = 1}^{n}\lambda_{j,k,\ell}X_\ell,$$ 
where all $\lambda_{j,k,\ell}$ are random elements in
$\KK$. Then, for $j=1,\dots,q$, we define
$$\lambda_j = \prod_{k=1}^{\delta_j}\lambda_{j,k},$$
and we let  $\mL$ be the matrix
\[\mL = 
\left( \begin{matrix}
\lambda_1 & 2\lambda_2 & \cdots & q\lambda_{q}\\
\lambda_1 & 2^2\lambda_2 & \cdots & q^2\lambda_q\\
\vdots & \vdots &  & \vdots \\
\lambda_1 & 2^p\lambda_2 & \cdots & q^p\lambda_q
\end{matrix} \right) \in \KK[\bX]^{p\times q}.
\]
For $i=1,\dots,s$ and $k=1,\dots,\gamma_i$, let us further define
$$\mu_{i,k} =  \mu_{i,k,0} + \sum_{\ell = 1}^{n}\mu_{i,k,\ell}X_\ell,$$ where
all $\mu_{i,k,\ell}$ are random elements in $\KK$; then, we let
$$m_i=\prod_{k=1}^{\gamma_i} \mu_{i,k}.$$ We can thus define the system
of equations $\bA=(a_1,\dots,a_s,\dots,a_m)$, where $a_i=m_i$ for
$i=1,\dots,s$, and $(a_{s+1},\dots,a_{m})$ are the $p$-minors of
$\mL$ (taken in the same order as those in the system $\bC$).

Finally, we let $T$ be a new variable, we define the matrix
$\mU=(1-T)\, \mL + T \, \mF \in \KK[T,\bX]^{p\times q}$, and we let $\bB$ be the polynomials in
$\KK[T,\bX]$ given by $\bB=(b_1,\dots,b_s,\dots,b_m)$, where
\begin{itemize}
\item $b_i=(1-T) m_i + T g_i$ for $i=1,\dots,s$
\item $(b_{s+1},\dots,b_{m})$ are the $p$-minors of $\mU$, taken in
  the same order as those in $\bC$.
\end{itemize}
We can then define $J$ as the ideal generated by $\bB$ in
$\KKbar[T,\bX]$. Using the notation of Section~\ref{sec:homotopy}, we
see that $\bB_0=\bA$ and $\bB_1=\bC$. Having in mind
to apply Proposition~\ref{prop:compute_isolated} to compute the
isolated points of $V(\bC)=S_{\mF,\mG}$, we now verify that all
required assumptions are satisfied.

\paragraph{Properties $\assA_1$ and $\assA_2$.}
These follow from Proposition~\ref{prop:KH1H2}.

\paragraph{Property $\assG_1(0)$.} We have to prove that for $i=1,\dots,m$,
$\deg_\bX(b_i)=\deg_\bX(a_i)$. 

For $i=1,\dots,s$, this amounts to proving that $\deg_\bX((1-T) m_i +
T g_i)=\deg_\bX(m_i)$. The latter is by construction equal to
$\gamma_i$. The former is at most $\gamma_i$ (since $b_i$ is the sum
of two polynomials of degree $\gamma_i$ in $\bX$), but since
evaluating $T$ at $0$ in $b_i$ gives us $g_i$, its degree in $\bX$
must be exactly $\gamma_i$.

To each index $i=m+1,\dots,s$ corresponds a sequence
$\bj_i=(j_{i,1},\dots,j_{i,p})$ such that $b_i$ and $a_i$ are the
minors built with columns index by $\bj_i$ in respectively
$\mU=(1-T)\, \mL + T \, \mF$ and $\mL$. In view of the shape of $\mL$,
the polynomial $a_i$ is equal to $c_i\lambda_{j_{i,1}}\cdots
\lambda_{j_{i,p}}$, where $c_i$ is a non-zero constant, so it has degree
$\delta_{j_{i,1}} + \cdots + \delta_{j_{i,p}}$.  Since the columns
$(j_{i,1},\dots,j_{i,p})$ of $U$ have respective degrees at most
$(\delta_{j_{i,1}},\dots,\delta_{j_{i,p}})$, $b_i$ has degree at most
$\delta_{j_{i,1}} + \cdots + \delta_{j_{i,p}}$. However, evaluating
$T$ at $0$ in $b_i$ gives us back the polynomial $a_i$, so $b_i$ must
have degree exactly $\delta_{j_{i,1}} + \cdots + \delta_{j_{i,p}}$.

\paragraph{Property $\assG_2(0)$.} We have to prove that the homogenization
of the system $\bA$ has no root at infinity. Thus, let $X_0$ be a new
variable, and let $\bA^H=(a_1^H,\dots,a_m^H)$ be the homogenization
of $\bA$. For $i=1,\dots,s$, we have
$$a_i^H=\prod_{k=1}^{\gamma_i} \mu^H_{i,k} \quad\text{with}\quad \mu^H_{i,k}=(\mu_{i,k,0}X_0 + \sum_{\ell = 1}^{n}\mu_{i,k,\ell}X_\ell),$$
whereas for $i=s+1,\dots,m$, 
$$a_i^H=c_i \lambda^H_{j_{i,1}}\ldots \lambda^H_{j_{i,p}}, \quad \text{~for~} \bj_i=(j_{i,1},\dots,j_{i,p}) \text{~as above},$$
where for $j=1,\dots,q$ we set 
$\lambda^H_j = \prod_{k=1}^{\delta_j}\lambda^H_{j,k}$,
with
$$\lambda^H_{j,k}=\lambda_{j,k,0}X_0 + \sum_{\ell = 1}^{n}\lambda_{j,k,\ell}X_\ell.$$
To prove  $\assG_1(0)$, we start by writing down all projective
solutions of this system (this will be of use below), before adding
the constraint $X_0=0$.

Since all $a_i^H$ are products of linear forms, we find the solutions
of $\bA^H$ by setting some of these linear forms to zero. In order to
cancel $a_1^H,\dots,a_s^H$, we choose indices $\bu=(u_1,\dots,u_s)$,
with $u_1\in\{1,\dots,\gamma_1\}$, \dots,
$u_s\in\{1,\dots,\gamma_s\}$, and we consider the equations 
$$\mu^H_{i,u_i}=0, \quad \text{~that is,~} \quad \mu_{i,u_i,0}X_0 + \sum_{\ell = 1}^{n}\mu_{i,u_i,\ell}X_\ell =0,$$ for $i=1,\dots,s$.
In what follows, we fix such an $\bu$.
Then, for a generic choice of coefficients $\mu_{i,k,\ell}$, these equations
are equivalent to
$$X_{n-s+1}=\mathfrak{h}_{n-s+1,\bu}(X_0,\dots,X_{n-s}),\dots,X_{n}=\mathfrak{h}_{n,\bu}(X_0,\dots,X_{n-s}),$$
for some homogeneous linear forms $\mathfrak{h}_{n-s+1,\bu},\dots,\mathfrak{h}_{n,\bu}$.
After applying this substitution, for all $j=1,\dots,q$, $\lambda^H_j$ can be rewritten as
$$\lambda^H_{j,\bu}=\prod_{k=1}^{\delta_j}\lambda^H_{j,k,\bu},$$
where 
$$\lambda^H_{j,k,\bu}=\lambda_{j,k,0}X_0 + \sum_{\ell =
  1}^{n-s}\lambda_{j,k,\ell}X_\ell + \sum_{\ell =
  n-s+1}^{n}\lambda_{j,k,\ell}
\mathfrak{h}_{\ell,\bu}(X_0,\dots,X_{n-s}).$$ Then,
$\bx=(x_0,\dots,x_n)$ cancels $a^H_{s+1},\dots,a^H_m$ if and only if
$\bx'=(x_0,\dots,x_{n-1})$ cancels the product
$\lambda^H_{j_1,\bu}\cdots \lambda^H_{j_p,\bu},$ for any choice of columns
$\bj=(j_1,\dots,j_p)$.

\begin{lemma}
  For $\bx'$ in $\P^{n-s}(\KKbar)$, the products
  $\lambda^H_{j_1,\bu}(\bx')\cdots \lambda^H_{j_p,\bu}(\bx')$
  vanish for all choices of columns $\bj=(j_1,\dots,j_p)$ if and only
  if there exist $\{j_1,\dots,j_{n-s}\} \subset \{1,\dots,q\}$ such 
  that $\lambda^H_{j_1,\bu}(\bx')=\cdots=\lambda^H_{j_{n-s},\bu}(\bx')=0$.
\end{lemma}
\begin{proof}
  Take an arbitraty representative $\bx^*$ of $\bx'$ in
  $\KKbar{}^{n+1}$, aonsider the polynomial
  $(1+\lambda^H_{1,\bu}(\bx^*)Y_1) \cdots (1+\lambda^H_{q,\bu}(\bx^*)Y_q),$
  for new variables $Y_1,\dots,Y_q$. The products
  $\lambda^H_{j_1,\bu}(\bx^*)\cdots \lambda^H_{j_p,\bu}(\bx^*)$ are all zero
  if and only if this polynomial has degree less than $p$, that is, if
  and only if $q-p+1$ terms among
  $\lambda^H_{1,\bu}(\bx^*),\dots,\lambda^H_{q,\bu}(\bx^*)$ vanish.
\end{proof}


For generic coefficients $\lambda_{j,k,\ell}$ and $\mu_{i,k,\ell}$,
the condition of the lemma holds if and only if there exist
$\bj=\{j_1,\dots,j_{n-s}\} \subset \{1,\dots,q\}$ and
$\bv=(v_1,\dots,v_{n-s})$, with $v_k$ in $\{1,\dots,\delta_k\}$ for all
$k$, such that $\lambda^H_{j_k,v_k,\bu}(\bx')=0$ 
for $k=1,\dots,n-s$.

This implies that for a fixed $\bu$, the possible values of $\bx'$ are
in finite number (we will write it down precisely below), determined
as solutions of a linear system of size $n-s$ with generic
coefficients. In particular, none of these points satisfies $X_0=0$,
so that $\assG_1(0)$ holds.

\paragraph{Property $\assG_3(0)$.} From $\assG_2(0)$, we know that
the projective variety defined by $\bA^\mH$ has no point at infinity,
so it is finite; as a result, the affine algebraic set defined by
$\bA$ is finite as well. In addition, all the affine solutions to
$\bA$ are obtained by setting $X_0=1$ in the projective solutions of
$\bA^\mH$. In other words, they are obtained by choosing indices
$\bu=(u_1,\dots,u_s)$ with $u_k$ in $\{1,\dots,\gamma_k\}$ for all $k$,
column indices $\bj=(j_1,\dots,j_{n-s})$, and
$\bv=(v_1,\dots,v_{n-s})$, with $v_k$ in $\{1,\dots,\delta_k\}$
for all $k$, solving the affine linear system
$$\lambda_{j_1,v_1,\bu}(X_1,\dots,X_{n-s})=\cdots=\lambda_{j_{n-s},v_{n-s},\bu}(X_1,\dots,X_{n-s})=0$$ 
and using the expressions
$$X_{n-s+1}=\mathfrak{h}_{n-s+1,\bu}(1,X_1,\dots,X_{n-s}),\dots,X_{n}=\mathfrak{h}_{n,\bu}(1,X_1,\dots,X_{n-s}).$$
To prove that the ideal generated by $\bA$ is radical, we prove that
at any point as described above, the Jacobian matrix of $\bA$ with
respect to $X_1,\dots,X_n$ has full rank.

Let thus $\bu$, $\bj$ and $\bv$ be as above, let $\bx \in \KKbar{}^n$
be the corresponding point in $V(\bA)$, and consider equations
$(a_1,\dots,a_s)$ first. Each such equation is a product of linear forms
as $a_i=\prod_{k=1}^{\gamma_i} \mu_{i,k}$, with $\mu_{i,u_i}(\bx)=0$.
Since the coefficients $\mu_{i,k,\ell}$ are chosen generically, for
$i=1,\dots,s$ and $k \ne u_i$, $\mu_{i,k}(\bx)$ is non-zero; as a
result, in the local ring at $\bx$, the polynomials $(a_1,\dots,a_s)$
are equal (up to units) to the linear forms
$(\mu_{1,u_1},\dots,\mu_{s,u_s})$.

Next, we consider the $p$-minors of $\mA$; in what follows, we 
write $\bx'=(x_1,\dots,x_{n-1})$. Our starting point is that due to 
the genericity of the coefficients $\lambda_{j,k,\ell}$, since 
$$\lambda_{j_1,v_1,\bu}(\bx')=\cdots=\lambda_{j_{n-s},v_{n-s},\bu}(\bx')=0,$$ 
none of the other linear forms $\lambda_{j,k,\bu}$ vanishes at $\bx'$.

Recall that $n=q-p+s+1$, so that $n-s = q-(p-1)$. Hence, there are
exactly $p-1$ columns of $\mA$ not indexed by $\bj=(j_1,\dots,j_{n-s})$; call
them $\bj'=(j'_1,\dots,j'_{p-1})$. We can then consider the 
products
$$ \lambda_{j_1} \lambda_{j'_1} \cdots \lambda_{j'_{p-1}},\dots, \lambda_{j_{n-s}}
\lambda_{j'_1} \cdots \lambda_{j'_{p-1}};$$ each of them (up to a non-zero
constant) is a $p$-minor of $\mA$, so they appear as elements in the
sequence $(a_{s+1},\dots,a_m)$, say as
$(a_{e_1},\dots,a_{e_{n-s}})$. By the remark of the previous
paragraph, in the local ring at $\bx$, up to non-zero constants, these
polynomials are respectively equal to the linear forms
$\lambda_{j_1,v_1},\dots,\lambda_{j_{n-1},v_{n-s}}$.  

To summarize, we have found that the linear equations
$(\mu_{1,u_1},\dots,\mu_{s,u_s})$ and
$(\lambda_{j_1,v_1},\dots,\lambda_{j_{n-1},v_{n-s}})$ belong to the
ideal $\langle \bA \rangle_\m$, where $\m$ is the maximal ideal at
$\bx$. As a result, the Jacobian matrix of $\mA$ must be invertible
at $\bx$, and $\assG_3(0)$ holds.


\medskip

At this stage, we have established all assumptions necessary to apply
Proposition~\ref{prop:degree_fiber}. Since $\bB$ satisfies
$\assA_1,\assA_2$ and $\bA=\bB_0$ satisfies $\assG_1,\assG_2,\assG_3$,
we deduce that the sum of the multiplicities of the isolated solutions
of $\bC=\bB_1$ is at most $c$, where $c$ is the number of solutions of
$\bA$. To estimate $c$, note first that there are $\gamma_1\cdots
\gamma_s$ choices of $\bu$. For each choice of $\bu$, there are
$\delta_{\rm col}(\mF)=E_{n-s}(\delta_1, \ldots, \delta_q)$ ways to
choose $\bj$ and $\bv$, where $E_{n-s}$ denotes the elementary
symmetric polynomials of degree $n-s$. As a result, $c=\gamma_1\cdots
\gamma_s \delta_{\rm col}(\mF)$. We can now inspect assumptions
$\assD_1,\dots,\assD_4$ that are needed to apply 
the algorithm of Proposition~\ref{prop:compute_isolated}.

\paragraph{Property $\assD_1$.} We know that $\assG_1(0),\assG_2(0),\assG_3(0)$,
so we are going to compute a zero-dimensional parametrization of
$V(\bA)$.  We do this by following the description of the solutions of
$\bA$ given in the previous paragraph: for any choice of indices
$\bu$, $\bj$ and $\bv$ as above, the corresponding point $\bx \in
\KKbar{}^n$ in $V(\bA)$ can be found by solving a linear system of
size $n$, so in time $O(n^3)$. We repeat this procedure
$c=\gamma_1\cdots \gamma_s \delta_{\rm col}(\mF)$ times, using a total
of $O(c n^3)$ operations in $\KK$.

Knowing all the points in $V(\bA)$, we can reconstruct a zero-dimensional
parametrization $\scrR_0$ such that $Z(\scrR_0)=V(\bA)$ in time
$O\tilde{~}(n c)$ by means of fast interpolation~\cite[Chapter~10]{GaGe03}.
(Note that for practical purposes, we may modify the algorithm
of Proposition~\ref{prop:compute_isolated} to take into account the
fact that all points in $V(\bA)$ are in $\KK^n$.)

\paragraph{Property $\assD_2$.} Next, we need to determine an upper bound 
$e$ on the degree of the curve $V(J')$, where $J'$ is the union of the
one-dimensional irreducible components of $V(\bB) \subset
\KKbar{}^{n+1}$ whose projection on the $T$-axis is dense.

Let us write $V(\bB)=V(J') \cup V' \cup V''$, where $V''$ is the union
of the other components of dimension one of $V(\bB)$ and $V''$ is the
union of the components of higher dimension (by $\assA_1$, $V(\bB)$
has no isolated points), and let $H$ be a generic hyperplane in
coordinates $T,X_1,\dots,X_n$. Then, $(V(J') \cup V') \cap V(H)$ is a
finite set consisting of $\deg(V(J')) + \deg(V')$ points, whereas $V''
\cap V(H)$ consits only on components of positive dimension; these two
sets are disjoint. Thus, we can take for $e$ the number of isolated 
points of $V(\bB)\cap V(H)$.

The hyperplane $H$ is defined by an equation $h_0 + h_1 X_1 + \cdots +
h_{n+1}X_{n+1} + h_{n+2} T=0$. This equation allows us to rewrite $T$
as $\eta(X_1,\dots,X_n)=-(h_0 + h_1 X_1 + \cdots +
h_{n+1}X_{n+1})/h_{n+2}$; the points in $V(\bB)\cap V(H)$ are thus in
one-to-one correspondence with the solutions of the system
$(\beta_1,\dots,\beta_s,\dots,\beta_m)$, where $\beta_i=(1-\eta) m_i +
\eta g_i$, for $i=1,\dots,s$, and $\beta_s,\dots,\beta_m$ are the
$p$-minors of the matrix ${\cal U}=(1-\eta)\, \mL + \eta \, \mF $.  Now, the
polynomials $(\beta_1,\dots,\beta_s)$ have respective degrees at most
$(\gamma_1+1),\dots,(\gamma_s+1)$, and the column degrees of 
${\cal U}$ are $\delta_1+1,\dots,\delta_q+1$. 

We can then apply Proposition~\ref{prop:degree_fiber}, which shows we
can take for $e$ the integer $(\gamma_1+1)\cdots(\gamma_s+1)
\tilde\delta_{\rm row}(\mF)$.

\paragraph{Property $\assD_3$.} Finally, we need to give an estimate on
the size of a straight-line program that computes the polynomials
$\bB=(b_1,\dots,b_m)$, assuming that we are given a straight-line
program of size $E$ that computes polynomials $G=(g_1,\dots,g_s)$ and
the entries of $\mF$.

First, we estimate the complexity of computing the polynomials
$(b_1,\dots,b_s)$. For $i \le s$, the $i$th polynomial $b_i$ is equal
to $(1-T)m_i + T g_i$, where $m_i$ is a product of $\gamma_i$ linear
forms in $n$ variables. This polynomial can be computed in $O(n
\gamma_i)$ operations in $\KK$, hence for a total of $O(n \sum_i
\gamma_i)$ operations for $(m_1,\dots,m_s)$, and $O(E+n \sum_i
\gamma_i)$ for $(b_1,\dots,b_s)$. The polynomials
$(b_{s+1},\dots,b_m)$ are the $p$-minors of $\mU=(1-T)\mL+T\mF$.

The polynomials $\lambda_1,\dots,\lambda_q$ can be computed in $O(n \sum_j
\delta_j)$ operations, so that the entries of $\mU$ can be computed in
$O(n^2 + E + n \sum_j \delta_j)$ operations. From that, all $p$-minors
of $\mU$ can be deduced in $O({q \choose p} n^3)$ further steps.
To summarize, all polynomials in $\bB$ can be computed by a straight-line
program of size
$E'=O(E + {q \choose p} n^3 +n \sum_i \gamma_i + n \sum_j \delta_j)$.

\medskip

We can then finally apply Proposition~\ref{prop:compute_isolated},
whose runtime is $\softO(c^5 m n^2 + c(e+c^5 n)(E' + n^3))$ operations
in~$\KK$; since $m \le n + {q \choose p}$, this can be simplified as
$$\softO\left (c(e+c^5 n)\left(E + {q \choose p} n^3 +n \sum_i
\gamma_i + n \sum_j \delta_j\right)\right).$$ Since all of $ {q
  \choose p}, \sum_i \gamma_i$ and $\sum_j \delta_j$ are at most $cn$, this becomes $\softO\left (c(e+c^5 n)(E + cn)\right),$
with $c=\gamma_1\cdots\gamma_s \delta_{\rm row}(\mF)$ and
$e=(\gamma_1+1)\cdots(\gamma_s+1) \tilde\delta_{\rm row}(\mF)$.
This is also  $\softO\left (c(e+c^5 )(E + c)\right).$

%% case discussion on e and c^5 n.

%%%%%%%%%%%%%%%%%%%%%%%%%%%%%%%%%%%%%%%%%%%%%%%%%%%%%%%%%%%%
%%%%%%%%%%%%%%%%%%%%%%%%%%%%%%%%%%%%%%%%%%%%%%%%%%%%%%%%%%%%
%%%%%%%%%%%%%%%%%%%%%%%%%%%%%%%%%%%%%%%%%%%%%%%%%%%%%%%%%%%%

\section{The row-degree homotopy}

%%%%%%%%%%%%%%%%%%%%%%%%%%%%%%%%%%%%%%%%%%%%%%%%%%%%%%%%%%%%

\subsection{Preliminaries}

In this subsection, we work with two families of matrices in
$\KK[X_0,\dots,X_t]^{p\times q}$, with $t=q-p+1$. These matrices are
\begin{align}\label{eqdef:type1}
\mN^H= \left( \begin{matrix}
\lambda^H_{1,1} & 0 & \cdots & 0 & \lambda^H_{1,p+1} & \cdots & \lambda^H_{1, q}\\
0 & \lambda^H_{2,2} & \cdots & 0 & \lambda^H_{2,p+1} & \cdots & \lambda^H_{2, q}\\
\vdots & \vdots & \ddots & \vdots & \vdots & \ddots & \vdots\\
0 & 0 & \cdots & \lambda^H_{p,p} & \lambda^H_{p,p+1} & \cdots & \lambda^H_{p, q}
\end{matrix} \right),
\end{align}
and matrices of a more general form such as
\begin{align}\label{eqdef:type2}
\mM^H= \left( \begin{matrix}
\lambda^H_{1,1} & \lambda^H_{1,2} & \cdots & \lambda^H_{1, q}\\
 \lambda^H_{2,1} &  \lambda^H_{2,2} & \cdots & \lambda^H_{2, q}\\
 \vdots & & & \vdots\\
 \lambda^H_{p,1} &  \lambda^H_{p,2}& \cdots & \lambda^H_{p, q}
\end{matrix} \right),
\end{align}
where the ${}^H$ superscript indicates that all entries are
homogenous.  In both cases, for all $i,j$, the entry $\lambda^H_{i,j}$
is a product of $\alpha_i$ homogeneous linear forms in $t+1$ variables
$X_0,\dots,X_t$ with generic coefficients (except when
$\lambda^H_{i,j}$ is explicitly set to zero in the first case), that
is, $\lambda^H_{i,j}=\prod_{k=1}^{\alpha} \lambda^H_{i,j,k}$. In
appendix, we prove the following properties.

\begin{lemma}\label{lemma:appendix}
  For generic choices of the coefficients of the linear forms
  $\lambda^H_{i,j,k}$, the following holds:
  \begin{itemize}
  \item the projective algebraic sets defined in $\P^t(\KKbar)$ by
    $I_p(\mN^H)$ and $I_p(\mM^H)$ have no solution at infinity (that is,
    with $X_0=0$);
  \item the Jacobian matrix of $I_p(\mN^H)$ and $I_p(\mM^H)$ with
    respect to $(X_0,\dots,X_t)$ has rank $t$ at every point of the above
    sets.
\end{itemize}
\end{lemma}
We are interested in describing the projective algebraic set defined
in $\P^t(\KKbar)$ by $I_p(\mN^H)$ (note that these minors are all
homogeous). One easily verifies that $\mN^H$ has rank less than $p$ at
$\tilde \bx \in \P^t(\KKbar)$ if and only if there exists a
subsequence $\bi=(i_1,\dots,i_\kappa)$ of $(1,\dots,p)$ of length
$\kappa\in \{1,\dots,\min(t,p)\}$ such that
$$ \lambda^H_{i_1,i_1}(\tilde\bx) =
\cdots = \lambda^H_{i_\kappa,i_\kappa}(\tilde\bx) = 0
\text{~and~}
\rank(\mN^H_{\bi;(p+1,\dots,q)}(\tilde\bx)) < \kappa.$$
In turn, $\lambda^H_{i_1,i_1},\dots,\lambda^H_{i_\kappa,i_\kappa}$
vanish at $\tilde\bx$ if and only if there exist
$\br=(r_1,\dots,r_\kappa)$, with $r_j$ in $\{1,\dots,\alpha_j\}$ 
for all $j$, such that $\lambda^H_{i_k,i_k,r_k}(\tilde\bx)=0$ 
for $k=1,\dots,\kappa$. For a generic choice of the coefficients 
$\lambda_{i,j,k,\ell}$, the latter linear equations can be rewritten
as 
$$X_{t-\kappa+1}=f_{\bi,\br,t-\kappa+1}(X_0,\dots,X_{t-\kappa}),\dots,
X_{t}=f_{\bi,\br,t}(X_0,\dots,X_{t-\kappa}),$$ for some homogeneous
linear forms $f_{\bi,\br,t-\kappa+1},\dots,f_{\bi,\br,t}$ that depend
on the choice of the indices $\bi$ and $\br$. Applying these
substitutions in the submatrix $\mN^H_{\bi;(p+1,\dots,q)}$ gives us a
matrix $\mM^H_{\bi,\br}$ of size $\kappa \times (t-1)$ whose entries
are products of linear forms, of degrees
$\alpha_{i_1},\dots,\alpha_{i_\kappa}$ on rows $1,\dots,\kappa$. We
deduce that the projective set $V_p(\mN^H)$ is the union of the set
$$V_{\bi,\br}=\{ (x_0,\dots,x_{t-\kappa},f_{\bi,\br,t-\kappa+1}(x_0,\dots,_{t-\kappa}),\dots,
f_{\bi,\br,t}(x_0,\dots,x_{t-\kappa})) \mid (x_0,\dots,x_{t-\kappa}) \in V_\kappa(\mM^H_{\bi,\br}) \}.$$
For generic coefficients, the union is disjoint.


%%%%%%%%%%%%%%%%%%%%%%%%%%%%%%%%%%%%%%%%%%%%%%%%%%%%%%%%%%%%

\subsection{The main algorithm}

We now give an algorithm to solve Problem~\ref{problem2} whose runtime
will depend on the row-degrees of the input matrix $\mF$. This
algorithm is more complex than the one in the previous section, due to
a recursive nature; this boils down to the fact that the start system
we use for the homotopy must itself be solved by means of several
homotopies of smaller size.

Again, we are given a matrix $\mF =[f_{i,j}]\in \KK[X_1,\dots,X_n]^{p
  \times q}$ and polynomials $G=(g_1,\dots,g_s)$ in
$\KK[X_1,\dots,X_n]$, with $p \leq q$ and $n = q-p+s+1$, and we want to
compute the isolated points of $S_{\mF,G}$, with
$$S_{\mF,G} = \{\bx \in \KKbar{}^n \mid \mathrm{rank}(\mF({\bx})) < p
\text{~and~} g_1(\bx)=\cdots=g_s(\bx)=0\}.$$ We are now interested in
designing an algorithm whose cost depends on the row degrees
$\alpha_1=\rdeg(\mF,1),\dots,\alpha_p=\rdeg(\mF,p)$; with this
notation, $\deg(f_{i,j}) \leq \alpha_i$ holds for all $i,j$. As in the
previous section, we write
$\gamma_1=\deg(g_1),\dots,\gamma_s=\deg(g_s)$.

\begin{proposition}\label{prop:rowdegree}
  There exists a randomized algorithm that computes the isolated
  points of $S_{\mF,G}$ using $\softO\left (\ \right)$ operations in
  $\KK$, with $d=\gamma_1\cdots\gamma_s \delta_{\rm row}(\mF)$ and
  $f=(\gamma_1+1)\cdots(\gamma_s+1) \tilde\delta_{\rm row}(\mF)$.
\end{proposition}
%% Remark that if all $\gamma_i$'s and $\delta_j$'s are at least equal to
%% $2$, we have the inequality $e \le c^2$, and the runtime becomes
%% $\softO\left (c^6 (E + c)\right)$.

For any ring $R$ and any matrix $\mM \in R^{p\times q}$, if $S$ is a
subsequence of $(1,\dots,p)$ and $T$ a subsequence of $(1,\dots,q)$,
the $\mM_{S,T}$ is the submatrix obtained by keeping rows indexed by
$S$ and columns indexed by $T$.


% \begin{proof}
%   If $\mN(\by)$ has rank less than~$p$, then some diagonal terms must
%   vanish at $\by$; thus, say
%   $\lambda^H_{i_1,i_1},\dots,\lambda^H_{i_\kappa,i_\kappa}$, , all
%   other ones being non-zero. Thus, we must have $\kappa \ge 1$; on the
%   other hand, since the coefficients of the linear forms making up the
%   polynomials $\lambda^H_{i,i,\bu}$ are random, any system made of
%   more than $n-s$ of these polynomials has no solution. Hence,
%   $\kappa$ is in $\{1,\dots,\min(n-s,p)\}$.

%   Let us write $\bi'=(i'_1,\dots,i'_{p-\kappa})$ for the complement of
%   $\bi$ in $(1,\dots,p)$. Then, for any $\kappa$-minor $\zeta$ of $\mL^H_{\bi;(p+1,\dots,q)}$,
%   the polynomial
%   $\lambda^H_{i'_1,i'_1,\bu}\cdots\lambda^H_{i'_{p-\kappa},i'_{p-\kappa},\bu}\zeta$
%   is a $p$-minor of $\mL^H_\bu$, so that $\zeta(\bx')$ must vanish.
%   As a result, $\mL^H_{\bi;(p+1,\dots,q)}$ has rank less thank $\kappa$.

%   Conversely, if $\lambda^H_{i_1,i_1}(\bx') = \cdots =
%   \lambda^H_{i_\kappa,i_\kappa}(\bx') = 0$ and
%   $\mL^H_{\bi;(p+1,\dots,q)}(\bx')$, the $(\kappa \times q)$-submatrix ${\mL^H_\bu}{}_{\bi;(1,\dots,q)}$ 
%   has rank less than $\kappa$ and $\mL^H_\bu$ has rank less than $p$.
% \end{proof}


\paragraph{Setting up the homotopy.}
Let $\mF$ and $G$ be as in Proposition~\ref{prop:rowdegree}.
To solve Problem~\ref{problem2}, we are again going to rely on the
algorithm of Section~\ref{sec:homotopy}. As in the previous section,
we let $\bC=(c_1,\dots,c_s,\dots,c_m)$ be such that
$(c_1,\dots,c_s)=(g_1,\dots,g_s)$ and $(c_{s+1},\dots,c_m)$ are the
$p$-minors of $\mF$. Our main concern is to design a sequence of
polynomials $\bB=(b_1,\dots,b_s,\dots,b_m)$ in $\KK[T,\bX]$ such that
$\bC=\bB_1$, such that we can solve efficiently the system
$\bA=\bB_0$, and with the same degree profile as our target system
$\bC$.

The polynomials $(b_1,\dots,b_s)$ are defined as in the previous
section, letting $a_i$ be a product of $\gamma_i$ linear forms
$\mu_{i,k}$ with randomly chosen coefficients, of the form
$$a_i=\prod_{k=1}^{\gamma_i} \mu_{i,k},\quad\text{with}\quad
\mu_{i,k}=\mu_{i,k,0} + \sum_{\ell=1}^n \mu_{i,k,\ell}X_\ell.$$ The
difference will lie in the construction of the matrix $\mL$ in
$\KK[\bX]^{p\times q}$, which will then allow us to define
$\mU=(1-T)\,\mL + T\, \mF$, and finally the polynomials
$(b_{s+1},\dots,b_m)$ as the $p$-minors of $\mU$.

The construction of the matrix $\mL$ presented in the previous section
does not carry over if we want to take row degrees into account (one
could imagine a matrix with $i$th row of the form $[1^i
  \lambda_i~\cdots~q^i \lambda_i]$, but the corresponding determinantal
variety has positive dimension). Instead, we use a deformation that
cancels out many off-diagonal terms, and we define
\[ \mL = \left( \begin{matrix}
\lambda_{1,1} & 0 & \cdots & 0 & \lambda_{1,p+1} & \cdots & \lambda_{1, q}\\
0 & \lambda_{2,2} & \cdots & 0 & \lambda_{2,p+1} & \cdots & \lambda_{2, q}\\
\vdots & \vdots & \ddots & \vdots & \vdots & \ddots & \vdots\\
0 & 0 & \cdots & \lambda_{p,p} & \lambda_{p,p+1} & \cdots & \lambda_{p, q}
\end{matrix} \right), \] 
where for all $i,$, $\lambda_{i,j}$ is a product of $\alpha_i$ linear forms with generic
coefficients, of the form
$$\lambda_{i,j}= \prod_{k=1}^{\alpha_i}\lambda_{i,j,k},
\quad\text{with}\quad
\lambda_{i,j,k} =\lambda_{i,j,k,0} + \sum_{\ell=1}^n \lambda_{i,j,k,\ell}X_\ell.
$$ Then, as said above, we define $(b_{s+1},\dots,b_m)$ as the $p$-minors of
$\mU=(1-T)\,\mL + T\, \mF$, and $\bB=(b_1,\dots,b_m)$: the polynomials
$\bA=(a_1,\dots,a_m)$ are defined by letting $T=0$ in $\bB$ (that is,
$\bA=\bB_0$); on the other hand, we also have $\bC=\bB_1$.  Our next
step is to prove that all assumptions of
Propositions~\ref{prop:degree_fiber} and~\ref{prop:compute_isolated}
are satisfied for $\bB$ and $\bA=\bB_0$.

\paragraph{Properties $\assA_1$ and $\assA_2$.}
These follow from Proposition~\ref{prop:KH1H2}.

\paragraph{Property $\assG_1(0)$.} We have to prove that for $i=1,\dots,m$,
$\deg_\bX(b_i)=\deg_\bX(a_i)$. We already established it in the
previous section for indices $i=1,\dots,s$. For $i=s+1,\dots,m$, we
know that the degree of $b_i$ in $\bX$ is at most $\alpha_1 + \cdots +
\alpha_p$, so it suffices to prove that the degree of all $p$-minors
$(a_{s+1},\dots,a_m)$ of $\mL$ is $\alpha_1 + \cdots + \alpha_p$.

Indeed, any $p$-minor of $\mL$ is of the form $\lambda_{i_1,i_1}
\cdots \lambda_{i_\kappa,i_\kappa} \zeta$, for some sequence
$\bi=(i_1,\dots,i_\kappa) \subset (1,\dots,p)$ of length $\kappa \in
\{0,\dots,p\}$ and some $(p-\kappa)$-minor $\zeta$ of $\mL_{\bi
  \mathbf{;}(p+1,\dots,q)}$.  Since the entries of $\mL_{\bi
  \mathbf{;}(p+1,\dots,q)}$ are products of linear form with randomly
chosen coefficients $(\lambda_{i,j,k,\ell})$, for a generic choice of
these coefficients, the determinant $\zeta$ has degree
$\sum_{i' \notin \bi} \alpha_{i'}$. As a result, the corresponding
$p$-minor of $\mL$ has degree $\alpha_1 + \cdots + \alpha_p$, as
claimed.

\paragraph{Property $\assG_2(0)$.} Next, we prove that the system $\bA=\bB_0$ has no solution 
at infinity. As in the previous section, we introduce a homogenization
variable $X_0$, and we consider the system $\bA^H=(a_1^H,\dots,a_s^H,\dots,a_m^H)$ obtained 
by homogenizing all equations in $\bA$. Thus we have
$$a_i^H=\prod_{k=1}^{\gamma_i} \mu^H_{i,k} \quad\text{with}\quad \mu^H_{i,k}=(\mu_{i,k,0}X_0 + \sum_{\ell = 1}^{n}\mu_{i,k,\ell}X_\ell)$$
for $i=1,\dots,s$, whereas $a_{s+1}^H,\dots,a_m^H$ are the $p$-minors of the matrix
\begin{align*}
\mL^H = \left( \begin{matrix}
\lambda^H_{1,1} & 0 & \cdots & 0 & \lambda^H_{1,p+1} & \cdots & \lambda^H_{1, q}\\
0 & \lambda^H_{2,2} & \cdots & 0 & \lambda^H_{2,p+1} & \cdots & \lambda^H_{2, q}\\
\vdots & \vdots & \ddots & \vdots & \vdots & \ddots & \vdots\\
0 & 0 & \cdots & \lambda^H_{p,p} & \lambda^H_{p,p+1} & \cdots & \lambda^H_{p, q}
\end{matrix} \right),  
\end{align*}
where $\lambda^H_{i,k}$ is the homogenization of
$\lambda_{i,j}$. (This latter property requires genericity of the
coefficients of the linear forms $\lambda^H_{i,k}$; it is enough that 
each $p$-minor of $\mL$ have degree $\alpha_1 + \cdots + \alpha_p$.)

The solutions of $\bA^H$ in $\P^n(\KKbar)$ are found by first solving
the equations $(a^H_1,\dots,a^H_s)$. As in the previous section, all
$a_i^H$ are products of linear forms, so any solution of
$(a^H_1,\dots,a^H_s)$ is obtained by setting some of these linear forms
to zero (at least one for each $i=1,\dots,s$). We choose indices $\bu=(u_1,\dots,u_s)$, with
$u_1\in\{1,\dots,\gamma_1\}$, \dots, $u_s\in\{1,\dots,\gamma_s\}$, and
we solve
$$\mu^H_{i,u_i}=0, \quad \text{~that is,~} \quad \mu_{i,u_i,0}X_0 + \sum_{\ell = 1}^{n}\mu_{i,u_i,\ell}X_\ell =0,$$ for $i=1,\dots,s$.
In what follows, we fix such an $\bu$.
Then, for a generic choice of coefficients $\mu_{i,k,\ell}$, these equations
are equivalent to
$$X_{n-s+1}=\mathfrak{h}_{n-s+1,\bu}(X_0,\dots,X_{n-s}),\dots,X_{n}=\mathfrak{h}_{n,\bu}(X_0,\dots,X_{n-s}),$$
for some linear forms $\mathfrak{h}_{n-s+1,\bu},\dots,\mathfrak{h}_{n,\bu}$.
After applying this substitution, for all $i,j$,
$\mL^H$ can be rewritten as 
\begin{align*}
 \mL^H_\bu = \left( \begin{matrix}
\lambda^H_{1,1,\bu} & 0 & \cdots & 0 & \lambda^H_{1,p+1,\bu} & \cdots & \lambda^H_{1, q,\bu}\\
0 & \lambda^H_{2,2,\bu} & \cdots & 0 & \lambda^H_{2,p+1,\bu} & \cdots & \lambda^H_{2, q,\bu}\\
\vdots & \vdots & \ddots & \vdots & \vdots & \ddots & \vdots\\
0 & 0 & \cdots & \lambda^H_{p,p,\bu} & \lambda^H_{p,p+1,\bu} & \cdots & \lambda^H_{p, q,\bu}
\end{matrix} \right),
\end{align*}
with
$$\lambda^H_{i,j,\bu}=\prod_{k=1}^{\alpha_i}\lambda^H_{i,j,k,\bu},
\quad\text{and}\quad \lambda^H_{i,j,k,\bu}=\sum_{\ell =
  0}^{n-s}\lambda_{i,j,k,\ell}X_\ell + \sum_{\ell =
  n-s+1}^{n}\lambda_{i,j,k,\ell}
\mathfrak{h}_{\ell,\bu}(X_0,\dots,X_{n-s}).$$ 

Remark that the entries of $\mL^H_\bu$ are products of homogeneous
linear forms in $(n-s)+s$ variables $(X_0,\dots,X_{n-s})$, so that
this matrix has the form seen in~\eqref{eqdef:type1}. As a result, for
a generic choice of the coefficients $\mu_{i,k,\ell}$ and
$\lambda_{i,j,k,\ell}$, the first item in Lemma~\ref{lemma:appendix}
implies that there is no projective solution to $I_p(\mL^H_\bu)$
satisfying $X_0=0$. Taking into account all possible choices of $\bu$,
we deduce that there is no projective solution to $\bA^H$ satisfying
$X_0=0$, and  $\assG_2(0)$ is proved.

\paragraph{Property $\assG_3(0)$.} Finally, we have to prove that the Jacobian
matrix of $\bA$ has full rank $n$ at any point in $V(\bA) \subset
\KKbar{}^n$. Let thus $\bx=(x_1,\dots,x_n)$ be in $V(\bA)$; in
particular, $\tilde \bx=(1,x_1,\dots,x_n)$ is a projective solution of
$\bA^H$.  Thus, there exists $\bu=(u_1,\dots,u_s)$ as above such that
$$x_{n-s+1}=\mathfrak{h}_{n-s+1,\bu}(1,x_1\dots,x_{n-s}),\dots,x_{n}=\mathfrak{h}_{n,\bu}(1,x_1,\dots,x_{n-s}),$$
and $\mL^H_\bu$ has rank less than $p$ at $\tilde\bx'=(1,x_1,\dots,x_{n-1})$.  The second item of
Lemma~\ref{lemma:appendix} shows that the Jacobian matrix of
$I_p(\mL^H_\bu)$ with respect to $X_0,\dots,X_{n-1}$ has rank $n-s$
at $\tilde \bx'$. Since the first coordinate of $\tilde\bx'$ is non-zero,
and all generating polynomials of $I_p(\mL^H_\bu)$ are homogeneous,
Euler's relation implies that the Jacobian matrix of $I_p(\mL_\bu)$
with respect to $X_1,\dots,X_{n-s}$
has full rank $n-s$ at $\bx'=(x_1,\dots,x_{n-s})$, where  
\begin{align}\label{eqdef:Lu}
 \mL_\bu = \left( \begin{matrix}
\lambda_{1,1,\bu} & 0 & \cdots & 0 & \lambda_{1,p+1,\bu} & \cdots & \lambda_{1, q,\bu}\\
0 & \lambda_{2,2,\bu} & \cdots & 0 & \lambda_{2,p+1,\bu} & \cdots & \lambda_{2, q,\bu}\\
\vdots & \vdots & \ddots & \vdots & \vdots & \ddots & \vdots\\
0 & 0 & \cdots & \lambda_{p,p,\bu} & \lambda_{p,p+1,\bu} & \cdots & \lambda_{p, q,\bu}
\end{matrix} \right),
\end{align}
with
$$\lambda_{i,j,\bu}=\prod_{k=1}^{\alpha_i}\lambda_{i,j,k,\bu},
\quad\text{and}\quad \lambda_{i,j,k,\bu}=\lambda_{i,j,k,0}+\sum_{\ell
  = 1}^{n-s}\lambda_{i,j,k,\ell}X_\ell + \sum_{\ell =
  n-s+1}^{n}\lambda_{i,j,k,\ell}
\mathfrak{h}_{\ell,\bu}(1,X_1,\dots,X_{n-s}).$$ We now  prove
that the Jacobian matrix of $\bA$ with respect to $X_1,\dots,X_n$ has
full rank at $\bx$.

The first step is similar to what we did in the previous section.  For
$i=1,\dots,s$, $a_i$ is a product of linear forms of the form
$a_i=\prod_{k=1}^{\gamma_i} \mu_{i,k}$, with $\mu_{i,u_i}(\bx)=0$.
Since the coefficients $\mu_{i,k,\ell}$ are chosen generically, for
$i=1,\dots,s$ and $k \ne u_i$, $\mu_{i,k}(\bx)$ is non-zero; as a
result, in the local ring at $\bx$, the polynomials $(a_1,\dots,a_s)$
are equal (up to units) to the linear forms
$(\mu_{1,u_1},\dots,\mu_{s,u_s})$. This further implies that
\begin{align}\label{eq:subst}
X_{n-s+1}-\mathfrak{h}_{n-s+1,\bu}(1,X_1,\dots,X_{n-s}),\dots,X_{n}-\mathfrak{h}_{n,\bu}(1,X_1,\dots,X_{n-s})
\end{align}
belong to the ideal generated by $(a_1,\dots,a_s)$ in the local 
ring at $\bx$.

Next, we consider the $p$-minors $(a_{s+1},\dots,a_m)$ of $\mA$. Let
$\zeta \in\KK[X_1,\dots,X_n]$ be a $p$-minor of $\mA$, and let
$\zeta_\bu \in \KK[X_1,\dots,X_{n-s}]$ be the polynomial obtained
after applying the substitution in~\eqref{eq:subst} in $\mA$. Since
$\zeta$ and all polynomials in~\eqref{eq:subst} are in the ideal
$\langle \bA \rangle \cdot \mathcal{O}_\bx$, the polynomial
$\zeta_\bu$ is in this ideal as well. Now, note that $\zeta_\bu$ is a
$p$-minor of $\mL_\bu$ as defined in~\eqref{eqdef:Lu}, and that all
its $p$-minors are obtained this way. We pointed out above that the
Jacobian matrix of these equations with respect to $X_1,\dots,X_{n-s}$
has full rank $n-s$. As a result, taking all $\zeta_\bu$ into account,
together with the equations in~\eqref{eq:subst}, we obtain 
a family of polynomials in $\langle \bA \rangle \cdot \mathcal{O}_\bx$ 
whose Jacobian matrix has rank $n$, and $\assG_3(0)$ is proved.

We can then apply Proposition~\ref{prop:degree_fiber}. Since $\bB$
satisfies $\assA_1,\assA_2$ and $\bA=\bB_0$ satisfies
$\assG_1,\assG_2,\assG_3$, we deduce that the sum of the
multiplicities of the isolated solutions of $\bC=\bB_1$ is at most
$c$, where $c$ is the number of solutions of $\bA$. Our next step
is to establish the value of $c$

\begin{lemma}
  For a generic choice of coefficients $\mu_{i,k,\ell}$ and
  $\lambda_{i,j,k,\ell}$, the cardinality $c$ of the algebraic set
  $V(\mA)$ is $\gamma_1 \cdots \gamma_s
  S_{n-s}(\alpha_1,\dots,\alpha_p)$.
\end{lemma}
\begin{proof}
  For a sequence $\bu=(u_1,\dots,u_s)$ as above, let $V_\bu$ be the
  subset of $V(\bA)$ consisting of all those points $\bx$ such that
  $\mu_{i,u_i}(\bx)=0$ for all $i$. Remark first that the sets $V_\bu$
  are (generically) pairwise disjoint: for $\bx$ in $V_\bu$, any index
  $i$ and any $k \ne u_i$, $\mu_{i,k}(\bx)$ is non-zero.
  
  We will prove in the following paragraphs that for any $\bu$,
  $V_\bu$ has cardinality $S_{n-s}(\alpha_1,\dots,\alpha_p)$; in view of 
  the remark above, this is enough to conclude the proof of the lemma.

  Let us thus fix $\bu=(u_1,\dots,u_s)$. The cardinality of $V_\bu$ is
  equal to the number of points in $V_p(\mL_\bu)$, which is a
  polynomial matrix of size $p \times q$, with entries that are
  products of linear forms in $t=n-s$ variables, with row degrees
  $\alpha_1,\dots,\alpha_p$. We prove by induction that for generic
  choices of the coefficients of these linear forms, $V_\bu$ has
  cardinality $S_{t}(\alpha_1,\dots,\alpha_p)$.
  
  
\end{proof}





% estimate $c$,
% note first that there are $\gamma_1\cdots \gamma_s$ choices of
% $\bu$. For each choice of $\bu$, there are $\delta_{\rm
%   col}(\mF)=E_{n-s}(\delta_1, \ldots, \delta_q)$ ways to choose $\bj$
% and $\bv$, where $E_{n-s}$ denotes the elementary symmetric
% polynomials of degree $n-s$. As a result, $c=\gamma_1\cdots \gamma_s
% \delta_{\rm col}(\mF)$. We can now inspect assumptions
% $\assD_1,\dots,\assD_4$ that are needed to apply the algorithm of
% Proposition~\ref{prop:compute_isolated}.
% To summarize, we have found that the linear equations
% $(\mu_{1,u_1},\dots,\mu_{s,u_s})$ and
% $(\lambda_{j_1,v_1},\dots,\lambda_{j_{n-1},v_{n-s}})$ belong to the
% ideal $\langle \bA \rangle_\m$, where $\m$ is the maximal ideal at
% $\bx$. As a result, the Jacobian matrix of $\mA$ must be invertible
% at $\bx$, and $\assG_3(0)$ holds.






%% Then,
%% $\bx=(x_0,\dots,x_n)$ cancels $a^H_{s+1},\dots,a^H_m$ if and only if
%% $\bx'=(x_0,\dots,x_{n-1})$ cancels the product
%% $\lambda^H_{j_1,\bu}\cdots \lambda^H_{j_p,\bu},$ for any choice of columns
%% $\bj=(j_1,\dots,j_p)$.


%% \paragraph{Property $\assG_3(0)$.}



%% \begin{proof} Without of loss of generality, we prove this result for $(i_1, \ldots, i_k) = (1, \ldots, k)$. This means we have $g_{1}(\bx) = \cdots = g_{k}(\bx) = 0$ and $g_{j}(\bx) \ne 0$ for any $j \in \{k+1, \ldots, p\}$. Let $G_{1:p;*} \in \KK[\bX]^{p \times p}$ be the submatrix of $G$ consisting $p$ rows and the columns $k+1, \ldots, p, j_1, \ldots, j_k$ of $G$, where $p+1 \leq j_i \leq q$. Let $f_{1:p;*}$ is the determinant of $G_{1:p;*}$. Then, for any solution $\bx$ of $\g$, $f_{1:p;*}(\bx) = 0$. Moreover, $f_{1:p;*} = \det(G_{1:p;*}) = g_{k+1,k+1} \ldots g_{p,p} \det(G_{1:k;j_1:j_k})$. Therefore, $\det(G_{1:k;j_1:j_k})(\bx) = 0$. This holds for any $(j_1, \ldots, j_k) \in \{p+1, \ldots, q\}^{k}$. This implies that $\mathrm{rank}(G_{i_1:i_k\mathbf{;}p+1:q}({\bx})) \leq k-1$. 
%% \end{proof}








\bibliographystyle{plain} \bibliography{roadmap}


\section*{Append: proof of~\ref{xxx}}

Fix positive integers $p,q$, with $p \le q$, and let $n=q-p+1$. Let
$\bD=(\delta_1,\dots,\delta_p)$ be positive integers. Let $N=q(n+1)(\delta_1 + \cdots
+ \delta_p)$; this is the number of coefficients needed to define affine
linear forms in $X_1,\dots,X_n$ (or equivalently, homogeneous linear
forms in $X_0,\dots,X_n$), for $i=1,\dots,p$, $j=1,\dots,q$ and
$k=1,\dots,\delta_i$. If needed, we will write $N=N(\bD,q)$ to make the
dependency explicit.


Let then $\mathfrak{ G}$ be the sequence of $N$ indeterminates
$\mathfrak{G}=(\mathfrak{ g}_{i,j,k,r})$, for $i,j,k$ as above and
$r=0,\dots,n$, and define
$$\mathfrak{l}_{i,j,k} = \frak{g}_{i,j,k,0}X_0 + \frak{g}_{i,j,k,1} X_1 +\cdots + \frak{g}_{i,j,k,n} X_n,$$
as well as 
$$\mathfrak{L}_{i,j} = \mathfrak{ l}_{i,j,1} \cdots \mathfrak{l}_{i,j,\delta_i} \in \KK[\mathfrak{G}][\tilde\bX],$$
with $\tilde\bX=(X_0,X_1,\dots,X_n)$. We can then define the
matrix
\begin{align}\label{eq:matM}
\mathfrak{M}_{\bD,q}=\left [\begin{matrix}
\mathfrak{L}_{1,1} & \cdots & \mathfrak{L}_{1,q}\\
 \vdots & & \vdots\\
\mathfrak{L}_{p,1} & \cdots & \mathfrak{L}_{p,q}
  \end{matrix}\right ]\in \KK[\mathfrak{ G}][\tilde\bX]^{p\times q}.  
\end{align}
Remark that for all $i,j$, the $(i,j)$-th entry of
$\mathfrak{M}_{\bD,q}$ has degree $\delta_i$ in $\tilde \bX$.

Given $G=(g_{i,j,k,r})\in \KKbar{}^N$, for any polynomial
$\mathfrak{F}$ in $\KK(\mathfrak{G})[\tilde \bX]$ with entries in
$\KK(\mathfrak{ G})[\tilde \bX]$, we write $\mathfrak{F}(G)$ for the
polynomial obtained by evaluating $\mathfrak{g}_{i,j,k,r}$ at
$g_{i,j,k,r}$, for all indices $i,j,k,r$ as above, as long as no
denominator vanishes through this evaluation; the notation extends to
polynomial matrices. More generally, for a field $\mathfrak{L}$
containing $\KK$, and $\mathfrak{H}$ in $\mathfrak{L}^N$, the notation
$\mathfrak{F}(\mathfrak{H})$ is defined similarly.

Let next $N'=n(n+1)(\delta_1+\cdots+\delta_p)$; as above, when needed, we will
write $N'=N'(\bD,q)$. Let $\mathfrak{ G}'\subset \mathfrak{G}$ be
the sequence of $N'$ indeterminates $\mathfrak{ G}'=(\mathfrak{
  g}_{i,j,k,r})$, for indices $i,j,k,r$ as follows: $i$ is in
$\{1,\dots,p\}$, $j$ is in $\{i,p+1,\dots,q\}$, and as previously, $k$
is in $\{1,\dots,\delta_i\}$ and $r$ is in $\{0,\dots,n\}$. Remark that the
polynomials $\mathfrak{L}_{i,j}$, for $i,j$ as above, are in
$\KK[\mathfrak{G}'][\tilde\bX] \subset \KK[\mathfrak{G}][\tilde \bX]$, and allow
us to define
\begin{align}\label{eq:matMprime}
\mathfrak{N}_{\bD,q}=\left [\begin{matrix} \mathfrak{L}_{1,1} & 0 & 0
    &\mathfrak{L}_{1,p+1} & \cdots & \mathfrak{L}_{1,q}\\ 0 & \ddots &
    0 & \vdots & & \vdots\\ 0&0& \mathfrak{L}_{p,p}
    &\mathfrak{L}_{p,p+1} & \cdots & \mathfrak{L}_{p,q}
  \end{matrix}\right ]\in \KK[\mathfrak{ G}'][\tilde\bX]^{p\times q}.
\end{align}
For $G' \in \KKbar{}^{N'}$, the notation $\mathfrak{F}(G')$ is defined
as in the case of polynomials over $\KK(\mathfrak{G}')$ described
previously.


\paragraph{Setting up the recurrences.}
The basic idea behind most proofs is the following: to prove that a
property holds for a matrix $\mathfrak{M}_{\bD,q}$, we prove that it
holds for a matrix of the form $\mathfrak{N}_{\bD,q}$, and use an
openness property. To prove the required property for the latter
matrices, we proceed by induction, relying on the presence of the
left-hand diagonal block. Indeed, of a matrix such as, say,
$\mathfrak{N}_{\bD,q}$ to be rank-deficient at $\bx \in \P^n(\KKGpbar)$,
one of $\mathfrak{L}_{1,1},\dots,\mathfrak{L}_{p,p}$ must vanish at
$\bx$.

Suppose for instance that
$\mathfrak{L}_{1,1}(\bx)=\mathfrak{L}_{2,2}(\bx)=0$, while all other
terms are non-zero. Then, the $((1,2),(p+1,\dots,q))$-submatrix of
${\mathfrak{M}_{\bD,q}'}(\bx)$ itself must be rank-deficient.  The
constraints $\mathfrak{L}_{1,1}(\bx)=\mathfrak{L}_{2,2}(\bx)=0$ give
us two linear equations, which allow us to eliminate two coordinates
of $\bx$, say $X_{n-1}$ and $X_n$. We can perform the corresponding
substitution in the above submatrix, and we are left with a matrix of
size $2 \times (n-1)$ that is of the form
$\mathfrak{M}_{(\delta_1,\delta_2),n-1}(\mathfrak{H})$, with entries depending
on $X_1,\dots,X_{n-2}$, for some vector of coefficients $\mathfrak{H}$
obtained through the elimination of $X_{n-1}$ and $X_n$. We can then
invoke our induction assumption on the latter matrix.

To formalize this process, for $1 \le m \le m' \le p$ we let
$S_{m,m'}$ be the set of all subsequences of $(1,\dots,p)$ of
cardinality in $\{m,\dots,m'\}$. For $s=(s_1,\dots,s_t)$ in
$S_{m,m'}$, we let $R_s$ be the set of all tuples $r=(r_1,\dots,r_t)$,
with $r_1$ in $\{1,\dots,\delta_{s_1}\}$, \dots, $r_t$ in
$\{1,\dots,\delta_{s_t}\}$; $r_i$ will be the index of the factor
$\mathfrak{l}_{s_i,s_i,r_i}$ of $\mathfrak{L}_{s_i,s_i}$ we
cancel. For given $s=(s_1,\dots,s_t)$, with $t \le n$, and $r$, let
$\mathfrak{G}'_{s,r} \subset \mathfrak{G}'$ be the indeterminates
corresponding to the coefficients of
$\mathfrak{l}_{s_1,s_1,r_1},\dots,\mathfrak{l}_{s_t,s_t,r_t}$, and of
all entries $\mathfrak{L}_{s_1,p+1},\dots,\mathfrak{L}_{s_t,q}$ of the
submatrix associated to $s$ in $\mathfrak{N}_{\bD,q}$.


By Gaussian elimination, we can rewrite the homogeneous linear
equations
$\mathfrak{l}_{s_1,s_1,r_1}=\dots=\mathfrak{l}_{s_t,s_t,r_t}=0$ as
\begin{align}\label{eq:f_sr}
X_{n-t+1}=\mathfrak{f}_{s,r,n-t+1}(X_0,\dots,X_{n-t}),\dots,X_{n}=\mathfrak{f}_{s,r,n}(X_0,\dots,X_{n-t}),  
\end{align}
for some linear forms
$\mathfrak{f}_{s,r,n-t+1},\dots,\mathfrak{f}_{s,r,n}$ of
$(X_0,\dots,X_{n-t})$ with coefficients in
$\KK(\mathfrak{G}'_{s,r})$. Applying this substitution in the entries
of the submatrix of ${\mathfrak{N}_{\bD,q}}$ associated to $s$ gives
us the $t \times (n-1)$ matrix
$\mathfrak{M}_{\b\delta_s,n-1}(\mathfrak{H}_{s,r})$, with
$\b\delta_s=(\delta_{s_1},\dots,\delta_{s_t})$, whose entries are products of linear
forms, and where $\mathfrak{H}_{s,r}$ is a vector of $N$ elements in
$\KK(\mathfrak{G}'_{s,r})$.  The main result we will use in this
section is the following lemma.
\begin{lemma}\label{lemma:union}
  For $k$ in $\{1,\dots,p\}$, $V_p(\mathfrak{N}_{\bD,q})$ is the
  union of the sets
 \begin{align}\label{eq:union}
 \left \{(\tilde\bx',\mathfrak{f}_{s,r,n-t+1}(\tilde\bx'),\dots,\mathfrak{f}_{s,r,n}(\tilde\bx')) \mid \tilde\bx' \in
  V_{t-(p-k)}(\mathfrak{M}_{\b\delta_s,n-1}(\mathfrak{H}_{s,r})) \subset \P^{n-t}(\KKGpbar)\right \},   
 \end{align}
 for $s=(s_1,\dots,s_t)$ in $S_{p-k+1,\min(p,n-1)}$ and $r$ in $R_s$, together with
 $\left \{
 (1,\mathfrak{f}_{s,r,1}(1),\dots,\mathfrak{f}_{s,r,n}(1))\right
 \},$ if $k=p$, with $s=(s_1,\dots,s_n)$ and $r$ in $R_s$.
\end{lemma}
\begin{proof}
  A point $\tilde\bx \in \P^n(\KKGpbar)$ belongs to
  $V_k(\mathfrak{N}_{\bD,q})$ if and only if some diagonal terms of
  $\mathfrak{N}_{\bD,q}$ vanish at $\tilde\bx$, say
  $\mathfrak{L}_{s_i,s_i}$ for $i=1,\dots,t$, and if the
 submatrix of $\mathfrak{N}_{\bD,q}$ associated to $s$ has rank
  less than $t-(p-k)$ at $\tilde \bx$, with $s=(s_1,\dots,s_t)$.  In
  particular, we must have $t-(p-k) > 0$, that is, $t \ge p-k+1$.

  For $i=1,\dots, t$, $\mathfrak{L}_{s_i,s_i}(\tilde\bx)=0$ if and
  only if there exists $r_i$ in $\{1,\dots,\delta_{s_i}\}$ such that
  $\mathfrak{l}_{s_i,s_i,r_i}(\tilde\bx)=0$. Thus, $\tilde\bx=(x_0,\dots,x_n)$ is in
  $V_k(\mathfrak{N}_{\bD,q})$ if and only if there exists
  $s=(s_1,\dots,s_t)$ in $S_{p-k+1,p}$ and $r$ in $R_s$ such that
  $\mathfrak{l}_{s_1,s_1,r_1}(\tilde\bx)=\cdots=\mathfrak{l}_{s_t,s_t,r_t}(\tilde\bx)=0$
  and the submatrix of
  $\mathfrak{N}_{\bD,q}$ associated to $s$ has rank less than $t-(p-k)$ at $\tilde\bx$.  Applying Gaussian
  elimination to the equations
  $\mathfrak{l}_{s_1,s_1,r_1},\dots,\mathfrak{l}_{s_t,s_t,r_t}$  allows us to rewrite them as
  \begin{align*}
    x_{n-t+1}=\mathfrak{f}_{s,r,n-t+1}(\tilde\bx'),\dots,x_{n}=\mathfrak{f}_{s,r,n}(\tilde\bx'),
  \end{align*}
  with $\tilde\bx'=(x_0,\dots,x_{n-t})$.  In particular, $t \le n$,
  since otherwise this linear system would have no solution (recall that 
  the coefficients are algebraically independent indeterminates); hence,
  $s$ is in $S_{p-k+1,\min(n,p)}$. Remark also that $\tilde\bx'$ is a
  well-defined element of $\P^{n-t}(\KKGpbar)$; otherwise, $\tilde\bx$
  would vanish.

  For $s=(s_1,\dots,s_t)$ with $t < n$, applying the above
  substitution in the submatrix of $\mathfrak{N}_{\bD,q}$ associated
  to $s$ (which has size $t \times (n-1)$), the rank condition above
  becomes that $\mathfrak{M}_{\b\delta_s,n-1}(\mathfrak{H}_{s,r})$ has rank
  less than $t-(p-k)$ at $\tilde \bx'$, that is, $\tilde \bx'$ is in
  $V_{t-(p-k)}(\mathfrak{M}_{\b\delta_s,n-1}(\mathfrak{H}_{s,r}))$.  

  When $t=n$, that is, $s=(s_1,\dots,s_n)$, the linear equations above
  determine $\tilde\bx$ entirely; setting $x_0=1$, we obtain
  $x_{1}=\mathfrak{f}_{s,r,1}(1),\dots,x_{n}=\mathfrak{f}_{s,r,n}(1).$
  In this case, the submatrix of $\mathfrak{N}_{\bD,q}$ associated to
  $s$ has size $n \times (n-1)$. Using the specialization of the
  coefficients that sets the off-diagonal entry to $0$ and the $i$th
  diagonal entries to $X_0^{\delta_i}$, $i=1,\dots,n-1$, we see that its
  evaluation at $\tilde\bx$ has rank $n-1$; as a result
  $\mathfrak{N}_{\bD,q}$ has rank $p-1$ at $\tilde \bx$. Thus, we need
  to take $t=n$ into account only if $k=p$; in this case, we add the
  point $\left \{
  (1,\mathfrak{f}_{s,r,1}(1),\dots,\mathfrak{f}_{s,r,n}(1))\right \}$
  to our list.
\end{proof}


\paragraph{Solutions with higher rank defect.} 
First, we discuss the case $k=p-1$.  We take parameters
$\bD=(\delta_1,\dots,\delta_p)$ and $q$, with $2 \le p \le q$, and we write
$N=N(\bD,q)$ and $N'=N'(\bD,q)$; we will establish the following
properties.
\begin{description}[leftmargin=*]
\item[$\assI_1(\bD,q).$] The projective algebraic set  $V_{p-1}(\mathfrak{M}_{\bD,q})\subset\P^n(\KKGbar)$
  is empty.
\item[$\assJ_1(\bD,q).$] The projective algebraic set 
 $V_{p-1}(\mathfrak{N}_{\bD,q})\subset\P^n(\KKGpbar)$ is empty.
\end{description}
The first step of the proof is to establish that for $\bD$ and $q$ as above,
$\assJ_1(\bD,q)$ implies $\assI_1(\bD,q)$. Let us thus fix $\bD$
and $q$.  Assumption $\assJ_1(\bD,q)$ implies that
$V_{p-1}(\mathfrak{N}_{\bD,q}(G'))$ is empty for a generic $G'$
in~$\KKbar{}^{N'}$.  We will prove that
$V_{p-1}(\mathfrak{M}_{\bD,q}(G))$ is empty for a generic $G$
in~$\KKbar{}^N$, which in turn establishes $\assI_1(\bD,q)$.

Consider the ideal $I_{p-1}(\mathfrak{M}_{\bD,q})$ in the polynomial
ring $\KK[\mathfrak{G},\tilde\bX]$ in $N+n+1$ variables. This ideal
defines an algebraic set $Z_{\bD,q}$ in $\KKbar{}^N \times
\P^n(\KKbar)$, and we let $\Delta_{\bD,q} \subset \KKbar{}^N$ be its
projection on the first factor: this is the set of all $G$ such that
$V_{p-1}(\mathfrak{M}_{\bD,q}(G))$ is not empty. Because the source is
a projective space, $\Delta_{\bD,q}$ is closed (so its complement is
open), and we just have to verify that it is not equal to the whole
$\KKbar{}^N$. This follows readily from property $\assJ_1(\bD,q)$,
which proves that generic matrices of the form
$\mathfrak{N}_{\bD,q}(G')$ do not belong to $\Delta_{p,q}$, so 
$\assI_1(\bD,q)$ holds.

We finish the proof by induction. We first take $p=q$ and consider
$\assJ_1(\bD,q)$.  In this case, $n=1$ and $\mathfrak{N}_{\bD,q}$ is
a diagonal matrix, whose diagonal entries are products of linear forms
in $(X_0,X_1)$ with indeterminate coefficients. Hence, no pair of
entries $\mathfrak{N}_{\bD,q}$ have any common solution in
$\P^1(\KKGpbar)$, so the rank of $\mathfrak{N}_{\bD,q}$ is at least
$p-1$. Hence, $\assJ_1(\bD,p)$ holds, and so does $\assI_1(\bD,p)$, by the claim in the previous paragraph.

Consider next a pair $(\bD,q)$, with $\bD=(\delta_1,\dots,\delta_p)$ and $2 \le
p \le q$, and suppose that $\assI_1(\bD',q')$ holds for all
$(\bD',q')$ with $\bD'=(D'_1,\dots,D'_{p'})$, $2 \le p' \le q'$, $p'
\le p$ and $q' < q$; we prove that $\assJ_1(\bD,q)$ holds
(as above, this will also imply $\assI_1(\bD,q)$).

Take $k=p-1$ in Lemma~\ref{lemma:union}. Then, the parameters
$(t-(p-k),\b\delta_s,n-1)$ used in each contribution to~\eqref{eq:union}
are of the form $(t-1,\b\delta_s,n-1)$, with $2 \le t \le \min(p,n-1)$.
Since the entries of $\mathfrak{H}_{s,r}$ are algebraically
independent, $\b\delta_s$ has length $t\ge 2$, with also $t \le n-1$, $t
\le p$ and $n-1 < q$, we can apply the induction hypothesis and deduce
that all $V_{t-1}(\mathfrak{M}_{\b\delta_s,n-1}(\mathfrak{H}_{s,r}))$
appearing in Lemma~\ref{lemma:union} are empty. As a consequence,
$V_{p-1}(\mathfrak{N}_{\bD,q})$ is empty, as claimed.


\paragraph{Solutions at infinity.} Next, we focus on the case $k=p$.
We take parameters $\bD=(\delta_1,\dots,\delta_p)$ and $q$, with $1 \le p \le
q$, and we write $N=N(\bD,q)$ and $N'=N'(\bD,q)$; then, we prove the following properties.
\begin{description}[leftmargin=*]
\item[$\assI_2(\bD,q).$] The projective algebraic set
  $V_p(\mathfrak{M}_{\bD,q}) \subset \P^n(\KKGbar)$ has no point with $X_0=0$.
\item[$\assJ_2(\bD,q).$] The projective algebraic set
  $V_p(\mathfrak{N}_{\bD,q}) \subset \P^n(\KKGpbar)$ has no point with $X_0=0$.
\end{description}
 In particular, this implies that these sets are finite. We will prove
 these properties as we did in the previous paragraph; the first step
 is thus to establish that for $\bD$ and $q$ as above, $\assJ_2(\bD,q)$
 implies $\assI_2(\bD,q)$.

Let us thus fix $\bD$ and $q$, and assume that $\assJ_2(\bD,q)$
holds. We prove that $V_p(\mathfrak{M}_{\bD,q}(G))$ has no point at
infinity for a generic $G$ in $\KKbar{}^N$; this will imply $\assI_2(\bD,q)$. Consider the ideal generated by
$I_{p}(\mathfrak{M}_{\bD,q})$ and $X_0$ in the polynomial ring
$\KK[\mathfrak{G},\tilde\bX]$ in $N+n+1$ variables. This ideal defines
an algebraic set $Z'_{\bD,q}$ in $\KKbar{}^N \times \P^n(\KKbar)$, and
we let $\Delta'_{\bD,q} \subset \KKbar{}^N$ be its projection on the
first factor: this is thus the set of all $G$ in $\KKbar{}^N$ such
that $V_p(\mathfrak{M}_{\bD,q}(G))$ has a point at infinity. Because
the source is a projective space, $\Delta'_{p,q}$ is closed (so its
complement is open), and we just have to verify that it is not equal
to the whole $\KKbar{}^N$. This follows from property $\assJ_2(\bD,q)$, which implies that matrices of the form
$\mathfrak{N}_{\bD,q}(G')$, for generic $G'$ in $\KKbar{}^{N'}$, do
not belong to $\Delta'_{\bD,q}$.

Again, we finish the proof by induction. We first take $p=q$, and we
prove that $\assJ_2(\bD,q)$ holds ($\assI_2(\bD,q)$ will follow,
by the previous paragraph). In this case, $n=1$ and
$\mathfrak{N}_{\bD,q}$ is a diagonal matrix, whose diagonal entries
are products of linear forms in $(X_0,X_1)$ with indeterminate
coefficients. Then, $\mathfrak{N}_{\bD,q}$ has rank less than $p$ at
$\tilde\bx\in\P^1(\KKGpbar)$ if and only if one of the linear factors
of some diagonal term vanishes at $\tilde \bx$. None of this linear
forms has a projective root at infinity, so we are done.

Consider next a pair $(\bD,q)$, with $\bD=(\delta_1,\dots,\delta_p)$ and $1 \le p
\le q$ and suppose that $\assI_2(\bD',q')$ holds for all $(\bD',q')$
with $\bD'=(D'_1,\dots,D'_{p'})$, $1 \le p' \le q'$, $p' \le p$ and $q' <
q$; we prove that $\assJ_2(\bD,q)$ holds; as above, this will imply
 $\assI_2(\bD,q)$.

Take $k=p$ in Lemma~\ref{lemma:union}. We first deal with the last
contribution, corresponding to $s=(s_1,\dots,s_n)$, and thus $t=n$: by
design, the corresponding point is not at infinity. For the other
contributions, the parameters $(t-(p-k),\b\delta_s,n-1)$ used
in~\eqref{eq:union} are of the form $(t,\b\delta_s,n-1)$, with $\b\delta_s$ of
length $t \in \{1,\dots, \min(p,n-1)\}$; since all conditions $ 1 \le
t \le n-1$, $t \le p$ and $n-1 < q$ are satisfied, we can invoke the
induction assumption. Since the coefficients $\mathfrak{H}_{s,r}$ are
algebraically independent, we deduce that none of the
projective sets $V_t(\mathfrak{M}_{\b\delta_s,n-1}(\mathfrak{H}_{s,r}))$
appearing in Lemma~\ref{lemma:union} has any point with $X_0=0$. As a
consequence, $V_p(\mathfrak{N}_{\bD,q})$ has no point at infinity
either, as claimed.


\paragraph{Refining $\assI_1$.} The following  is a strengthening of 
property $\assI_1$ above. Again, we consider $\bD=(\delta_1,\dots,\delta_p)$
and $q$, with $1 \le p \le q$, together with a matrix
$\mathfrak{m}_{\bD,q}$, built as $\mathfrak{M}_{\bD,q}$ before, but
using products of homogeneous linear forms in $n=q-p+1$ variables $X_0,\dots,X_{n-1}$, instead of
$n+1$ variables $X_0,\dots,X_n$. Such a matrix takes the form
\begin{align}\label{eq:matM2}
\mathfrak{m}_{\bD,q}=\left [\begin{matrix}
\mathfrak{P}_{1,1} & \cdots & \mathfrak{P}_{1,q}\\
 \vdots & & \vdots\\
\mathfrak{P}_{p,1} & \cdots & \mathfrak{P}_{p,q}
  \end{matrix}\right ]\in \KK[\mathfrak{C}][X_0,\dots,X_{n-1}]^{p\times q},
\end{align}
with 
$$\mathfrak{p}_{i,j,k} = \frak{c}_{i,j,k,0}X_0 + \frak{c}_{i,j,k,1} X_1 +\cdots + \frak{c}_{i,j,k,n-1} X_{n-1},$$
and
$$\mathfrak{P}_{i,j} = \mathfrak{p}_{i,j,1} \cdots
\mathfrak{p}_{i,j,\delta_i} \in \KK[\mathfrak{C}][X_0,\dots,X_{n-1}],$$
where $\mathfrak{C}=(\mathfrak{c}_{i,j,k,\ell})$ are indeterminates,
for $i=1,\dots,p$, $j=1,\dots,q$, $k=1,\dots,\delta_i$ and
$\ell=0,\dots,n-1$; we let $M=q n (\delta_1+\cdots +\delta_p)$ be the total
number of coefficients $\mathfrak{c}_{i,j,k,\ell}$ involved. In this context, the
following property could be proved by induction as in the other cases,
but a direct proof is available.

\begin{description}[leftmargin=*]
\item[$\assI_3(\bD,q).$] The projective algebraic set
  $V_p(\mathfrak{m}_{\bD,q}) \subset \P^{n-1}(\KKCbar)$ is empty.
\end{description}
To prove this property, take $\bD$ and $q$ as above. If $q=p$, we have
$n=1$, so the $i,j$ entry of $\mathfrak{m}_{\bD,q}$ has the form
$\mathfrak{c}_{i,j,1,0}\cdots\mathfrak{c}_{i,j,1,\delta_i} X_0^{\delta_i}$; hence, its determinant is non-zero,
and the claim follows.

We can thus suppose $q > p$, so that $q-1 \ge p$.  Then, the
$((1,\dots,p),(1,\dots,q-1))$-submatrix of $\mathfrak{m}_{\bD,q}$ is
of the form $\mathfrak{M}_{\bD,q-1}$, with entries depending on
$N(\bD,q-1)$ parameters.  Let $(\Delta_i)_{i \in I}$ be the $p$-minors
of $\mathfrak{m}_{\bD,q}$ built by taking $p-1$ of the first $q-1$
columns of $\mathfrak{m}_{\bD,q}$, together with its last column.
Any such minor can be expanded along the last column as $\Delta_i =
\mathfrak{P}_{1,q} \delta_{i,1} + \cdots + \mathfrak{P}_{p,q}
\delta_{i,p}$, where $\mathfrak{P}_{1,q},\dots,\mathfrak{P}_{p,q}$ are
the entries of the last column, and $\delta_{i,1},\dots,\delta_{i,p}$
are $(p-1)$-minors from $\mathfrak{M}_{\bD,q-1}$. Remark that
$(\delta_{i,j})_{i \in I, 1 \le j \le p}$ are {\em all} $(p-1)$-minors
of $\mathfrak{M}_{\bD,q-1}$ (if $p=1$, we have $I=\{1\}$ and
$\Delta_1=\mathfrak{P}_{1,q}$, with $\delta_{1,1}=1$).

By $\assI_2(\bD,q-1)$, we deduce that $V_p(\mathfrak{M}_{\bD,q-1})
\subset \P^{n-1}(\KKCbar)$ is finite. For all other points $\tilde\bx$
in $\P^{n-1}(\KKCbar)$, $\mathfrak{M}_{\bD,q-1}$ has full rank $p$ at
$\tilde\bx$, and thus so does $\mathfrak{m}_{\bD,q}$. Hence, we can we
can focus on the points in $V_p(\mathfrak{M}_{\bD,q-1})$.  Consider a
point $\tilde\bx$ in this set (in particular, by $\assI_2(\bD,q-1)$,
we can take it first coordinate $x_0$ equal to $1$). Using $\assI_1(\bD,q-1)$, together with our remark on the minors of
$\mathfrak{M}_{\bD,q-1}$, we deduce that not all minors
$(\delta_{i,j})_{i \in I, 1 \le j \le p}$ vanish at
$\tilde\bx$. Suppose thus that $\delta_{i_0,j_0}(\tilde\bx) \ne 0$; we
prove that $\Delta_{i_0}(\tilde\bx) \ne 0$.

Let us split the $M$ indeterminates $\mathfrak{C}$ into
$\mathfrak{C}_1$ and $\mathfrak{C}_2$, where $\mathfrak{C}_1$ has
cardinality $M_1=N(\bD,q-1)$ and corresponds to the coefficients used
in the entries $\mathfrak{P}_{1,1},\dots,\mathfrak{P}_{p,q-1}$ in
$\mathfrak{m}_{\bD,q}$, and $\mathfrak{C}_2$ of cardinality
$M_2=M-M_1$ stands for the coefficients of the entries
$\mathfrak{P}_{1,q},\dots,\mathfrak{P}_{p,q}$ in the last column of
$\mathfrak{m}_{\bD,q}$.  Let us further write
$\Delta_{i_0}(\tilde\bx)= \mathfrak{P}_{1,q}(\tilde\bx)
\delta_{i_0,1}(\tilde\bx) + \cdots + \mathfrak{P}_{p,q}(\tilde\bx)
\delta_{i_0,p}(\tilde\bx)$.  Since $x_0=1$, the
polynomial $\mathfrak{P}_{i,q}(\tilde\bx)$ admits
$\mathfrak{c}_{i,q,1,0}\cdots \mathfrak{c}_{i,q,\delta_i,0}$ 
as a specialization (setting all other coefficients to zero), so that
$\Delta_{i_0}(\tilde\bx)$ admits 
$$\mathfrak{c}_{1,q,1,0}\cdots \mathfrak{c}_{1,q,\delta_i,0}
\delta_{i_0,1}(\tilde\bx) + \cdots + \mathfrak{c}_{p,q,1,0}\cdots
\mathfrak{c}_{p,q,\delta_i,0} \delta_{i_0,p}(\tilde\bx)$$ as a
specialization. The coefficients
$\delta_{i_0,j}(\tilde\bx)$ are not all zero, and since
$V_p(\mathfrak{M}_{\bD,q-1})$ is finite, algebraic over
$\KK(\mathfrak{C}'_1)$. Since the entries of $\mathfrak{C}'_2$ are
algebraically independent over $\KK(\mathfrak{C}'_1)$, the sum
$\Delta_{i_0}(\tilde\bx)$ is non-zero, as claimed.

\paragraph{Multiplicity of the solutions.} 
Again, we take parameters $\bD=(\delta_1,\dots,\delta_p)$ and $q$, with $1 \le p
\le q$, and we write $N=N(\bD,q)$ and $N'=N'(\bD,q)$; we will prove
consider the following properties.
\begin{description}[leftmargin=*]
\item[$\assI_4(\bD,q).$] The Jacobian matrix of
  $M_p(\mathfrak{M}_{\bD,q})$ has rank $n$ at all points in
  $V_p(\mathfrak{M}_{\bD,q})$.
\item[$\assJ_4(\bD,q).$] The Jacobian matrix of
  $M_p(\mathfrak{N}_{\bD,q})$ has rank $n$ at all points in
  $V_p(\mathfrak{N}_{\bD,q})$.
\end{description}
As for other proofs involving both $\mathfrak{M}_{\bD,q}$ and
$\mathfrak{N}_{\bD,q}$, we first show that $\assJ_4(\bD,q)$ implies
$\assI_4(\bD,q)$.

We fix $\bD$ and $q$, and we assume that $\assJ_4(\bD,q)$
holds. Consider the ideal of the polynomial ring
$\KK[\mathfrak{G},\tilde\bX]$ in $N+n+1$ variables generated by the
$p$-minors of $\mathfrak{M}_{\bD,q}$, together with the $n$-minors of
the Jacobian matrix of these equations with respect to
$X_0,\dots,X_n$. This ideal defines an algebraic set $Z''_{\bD,q}$ in
$\KKbar{}^N \times \P^n(\KKbar)$, and we let $\Delta''_{\bD,q} \subset
\KKbar{}^N$ be its projection on the first factor. By construction,
for $G$ in $\KKbar{}^N-Z''_{\bD,q}$, the Jacobian matrix of
$M_p(\mathfrak{M}_{\bD,q}(G))$ has rank $n$ at any $\bx$ in
$V_p(\mathfrak{M}_{\bD,q}(G))$. As before, because the source is a
projective space, $\Delta''_{p,q}$ is closed (so its complement is
open), and we just have to verify that it is not equal to the whole
$\KKbar{}^N$. This follows from property $\assJ_4(\bD,q)$, which
proves that generic matrices of the form $\mathfrak{N}_{\bD,q}$ do not
belong to $\Delta''_{\bD,q}$.

Again, we finish the proof by induction. We first take $p=q$, and we
prove that $\assJ_4(\bD,q)$ holds ($\assI_4(\bD,q)$ will follow,
by the previous paragraph). In this case, $n=1$ and
$\mathfrak{N}_{\bD,q}$ is a diagonal matrix, whose diagonal entries
are products of linear forms $\mathfrak{L}_{i,i}$ depending on
$(X_0,X_1)$ and with indeterminate coefficients. For $G'$ in
$\KKbar{}^N$, the ideal $I_p(\mathfrak{N}_{\bD,q}(G'))$ is generated by
the product of the terms $\mathfrak{L}_{i,i}(G')$; for generic $G'$, it
admits no repeated factors, and the conclusion follows.

Consider next a pair $(\bD,q)$, with $\bD=(\delta_1,\dots,\delta_p)$ and $1 \le p
\le q$ and suppose that $\assI_4(\bD',q')$ holds for all $(\bD',q')$
with $\bD'=(D'_1,\dots,D'_{p'})$, $1 \le p' \le q'$, $p' \le p$ and $q' <
q$; we prove that $\assJ_4(\bD,q)$ holds; this will imply
 $\assI_4(\bD,q)$. 

We take $k=p$ in the formula of the lemma, and we first deal with the
terms in~\eqref{eq:union}.  Thus, we choose a subsequence
$s=(s_1,\dots,s_t)$ of $(1,\dots,p)$, with $1 \le t\le \min(p,n-1)$,
and indices $R=(r_1,\dots,r_t)$, with $ 1\le r_i \le \delta_{s_i}$ for all
$i$; for simplicity, we write the proof with $s=(1,\dots,t)$, so that
we have $\b\delta_s=(\delta_1,\dots,\delta_t)$. We take $\tilde\bx=(x_0,\dots,x_n)$
in $\P^n(\KKGpbar)$ such that $\tilde \bx'=(x_0,\dots,x_{n-t})$
is in $V_t(\mathfrak{M}_{\b\delta_s,n-1}(\mathfrak{H}_{s,r})) \subset
\P^{n-t}(\KKGpbar)$, and such that 
\begin{align}\label{eq:subsX}
  x_{n-t+1}=\mathfrak{f}_{s,r,n-t+1}(x_0,\dots,x_{n-t}),\dots,x_{n}=\mathfrak{f}_{s,r,n}(x_0,\dots,x_{n-t});
\end{align}
we then prove that the Jacobian matrix of $M_p(\mathfrak{N}_{\bD,q})$
with respect to $X_0,\dots,X_n$ has rank $n$ at $\tilde\bx$.  By
Lemma~\ref{lemma:union}, taking all such $\tilde\bx$ into account, for
all $s$ and $r$, will cover all points in $V_p(\mathfrak{N}_{\bD,q})$, up to the
exception of those points obtained from $t=n$, which will admit a
simpler treatment.

We are going to exhibit some ``nice'' polynomials that belong to
$I_p(\mathfrak{N}_{\bD,q})$, for which we can control the rank of the
Jacobian at $\tilde\bx$. First, we prove that for $i\in\{1,\dots,t\}$
and $r \in \{1,\dots,\delta_i\}-\{r_i\}$, as well as $i\in\{t+1,\dots,p\}$
and $r \in \{1,\dots,\delta_i\}$, $\mathfrak{l}_{i,i,r}(\tilde\bx)$ is
non-zero.  We subdivide the coordinates $\mathfrak{G}'$ into
$\mathfrak{G}'_1$ and $\mathfrak{G}'_2$, where $\mathfrak{G}'_1$
corresponds to the coefficients involved in $\mathfrak{l}_{i,i,r_i}$,
for $i=1,\dots,t$, and in the submatrix of $\mathfrak{N}_{\bD,q}$
associated to $s$, and $\mathfrak{G}'_2$ are the other coordinates.
By $\assI_2(\b\delta_s,n-1)$,  $V_t(\mathfrak{M}_{\b\delta_s,n-1}(\mathfrak{H}_{s,r}))$ 
is finite; as a result, all coordinates of $\tilde\bx$ are algebraic
over $\KK(\mathfrak{G}'_1)$. For $i,r$ as described above, the coefficients of the equation
$$\mathfrak{l}_{i,i,r} = \frak{g}_{i,i,r,0} X_0+ \frak{g}_{i,i,r,1}
X_1 +\cdots + \frak{g}_{i,i,r,n} X_n$$ are thus algebraically independent
over the field of definition of $\bx'$, so that $\mathfrak{l}_{i,i,r}(\tilde\bx)$
is non-zero.

Next, we take $i$ in $\{1,\dots,t\}$. We can then define
$s^*=(1,\dots,i-1,i+1,\dots,t)$,
$\bD^*=(\delta_1,\dots,\delta_{i-1},\delta_{i+1},\dots,\delta_t)$, and we call
$\mathfrak{N}_{\bD,q}^*$ the submatrix of
$\mathfrak{N}_{\bD,q}$ associated to $s^*$. We prove that there exists a $(t-1)$-minor
$\delta_i$ of $\mathfrak{N}_{\bD,q}^*$ such that $\delta_i(\tilde\bx)\ne 0$.  Let indeed
$\mathfrak{H}^*$ be the coefficients of the linear forms obtained by
applying the substitution~\eqref{eq:subsX} in
$\mathfrak{N}_{\bD,q}^*$. Since the coefficients of $\mathfrak{H}^*$
are algebraically independent over $\KK$, we can apply $\assI_3(\bD^*,n-1)$ to $\mathfrak{N}_{\bD,q}^*(\mathfrak{H}^*)$,
and deduce that this matrix has full rank $t-1$ at $\tilde\bx'$.
Thus, $\mathfrak{N}_{\bD,q}^*$ has rank $t-1$ at $\tilde\bx$, 
from which the existence of the minor $\delta_i$ follows.

We next show that these properties imply that for $i\in\{1,\dots,t\}$,
there exists a polynomial of the form $B_{i} \mathfrak{l}_{i,i,r_i}$
in the ideal $I_p(\mathfrak{N}_{\bD,q})$, with $B_{i}(\tilde\bx)\ne
0$. Indeed, we consider the $p$-minor of $\mathfrak{N}_{\bD,q}$
obtained by taking the columns $i$,$t+1,\dots,p$, and all $t-1$ columns
in $\delta_i$. Using the factorization
$$\mathfrak{L}_{i,i} = b_i \mathfrak{l}_{i,i,r_i},\quad\text{with}\quad
b_i=\mathfrak{l}_{i,i,1}\cdots \mathfrak{l}_{i,i,r_i-1}\mathfrak{l}_{i,i,r_i+1}\cdots \mathfrak{l}_{i,i,\delta_i},$$
that minor evaluates to 
$$B_i \mathfrak{l}_{i,i,r_i}\quad\text{with}\quad B_i = b_i
\mathfrak{L}_{t+1,t+1}\cdots \mathfrak{L}_{p,p}\delta_i.$$ Hence, $B_i\,
\mathfrak{l}_{i,i,r_i}$ belongs to $I_p(\mathfrak{N}_{\bD,q})$, and by
the discussion of the two previous paragraphs, $B_i(\tilde\bx)\ne 0$,
as claimed.

For the same reasons, for every $t$-minor $\eta$ of the submatrix of
$\mathfrak{N}_{\bD,q}$ associated to $s$, there exists a polynomial
$C_\eta$ such that $C_\eta\, \eta$ belongs to
$I_p(\mathfrak{N}_{\bD,q})$, and $C_\eta(\tilde\bx)\ne 0$. 
Using the
fact that (up to the factors $B_i$) the equations
$\mathfrak{l}_{i,i,r_i}$ are in $I_p(\mathfrak{N}_{\bD,q})$, we deduce
the existence of cofactors $\tilde C_\eta$ such that $C_\eta\, \eta(\mathfrak{H}_{s,r})$
belongs to $I_p(\mathfrak{N}_{\bD,q})$, and $C_\eta(\tilde\bx)\ne 0$.




\end{document}
